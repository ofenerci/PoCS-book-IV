%% add this:
%% http://www.babycenter.com/0_unusual-baby-names-of-2012_10375911.bc

%% Add section on dragon kings
%% 
%% Add section on stable distributions
%% 
%% Add an assignment question or two
%% 
%% convolve distributions
%% 
%% Cauchy
%% 
%% Inverse Gaussian

%% add zipf's law from
%% estoup, 1912

%% add polya urn stuff

%% add stuff from clauset's lectures?

%% todo
%% 
%% good figure...
%% http://www.noop.nl/2009/07/your-project-will-suffer-from-power-laws.html

%% add gaussian distribution
%% 

%% have students sketch the curve

%% Cancer?
%% http://mskcc.convio.net/pdf/cycle_for_survival/cfs_cancer_fact_sheet1.pdf
%% Maybe make this an assignment question

%%%%%%%%%%%%%%%%%%%%%%%%%%%%%%%%%%%%%%%%%%%%%%%%%%%%%%%%%%%%%%%%%%%%%%%%%%%%

We (The Humans)\footnote{Whenever someone starts talking about humans ...}
have many aptitudes.
We are unparalleled at running long distances,
and we can throw with remarkable power and accuracy.


We can guess~\cite{FACES,deceptionpaper}
An experience visual-story-consumer can 
often predict what's about to happen in a movie,

We are also very good at saying how very good we are,
with a strong tradition in explaining how 
much better than animals from which we are so clearly separate.

Slowly, we've awakened to the brilliance of others.
Octopuses\footnote{Look: this sounds better than Octopi which really should
  be Octopodes} are worryingly clever\footnote{If they ever stop
  eating each other, we're in a lot of trouble.\footnote{Possibly what
the dinosaurs said about mammals}}

But let's make an admission: We (humans) are not especially
good with thinking probabilistically.
Sure, we have skill in inference.




We are 



While multiplying many digit numbers together

\section{Our\ Intuition}

  \textbf{Let's test our collective intuition:}

  
      
      
      \includegraphics[width=\textwidth]{Onedolar2009series.png}
      
      Money \\
      $\equiv$ \\
      Belief
      
      

  \textbf{Two questions about wealth distribution in the United States:}
    
    
      Please estimate the percentage of all wealth 
      owned by individuals when grouped into quintiles.
    
      Please estimate what you believe each quintile
      should own, ideally.
    
      Extremes: 100, 0, 0, 0, 0 and 20, 20, 20, 20, 20
    
  
  


  \textbf{\small Wealth distribution in the United States:\cite{norton2011a}}
    \includegraphics[width=\textwidth]{norton2011a_fig2-tp-10}
  
  \small
  ``Building a better America---One wealth quintile at a time''\newline
  Norton and Ariely, 2011.\cite{norton2011a}
  


  \textbf{\small Wealth distribution in the United States:\cite{norton2011a}}
    \includegraphics[height=0.8\textheight]{norton2011a_fig3-tp-10}
    \small
    \small
    A highly watched video based on this research is
    \wordwikilink{http://www.youtube.com/watch?v=QPKKQnijnsM}{here.}
  
  


      
    
    Actual:\\
    {\small Fall 2013:}\\
    \includegraphics[width=\textwidth]{figquintiles2013-08_001_1_noname.pdf}\\
    {\small Spring 2013:}\\
    \includegraphics[width=\textwidth]{figquintiles001_1_noname.pdf}\\
    
    
    
    Ideal:\\
    {\small Fall 2013:}\\
    \includegraphics[width=\textwidth]{figquintiles2013-08_001_2_noname.pdf}\\
    {\small Spring 2013:}\\
    \includegraphics[width=\textwidth]{figquintiles001_2_noname.pdf}\\
    
  
\changelogo{.18}{2013-01-17power-law-cartoons001-tp-3.pdf}

\section{Definition}

%%  \textbf{Size distributions:}
  
  
    The sizes of many systems' elements
    appear to obey an\\
    \alertb{inverse power-law 
      size distribution}:
    $$P(\mbox{size}=x) \sim  c\, x^{-\gamma}$$
    $$
    \mbox{where} 
    \quad 
    0 < x_{\textrm{min}} < x < x_{\textrm{max}}
    \quad 
    \mbox{and} 
    \quad 
    \gamma > 1.
    $$
    
    
      Exciting class exercise: sketch this function.
    
  

  
    
    
      $x_{\textrm{min}}$ = lower cutoff, 
      $x_{\textrm{max}}$ = upper cutoff
    
      Negative linear relationship in log-log space:
      $$ \log_{10} P(x) = \log_{10} c - \gamma \log_{10} x $$
    
      We use base 10 because we are \alertg{good people}.
    
  

  
    
    
      power-law decays in probability:  \\
      The \alertb{Statistics of Surprise}.
    
  


  \textbf{Size distributions:}

  \textbf{Usually, only the tail of the distribution obeys a power law:}
    $$\alertb{P(x) \sim  c\, x^{-\gamma}} \ \mbox{for}\ x \ \mbox{large}.$$
    
     
      Still use term `power-law size distribution.'
     
      Other terms: 
      
       
        \alertg{Fat-tailed} distributions.
       
        \alertg{Heavy-tailed} distributions.
      
    

    \textbf{Beware:}
      
       Inverse power laws aren't the only ones:\\
          \wordwikilink{http://en.wikipedia.org/wiki/Log-normal\_distribution}{lognormals}, 
          \wordwikilink{http://en.wikipedia.org/wiki/Weibull\_distribution}{Weibull distributions}, \ldots
      
    

  


  \textbf{Size distributions:}

  \textbf{Many systems have discrete sizes $k$:}
    
    
      Word frequency
    
      Node degree in networks: \# friends, \# hyperlinks, etc.
    
      \# citations for articles, court decisions, etc.
    
  

  
    $$P(k) \sim  c\, k^{-\gamma}$$
    $$\mbox{where} \ \  k_{\textrm{min}} \le k \le k_{\textrm{max}}$$
    
     
      Obvious fail for $k=0$.
     
      Again, typically a description of distribution's tail.
    
  



\section{Examples}

\textbf{The statistics of surprise---words:}

\textbf{\wordwikilink{http://en.wikipedia.org/wiki/Brown\_Corpus}{Brown Corpus} ($\sim 10^6$ words):}
  \begin{tabular}{|rrr|}
    \hline
    rank & word & \% q \\
    \hline
    1. &     the   &    6.8872 \\
    2. &     of    &    3.5839 \\
    3. &     and   &    2.8401 \\
    4. &     to    &    2.5744 \\
    5. &     a     &    2.2996 \\
    6. &     in    &    2.1010 \\
    7. &     that  &    1.0428 \\
    8. &     is    &    0.9943 \\
    9. &     was   &    0.9661 \\
    10.&     he    &    0.9392 \\
    11.&     for   &    0.9340 \\
    12.&     it    &    0.8623 \\
    13.&     with  &    0.7176 \\
    14.&     as    &    0.7137 \\
    15.&     his   &    0.6886 \\
    \hline
  \end{tabular}\quad\begin{tabular}{|rrr|}
    \hline
    rank & word & \% q \\
    \hline
    1945. &  apply   &      0.0055 \\
    1946. &  vital   &      0.0055 \\
    1947. &  September&      0.0055 \\
    1948. &  review  &      0.0055 \\
    1949. &  wage    &      0.0055 \\
    1950. &  motor   &      0.0055 \\
    1951. &  fifteen &      0.0055 \\
    1952. &  regarded&      0.0055 \\
    1953. &  draw    &      0.0055 \\
    1954. &  wheel   &      0.0055 \\
    1955. &  organized&      0.0055 \\
    1956. &  vision  &      0.0055 \\
    1957. &  wild    &      0.0055 \\
    1958. &  Palmer  &      0.0055 \\
    1959. &  intensity&      0.0055 \\
    \hline
  \end{tabular}



  \textbf{Jonathan Harris's \wordwikilink{http://wordcount.org}{Wordcount:}}

  \textbf{A word frequency distribution explorer:}
  \includegraphics[width=.95\textwidth]{2011-09-13jonathanharris-wordcount.pdf}\\
  \smallskip
  \includegraphics[width=.95\textwidth]{2011-09-13jonathanharris-wordcount-alt.pdf}
  
  



  \textbf{The statistics of surprise---words:}

  \textbf{First---a Gaussian example:}
    $$
    P(x) \dee{x} = 
    \frac{1}{\sqrt{2\pi} \sigma}
    e^{-(x-\mu)^2/2\sigma} 
    \dee{x}
    $$
          
      
      linear:
      \includegraphics[width=\textwidth]{figgaussian2_noname}
      
      log-log\\
      \includegraphics[width=\textwidth]{figgaussian1_noname}
      
        mean $\mu=10$, variance $\sigma^2$ = 1.
  

  
    
    
      Sketch $n_q \sim q^{-2}$ for $q=1$ to $q=10^6$ 
      where  $q_w$ = percentage of all words that are a given word $w$.
    
  



  \textbf{The statistics of surprise---words:}

  \textbf{Raw `probability' (binned) for Brown Corpus:}
          
      
      linear:
      \includegraphics[width=\textwidth]{figwords5_noname} % 5
      
              
          
        log-log\\
        \includegraphics[width=\textwidth]{figwords6_noname} % 6
            
        $q_{w}$ = percentage of all words that are a given word $w$.



  \textbf{The statistics of surprise---words:}

  \textbf{`Exceedance probability' $N_{> q}$:}
          
      
      linear:
      \includegraphics[width=\textwidth]{figwords4_noname}
      
      log-log\\
      \includegraphics[width=\textwidth]{figwords1_noname}
      
      



  \textbf{My, what big words you have...}

  
    \includegraphics[width=\textwidth]{2011-08UVM-300test.png}

    
    
      Test capitalizes on word frequency following a
      heavily skewed frequency distribution
      with a decaying power-law tail.
     
      \wordwikilink{http://testyourvocab.com/}{Let's do it
        collectively \ldots}
    
  



\textbf{The statistics of surprise:}


\textbf{\wordwikilink{http://en.wikipedia.org/wiki/Gutenberg-Richter\_law}{Gutenberg-Richter law}}
      
    
    \includegraphics[width=\textwidth]{christensen2002a_fig2-tp-5.pdf}
    
    
     
      Log-log plot
     
      Base 10
     
      Slope = -1
    
    $N(M>m) \propto m^{-1}$
    
  


  
   
    From \alertb{both} the very awkwardly similar Christensen et al. and Bak et al.:\\
    ``Unified scaling law for earthquakes''\cite{christensen2002a,bak2002a}
  




\textbf{The statistics of surprise:}

\textbf{From:
\wordwikilink{http://www.nytimes.com/2011/03/14/world/asia/14seismic.html}{``Quake Moves Japan Closer to U.S. and Alters Earth's Spin''}
by Kenneth Chang, March 13, 2011, NYT:
}
`What is perhaps most surprising about the Japan earthquake is how
misleading history can be. In the past 300 years, no earthquake nearly
that large---nothing larger than magnitude eight---had struck in the
Japan subduction zone. That, in turn, led to assumptions about how
large a tsunami might strike the coast.'



```It did them a giant disservice,'' said Dr. Stein of the geological
survey. That is not the first time that the earthquake potential of a
fault has been underestimated. Most geophysicists did not think the
Sumatra fault could generate a magnitude 9.1 earthquake, \ldots'
%% and a
%% magnitude 7.3 earthquake in Landers CA in 1992 also caught
%% earthquake experts by surprise.





  \textbf{Well, that's just great:}

  \textbf{Two things we have poor cognitive understanding of:}
    
     
      Probability
      
       
        Ex. \wordwikilink{http://en.wikipedia.org/wiki/Monty\_Hall\_problem}{The Monty Hall Problem.}
      
        Ex. \wordwikilink{http://www.sciencenews.org/view/generic/id/60598/title/When\_intuition\_and\_math\_probably\_look\_wrong}{Daughter/Son born on Tuesday.} \newline
        {\small(see asides; Wikipedia entry \wordwikilink{http://en.wikipedia.org/wiki/Boy\_or\_Girl\_paradox}{here}.)}
      
      
      Logarithmic scales.
    
  

  \textbf{On counting and logarithms:}
                
            
      \includegraphics[width=\textwidth]{radiolab_134345_medium_image.jpg}
      
      
       
        Listen to Radiolab's\\ 
        \wordwikilink{http://www.radiolab.org/2009/nov/30/}{``Numbers.''}.
       
        Later: \wordwikilink{http://en.wikipedia.org/wiki/Benford's\_law}{Benford's Law}.
      
      
      
  

  %% example
  \includegraphics[width=0.8\textwidth]{zhu2013a}

  
   
    From ``Geography and Similarity of Regional Cuisines in China''\cite{zhu2013a},
    Zhu et al., PLoS ONE, 2013.
   
    Fraction of ingredients that appear in at least $k$ recipes.
   
    Bad notation: $P(k)$ is the Complementary Cumulative Distribution $P_{\ge}(k)$ 
  



%%  \textbf{Size distributions:}

  
    \includegraphics[height=.95\textheight]{newman2005b_fig4.pdf}
    \includegraphics[width=.95\textheight,angle=90]{newman2005b_fig4caption.pdf}
  


  \textbf{Size distributions:}

  \textbf{Examples:}
    
     
      Earthquake magnitude (\wordwikilink{http://en.wikipedia.org/wiki/Gutenberg-Richter\_law}{Gutenberg-Richter law}):\cite{gutenberg1942a,bak2002a} $P(M) \propto M^{-2}$
     
      \alertg{\# war deaths:\cite{richardson1949a} $P(d) \propto d^{-1.8}$}
     
      Sizes of forest fires\cite{grassberger2002a}
     
      \alertg{Sizes of cities:\cite{simon1955a} $P(n) \propto n^{-2.1}$}
     
      \# links to and from websites\cite{barabasi1999a}
    
  

  
    
     
      See in part Simon\cite{simon1955a} and 
      M. E. J. Newman\cite{newman2005b}
      ``Power laws, Pareto distributions and Zipf's law''
      and Clauset, Shalizi, and Newman\cite{clauset2009b} for more.
     
      Note: Exponents range in error
    
  



  \textbf{Size distributions:}

  \textbf{Examples:}
    
     
      \# citations to papers:\cite{price1965a,price1976a} $P(k) \propto k^{-3}$.
     
      \alertg{Individual wealth (maybe): $P(W) \propto W^{-2}$.}
     
      Distributions of tree trunk diameters: $P(d) \propto d^{-2}$.
     
      \alertg{The gravitational force at a random point in the universe:\cite{holtsmark1917a} $P(F) \propto F^{-5/2}$.}
      (see the 
      \wordwikilink{http://en.wikipedia.org/wiki/Holtsmark\_distribution}{Holtsmark distribution}
      and 
      \wordwikilink{http://en.wikipedia.org/wiki/Stable\_distribution}{stable distributions}
     
      Diameter of moon craters:\cite{newman2005b} $P(d) \propto d^{-3}$.
     
      \alertg{Word frequency:\cite{simon1955a} e.g., $P(k) \propto k^{-2.2}$ (variable).}
     
      \# religious adherents in cults:\cite{clauset2009b} 
      $P(k) \propto k^{-1.8 \pm 0.1}.$
    
      \alertg{\# sightings of birds per species (North American
      Breeding Bird Survey for 2003):\cite{clauset2009b} 
      $P(k) \propto k^{-2.1 \pm 0.1}.$}
    
      \# species per genus:\cite{yule1924a,simon1955a,clauset2009b} 
      $P(k) \propto k^{-2.4 \pm 0.2}.$
    
  



%%    \wordwikilink{http://www.arxiv.org/cond-mat/0412004}{arxiv.org/cond-mat/0412004}

% ??? add exponents




  \textbf{Table 3 from Clauset, Shalizi, and Newman\cite{clauset2009b}:}
    \includegraphics[width=1.2\textwidth]{clauset2009b_tab3.pdf}
  

  
   
    We'll explore various exponent measurement techniques in assignments.
  
  

%% %%   \textbf{Size distributions:}
%% 
%%   \textbf{Power-law distributions are..}
%%     
%%      
%%       often called `heavy-tailed'
%%      
%%       or said to have `fat tails'
%%     
%%   

%% 
\section{Wild\ vs.\ Mild}

  \textbf{power-law distributions}

  \textbf{Gaussians versus power-law distributions:}
    
    
      \alertb{Mediocristan} versus \alertb{Extremistan}\\
    
      Mild versus Wild (Mandelbrot)
    
      Example: Height versus wealth.
    
  

  
          
      
      \includegraphics[width=\textwidth]{blackswan_cover.pdf}
      
      
       
        See ``The Black Swan'' by Nassim Taleb.\cite{taleb2007a}
      
      
      


  \textbf{Turkeys...}

  \begin{center}
    \includegraphics[angle=-0.3,width=\textwidth]{taleb_turkey.pdf}
  \end{center}

  {\small From ``The Black Swan''\cite{taleb2007a}}


  \textbf{Taleb's table\cite{taleb2007a}}

{\small
  \textbf{Mediocristan/Extremistan}
    
     
      \alertg{Most typical member is} \alertb{mediocre}/Most typical is either \alertb{giant or tiny}
     
      \alertg{Winners get a small segment}/Winner take almost all effects 
     
      \alertg{When you observe for a while, you know what's going on}/It
      takes a \alertr{very long time} to figure out what's going on
     
      Prediction is \alertb{easy}/Prediction is \alertr{hard}
     
      \alertg{History crawls}/History makes jumps
     
      \alertg{Tyranny of the collective}/Tyranny of the rare and accidental
    
  
}


  \textbf{Size distributions:}

      
    \includegraphics[width=\textwidth]{Vilfredo_Pareto.jpg}
    
  
    Power-law size distributions 
    are sometimes called\\ 
    \wordwikilink{http://en.wikipedia.org/wiki/Pareto_distribution}{Pareto distributions}
    after Italian scholar 
    \wordwikilink{http://en.wikipedia.org/wiki/Vilfredo\_Pareto}{Vilfredo Pareto.}
  

  
    
    
      Pareto noted wealth in Italy was distributed unevenly
      (80--20 rule; misleading).
    
      Term used especially by practitioners of the 
      \wordwikilink{http://en.wikipedia.org/wiki/The\_dismal\_science}{Dismal Science}.
    
  
  


  \textbf{Devilish power-law size distribution details:}

  \textbf{Exhibit A:} 
    
     
      Given $P(x) = c x^{-\gamma}$ with $0 < \xmin < x < \xmax$,\\
      the mean is ($\gamma \ne 2$):
    
    $$ \avg{x} = \frac{c}{2-\gamma} \left( \xmax^{2-\gamma} - \xmin^{2-\gamma} \right). $$
  

  
    
    
      Mean `blows up' with upper cutoff if $\gamma < 2$.
    
      Mean depends on lower cutoff if $\gamma > 2$.
    
      \alertr{$\gamma < 2$}: Typical sample is large.
    
      \alertr{$\gamma > 2$}: Typical sample is small.
    
  

  \insertassignmentquestionsoft{01}{1}



  \textbf{And in general...}

  \textbf{Moments:}
    
    
      All moments depend only on cutoffs.
    
      \alertb{No internal scale} that dominates/matters.
    
      Compare to a Gaussian, exponential, etc.
    
  

  \textbf{For many real size distributions: $ 2 < \gamma < 3 $}
    
     
      mean is finite (depends on lower cutoff)
     
      $\sigma^2$ = variance is `infinite' (depends on upper cutoff)    
     
      Width of distribution is `infinite'    
    
      If $\gamma > 3$, distribution is less terrifying and may
      be easily confused with other kinds of distributions.
    
  

  \insertassignmentquestionsoft{01}{1}


%% %%   \textbf{Moments}
%% 
%%   \textbf{The variance:}
%%     $$ \sigma^2 = \avg{(x-\avg{x})^2} $$
%%     $$ = \int_{\xmin}^{\xmax} (x-\avg{x})^2 P(x) \dee{x} $$
%%     $$ = \tavg{x^2} - \tavg{x}^2 $$
%%   
%% 
%% 
  \textbf{Moments}

  \textbf{Standard deviation is a mathematical convenience:}
    
     
      Variance is nice analytically...
     
      Another measure of distribution width:
      $$
      \mbox{Mean average deviation (MAD)} =
      \avg{\left| x - \avg{x} \right|}
      $$
    
      For a pure power law with $2 < \gamma < 3$:
      $$\avg{\left| x - \avg{x} \right|} \ \mbox{is finite.}$$
    
      But MAD is mildly unpleasant analytically...
    
      We still speak of infinite `width' if $\gamma < 3$.
    
  

  \insertassignmentquestionsoft{02}{2}


%% %%   \textbf{Moments}
%% 
%% 
%%   Still, we say such a distribution has infinite `width'
%% 
%% 
  \textbf{How sample sizes grow...}

  \textbf{Given $P(x) \sim c x^{-\gamma}$:}
    
     
      We can show that after $n$ samples,
      we expect the largest sample to be 
      $$ x_{1} \gtrsim c' n^{1/(\gamma-1)} $$
     
      Sampling from a 
      finite-variance distribution 
      gives a much slower growth with $n$.
     
      e.g., for $P(x) = \lambda e^{-\lambda x}$,
      we find
      $$ x_{1} \gtrsim \frac{1}{\lambda} \ln n. $$
    
  

  \insertassignmentquestionsoft{02}{2}


\section{CCDFs}

  \textbf{\small Complementary Cumulative Distribution Function:}

  \textbf{CCDF:}
    
    
      $$ P_{\ge}(x) = P(x' \ge x)  = 1 - P(x'<x) $$
    
      $$ = \int_{x' = x}^{\infty} P(x') \dee{x'}  $$
    
      $$ \propto \int_{x' = x}^{\infty} (x')^{-\gamma} \dee{x'}  $$
    
      $$ = \left. \frac{1}{-\gamma+1}(x')^{-\gamma+1} \right|_{x' = x}^{\infty} $$
    
      $$ \propto x^{-\gamma+1} $$
    
  


  \textbf{\small Complementary Cumulative Distribution Function:}

  \textbf{CCDF:}
    
    
      $$ P_{\ge}(x) \propto x^{-\gamma+1} $$
    
      Use when tail of $P$ follows a power law.
    
      Increases exponent by one.
    
      Useful in cleaning up data.
    
  

      
      
    
              
        
        PDF:\\
        \includegraphics[width=\textwidth]{figwords6_noname}
        
        CCDF:\\
        \includegraphics[width=\textwidth]{figwords1_noname}
        
          
  

  \textbf{\small Complementary Cumulative Distribution Function:}

  
    
    
      Discrete variables:
      $$ P_\ge(k) = P(k' \ge k) $$
    {
      $$ = \sum_{k'=k}^{\infty} P(k) $$
      }
    {
      $$ \propto k^{-\gamma+1} $$
      }
    
      Use integrals to approximate sums.
    
  


\section{Zipf's\ law}

  \textbf{Zipfian rank-frequency plots}

  \textbf{George Kingsley Zipf:}
    
    
      Noted various rank distributions\\ 
      have power-law tails, often with exponent -1\\
      (word frequency, city sizes...)
    
      Zipf's 1949 \wordwikilink{http://en.wikipedia.org/wiki/Principle\_of\_least\_effort}{Magnum Opus}:\\
    
          
      
      \amazonbook{zipf1949a}
        
     We'll study Zipf's law in depth...
    
  

  %% \alertb{``Human Behaviour and the Principle of Least-Effort''}\cite{zipf1949a}
  %% {\small Addison-Wesley, Cambridge MA, 1949.}


  \textbf{Zipfian rank-frequency plots}

  \textbf{Zipf's way:}
  
  
    Given a collection of entities, rank them by
    size, largest to smallest.
  
    $x_\rank$ = the size of the $\rank$th ranked entity.
   
    $r=1$ corresponds to the largest size.
   
    Example: $x_1$ could be the frequency of occurrence of
    the most common word in a text.
  
    Zipf's observation:
    $$ x_r \propto r^{-\alpha} $$
  
  


\section{Zipf\ \texorpdfstring{$\Leftrightarrow$}{is\ equivalent\ to}\ CCDF}

  \textbf{Size distributions:}

  \textbf{Brown Corpus (1,015,945 words):}
    \medskip
          
      
      CCDF:\\
      \includegraphics[width=\textwidth]{figwords1_noname}
      
      Zipf:\\
      \includegraphics[width=\textwidth]{figwords2_noname}
      
      

  
    
     
      The, of, and, to, a, ...  = `objects'
     
      `Size' = word frequency
    
      {
        \alertr{Beep:} (Important) CCDF and Zipf plots are related...}
    
  


  \textbf{Size distributions:}

  \textbf{Brown Corpus (1,015,945 words):}
          
      
      CCDF:\\
      \includegraphics[width=\textwidth]{figwords1_noname}
      
      Zipf:\\
      \includegraphics[width=\textwidth]{figwords3_noname}
      
    
  

  
    
     
      The, of, and, to, a, ...  = `objects'
     
      `Size' = word frequency
    
      {
        \alertr{Beep:} (Important) CCDF and Zipf plots are related...}
    
  



%%  \textbf{Size distributions:}

  \textbf{Observe:}
    
     $ NP_\ge(x) = $ the number of objects with size at least $x$\\
      where $N$ = total number of objects.
     If an object has size $x_\rank$, then $NP_\ge(x_\rank)$ is its rank $\rank$.
     So 
      $$\alertb{x_\rank \propto \rank^{-\alpha} = (NP_\ge(x_\rank))^{-\alpha}}$$
    [] 
      $$
      \alertb{ \propto x_\rank^{(-\gamma+1)(-\alpha)}}
      \mbox{\ since $P_\ge(x) \sim x^{-\gamma+1}$}.
      $$
    [] 
      We therefore have $1=(-\gamma+1)(-\alpha)$ or:
      $$
      \alertb{
        \boxed{\alpha = \frac{1}{\gamma-1}}
      }
      $$
     
      A rank distribution exponent of $\alpha = 1$ 
      corresponds to a size distribution exponent $\gamma=2$.
    
  



%% %%   \textbf{Details on the lack of scale:}
%% 
%%   \textbf{Let's find the mean:}
%%     
%%      
%%       $$ \avg{x} = \int_{x=\xmin}^{\xmax} x P(x) \dee{x} $$
%%     [] 
%%       $$ = c \int_{x=\xmin}^{\xmax} x x^{-\gamma} \dee{x} $$
%%     [] 
%%       $$ = \frac{c}{2-\gamma} \left(
%%         \xmax^{2-\gamma} - \xmin^{2-\gamma} \right).
%%       $$
%%     
%%   
%% 
%% 

  \textbf{\wordwikilink{http://en.wikipedia.org/wiki/Donald_Bradman}{The Don.}}

  \textbf{Extreme deviations in \alertg{test cricket:}}
      
    \begin{center}
      \includegraphics[width=.2\textwidth]{bradman.jpg}
      \includegraphics[angle=-90,width=0.6\textwidth]{CricketBattingAverageHistogram.pdf}
      \includegraphics[width=.05\textwidth]{wikipedia-tp.pdf}
    \end{center}
    
    
      {Don Bradman's \wordwikilink{http://en.wikipedia.org/wiki/Batting\_average}{batting average} \\ = \alertr{166\%} next best.}
    
      That's pretty solid.
    
      Later in the course: Understanding success--- \newline
      is the Mona Lisa like Don Bradman?
    
  


