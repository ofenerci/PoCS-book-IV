\section{Introduction}

\begin{frame}
  \frametitle{Overview of I. Generalized Contagion}

  \begin{enumerate}
  \item Independent Interaction models.
  \item Threshold models.
%  \item Other models of contagion.
  \item Generalized model.
  \item Results: analysis, numerics, and model simulations.
  \item Conclusions.
  \end{enumerate}

\end{frame}

%% \begin{frame}
%%   \frametitle{Independent Interaction models}
%% 
%%   \begin{block}<1->{Many variants of the SIR model:} 
%%     \begin{itemize}
%%     \item<2-> \alert{SIS}: susceptible-infective-susceptible 
%%     \item<3-> \alert{SIRS}: susceptible-infective-recovered-susceptible
%%     \item<4-> compartment models (age or gender partitions)
%%     \item<5-> more categories such as `exposed' (\alert{SEIRS})
%%     \item<6-> recruitment (migration, birth)
%%     \end{itemize}
%%   \end{block}
%% 
%% \end{frame}


\begin{frame}
  \frametitle{Generalized model}

  Important quantities:
  $$  
  P_k
  =
  \int_{0}^{\infty} \dee{\dstar}
  g(\dstar)
  P\left(
%    \textstyle{\sum_{i=1}^{k} d_i \ge \dstar}
    \sum_{j=1}^{k} d_j \ge \dstar
    \right)
  $$
  where $1 \le k \le T$.

  $P_k$ = Probability that the threshold of \\
\quad \quad \  a randomly selected individual\\
\quad \quad \  will be exceeded by $k$ doses.

e.g., \\
$P_1$ = Probability that \underline{\tc{blue}{one dose}} will exceed\\
\quad \quad \ the threshold of a random individual\\
\ \quad  = Fraction of \underline{\tc{blue}{most vulnerable}} individuals.

\end{frame}

\subsection{Dose response}

\begin{frame}
  \frametitle{Dose response curves:}

  Contact with $K$ infected individuals in last $T$ time steps:
  $$
  P_{\textnormal{inf}} 
  = \sum_{i=1}^K
  \binom{K}{i} 
  p^i (1 -p)^{K-i} P_i.
  $$
  \includegraphics[width=\textwidth]{figdoseresponse11b_noname}\\

  \textbf{A.} Independent Interaction model
  \hfill 
  {\small $p=0.3$, 
    $d = 1$, 
    $\dstar=1$}\\
  \smallskip
  \textbf{B.} Stochastic threshold model
  \hfill
  {\small $p=1$, 
    $f(d) \sim$ logn,
    $\dstar=5$}\\
  \mbox{}
  \hfill
  {\small $\tavg{d} = 1$, $\sigma_d^2 = 0.4333$}\\
  \smallskip
  \textbf{C.} Deterministic threshold model
  \hfill
  {\small $p=1$, 
    $d=1$,
    $\dstar=5$}

\end{frame}


%%%%%%%%%%%%%% 5 mins
% 3b. our model---results

% homogeneous

% heterogeneous

\begin{frame}
  \frametitle{Generalized mean-field model}

  SIS-type contagion.

  Recovered individuals are immediately susceptible again.\\
  ($r=\rho=1$)
  
  Look for steady-state behavior (fixed points $\phifix$)\\
  as a function of exposure probability $p$.

%  \tc{blue}{Homogeneous version:}\\
%  \ding{228} All individuals have threshold $\bar{\dstar}$\\
%  \ding{228} All dose sizes are equal: $d=1$

\end{frame}


\begin{frame}
  \frametitle{Generalized model}

  \tc{blue}{Model details:}

  Mean field.

  Population size $N = 10^5$

  Number of time steps $N_t = 10^4$.

  Initialize with fixed $\phi$.

\end{frame}

\begin{frame}
  \frametitle{Calculations---Fixed points for \tc{blue}{$r \le 1$}, $\dstar=1$, and \tc{blue}{$T \ge 1$}}

  Add fraction who did not receive any infections in last T time steps
  and have not recovered from a previous infection.
  $$H_1 = \{ \ldots, d_{t-T-2}, d_{t-T-1}, 1, \underbrace{0, 0, \ldots, 0, 0}_{\mbox{$T$ 0's}} \},$$
  $$H_{m+1} = \{ \ldots, d_{t-T-m-1}, 1, \underbrace{0, 0, \ldots, 0, 0}_{\mbox{$m$ 0's}}, \underbrace{0, 0, \ldots, 0, 0}_{\mbox{$T$ 0's}} \};$$

  $$P(H_1) = p\phifix (1 -p\phifix)^T (1-r), $$
  $$P(H_{m+1}) = p\phifix (1 -p\phifix)^{T+m} (1-r)^{m+1}. $$

\end{frame}

\begin{frame}
  \frametitle{Calculations---Fixed points for $\dstar=1$, \tc{blue}{$r \le 1$}, and \tc{blue}{$T \ge 1$}}

  Fixed point equation:

  $$
  \phifix =
 1 - \frac{r (1-p\phifix)^T }
    {1 - (1-p\phifix)(1-r)}.
  $$

  $$
  \phifix \rightarrow 0
  \quad \Rightarrow \quad 
  \tc{red}{p_c} = \frac{1}{T + 1/r - 1} \tc{red}{= \frac{1}{T + \tau}}.
  $$

  \hfill $\tau$ = mean recovery time.

\end{frame}


\begin{frame}
  \frametitle{Calculations---Homogeneous Model}

  Fixed points for $\dstar=1$, \tc{blue}{$r \le 1$},  and \tc{blue}{$T \ge 1$}.
  \begin{columns}
    \column{0.5\textwidth}
    $
    \begin{array}{l}
      \phifix =
      1 - \frac{r (1-p\phifix)^T }
      {1 - (1-p\phifix)(1-r)}\\
      \\
      p_c = 1/(T+\tau)\\
      \\
      \mbox{from} \ \ p (T+\tau) \ge 1
    \end{array}
    $
    \column{0.5\textwidth}
  \includegraphics[width=\textwidth]{figgc_r0p50_k1_T2_paper2_noname.pdf}
  \end{columns}

  $\tau = 1/r - 1$ = characteristic recovery time\\
  $T + \tau \simeq $  total memory in system\\
  %%  \hfill $T=2=1/r \rightarrow p_c = 1/3$ \qquad \mbox{}
  \hfill $\Rightarrow$  \tc{blue}{Epidemic Threshold Models}

\end{frame}



%%%%%%%%%%%%%%%
%3d.  SIS, r=1, dstar>1, T>=dstar

\begin{frame}
  \frametitle{Calculations---Homogeneous models}

  Fixed points for $\dstar > 1$, \tc{blue}{$r = 1$},  and \tc{blue}{$T \ge 1$}.\\
  $$
  \begin{array}{l}
  \phifix = \\
  \sum_{i=\dstar}^{T}
  \binom{T}{i}
  (p\phifix)^{i} (1 - p\phifix)^{T-i}\\
  \\
  p_c = \infty\\
  \end{array}
   \raisebox{-4cm}{
    \includegraphics[width=0.425\textwidth]{figgc_r1_k3_T12_paper2_noname.pdf}}
  $$
%  $\dstar=3$, $T=12$ 
  $d=1$, $\bar{\dstar}=3$, $T=12$ \hfill $\Rightarrow$ Critical mass models\\
  \hfill \ldots are there any other types?
  
\end{frame}

\begin{frame}
  \frametitle{Calculations---Homogeneous models}

  \ding{228} Nontrivial threshold ($\dstar > 1$) \\
  \ding{228} Immediate recovery ($r=1$)

  To be infected,\\ must have at least $\dstar$
  exposures in last $T$ time steps:

  $$
  \phifix = 
  \sum_{i=\dstar}^{T}
  \binom{T}{i}
  (p\phifix)^{i} (1 - p\phifix)^{T-i}.
  $$

\end{frame}

\begin{frame}
  \frametitle{Calculations---Homogeneous models}
  \centering
  Critical Mass Models\\
  \includegraphics[width=0.6\textwidth]{figgc_r1_k3_T12_paper2_noname.pdf}\\
%  $\dstar=3$, $T=12$ 
  $d=1$, $\bar{\dstar}=3$, $T=12$ \hfill Saddle-node bifurcation
  
\end{frame}


\begin{frame}
  \frametitle{Calculations---Fixed points for $r = 1$, \tc{blue}{$\dstar > 1$}, and \tc{blue}{$T \ge 1$}}

  Solvable for small $T$; e.g., for $\dstar=2$, $T=3$:

  \includegraphics[width=0.4\textwidth]{figgc_r1_k2_T3_noname.pdf}

 $\phifix = 3 p^2 \phifix (1 - p \phifix) + p^3 \phifix^2$
 $\Rightarrow$ $(p_b,\phifix)=(8/9,27/32)$.

\end{frame}

\begin{frame}
 \frametitle{Calculations---Fixed points for $r = 1$, \tc{blue}{$\dstar > 1$}, and \tc{blue}{$T \ge 1$}}

\begin{center}
 \includegraphics[width=0.45\columnwidth]{figgc_T24_kvar_r1_noname.pdf} 
 \includegraphics[width=0.45\textwidth]{figgc_bipts_r1_noname.pdf}
 $T=96$ ($\vartriangle$).
 $T=24$ ($\triangleright$),
 $T=12$ ($\triangleleft$),
 $T=6$ ($\Box$),
 and
 $T=3$ ($\bigcirc$), 
$T=24$ \hfill Bifurcation points \\
\mbox{} \hfill $T=3$, 6, 12, 24, 96; 
 \end{center}
\end{frame}


\begin{frame}
  \frametitle{Calculations---Homogeneous models}

  \centering
  
  Two classes of contagion models:

\includegraphics[width=0.475\textwidth]{figgc_r0p50_k1_T2_paper2_noname}
\hfill      
\includegraphics[width=0.475\textwidth]{figgc_r1_k3_T12_paper2_noname.pdf}\\

 \includegraphics[width=0.45\columnwidth]{figgc_T24_kvar_r1_noname.pdf}
 $T=96$ ($\vartriangle$).
 $T=24$ ($\triangleright$),
 $T=12$ ($\triangleleft$),
 $T=6$ ($\Box$),
 and
 $T=3$ ($\bigcirc$), 
 $T=24$\\

  \qquad I. Epidemic threshold models \hfill
  II. Critical mass models

  \ding{228} Shift from $\dstar=1$ to $\dstar>1$ causes fundamental change.

\end{frame}


%%%%%%%%%%%%%
e   SIS, some specific cases


\begin{frame}
  \frametitle{Calculations---Fixed points for \tc{blue}{$r < 1$}, \tc{blue}{$\dstar > 1$}, and \tc{blue}{$T \ge 1$}}

  $D_i(t) = \sum_{t'=t-T+1}^{t} d_i(t')$ \hfill 
  Partially summed random walk.

  \centering
  \includegraphics[width=0.7\columnwidth]{figrandomwalkcalc_noname.pdf}
  
  $T=24$, $\dstar=14$.
\end{frame}

\begin{frame}
  \frametitle{Calculations---Fixed points for \tc{blue}{$r < 1$}, \tc{blue}{$\dstar > 1$}, and \tc{blue}{$T \ge 1$}}

  $\gamma_m$ = fraction of individuals for whom $D(t)$ last equaled the threshold $m$ time steps ago.

  Fraction not recovered:
  $$
  \Gamma(p,\phifix;r) = \sum_{m=1}^{\infty} (1-r)^m \gamma_m(p,\phifix).
  $$

  Fixed point equation:
$$
  \phifix = \Gamma(p,\phifix;r) 
  + \sum_{i=\dstar}^{T}
  \binom{T}{i}
  (p\phifix)^{i} (1 - p\phifix)^{T-i}.
$$

\end{frame}


\begin{frame}
  \frametitle{Calculations---Fixed points for \tc{blue}{$r < 1$}, \tc{blue}{$\dstar > 1$}, and \tc{blue}{$T \ge 1$}}

$T=3$, $\dstar=2$:

Two sequences lead to being below threshold:
$$D_n=2 \Rightarrow D_{n+1}=1$$
\tc{blue}{\qquad \qquad \qquad $\{d_{n-2},d_{n-1},d_{n},d_{n+1}\} = \{1,1,0,\tc{red}{0}\}$} \hfill  \\
\tc{blue}{\qquad \qquad \qquad $\{d_{n-2},d_{n-1},d_{n},d_{n+1},d_{n+2}\} = \{1,0,1,\tc{red}{0},\tc{red}{0}\}$} 

  Acceptable following sequences composed of
  $$ a=\{0\} \quad \mbox{and} \quad b=\{1,0,0\}. $$
\end{frame}

\begin{frame}
  \frametitle{Calculations---Fixed points for \tc{blue}{$r < 1$}, \tc{blue}{$\dstar > 1$}, and \tc{blue}{$T \ge 1$}}

  $N_a$ = number of $a$ subsequences.\\
  $N_b$ = number of $b$ subsequences.

  $$
  m = N_a \cdot 1 + N_b \cdot 3
  $$

  $$
  N_b = 0, 1, \ldots, [m/3].
  $$

  $$
  N_a = m, m-3, m-6, \ldots, m-3[m/3].
  $$

\end{frame}

\begin{frame}
  \frametitle{Calculations---Fixed points for \tc{blue}{$r < 1$}, \tc{blue}{$\dstar > 1$}, and \tc{blue}{$T \ge 1$}}

  How many ways to arrange $N_a$ $a$'s and $N_b$ $b$'s?

  Think of overall sequence in terms of subsequences:
  $$ \{Z_1, Z_2, \ldots, Z_{N_a + N_b} \} $$

  $N_a + N_b$ slots for subsequences.

  Choose positions of $a$'s or $b$'s:
  $$
  \binom{N_a + N_b}{N_a} = \binom{N_a + N_b}{N_b}.
  $$

\end{frame}

\begin{frame}
  \frametitle{Calculations---Fixed points for \tc{blue}{$r < 1$}, \tc{blue}{$\dstar > 1$}, and \tc{blue}{$T \ge 1$}}
  Total number of allowable sequences of length $m$:
  $$
  \sum_{N_b=0}^{[m/3]}
  \binom{N_b+N_a}{N_b}
  = 
  \sum_{k=0}^{[m/3]}
  \binom{m-2k}{k}
  $$

  $P(a) = (1-p\phifix)$ and $P(b) = p\phifix(1-p\phifix)^2$
  
  Total probability of allowable sequences of length $m$:
  $$
  \chi_m(p,\phifix)
  = 
  \sum_{k=0}^{[m/3]}
  \binom{m-2k}{k}
  (1-p\phifix)^{m-k}
  (p\phifix)^k.
  $$

\end{frame}

\begin{frame}
  \frametitle{Calculations---Fixed points for \tc{blue}{$r < 1$}, \tc{blue}{$\dstar > 1$}, and \tc{blue}{$T \ge 1$}}

  Three endings $\Rightarrow$ Six possible sequences:
  $$ D_1 = \{1, 1, 0, 0, D_{m-1}^{a,b}\} $$
  $$ D_2 = \{1, 1, 0, 0, D_{m-2}^{a,b}, 1\} $$
  $$ D_3 = \{1, 1, 0, 0, D_{m-3}^{a,b}, 1, 0\}$$
  $$ D_4 = \{1, 0, 1, 0, 0, D_{m-2}^{a,b}\} $$
  $$ D_5 = \{1, 0, 1, 0, 0, D_{m-3}^{a,b}, 1\} $$
  $$ D_6 = \{1, 0, 1, 0, 0, D_{m-4}^{a,b}, 1, 0\} $$


\end{frame}

\begin{frame}
  \frametitle{Calculations---Fixed points for \tc{blue}{$r < 1$}, \tc{blue}{$\dstar > 1$}, and \tc{blue}{$T \ge 1$}}

  Their corresponding probabilities:
  $$ P_1 = (p\phi)^2 (1-p\phi)^2 \chi_{m-1}(p,\phi) $$
  $$ P_2 = (p\phi)^3 (1-p\phi)^2 \chi_{m-2}(p,\phi) $$
  $$ P_3 = (p\phi)^3 (1-p\phi)^3 \chi_{m-3}(p,\phi) $$
  $$ P_4 = (p\phi)^2 (1-p\phi)^3 \chi_{m-2}(p,\phi) $$
  $$ P_5 = (p\phi)^3 (1-p\phi)^3 \chi_{m-3}(p,\phi) $$
  $$ P_6 = (p\phi)^3 (1-p\phi)^4 \chi_{m-4}(p,\phi) $$
\end{frame}



\begin{frame}
  \frametitle{Calculations---Fixed points for \tc{blue}{$r < 1$}, \tc{blue}{$\dstar > 1$}, and \tc{blue}{$T \ge 1$}}

  \begin{center}
      \includegraphics[width=0.625\textwidth]{figgc_T3_k2_bif_theorycomp2_noname.pdf}\\
      $T=3$, $\dstar=2$;
      \hfill
      $r=0.01, 0.05, 0.10, 0.15, \tc{blue}{0.20},\ldots, 1.00$.
  \end{center}
\end{frame}

\begin{frame}
  \frametitle{Calculations---Fixed points for \tc{blue}{$r < 1$}, \tc{blue}{$\dstar > 1$}, and \tc{blue}{$T \ge 1$}}

  \begin{center}
      \includegraphics[width=0.625\textwidth]{figgc_T2_k2_bif_theorycomp_noname.pdf}\\
      $T=2$, $\dstar=2$;
      \hfill
      $r=0.01, 0.05, 0.10,\ldots, 0.3820 \pm 0.0001$.
  \end{center}
\end{frame}


\begin{frame}
  \frametitle{Generalized model---heterogeneity, $r=1$}

%  Generalize to \tc{red}{heterogeneous case} ($r=1$).

%  \ding{228} doses and thresholds arbitrarily distributed.

  Fixed point equation:
  $$
  \phifix
  =
  \sum_{k=1}^{T}
  \binom{T}{k}
  (p\phifix)^{k}
  (1-p\phifix)^{T-k}
  \underline{\tc{blue}{P_k}}
  $$
  $$ \hfill \ \mbox{where} \
  P_k
  =
  \int_{0}^{\infty} \dee{\dstar}
  g(\dstar)
  P\left(
%    \textstyle{\sum_{i=1}^{k} d_i \ge \dstar}
    \sum_{j=1}^{k} d_j \ge \dstar
    \right)
  $$

  Spread from single seed if $p P_1 T \ge 1 \Rightarrow p_c = 1/(TP_1)$.

  Find \underline{\tc{blue}{three}} universal classes
  of contagion models...

\end{frame}

\begin{frame}
  \frametitle{Heterogeneous case---Three universal classes}

  \begin{center}
    \includegraphics[width=0.95\textwidth]{figgc_paperfigs_classes6c_noname}
  \end{center}

  Epidemic threshold: \hfill $P_1 > P_2/2$, $p_c = 1/(TP_1) < 1$

  Vanishing critical mass: \hfill $P_1 < P_2/2$, $p_c = 1/(TP_1) < 1$

  Pure critical mass: \hfill $p_c = 1/(TP_1) > 1$

\end{frame}

% \begin{frame}
%   \frametitle{Heterogeneous case---Transitions between classes}
% 
%   \begin{center}
%     \includegraphics[width=0.9\textwidth]{figgc_paperfigs_classes7a_noname.pdf}
%   \end{center}
% 
%   I--II: $P_1 = P_2/2$, $p_c = 1/(TP_1) < 1$
% 
%   \hfill II---III: $p_c = 1/(TP_1) = 1$
% \end{frame}

\begin{frame}
  \frametitle{Calculations---Fixed points for \tc{blue}{$r < 1$}, \tc{blue}{$\dstar = 2$}, and \tc{blue}{$T = 3$}}
  $$
  \mbox{F.P.\ Eq:} \ \
  \phifix = \Gamma(p,\phifix;r) 
  + \sum_{i=\dstar}^{T}
  \binom{T}{i}
  (p\phifix)^{i} (1 - p\phifix)^{T-i}.
  $$
  \begin{eqnarray*}
  \Gamma(p,\phifix;r) & = & (1-r)(p\phi)^2 (1-p\phi)^2 
   +  \sum_{m=1}^{\infty} (1-r)^m 
  (p\phi)^2 (1-p\phi)^2 \times
  \\ \nonumber
  \\ \nonumber
  &  & 
  \left[ 
    \chi_{m-1} + 
    \chi_{m-2} + 
    2p\phi (1-p\phi)\chi_{m-3} + 
    p\phi (1-p\phi)^2\chi_{m-4}
  \right]
  \\ \nonumber
  \end{eqnarray*}
  $$
  \mbox{where} \ \  \chi_m(p,\phifix)
  = 
  \sum_{k=0}^{[m/3]}
  \binom{m-2k}{k}
  (1-p\phifix)^{m-k}
  (p\phifix)^k.
  $$

\end{frame}


\begin{frame}
  \frametitle{SIS model}

  Now allow $r<1$: \hfill
  \raisebox{-4cm}{\includegraphics[width=0.49\textwidth]{figgc_T3_k2k1_bif_theorycomp2b_noname}}

II-III transition generalizes: $p_c = 1/[P_1 (T + \tau)]$ 

(I-II transition less pleasant analytically)

\end{frame}

\begin{frame}
  \frametitle{More complicated models}

  \includegraphics[width=0.95\textwidth]{figgc_manybif_thy_comb03_noname}

  \ding{228} Due to heterogeneity in individual thresholds.

  \ding{228} Same model classification holds:  I, II, and III.
\end{frame}

\begin{frame}
  \frametitle{Hysteresis in vanishing critical mass models}

  \centering
  \includegraphics[width=0.65\textwidth]{figgc_manybif_thy_comb03b_noname}

\end{frame}

 

\begin{frame}
  \frametitle{SIRS model}

  And $\rho<1$\ldots \hfill
  \raisebox{-4cm}{\includegraphics[width=0.49\textwidth]{figgcSIR_logn_070b_noname}}

  Type II models become type I as $\rho$ decreases.

  (Type I and III models stay in same class)


\end{frame}

\begin{frame}
  \frametitle{SIR model}
  
%  Once immune state R is reached, no return to S: $\rho=0$.
  All individuals end in state R: $\rho=0$.

  But: for $T>1$ individuals remain infected\\ 
  for relatively long times:

  $$ \tmax \propto (1-p)^{-(T-1)} $$

  $\Rightarrow$ More `sick days.'

  $\Rightarrow$ More chance of spreading to
  other populations.
\end{frame}

% MUST DO BETTER THAN THIS!
%\begin{frame}
%  \frametitle{SIRS indicates SIR behavior}
%
%  \centering
%  \includegraphics[width=0.6\textwidth]{figgc_manybif_thy_comb03c_noname.pdf}\\
%  
%  Total fraction infected $\nearrow$ faster in SIS `growth' region.
% \end{frame}

%\begin{frame}
%  \frametitle{SIR model, $\rho=0$}
%  
%  $\tmax \propto (1-p)^{-(T-1)}$ \hfill
%  \raisebox{-4cm}{\includegraphics[width=0.65\textwidth]{figgcSIR_logn_050comb3_noname.pdf}}
%\end{frame}

 
%%%%%%%%%%%%%% 2 mins
% 4. conclusions
%    future work
%     SIR etc.
%     networks
%     memory may be important in disease

\begin{frame}
  \frametitle{Discussion}

  \ding{228} Memory is a natural ingredient. % emulsifying

  \ding{228} Three universal classes of contagion processes:\\
  \qquad \tc{blue}{I. Epidemic Threshold}\\
  \qquad \tc{blue}{II. Vanishing Critical Mass} \\ %\hfill $\Rightarrow P_1$ vs.\ $P_2/2$\\
  \qquad \tc{blue}{III. Critical Mass}

  \ding{228} Dramatic changes in behavior possible.

  \ding{228} To change kind of model: `adjust' memory, recovery, fraction
  of vulnerable individuals ($T$, $r$, $\rho$, $P_1$, and/or $P_2$).

  \ding{228} To change behavior given model: `adjust' probability
  of exposure ($p$) and/or initial number infected ($\phi_0$).
\end{frame}

\begin{frame}
  \frametitle{Discussion}

  Single seed infects others if $p P_1 (T + \tau) \ge 1$.

  Key quantity: $\tc{blue}{p_c = 1/[P_1 (T + \tau)]}$\\

  If $p_c < 1$ $\Rightarrow$ contagion can spread from single seed.
  
  Depends only on:\\
  1. \tc{blue}{System Memory ($T+\tau$).}\\
  2. \tc{blue}{Fraction of highly vulnerable individuals ($P_1$).}

  Details unimportant:\\
  Many threshold and dose distributions give same $P_k$.

  \ding{228} Most vulnerable/gullible population may be more important than
  small group of super-spreaders or influentials.

\end{frame}

\begin{frame}
  \frametitle{Future work/questions}
  \ding{228} Do any real diseases work like this?\\
  \mbox{}\newline
  \ding{228} Examine model's behavior on networks\\
  \mbox{}\newline
  \ding{228} Media/advertising + social networks model\\
  \mbox{}\newline
  \ding{228} Classify real-world contagions

\end{frame}


