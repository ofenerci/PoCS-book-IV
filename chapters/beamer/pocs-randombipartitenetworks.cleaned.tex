%% hand draw some slides

%% include paper on primes

\section{Introduction}

\framedisplaypaper{ahn2011a}{1}{fig1}

\framedisplaypaper{ahn2011a}{1}{fig2}

\framedisplaypaper{teng2012a}{1}{fig2}

\framedisplaypaper{hidalgo2007a}{2}{fig1b_mod}

  \textbf{Networks and creativity:}

      
    \includegraphics[width=\textwidth]{guimera2005b_fig2}
    
    
    
      Guimer\`{a} et al., Science 2005:\cite{guimera2005b}
      ``Team Assembly Mechanisms Determine Collaboration Network Structure and Team Performance''
     
      Broadway musical industry
     
      Scientific collaboration in Social Psychology, Economics, Ecology, and Astronomy.
    
  

\framedisplaypaper{goh2007a}{1}{fig2a}

\framedisplaypaper{garcia-perez2014a}{1}{fig1}

  \textbf{Random bipartite networks:}

  \textbf{We'll follow this paper:}
    \displaypaper{newman2001b}{1}
  

  \begin{center}
    \includegraphics[height=0.5\textheight]{newman2001b_fig2-tp-5}
  \end{center}


\begin{center}
      \includegraphics[height=\textwidth]{2014-04-08bipartite-network-example-cropped01-dark-tp-10.png}
  \end{center}


\section{Basic\ story}



  \textbf{Basic story:}
    
    
      An example of two inter-affiliated types:
      
      
        $\rbone$ = stories, 
      
        $\rbtwo$ = \wordwikilink{http://tvtropes.org/}{tropes}.
      
    
      Stories contain tropes, tropes are in stories.
    
      Consider a story-trope system with $N_{\rbone}$ = \# stories and $N_{\rbtwo}$ = \# tropes.
    
      $m_{\rbone,\rbtwo}$ = number of edges between $\rbone$ and $\rbtwo$.
    
      Let's have some underlying degree distributions:
      $P^{(\rbone)}_{k}$
      and
      $P^{(\rbtwo)}_{k}$.
    
      Average number of affiliations:
      $\tavg{k}_{\rbone}$ and $\tavg{k}_{\rbtwo}$.
      
      
        $\tavg{k}_{\rbone}$ = average number of tropes per story.
      
        $\tavg{k}_{\rbtwo}$ = average number of stories containing a given trope.
      
    
      Must have balance:
      $
      N_{\rbone} \cdot \tavg{k}_{\rbone}
      =
      m_{\rbone,\rbtwo}
      =
      N_{\rbtwo} \cdot \tavg{k}_{\rbtwo}.
      $
    
  
  


  \textbf{Usual helpers for understanding network's structure:}
    
     
      Randomly select an edge connecting 
      a $\rbone$ to a $\rbtwo$.
    
      Probability the $\rbone$ contains $k$ other tropes:
      $$
      R^{(\rbone)}_{k}
      =
      \frac{
        (k+1)P^{(\rbone)}_{k+1}
      }
      {
        \sum_{j=0}^{N_\rbone}(j+1)P^{(\rbone)}_{j+1}
      }
      =
      \frac{
        (k+1)P^{(\rbone)}_{k+1}
      }
      {
        \tavg{k}_\rbone
      }.
      $$
    
      Probability the $\rbtwo$ is in $k$ other stories:
      $$
      R^{(\rbtwo)}_{k}
      =
      \frac{
        (k+1)P^{(\rbtwo)}_{k+1}
      }
      {
        \sum_{j=0}^{N_\rbtwo}(j+1)P^{(\rbtwo)}_{j+1}
      }
      =
      \frac{
        (k+1)P^{(\rbtwo)}_{k+1}
      }
      {
        \tavg{k}_\rbtwo
      }.
      $$
    
  



  \textbf{Induced networks of $\rbone$ and $\rbtwo$:}
    
     
      $\Prboneind$ = probability a random $\rbone$ is connected
      to $k$ stories by sharing at least one $\rbtwo$.
     
      $\Prbtwoind$ = probability a random $\rbtwo$ is connected
      to $k$ tropes by co-occurring in at least one $\rbone$.
     
      $\Rrboneind$ = probability a random edge leads to a $\rbone$
      which is connected
      to $k$ other stories by sharing at least one $\rbtwo$.
     
      $\Rrbtwoind$ = probability a random edge leads to a $\rbtwo$ 
      which is connected
      to $k$ other tropes by co-occurring in at least one $\rbone$.
     
      Goal: find these distributions.
     
      Another goal: find the induced distribution of component sizes
      and a test for the presence or absence of a giant component.
     
      Unrelated goal: be 10\% happier/weep less.
    
  



  \textbf{Generating\ Function\ Madness}

  \textbf{Yes, we're doing it:}
    
     
      $
      F_{P^{(\rbone)}}(x)
      =
      \sum_{k=0}^{\infty}
      P^{(\rbone)}_{k} x^k
      $
     
      $
      F_{P^{(\rbtwo)}}(x)
      =
      \sum_{k=0}^{\infty}
      P^{(\rbtwo)}_{k} x^k
      $
     
      $
      F_{R^{(\rbone)}}(x)
      =
      \sum_{k=0}^{\infty}
      R^{(\rbone)}_{k} x^k
      =
      \frac{
        F'_{P^{(\rbone)}}(x)
      }
      {
        F'_{P^{(\rbone)}}(1)
      }
      $
     
      $
      F_{R^{(\rbtwo)}}(x)
      =
      \sum_{k=0}^{\infty}
      R^{(\rbtwo)}_{k} x^k
      =
      \frac{
        F'_{P^{(\rbtwo)}}(x)
      }
      {
        F'_{P^{(\rbtwo)}}(1)
      }
      $
    
  

  \textbf{The usual goodness:}
    
     
      Normalization:
      $
      F_{P^{(\rbone)}}(1) 
      =
      F_{P^{(\rbtwo)}}(1) 
      =
      1
      $.
    
      Means:
      $
      F'_{P^{(\rbone)}}(1)
      =
      \tavg{k}_{\rbone}
      $
      and
      $
      F'_{P^{(\rbtwo)}}(1)
      =
      \tavg{k}_{\rbtwo}.
      $
    
  



  \textbf{We strap these in as well:}
    
    
      $
      F_{\Prboneind}(x)
      =
      \sum_{k=0}^{\infty}
      \Prboneind x^k
      $
    
      $
      F_{\Prbtwoind}(x)
      =
      \sum_{k=0}^{\infty}
      \Prbtwoind x^k
      $
    
      $
      F_{\Rrboneind}(x)
      =
      \sum_{k=0}^{\infty}
      \Rrboneind x^k
      $
    
      $
      F_{\Rrbtwoind}(x)
      =
      \sum_{k=0}^{\infty}
      \Rrbtwoind x^k
      $
    
  

  \textbf{So how do all these things connect?}
    
    
      We're again performing sums of a randomly chosen number
      of randomly chosen numbers.
    
      We use one of our favorite sneaky tricks:
      $$
      W 
      = 
      \sum_{i=1}^{U} V^{(i)}
      \rightleftharpoons
      F_W(x)
      =
      F_U(F_V(x)).
      $$
    
  


  \textbf{Induced distributions are not straightforward:}
    \begin{center}
      \includegraphics[height=0.5\textheight]{newman2001b_fig7-tp-5_mod}
    \end{center}
    
     
      View this as $\Prboneind$ (the probability a story shares tropes
      with $k$ other stories).\cite{newman2001b}
     
      Result of purely random wiring.
     
      Parameters: 
      $N_\rbone = 10^4$,
      $N_\rbtwo = 10^5$,\newline
      $\tavg{k}_\rbone = 1.5$,
      and
      $\tavg{k}_\rbtwo = 15$.
    
  
  


  \textbf{Induced distributions for stories:}
    
    
      Randomly choose a $\rbone$, find its tropes ($U$),
      and then find how many other stories each of those tropes 
      are part of ($V$):
      $$
      F_{\Prboneind}(x)
      =
      F_{P^{(\rbone)}}
      \left(
        F_{R^{(\rbtwo)}}(x)      
      \right)
      $$
    
      Find the $\rbone$ at the end of a randomly chosen edge, find its tropes ($U$),
      and then find how many other stories each of those tropes 
      are part of ($V$):
      $$
      F_{\Rrboneind}(x)
      =
      F_{R^{(\rbone)}}
      \left(
        F_{R^{(\rbtwo)}}(x)      
      \right)
      $$
    
  



  \textbf{Induced distributions for tropes:}
    
    
      Randomly choose a $\rbtwo$, find the stories its part of ($U$),
      and then find how many other tropes are part of those stories ($V$):
      $$
      F_{\Prbtwoind}(x)
      =
      F_{P^{(\rbtwo)}}
      \left(
        F_{R^{(\rbone)}}(x)      
      \right)
      $$
    
      Find the $\rbtwo$ at the end of a randomly chosen edge, 
      find the
      stories that use it ($U$),
      and then find how many other tropes are in those stories
      ($V$):
      $$
      F_{\Rrbtwoind}(x)
      =
      F_{R^{(\rbtwo)}}
      \left(
        F_{R^{(\rbone)}}(x)      
      \right)
      $$
    
  


  \textbf{Let's do some good:}
    
    
      Average number of stories connected to a story
      through trope-space:
      $$
      \tavg{k}_{\rbone,\textnormal{ind}}
      =
      F'_{\Prboneind}(1)
      $$
    
      So:
      $$
      \tavg{k}_{\rbone,\textnormal{ind}}
      =
      \left.
      \diff{}{x}
      F_{P^{(\rbone)}}
      \left(
        F_{R^{(\rbtwo)}}(x)      
      \right)
      \right|_{x=1}
      $$
      $$
      {
      =
      F'_{R^{(\rbtwo)}}(1)
      F'_{P^{(\rbone)}}
      \left(
        F_{R^{(\rbtwo)}}(1)      
      \right)
      }
      {
      =
      F'_{R^{(\rbtwo)}}(1)
      F'_{P^{(\rbone)}}(1)
    }
      $$
    
      Similarly, the 
      average number of tropes connected to a random trope
      through stories:
      $$
      \tavg{k}_{\rbtwo,\textnormal{ind}}
      =
      F'_{R^{(\rbone)}}(1)
      F'_{P^{(\rbtwo)}}(1)
      $$
    
      In terms of the underlying distributions, we have:
      $
      \tavg{k}_{\rbone,\textnormal{ind}}
      = 
      \frac{\tavg{k(k-1)}_{\rbtwo}}
      {\tavg{k}_{\rbtwo}}
      \tavg{k}_{\rbone}
      $
      and
      $
      \tavg{k}_{\rbtwo,\textnormal{ind}}
      = 
      \frac{\tavg{k(k-1)}_{\rbone}}
      {\tavg{k}_{\rbone}}
      \tavg{k}_{\rbtwo}
      $
    
  
  


  \textbf{Next: is this thing connected?}
    
    
      Always about the edges: when following a random edge toward
      a $\rbone$, what's the expected number
      of new edges leading to other stories via tropes?
    
      We want to determine $\tavg{k}_{R,\rbone,\textnormal{ind}} = F'_{\Rrboneind}(1)$ 
      (and $F'_{\Rrbtwoind}(1)$ for the trope side of things).
    
      We compute with joy:
      $$
      \tavg{k}_{R,\rbone,\textnormal{ind}}
      =
      \left.
      \diff{}{x}
        F_{\Rrboneind}(x)
      \right|_{x=1}
      =
      {
      \left.
      \diff{}{x}
        F_{R^{(\rbone)}}
        \left(
          F_{R^{(\rbtwo)}}(x)      
        \right)
      \right|_{x=1}
    }
    $$
    $$
    {
      =
      F'_{R^{(\rbtwo)}}(1)
        F'_{R^{(\rbone)}}
        \left(
          F_{R^{(\rbtwo)}}(1)
        \right)
      }
      {
        =
        F'_{R^{(\rbtwo)}}(1)
        F'_{R^{(\rbone)}}(1)
      }
      $$
    
      Note symmetry. 
    
      \$happiness++;
    
  
  


  
    
    
      In terms of the underlying distributions:
      $$
      \tavg{k}_{R,\rbone,\textnormal{ind}}
      =
      \frac{\tavg{k(k-1)}_{\rbone}}
      {\tavg{k}_{\rbone}}
      \frac{\tavg{k(k-1)}_{\rbtwo}}
      {\tavg{k}_{\rbtwo}}
      $$
    
      We have a giant component in \alertb{both} 
      induced networks when 
      $$
      \tavg{k}_{R,\rbone,\textnormal{ind}}
      \equiv
      \tavg{k}_{R,\rbtwo,\textnormal{ind}}
      > 1
      $$.
    
      See this as the product of two gain ratios. \newline
      {\#excellent} 
      {\#physics}
    
      We can mess with this condition to make it mathematically pleasant
      and pleasantly inscrutable:
      $$
      \sum_{k=0}^{\infty}
      \sum_{k'=0}^{\infty}
      kk'(kk'-k-k') P^{(\rbone)}_{k} P^{(\rbtwo)}_{k'} = 0.
      $$
      {\#reallynotthebest}
    
  


  \textbf{To come:}
    
     
      Distributions of component size.
     
      Simpler computation for the giant component condition.
     
      Contagion.
     
      Testing real bipartite structures for departure from randomness.
    
  

\begin{comment}


  




  

  \begin{center}
    \includegraphics[height=0.7\textheight]{newman2001b_fig7-tp-5}
  \end{center}


  

  \begin{center}
    \includegraphics[height=0.7\textheight]{newman2001b_fig8-tp-5}
  \end{center}


  

  \begin{center}
    \includegraphics[height=0.7\textheight]{newman2001b_fig9-tp-5}
  \end{center}


  

  \begin{center}
    \includegraphics[height=0.7\textheight]{newman2001b_fig10-tp-5}
  \end{center}



\section{Nutshell}

\end{comment}
