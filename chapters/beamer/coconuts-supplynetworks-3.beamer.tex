\changelecturelogo{.18}{gastner2006c_fig6b-tp-1}

\section{Distributed\ Sources}

%% \subsection{Facility location}

\begin{frame}
  \frametitle{Many sources, many sinks}
  
  \begin{block}<1->{How do we distribute sources?}
    \begin{itemize}
    \item<2-> Focus on 2-d (results generalize to higher dimensions).
    \item<3-> Sources = hospitals, post offices, pubs, ...
    \item<4-> \alert{Key problem:} How do we cope with uneven population densities?
    \item<5-> Obvious: if density is uniform then sources are best distributed
      \alert{uniformly}.
    \item<6-> Which lattice is optimal? \uncover<7->{The \alert{hexagonal lattice}}
    \item<8-> \alert{Q2:} Given population density is uneven, what do we do?
    \item<9-> We'll follow work by Stephan (1977, 1984)\cite{stephan1977a,stephan1984a},
      Gastner and Newman (2006)\cite{gastner2006c}, 
      Um \etal (2009)\cite{um2009a} and work cited by them.
    \end{itemize}
  \end{block}

\end{frame}


\begin{frame}
  \frametitle{Optimal source allocation}

  \begin{block}<1->{Solidifying the basic problem}
    \begin{itemize}
    \item<1-> Given a region with some population distribution $\rho$, most likely uneven.
    \item<2-> Given resources to build and maintain $N$ facilities.
    \item<3-> \alert{Q:} How do we locate these $N$ facilities so as to
      \alert{minimize the average distance} between an \alertb{individual's residence} and 
      the \alertb{nearest facility}?
    \end{itemize}
  \end{block}
\end{frame}

\begin{frame}
  \frametitle{Optimal source allocation}

  {\centering
    \includegraphics[width=0.9\textwidth]{gastner2006c_fig1}
  }

  {\small 
    From Gastner and Newman (2006)\cite{gastner2006c}\\
    \begin{itemize}
    \item<1-> Approximately optimal location of 5000 facilities.
    \item<1-> Based on 2000 Census data.
    \item<1-> Simulated annealing + Voronoi tessellation.
    \end{itemize}
  }

\end{frame}

\begin{frame}
  \frametitle{Optimal source allocation}

  \begin{center}
    \includegraphics[width=0.7\textwidth]{gastner2006c_fig2}
  \end{center}
  {\small
    From Gastner and Newman (2006)\cite{gastner2006c}
  }
  \begin{itemize}
  \item<1-> Optimal facility density $\rhofac$ vs.\ population density $\rhopop$.
  \item<2-> Fit is $\rhofac \propto \rhopop^{0.66}$ with $r^2 = 0.94$.
  \item<3-> Looking good for a 2/3 power...
  \end{itemize}
  
\end{frame}


\subsection{Size-density\ law}

\begin{frame}
  \frametitle{Optimal source allocation}

  \begin{block}<1->{Size-density law:}
    \begin{itemize}
    \item<1->
    $$
    \boxed{\alert{\rhofac \propto \rhopop^{2/3}}}
    $$
    \item<2->
      Why?
    \item<3->
      Again: Different story to branching networks where
      there was either one source or one sink.
    \item<4->
      Now sources \& sinks are distributed 
      throughout region...
    \end{itemize}
  \end{block}

\end{frame}

\begin{frame}
  \frametitle{Optimal source allocation}

    \begin{itemize}
    \item<+-> 
      We first examine Stephan's treatment (1977)\cite{stephan1977a,stephan1984a}
    \item<+->
      \alertb{``Territorial Division: The Least-Time Constraint
        Behind the Formation of Subnational Boundaries''} (Science, 1977)
    \item<+->
      Zipf-like approach: invokes \alert{principle of minimal effort}.
    \item<+->
      Also known as the Homer principle.
    \end{itemize}

\end{frame}

\begin{frame}
  \frametitle{Optimal source allocation}

  \begin{itemize}
  \item<1-> 
    Consider a region of area $A$ and population $P$ with
    a single functional center that everyone needs to access
    every day.
  \item<2->
    Build up a general cost function based on time expended
    to \alert{access and maintain center}.
  \item<3->
    Write \alert{average travel distance} to center as $\bar{d}$ and 
    assume \alert{average speed of travel} is $\bar{v}$.
  \item<4->
    Assume \alertb{isometry}: average travel distance $\bar{d}$ will be on the length
    scale of the region which is $\sim$ \alertb{$A^{1/2}$}
  \item<5->
    Average time expended per person in accessing facility
    is therefore
    \alertb{
      $$
      \bar{d}/\bar{v} = c A^{1/2} / \bar{v}
      $$
    }
    where $c$ is an unimportant shape factor.
  \end{itemize}

\end{frame}

\begin{frame}
  \frametitle{Optimal source allocation}

  \begin{itemize}
  \item<1-> Next assume facility requires regular maintenance (person-hours per day)
  \item<2-> Call this quantity $\tau$
  \item<3-> If burden of mainenance is shared then average cost per person
    is \alert{$\tau/P$} where $P$ = population.
  \item<4-> Replace $P$ by $\rhopop A$ where $\rhopop$ is density.
  \item<5-> Total average time cost per person:
    $$
    T = \bar{d}/\bar{v} + \tau/(\rhopop A) 
    \uncover<6->{= c \alert{A^{1/2}}/\bar{v} + \tau/(\rhopop \alert{A}).}
    $$
  \item<7-> Now Minimize with respect to $A$...
  \end{itemize}

\end{frame}

\begin{frame}
  \frametitle{Optimal source allocation}

  \begin{itemize}
  \item<1-> Differentiating...
    $$
    \partialdiff{T}{A} = 
    \partialdiff{}{A} \left( c A^{1/2}/\bar{v} + \tau/(\rhopop A) \right)
    $$
    $$
    \uncover<2->{
      =
      \frac{c}{2\bar{v} A^{1/2}}
      -\frac{\tau}{\rhopop A^2}
    }
    \uncover<3->{
      \alert{ = 0 }
    }
    $$
  \item<4-> Rearrange:
    $$
    A = 
    \left(
      \frac{2 \bar{v} \tau}
      {c \rhopop}
    \right)^{2/3}
    \uncover<5->{
      \propto \rhopop^{-2/3}
    }
    $$
  \item<6-> \# facilities per unit area $\rhofac$:
    $$ 
    \rhofac
    \propto
    \alert{A^{-1}  \propto \rhopop^{2/3}}
    $$
  \item<7-> Groovy...
    
  \end{itemize}

\end{frame}

\begin{frame}
  \frametitle{Optimal source allocation}

  \begin{block}{An issue:}
    \begin{itemize}
    \item<1-> Maintenance ($\tau$) is assumed to be 
      \alert{independent} of population
      and area ($P$ and $A$)
    \end{itemize}
  \end{block}
  
\end{frame}

\begin{frame}
  \frametitle{Optimal source allocation}

  \begin{itemize}
  \item 
    Stephan's online book\\
    \alert{``The Division of Territory in Society''}
    is
    \wordwikilink{http://www.edstephan.org/Book/contents.html}{here}.
  \item 
    (It used to be
    \wordwikilink{http://www.ac.wwu.edu/~stephan/Book/contents.html}{here}.)
  \item 
    The 
    \wordwikilink{http://www.edstephan.org/Book/chap0/0.html}{Readme} 
    is well worth reading (1995).
  \end{itemize}

  %% winner of the first Zipf award!
  %% George Kingsley Zipf 
  %% Memorial Award
  %% 1984 Population Association of America
\end{frame}

\subsection{Cartograms}

\begin{frame}
  \frametitle{Cartograms}

  Standard world map:
  \includegraphics[width=\textwidth]{newman_world1024x512.png}

\end{frame}

\begin{frame}
  \frametitle{Cartograms}

  Cartogram of countries `rescaled' by population:
  \includegraphics[width=\textwidth]{newman_population1024x512.png}\\
  \includegraphics[width=0.25\textwidth]{newman_world1024x512.png}
\end{frame}

\begin{frame}
  \frametitle{Cartograms}

  \begin{block}<1->{Diffusion-based cartograms:}
    \begin{itemize}
    \item<2-> Idea of cartograms is to \alert{distort areas} to 
      more accurately represent
      some local density $\rhopop$ (e.g. population).
    \item<3-> Many methods put forward---typically involve
      some kind of physical analogy to \alert{spreading or repulsion}.
    \item<4-> Algorithm due to Gastner and Newman (2004)\cite{gastner2004a}
      is based on \alertb{standard diffusion}:
      $$ 
      \nabla^2 \rhopop - \partialdiff{\rhopop}{t} = 0. 
      $$
    \item<5-> Allow density to diffuse and trace the 
      movement of individual elements and boundaries.
    \item<6-> Diffusion is constrained by boundary condition
      of surrounding area having density $\bar{\rhopop}$.
    \end{itemize}
  \end{block}

\end{frame}

\begin{frame}
  \frametitle{Cartograms}

  Child mortality:
  \includegraphics[width=\textwidth]{newman_childmort1024x512.png}

\end{frame}

\begin{frame}
  \frametitle{Cartograms}

  Energy consumption:
  \includegraphics[width=\textwidth]{newman_energyconsump1024x512.png}
\end{frame}

\begin{frame}
  \frametitle{Cartograms}

  Gross domestic product:
  \includegraphics[width=\textwidth]{newman_gdp1024x512.png}
\end{frame}

\begin{frame}
  \frametitle{Cartograms}

  Greenhouse gas emissions:
  \includegraphics[width=\textwidth]{newman_greenhouse1024x512.png}
\end{frame}

\begin{frame}
  \frametitle{Cartograms}

  Spending on healthcare:
  \includegraphics[width=\textwidth]{newman_healthcare1024x512.png}
\end{frame}

\begin{frame}
  \frametitle{Cartograms}
  
  People living with HIV:
  \includegraphics[width=\textwidth]{newman_hiv1024x512.png}
\end{frame}


\begin{frame}
  \frametitle{Cartograms}

  \begin{itemize}
  \item<1-> The preceding sampling of Gastner \& Newman's cartograms
    lives \wordwikilink{http://www-personal.umich.edu/~mejn/cartograms/}{here}.
  \item<1->
    A larger collection can be found
    at \wordwikilink{http://www.worldmapper.org/}{worldmapper.org}.

    \bigskip

    \includegraphics[width=0.5\textwidth]{worldmapper.png}
  \end{itemize}

\end{frame}

\begin{frame}
  \frametitle{Size-density law}

  \includegraphics[width=\textwidth]{gastner2006c_fig3}

  \begin{itemize}
  \item <1-> \alert{Left:} population density-equalized cartogram.
  \item <2-> \alert{Right:} (population density)$^{2/3}$-equalized cartogram.
  \item <3-> Facility density is uniform for $\rhopop^{2/3}$ cartogram.
  \end{itemize}
  {\small
    From Gastner and Newman (2006)\cite{gastner2006c}
  }
\end{frame}

%% \begin{frame}
%%   \frametitle{}
%% 
%%   \includegraphics[width=\textwidth]{gastner2006c_fig4}
%% 
%%   From Gastner and Newman (2006)\cite{gastner2006c}
%% \end{frame}

\begin{frame}
  \frametitle{Size-density law}

  \includegraphics[width=\textwidth]{gastner2006c_fig5}

  {\small
    From Gastner and Newman (2006)\cite{gastner2006c}
  }
  \begin{itemize}
  \item Cartogram's Voronoi cells are somewhat hexagonal.
  \end{itemize}
  
\end{frame}

\subsection{A\ reasonable\ derivation}

\begin{frame}
  \frametitle{Size-density law}

  \begin{block}<1->{Deriving the optimal source distribution:}
    \begin{itemize}
    \item<2-> \alert{Basic idea:} Minimize the average distance
      from a random individual to the nearest facility.\cite{gastner2006c}
    \item<3-> Assume given a fixed population density $\rhopop$ defined
      on a spatial region $\Om$.
    \item<4-> Formally, we want to find the locations of 
      \alert{$n$ sources} $\{\vec{x}_1,\ldots,\vec{x}_n\}$
      that minimizes the \alert{cost function}
      $$
      F(\{\vec{x}_1,\ldots,\vec{x}_n\})
      =
      \int_{\Om}
      \alert{\rhopop(\vec{x})}
      \,
      \alertb{\min_{i}
      || \vec{x} - \vec{x}_i ||}
      \dee{\vec{x}}.
      $$
    \item<5-> Also known as the p-median problem.
    \item<6-> Not easy...  \uncover<6->{in fact this one is an NP-hard problem.\cite{gastner2006c}}
    \item<7-> Approximate solution originally due to
      Gusein-Zade\cite{gusein-zade1982a}.
    \end{itemize}
  \end{block}

\end{frame}

\begin{frame}
  \frametitle{Size-density law}

  \begin{block}{Approximations:}
    \begin{itemize}
    \item<1-> For a given set of source placements $\{\vec{x}_1,\ldots,\vec{x}_n\}$,
      the region $\Om$ is divided up into 
      \wordwikilink{http://en.wikipedia.org/wiki/Voronoi_diagram}{Voronoi cells},
      one per source.
    \item<2->
      Define \alert{$A(\vec{x})$} as the \alert{area} of the 
      Voronoi cell containing $\vec{x}$.
    \item<3-> As per Stephan's calculation, estimate
      typical distance from $\vec{x}$ to the nearest source (say $i$)
      as 
      $$
      \alertb{c_i A(\vec{x})^{1/2}}
      $$
      where $c_i$ is a shape factor for the $i$th Voronoi cell.
    \item<4-> 
      Approximate $c_i$ as a constant $c$.
    \end{itemize}
  \end{block}

\end{frame}

\begin{frame}
  \frametitle{Size-density law}

  \begin{block}{Carrying on:}
    \begin{itemize}
    \item<1-> The cost function is now
      $$
      F
      =
      c \int_{\Om}
      \alertb{\rhopop(\vec{x})}
      \alertb{ A(\vec{x})^{1/2}}
      \dee{\vec{x}}.
      $$
    \item<2-> We also have that the \alert{constraint} that
      Voronoi cells divide up the overall area
      of $\Om$:
      $
      \sum_{i=1}^{n} A(\vec{x}_i) = A_\Om.
      $
    \item<3-> Sneakily turn this into an integral constraint:
      $$
      \int_\Om
      \frac{\dee{\vec{x}}}
      {A(\vec{x})}
      = n.
      $$
    \item<4->
      Within each cell, $A(\vec{x})$ is constant.
    \item<5->
      So... integral over each of the $n$ cells equals 1.
    \end{itemize}
  \end{block}

\end{frame}

\begin{frame}
  \frametitle{Size-density law}

  \begin{block}{Now a Lagrange multiplier story:}
    \begin{itemize}
    \item<1-> By varying $\{\vec{x}_1,...,\vec{x}_n\}$, minimize
      $$
      G(A) = 
      c \int_{\Om}
      \alertb{\rhopop(\vec{x})}
      \alertb{ A(\vec{x})^{1/2}}
      \dee{\vec{x}}
      -
      \lambda
      \left(n -
        \int_\Om
        \left[A(\vec{x})\right]^{-1}
        \dee{\vec{x}}
      \right)
      $$
    \item<2->
      I Can Haz
      \wordwikilink{http://en.wikipedia.org/wiki/Calculus\_of\_variations}{Calculus of Variations}?
    \item<3->
      Compute
      $\delta G / \delta A$,
      the \wordwikilink{http://en.wikipedia.org/wiki/Functional_derivative}{functional derivative}
      of the functional $G(A)$.
    \item<4-> This gives
      $$
      \int_{\Om}
      \left[
        \frac{c}{2} \alertb{\rhopop(\vec{x})}
        \alertb{ A(\vec{x})^{-1/2}}
        -
        \lambda
        \left[A(\vec{x})\right]^{-2}
      \right]
      \dee{\vec{x}} = 0.
      $$
    \item<5-> Setting the integrand to be zilch, we have:
      $$
      \rhopop(\vec{x})
      =
      2\lambda
      c^{-1}
      A(\vec{x})^{-3/2}.
      $$
    \end{itemize}
  \end{block}

\end{frame}


\begin{frame}
  \frametitle{Size-density law}

  \begin{block}{Now a Lagrange multiplier story:}
    \begin{itemize}
    \item<1-> Rearranging, we have
      $$
      A(\vec{x}) = (2{\lambda} c^{-1})^{2/3} \rhopop^{-2/3}.
      $$
    \item<2->
      Finally, we indentify $1/A(\vec{x})$ as $\rhofac(\vec{x})$,
      an approximation of the local source density.
    \item<3-> Substituting $\rhofac=1/A$, we have
      $$
      \alert{\rhofac(\vec{x})
      = \left( 
        \frac{c}{2{\lambda}}
        \rhopop
    \right)^{2/3}}.
      $$
    \item<4-> Normalizing (or solving for $\lambda$):
      $$
      \alert{\rhofac(\vec{x})}
      =  n 
      \frac{[\rhopop(\vec{x})]^{2/3}}
      {\int_{\Om} [\rhopop(\vec{x})]^{2/3} \dee{\vec{x}}}
      \alert{\propto [\rhopop(\vec{x})]^{2/3}}.
      $$
    \end{itemize}
  \end{block}

\end{frame}

\subsection{Global\ redistribution}

\begin{frame}
  \frametitle{Global redistribution networks}

  \begin{block}<1->{One more thing:}
    \begin{itemize}
    \item<1-> How do we supply these facilities?
    \item<2-> How do we best redistribute mail?  People?
    \item<3-> How do we get beer to the pubs?
    \item<4-> Gaster and Newman model: cost is 
      a function of basic maintenance and travel time:
      $$
      C_{\textnormal{maint}} + \gamma C_{\textnormal{travel}}.
      $$
    \item<5-> Travel time is more complicated:
      Take `distance' between nodes to be a composite
      of shortest path distance $\ell_{ij}$ and 
      number of legs to journey:
      $$
      (1-\delta) \ell_{ij} + \delta (\# \mbox{hops}).
      $$
    \item<6-> When $\delta=1$, only number of hops matters.
      
      
    \end{itemize}
  \end{block}

\end{frame}

\begin{frame}
  \frametitle{Global redistribution networks}

  \includegraphics[width=\textwidth]{gastner2006c_fig6}

  From Gastner and Newman (2006)\cite{gastner2006c}
\end{frame}

\subsection{Public\ versus\ Private}

\begin{frame}
  \frametitle{Public versus private facilities}

  \begin{block}<1->{Beyond minimizing distances:}
    \begin{itemize}
    \item<2->
      ``Scaling laws between population and facility densities'' by
      Um et al., Proc. Natl. Acad. Sci., 2009.\cite{um2009a}
    \item<3->
      Um et al.\ find empirically and argue theoretically that the connection
      between facility and population density
      $$
      \rhofac \propto \rhopop^{\alpha}
      $$
      \alertb{does not universally hold} with $\alpha=2/3$.
    \item<4->
      \alert{Two idealized limiting classes}:
      \begin{enumerate}
      \item<4->
        For-profit, commercial facilities: \alertb{$\alpha = 1$};
      \item<5->
        Pro-social, public facilities: \alertb{$\alpha = 2/3$}.
      \end{enumerate}
    \item<5->
      Um et al.\ investigate facility locations in the United States
      and South Korea.
    \end{itemize}
  \end{block}
  
\end{frame}


\begin{frame}
  \frametitle{Public versus private facilities: evidence}

  \includegraphics[width=0.49\textwidth]{um2009a_fig1A.pdf}
  \includegraphics[width=0.49\textwidth]{um2009a_fig1B.pdf}

  \begin{itemize}
  \item<1->
    \alert{Left plot:} ambulatory hospitals in the U.S.
  \item<1-> 
    \alert{Right plot:} public schools in the U.S.
  \item<2->
    Note: break in scaling for public schools.
    Transition from $\alpha \simeq 2/3$ to 
    $\alpha = 1$ around $\rhopop \simeq 100$.
  \end{itemize}

\end{frame}

\begin{frame}
  \frametitle{Public versus private facilities: evidence}

  \begin{columns}
    \column{0.7\textwidth}
    \includegraphics[width=\textwidth]{um2009a_tab1A.pdf}\\
    \includegraphics[width=\textwidth]{um2009a_tab1B.pdf}
    \column{0.3\textwidth}
      {\small
        Rough \alertb{transition} between public and private at $\alpha \simeq 0.8$.

        \medskip

        Note: * indicates analysis is at state/province level; otherwise county level.}
  \end{columns}

\end{frame}

\begin{frame}
  \frametitle{Public versus private facilities: evidence}

  \includegraphics[width=\textwidth]{um2009a_fig2.pdf}

  \alert{A, C:} ambulatory hospitals in the U.S.;
  \alert{B, D:} public schools in the U.S.;
  \alert{A, B:} data; 
  \alert{C, D:} Voronoi diagram from model simulation.

\end{frame}

\begin{frame}
  \frametitle{Public versus private facilities: the story}

  \begin{block}<1->{So what's going on?}
    \begin{itemize}
    \item<1->
      Social institutions seek to \alertb{minimize distance of travel}.
    \item<2->
      Commercial institutions seek to \alertb{maximize the
      number of visitors}.
    \item<3->
      \alertb{Defns:} For the $i$th facility and its Voronoi cell $V_i$, define
      \begin{itemize}
      \item
        $n_i$ = population of the $i$th cell;
      \item
        $\tavg{r_i}$ = the average travel distance
        to the $i$th facility.
      \item
        $s_i$ = area of $i$th cell.
      \end{itemize}
    \item<4->
      Objective function to maximize for a facility (highly constructed):
      $$ 
      \alertb{v_i = n_i \tavg{r_i}^\beta}
      \
      \mbox{with}
      \
      0 \le \beta \le 1.
      $$
    \item<5->
      Limits:
      \begin{itemize}
      \item $\beta = 0$: purely commercial.
      \item $\beta = 1$: purely social.
      \end{itemize}
    \end{itemize}
  \end{block}

\end{frame}

\begin{frame}
  \frametitle{Public versus private facilities: the story}

  \begin{itemize}
  \item<1-> 
    Proceeding as per the Gastner-Newman-Gusein-Zade calculation,
    Um et al.\ obtain:
    $$
    \alert{\rhofac(\vec{x})}
    =  n
    \frac{[\rhopop(\vec{x})]^{2/(\beta+2)}}
    {\int_{\Om} [\rhopop(\vec{x})]^{2/(\beta+2)} \dee{\vec{x}}}
    \alert{\propto [\rhopop(\vec{x})]^{2/(\beta+2)}}.
    $$
  \item<2-> 
    For $\beta=0$, $\alpha=1$: commercial scaling is linear.
  \item<3-> 
    For $\beta=1$, $\alpha=2/3$: social scaling is sublinear.
  \item<3-> 
    You can try this too: \insertassignmentquestion{04}{4}{3}.
  \end{itemize}

\end{frame}

