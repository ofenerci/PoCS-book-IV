%%%%%%%%%%%%%%%%%%%%%%%%%%%%%%%%%%%%%
%% basic definitions
%%%%%%%%%%%%%%%%%%%%%%%%%%%%%%%%%%%%%

\changelogo{.18}{icons-lightbulb-tp.pdf}

\section{Defining\ Complexity}
  \textbf{Definitions}

  %% dictionary definition
  \alertb{Complex:} (Latin = with + fold/weave (com + plex))
  \hfill
  \includegraphics[width=.07\textwidth]{wikipedia-tp.pdf}

  \medskip

  \textbf{Adjective:}
    
     Made up of multiple parts; intricate or detailed.
     Not simple or straightforward.
  \textbf{Definitions}

  \textbf{Complicated versus Complex:}
      Complicated: Mechanical watches, airplanes, ...
     
      Engineered systems can be made to be \alertr{highly robust
        but not adaptable}.
    
      But engineered systems can become complex (power grid, planes).
     
      They can also \alertr{fail spectacularly}.
     
      Explicit distinction: \alertb{Complex Adaptive Systems}.
%% add this!!!!
%% http://www.nytimes.com/2013/01/17/business/faa-orders-grounding-of-us-operated-boeing-787s.html
%%  \textbf{Definitions}

  \textbf{
      A while ago:
      \wordwikilink{https://en.wikipedia.org/wiki/Complex\_systems}{The
        Wikipedia on Complex Systems:}
    }
    ``Complexity science is not a single theory: 
    it encompasses more than one theoretical framework and is highly
    interdisciplinary, seeking the answers to some fundamental questions
    about living, adaptable, changeable systems.''
  \textbf{
      \wordwikilink{https://en.wikipedia.org/wiki/Complex\_systems}{Now:}
      \hfill \includegraphics[width=.07\textwidth]{wikipedia-tp.pdf} 
    }   
    ``Complex systems present problems both in mathematical modelling
and philosophical foundations. The study of complex systems represents
a new approach to science that investigates how relationships between
parts give rise to the collective behaviors of a system and how the
system interacts and forms relationships with its environment.''
  \textbf{Definitions}

  \textbf{
      \alertb{Nino Boccara} in \textit{Modeling Complex Systems}:}\cite{boccara2004a}
    ``... there is no universally accepted definition
    of a complex system ... most researchers would describe
    a system of connected agents that exhibits
    an emergent global behavior not imposed by a central
    controller, but resulting from the interactions between
    the agents.''
  \textbf{\alertb{Philip Ball} in \textit{Critical Mass}:}\cite{ball2004a}
    ``...complexity theory seeks to understand how order and 
    stability arise from the interactions of many components
    according to a few simple rules.''
%%   \textbf{\alertb{Steve Strogatz} in \textit{Sync}:}
%%     ``... every decade or so, a grandiose theory comes along, bearing
%%     similar aspirations and often brandishing an ominous-sounding
%%     C-name. In the 1960s it was cybernetics. In the '70s it was
%%     catastrophe theory. Then came chaos theory in the '80s and complexity
%%     theory in the '90s.''
%%   

%%   \textbf{\alertb{Cosma Shalizi:}}
%%     ``The "sciences of complexity" are very much a potpourri, and while the
%%     name has some justification---chaotic motion seems more complicated
%%     than harmonic oscillation, for instance---I think the fact that it
%%     is more dignified than "neat nonlinear nonsense" has not been the
%%     least reason for its success.---That opinion wasn't exactly changed
%%     by working at the Santa Fe Institute for five years.''
%%   

%% 
%%   \textbf{Buzzword Definitions}
%% 
%%   \textbf{\alertb{Nonlinear} (OED)}
%%     1. a. Math. and Physics. Not linear; ...
%%     involving or possessing the property that the magnitude of an
%%     effect or output is not linearly or proportionally related to that
%%     of the cause or input.
%%     \medskip
%%     First cited use 1844.
%%   
%% 
%% 
%% 
%% 
%%   \textbf{Buzzword Definitions}
%% 
%%   \textbf{\alertb{Nonlinear} (OED)}
%%     b. \textit{colloq.} \textbf{to go non-linear:} 
%%     to lose one's head; to rave, esp. about a particular obsession.
%%     \medskip
%%     First cited use 1985.
%%   
%% 
%% 
  \textbf{Definitions}

  %%  \hfill
  %%  \includegraphics[width=.07\textwidth]{wikipedia-tp.pdf}

  \textbf{A working definition of a \alertb{Complex System}:}
      Distributed system of many interrelated (possibly networked) parts
      with no centralized control
      exhibiting 
      emergent behavior---`More is Different'\cite{anderson1972a}
  \textbf{Other features/aspects:}
      Explicit nonlinear relationships.
    
      Presence of feedback loops.
    
      Being open or driven, opaque boundaries.
    
      Memory.
    
      Modular (nested)/multiscale structure.
    
      Mechanisms range from being purely physical to purely algorithmic in nature.
  \textbf{Examples of Complex Systems:}
    
        
          human societies 
        
          financial systems
        
          cells     
        
          ant colonies 
        
          weather systems 
        
          ecosystems     
        
          power grids
      
          animal societies     
        
          disease ecologies    
        
          brains               
        
          social insects       
        
          geophysical systems  
        
          Internet + Web
    
  
  
    i.e., everything that's interesting \ldots
  \textbf{Relevant fields:}
    
       
        Physics
       
        Economics
       
        Sociology
       
        Psychology
       
        Information Sciences
      
        Cognitive Sciences
       
        Biology
       
        Ecology
       
        Geociences
       
        Geography
      
        Medical Sciences
       
        Systems Engineering
       
        Computer Science
       
      \ldots
  
  
    i.e., everything that's interesting \ldots
  \textbf{A visualized history of Complex Systemsish fields:}

  \includegraphics[width=\textwidth]{1200px-Complexity_Map.png}
  
  \attribution{``Complexity Map'' by Brian Castellani/Wiki}

  Online here:
  \wordwikilink{https://en.wikipedia.org/wiki/Complex\_systems\#History}{https://en.wikipedia.org/wiki/Complex\_systems\#History}
  \textbf{The Golden Age of Reductionism:}
  \includegraphics[width=\textwidth]{2014-08-16pocsmap-emergence-reductionism-manifesto_postcard2_polaroid.png}
\changelogo{.18}{2014-08-16pocsmap-emergence-reductionism-manifesto_postcard2_polaroid-logo.png}
  \textbf{Reductionism:}
    
    \includegraphics[width=\textwidth]{200px-Democritus2.jpg}\\
    
    \textbf{
        \wordwikilink{http://en.wikipedia.org/wiki/Democritus}{Democritus}\\
        (ca. 460 BC -- ca. 370 BC)
      }
        Atomic hypothesis
       
        Atom $\sim$ a (not) -- temnein (to cut)
       
        Plato allegedly wanted his books burned.
  

  \medskip
    
    \includegraphics[width=\textwidth]{240px-Dalton_John_desk.jpg}\\
    
    \textbf{
        \wordwikilink{http://en.wikipedia.org/wiki/John\_Dalton}{John Dalton}\\
        1766--1844
      }
        Chemist, Scientist
       
        Developed atomic theory
       
        First estimates of atomic weights
  
  %% \textbf{Reductionism:}

  \textbf{ \wordwikilink{http://en.wikipedia.org/wiki/Ludwig\_Boltzmann}{Ludwig
        Boltzmann}, 1844--1906. Atomic Theory.}
      \includegraphics[width=\textwidth]{225px-Boltzmann2.jpg}\\
      
      \small
      ``Boltzmann's kinetic theory of gases seemed to presuppose the
      reality of atoms and molecules, but almost all German philosophers and
      many scientists like Ernst Mach and the physical chemist Wilhelm
      Ostwald disbelieved their existence.''
  
    \small
    ``In 1904 at a physics conference in St. Louis most
    physicists seemed to reject atoms and he was not even invited
    to the physics section.
    {
      Rather, he was stuck in a section
      called "applied mathematics,"} 
    {
      he violently attacked
      philosophy, especially on allegedly Darwinian grounds
      }
    {
      but
      actually in terms of Lamarck's theory of the inheritance of
      acquired characteristics that people inherited bad philosophy
      }
    {
      from the past and that it was hard for scientists to overcome
      such inheritance.''
    }
    \\
    {\mbox{} \hfill
      See: \wordwikilink{https://en.wikipedia.org/wiki/Epigenetics}{epigenetics}.}
%%  \textbf{Reductionism:}
    
    \includegraphics[width=\textwidth]{220px-Einstein_1921_portrait2.jpg}\\
    
    \textbf{
    \wordwikilink{http://en.wikipedia.org/wiki/Albert\_Einstein}{Albert Einstein}
    1879--1955}
      \wordwikilink{http://en.wikipedia.org/wiki/Annus_Mirabilis_papers}{Annus Mirabilis paper:} ``the Motion of Small Particles Suspended in a Stationary Liquid, as Required by the Molecular Kinetic Theory of Heat''\cite{einstein1905a,einstein1956a}
     
      Showed \wordwikilink{http://en.wikipedia.org/wiki/Brownian_motion}{Brownian motion} 
      followed from an atomic model giving rise to diffusion.
  

  \medskip
    
    \includegraphics[width=\textwidth]{180px-Jean_Baptiste_Perrin.jpg}\\
    
    \textbf{
        \wordwikilink{http://en.wikipedia.org/wiki/Jean\_Perrin}{Jean Perrin}
        1870--1942}
        1908: Experimentally verified Einstein's work and Atomic Theory.
  
  \small

  \textbf{Feynmann:}
      
      ``If, in some cataclysm, all of scientific knowledge were to be
      destroyed, and only one sentence passed on to the next generation of
      creatures, what statement would contain the most information in the
      fewest words?
      
      \includegraphics[width=\textwidth]{Richard_Feynman_Nobel.jpg}
    \smallskip
    {
      ``I believe it is the atomic hypothesis that all things
      are made of atoms}{---little particles that move around in perpetual
      motion, attracting each other when they are a little distance apart,
      but repelling upon being squeezed into one another.
    } 
    {
      ``In that one
      sentence, you will see, there is an enormous amount of information
      about the world, if just a little imagination and thinking are
      applied.''
    }
\tiny
Snared from 
\wordwikilink{http://www.brainpickings.org/index.php/2012/09/11/richard-feynman-lectures-on-physics/}{brainpickings.org}
\changelogo{0.18}{2014-08-16pocsmap-emergence-reductionism-manifesto_postcard3_polaroid-logo.png}

\section{A\ Manifesto}
  %% manifesto
  \includegraphics[width=\textwidth]{2014-08-16pocsmap-emergence-reductionism-manifesto_postcard3_polaroid.png}
%%  
  \small

%%  {\fontsize{8}{10.2}\selectfont

  \textbf{\wordwikilink{http://www.uvm.edu/\%7Epdodds/fama/2015/06/04/complex-sytems-a-manifesto/}{The Science of Complex Systems Manifesto:}}
      Systems are ubiquitous and systems matter.
    
      Consequently, much of science is about understanding
      how pieces dynamically fit together.
    
      1700 to 2000 = Golden Age of Reductionism:\newline
      Atoms!, sub-atomic particles, DNA, genes, people, ...
    
      Understanding and creating systems (including new `atoms')
      is the greater part of science and engineering.
    
      \wordwikilink{http://en.wikipedia.org/wiki/Universality\_(dynamical_systems)}{Universality}: systems with quantitatively different micro details
      exhibit qualitatively similar macro behavior.
    
      Computing advances make the Science of Complex Systems possible:
        We can measure and record enormous amounts of data,
        research areas continue to transition from data scarce to data rich.
      
        We can simulate, model, and create complex systems
        in extraordinary detail.  
  
\neuralreboot{a6QHzIJO5a8}{}{}{Monotrematic Love}
