\section{Basic\ definitions}

\begin{frame}

  \begin{block}{}
    \includegraphics[width=\textwidth]{network_dictionary_cut.pdf}
  \end{block}

\end{frame}

\begin{frame}
%%  \frametitle

  \begin{block}{Thesaurus deliciousness:}
    \begin{center}
      \includegraphics[width=0.9\textwidth]{network_thesaurus_cut.pdf}
    \end{center}
  \end{block}

\end{frame}




%% \begin{frame}
%%   \frametitle{Basic definitions}
%% 
%%   \alert{Network:} (net + work, 1500's)
%%   \hfill
%%   \includegraphics[width=.07\textwidth]{wikipedia.jpg}
%% 
%%   \begin{block}<2->{\alert{Noun:}}
%%     \begin{enumerate}
%%     \item<2-> Any interconnected group or system
%%     \item<3-> Multiple computers and other devices connected together to share information
%%     \end{enumerate}
%%   \end{block}
%% 
%%   \begin{block}<4->{\alert{Verb:}}
%%     \begin{enumerate}
%%     \item<4-> To interact socially for the purpose of getting connections or personal advancement
%%     \item<5-> To connect two or more computers or other computerized devices
%%     \end{enumerate}
%%   \end{block}
%% 
%% \end{frame}

\begin{frame}
  \frametitle{Ancestry:}

  \begin{block}{
      From Keith Briggs's excellent
      \wordwikilink{http://keithbriggs.info/network.html}{etymological investigation:}}
    \begin{columns}
      \column{0.05\textwidth}
      \column{0.4\textwidth}
      \begin{itemize}
      \item<1-> 
        Opus reticulatum:
      \item<1-> 
        A Latin origin?
      \end{itemize}
      \column{0.55\textwidth}
      \includegraphics[width=\textwidth]{opus_reticulatum.jpg}\\
      {\tiny [http://serialconsign.com/2007/11/we-put-net-network]}
    \end{columns}
  \end{block}

\end{frame}

\begin{frame}
  \frametitle{Ancestry:}

  \begin{block}<1->{First known use: Geneva Bible, 1560}
    `And thou shalt make unto it a grate like networke of brass (Exodus xxvii 4).'
  \end{block}

  \begin{block}<2->{From the OED via Briggs:}
    \begin{itemize}
    \item<2-> 
      1658--: reticulate structures in animals
    \item<3-> 
      1839--: rivers and canals
    \item<4-> 
      1869--: railways
    \item<5-> 
      1883--: distribution network of electrical cables
    \item<6-> 
      1914--: wireless broadcasting networks
    \end{itemize}
  \end{block}

\end{frame}


\begin{frame}
  \frametitle{Ancestry:}

  \begin{block}{Net and Work are venerable old words:}
    \begin{itemize}
    \item
      \alert{`Net'} first used to mean spider web 
      {\small (King {\AE}lfr\'{e}d, 888)}.
    \item
      \alert{`Work'} appear to have long meant purposeful action.
    \end{itemize}
  \end{block}

  \begin{columns}
    \column{0.5\textwidth}
    \includegraphics[width=\textwidth]{briggs2005a_fig1}
    \column{0.5\textwidth}
    \includegraphics[width=\textwidth]{briggs2005a_fig2}
  \end{columns}

  \begin{block}{}
    \begin{itemize}
    \item<2->
      `Network' = something built
      based on the idea of natural, flexible lattice or web.
    \item<3-> 
      c.f., ironwork, stonework, fretwork.
    \end{itemize}
  \end{block}

\end{frame}

\begin{frame}
  \frametitle{Key Observation:}

  \begin{block}{}
  \begin{itemize}
  \item <1->
    Many \alert{complex systems}\\ 
    can be viewed as \alert{complex networks}\\
    of physical or abstract interactions.
  \item <2->
    Opens door to mathematical and numerical analysis.
  \item <3-> 
    Dominant approach of last decade of 
    a \alertb{theoretical-physics/stat-mechish} flavor.
  \item <4-> 
    Mindboggling amount of work published 
    on complex networks since 1998...
  \item <5-> 
    ... largely due to your typical theoretical physicist:
    \begin{overprint}
      \onslide<1-5 | handout:0 | trans: 0>
      \onslide<6- | handout:1 | trans: 1>
      \smallskip
      \begin{columns}
        \column{0.25\textwidth}
        \includegraphics[width=\textwidth]{piranha3-tp.pdf}
        \column{0.75\textwidth}
        \begin{itemize}
        \item \textit{Piranha physicus}
        \item<7-> Hunt in packs.
        \item<8-> Feast on new and interesting ideas \\
          {\small (see chaos, cellular automata, ...)}
        \end{itemize}
      \end{columns}
    \end{overprint}
  \end{itemize}
    
  \end{block}
\end{frame}

%%%%%%%%%%%%%%%%%%%%%%%%%%%%%%%%%%%%%
% popularity
%%%%%%%%%%%%%%%%%%%%%%%%%%%%%%%%%%%%%

\begin{frame}
  \frametitle{Popularity (according to Google Scholar)}

  \begin{block}<1->{``Collective dynamics of `small-world' networks''\cite{watts1998a}}
    \begin{itemize}
    \item[] 
      Duncan Watts and Steve Strogatz\\
      Nature, 1998
    \item[] 
      \wordwikilink{http://scholar.google.com/citations?view\_op=view\_citation\&hl=en\&user=LhOAiXMAAAAJ\&citation\_for\_view=LhOAiXMAAAAJ:u5HHmVD\_uO8C}
      {Times cited: \uncover<2->{\alert{$\sim 23,732$}} }
      {\tiny(as of September 23, 2014)}
        %% http://scholar.google.com/citations?view_op=view_citation&hl=en&user=LhOAiXMAAAAJ&citation_for_view=LhOAiXMAAAAJ:u5HHmVD_uO8C
        %% 
      \end{itemize}
    \end{block}

  \begin{block}<1->{``Emergence of scaling in random networks''\cite{barabasi1999a}}
    \begin{itemize}
    \item[] 
      L\'{a}szl\'{o} Barab\'{a}si and R\'{e}ka Albert\\
      Science, 1999
    \item[] 
      \wordwikilink{http://scholar.google.com/citations?view\_op=view\_citation\&hl=en\&user=vsj2slIAAAAJ\&citation\_for\_view=vsj2slIAAAAJ:u5HHmVD\_uO8C}
      {Times cited: \uncover<3->{\alert{$\sim 20,734$}}}
      {\tiny(as of September 23, 2014)}
      %% http://scholar.google.com/citations?view_op=view_citation&hl=en&user=vsj2slIAAAAJ&citation_for_view=vsj2slIAAAAJ:u5HHmVD_uO8C
    \end{itemize}
  \end{block}
\end{frame}


\begin{frame}
  \frametitle{Popularity (according to Google Scholar)}

  \begin{block}<1->{Review articles:}
    \begin{itemize}
    \item
      S. Boccaletti et al.,\\
      Physics Reports, 2006,\\
      \alertb{``Complex networks: structure and dynamics''}\cite{boccaletti2006a}\\
      \wordwikilink{http://scholar.google.com/citations?view\_op=view\_citation\&hl=en\&user=BEC76f4AAAAJ\&citation\_for\_view=BEC76f4AAAAJ:u5HHmVD\_uO8C}{Times
        cited: \alert{$\sim$ 4,925}} 
      {\tiny(as of September 23, 2014)}
      %% http://scholar.google.com/citations?view_op=view_citation&hl=en&user=BEC76f4AAAAJ&citation_for_view=BEC76f4AAAAJ:u5HHmVD_uO8C
    \item 
      M. Newman,\\
      SIAM Review, 2003,\\
      \alertb{``The structure and function of complex networks''}\cite{newman2003a}\\
      \wordwikilink{http://scholar.google.com/scholar?cites=12945060519911641528\&as\_sdt=5,46\&sciodt=0,46\&hl=en}
      {Times cited: \alert{$\sim$ 11,550}} 
      {\tiny(as of September 23, 2014)}
      %% http://scholar.google.com/scholar?cites=12945060519911641528&as_sdt=5,46&sciodt=0,46&hl=en
    \item 
      R.\ Albert and A.-L.\ Barab\'{a}si\\
      Reviews of Modern Physics, 2002,\\
      \alertb{``Statistical mechanics of complex networks''}\cite{albert2002a}\\
      \wordwikilink{http://scholar.google.com/citations?view\_op=view\_citation\&hl=en\&user=d27Ji6kAAAAJ\&citation\_for\_view=d27Ji6kAAAAJ:WF5omc3nYNoC}
      {Times cited: \alert{$\sim$ 14,298}} 
      {\tiny(as of September 23, 2014)}
      %% http://scholar.google.com/citations?view_op=view_citation&hl=en&user=d27Ji6kAAAAJ&citation_for_view=d27Ji6kAAAAJ:WF5omc3nYNoC
    \end{itemize}
  \end{block}

\end{frame}

%% \begin{frame}
%%   \frametitle{Popularity according to academic courses:}
%%
%% link to other courses around the world on the main site
%%
%% Lada Adamic
%%
%% David Easley and Jon Kleinberg (Economics and Computer Science, Cornell)
%% 
%%  Mark Newman (Physics, Michigan)\\
%%   \begin{block}<2->{Textbooks:}
%%     \begin{itemize}
%%       \item \small
%%         Mark Newman (Physics, Michigan)\\
%%         \alertb{``Networks: An Introduction''}
%%         \wikilink{http://www.amazon.com/Networks-Introduction-Mark-Newman/dp/0199206651}
%%       \item \small
%%         
%%         \alertb{``Networks, Crowds, and Markets: Reasoning About a Highly Connected World''}
%%         \wikilink{http://www.cs.cornell.edu/home/kleinber/networks-book/}
%%     \end{itemize}
%%   \end{block}
%% 
%% \end{frame}

\begin{frame}
  \frametitle{Popularity according to textbooks:}
  
  \begin{block}<2->{Textbooks:}
    \begin{itemize}
      \item \small
        Mark Newman (Physics, Michigan)\\
        \alertb{``Networks: An Introduction''}
        \wikilink{http://www.amazon.com/Networks-Introduction-Mark-Newman/dp/0199206651}
      \item \small
        David Easley and Jon Kleinberg (Economics and Computer Science, Cornell)\\
        \alertb{``Networks, Crowds, and Markets: Reasoning About a Highly Connected World''}
        \wikilink{http://www.cs.cornell.edu/home/kleinber/networks-book/}
    \end{itemize}
  \end{block}

\end{frame}


\begin{frame}
  \frametitle{Popularity according to books:}

  \showbook{tippingpoint.jpg}
  {The Tipping Point: How Little Things can make a Big Difference}
  {Malcolm Gladwell\cite{gladwell2000a}}

  \bigskip

  \showbook{nexus.jpg}
  {Nexus: Small Worlds and the Groundbreaking Science of Networks}
  {Mark Buchanan}

\end{frame}

\begin{frame}
  \frametitle{Popularity according to books:}

  \showbook{linked.jpg}
  {Linked: How Everything Is Connected to Everything Else and What It Means}
  {Albert-Laszlo Barab\'{a}si}

  \bigskip

  \showbook{sixdegrees.jpg}
  {Six Degrees: The Science of a Connected Age}
  {Duncan Watts\cite{watts2003a}}

\end{frame}

%% \begin{frame}
%%   \frametitle{Books}
%% 
%% \showbook{handbookgraphs.jpg}
%% {Handbook of Graphs and Networks}
%% {editors: Stefan Bornholdt and H. G. Schuster\cite{bornholdt2003a}}
%% 
%% \bigskip
%% 
%% \showbook{evolutionofnetworks.jpg}
%% {Evolution of Networks}
%% {S. N. Dorogovtsev and J. F. F. Mendes\cite{dorogovtsev2003a}}
%% 
%% \end{frame}
%% 
%% \begin{frame}
%%   \frametitle{Books}
%% 
%% \showbook{socialnetworkanalysis.jpg}
%% {Social Network Analysis}
%% {Stanley Wasserman and Kathleen Faust\cite{wasserman1994a}}
%% 
%% \bigskip
%% 
%% \showbook{inthebeatofaheart.jpg}
%% {In the Beat of a Heart: Life, Energy, and the Unity of Nature}
%% {John Whitfield}
%% 
%% \end{frame}

\begin{frame}
%%  \frametitle{}
  
  \small

  \begin{block}{Numerous others \ldots}
      \begin{itemize}
    \item 
      \alertb{Complex Social Networks}---F. Vega-Redondo\cite{vega-redondo2007a}
    \item 
      \alertb{Fractal River Basins: Chance and Self-Organization}---I. Rodr\'{\i}guez-Iturbe and A. Rinaldo\cite{rodriguez-iturbe1997a}
    \item 
      \alertb{Random Graph Dynamics}---R. Durette
    \item 
      \alertb{Scale-Free Networks}---Guido Caldarelli
    \item 
      \alertb{Evolution and Structure of the Internet: A Statistical Physics Approach}---Romu Pastor-Satorras and Alessandro Vespignani
    \item 
      \alertb{Complex Graphs and Networks}---Fan Chung
    \item 
      \alertb{Social Network Analysis}---Stanley Wasserman and Kathleen Faust
    \item 
      \alertb{Handbook of Graphs and Networks}---Eds: Stefan Bornholdt and H. G. Schuster\cite{bornholdt2003a}
    \item 
      \alertb{Evolution of Networks}---S. N. Dorogovtsev and J. F. F. Mendes\cite{dorogovtsev2003a}
    \end{itemize}
  \end{block}

\end{frame}

%%%%%%%%%%%%%%%%%%%%%%%%%
%% Observations
%%%%%%%%%%%%%%%%%%%%%%%%%

\begin{frame}
  \frametitle{More observations}

  \begin{block}{}
  \begin{itemize}
  \item<1->
    But surely \alert{networks aren't new}...
  \item<2->
    Graph theory is well established...
  \item<3->
    Study of social networks started in the 1930's...
  \item<4->
    So why all this `new' research on networks?
  \item<5->
    \alert{Answer:} \alertb{Oodles of Easily Accessible Data.}
  \item<6->
    We can now inform (alas) our theories \\
    with a much more measurable reality.$^\ast$
  \item<7-> 
    A worthy goal: establish \alertb{mechanistic explanations}.\\
    \medskip
    \visible<8>{
    {\small 
      $\mbox{}^\ast$\textit{If this is upsetting, maybe string theory is for you...}}
  }
  \end{itemize}
  \end{block}



\end{frame}

\begin{frame}
  \frametitle{More observations}

  \begin{block}{}
    \begin{itemize}
    \item<1->
      \alertb{Web-scale} data sets can be overly \alert{exciting}.
    \end{itemize}
  \end{block}
  
  \begin{block}<2->{Witness:}
    \begin{itemize}
    \item<2->
      The End of Theory: The Data Deluge Makes the Scientific Theory Obsolete (Anderson, Wired)
      \wikilink{http://www.wired.com/science/discoveries/magazine/16-07/pb\_theory\#}
    \item<3->
      ``The Unreasonable Effectiveness of Data,''\\ Halevy et al.\cite{halevy2009a}.
    \item<3->
      c.f. Wigner's ``The Unreasonable Effectiveness of Mathematics in the Natural Sciences''\cite{wigner1960a}
    \end{itemize}
  \end{block}

  \begin{block}<4->{But:}
  \begin{itemize}
  \item<4-> 
    For scientists, description is only part of the battle.
  \item<5-> 
    We still need to \alertb{understand}.
  \end{itemize}
  \end{block}

\end{frame}


%%%%%%%%%%%%%%%%%%%%%%%%
%% Basic definitions
%%%%%%%%%%%%%%%%%%%%%%%%

\begin{frame}
  \frametitle{Super Basic definitions}

  \begin{block}<1->{\alert{Nodes} = A collection of entities 
      which have properties that
      are somehow related to each other}
    \begin{itemize}
    \item <2-> 
      e.g., people, forks in rivers, proteins, webpages, organisms,...
    \end{itemize}
  \end{block}

  \begin{block}<3->{\alert{Links} = Connections between nodes}
    \begin{itemize}
    \item<4->
      \alert{Links} may be directed or undirected.
    \item<5->
      \alert{Links} may be binary or weighted.
    \end{itemize}
  \end{block}

  \uncover<6->{
    Other spiffing words: vertices and edges.
  }

\end{frame}

%% \begin{frame}
%%   \frametitle{Basic definitions}
%% 
%%   \begin{block}{\alert{Links} = Connections between nodes}
%%     \begin{itemize}
%%     \item<2->
%%       \alert{links}
%%       \begin{itemize}
%%       \item <3->
%%       may be real and fixed (rivers),
%%       \item <4->
%%       real and dynamic (airline routes), 
%%       \item <5->
%%       abstract with physical impact (hyperlinks),
%%       \item <6->
%%       or purely abstract (semantic connections between concepts).
%%       \end{itemize}
%%     \item<7->
%%       \alert{Links} may be directed or undirected.
%%     \item<8->
%%       \alert{Links} may be binary or weighted.
%%     \end{itemize}
%%   \end{block}
%% 
%% \end{frame}

\begin{frame}
  \frametitle{Super Basic definitions}

  \begin{block}{\alert{Node degree} = Number of links per node}
    \begin{itemize}
    \item<2-> Notation: Node $i$'s degree = $k_i$.
    \item<3-> $k_i$ = 0,1,2,\ldots.
    \item<4-> Notation: the average degree of a network = $\avg{k}$ \\
      \visible<5->{(and sometimes $z$)}
    \item<6->
      Connection between number of edges $m$ and average degree:
      $$
      \tavg{k} = \frac{2m}{N}.
      $$
    \item<7->
      \alertb{Defn:} ${\cal N}_i$ = the set of $i$'s $k_i$ neighbors
    \end{itemize}
  \end{block}

\end{frame}

\begin{frame}
  \frametitle{Super Basic definitions}

  \begin{block}{Adjacency matrix:}
    \begin{itemize}
    \item<1->
      We represent a directed network by a 
      matrix $A$ with link weight $a_{ij}$ for nodes $i$ and $j$
      in entry $(i,j)$.
    \item<2->
      e.g.,
      $$
      A = \left[
        \begin{array}{ccccc}
          0 & 1 & 1 & 1 & 0\\
          0 & 0 & 1 & 0 & 1\\
          1 & 0 & 0 & 0 & 0 \\
          0 & 1 & 0 & 0 & 1 \\
          0 & 1 & 0 & 1 & 0 \\
        \end{array}
      \right]
      $$
    \item<3->
      (n.b., for numerical work, we 
      always use sparse matrices.)
    \end{itemize}
  \end{block}

\end{frame}


%%%%%%%%%%%%%%%%%%%%%%%%%%%%%%%%%%%%%
% examples
%%%%%%%%%%%%%%%%%%%%%%%%%%%%%%%%%%%%%

\section{Examples\ of\ Complex\ Networks}

\begin{frame}
  \frametitle{Examples}

  \begin{block}{So what passes for a complex network?}
    \begin{itemize}
    \item<2-> Complex networks are \alert{large} (in node number)
    \item<3-> Complex networks are \alert{sparse} (low edge to node ratio)
    \item<4-> Complex networks are usually \alert{dynamic} and \alert{evolving}
    \item<5-> Complex networks can be social, economic, natural, informational, abstract, ...
    \end{itemize}
  \end{block}

\end{frame}

\begin{frame}
  \frametitle{Examples}

  \begin{block}<1->{Physical networks}
    \begin{columns}
      \column {0.5\textwidth}
        \begin{itemize}
        \item<1-> River networks
        \item<2-> Neural networks
        \item<3-> Trees and leaves
        \item<4-> Blood networks
        \end{itemize}

      \column{0.5\textwidth}
        \begin{itemize}
        \item<5-> The Internet
        \item<6-> Road networks
        \item<7-> Power grids
        \end{itemize}

    \end{columns}
  \end{block}

  \medskip

  \begin{columns}[t]
    \column{0.3\textwidth}
    \includegraphics<5->[height=.28\textheight]{opte1105841711-LGL-2D-4000x4000.png} 
      %%      {\centering \tiny (\url{opte.com})}
    \column{0.4\textwidth}
    \includegraphics<1->[height=.28\textheight]{Rivierescr.jpg}
    \column{0.3\textwidth}
    \includegraphics<3->[height=.28\textheight]{BoucleSach_imacr.jpg} 
  \end{columns}

  \begin{block}{}
    \begin{itemize}
    \item<8> \alert{Distribution} (branching) versus \alert{redistribution} (cyclical)
    \end{itemize}
  \end{block}

\end{frame}

\begin{frame}
  \frametitle{Examples}

  \begin{columns}
    \column{0.4\textwidth}
    \begin{block}{Interaction networks}
      \begin{itemize}
      \item<1->  The Blogosphere
      \item<2->  Biochemical networks
      \item<3->  Gene-protein networks
      \item<4->  Food webs: who eats whom
      \item<5->  The World Wide Web (?)
      \item<6->  Airline networks
      \item<7->  Call networks (AT\&T)
      \item<8->  The Media
      \end{itemize}
    \end{block}
    \column{0.6\textwidth}
    \includegraphics[width=\textwidth]{datamining-core-2006-06-27.png}\\
    {\tiny \wordwikilink{http://datamining.typepad.com}{datamining.typepad.com}}
  \end{columns}

\end{frame}

\begin{frame}
  \frametitle{topics:}

  \begin{block}{}
    \begin{columns}
      \column{0.4\textwidth}
      \begin{itemize}
      \item 
        Hidalgo et al.'s ``The Product Space Conditions the Development of Nations''\cite{hidalgo2007a}
      \item 
        How do products depend on each other, and how does this network evolve?
      \item 
        How do countries depend on
        each other for water, energy, people (immigration), investments?
      \end{itemize}      
      \column{0.6\textwidth}
      \includegraphics[width=\textwidth]{spacelabelslegends.pdf}
    \end{columns}
  \end{block}

\end{frame}

\begin{frame}
  \frametitle{Examples}

  \begin{columns}
    \column{0.5\textwidth}
    \begin{block}{Interaction networks: social networks}
      \begin{itemize}
      \item<1-> Snogging
      \item<2-> Friendships
      \item<3-> Acquaintances
      \item<4-> Boards and directors
      \item<5-> Organizations %% formal and informal ties
      \item<6-> 
        \wordwikilink{http://www.facebook.com}{facebook}
        \wordwikilink{http://www.twitter.com}{twitter},
      \end{itemize}
    \end{block}
    \column{0.5\textwidth}
    \includegraphics[width=\textwidth]{bearman_sex_network.jpg}\\
    {\tiny (Bearman \etal, 2004)} 
  \end{columns}

  \begin{block}{}
    \begin{itemize}
    \item<7->
      `Remotely sensed' by:
      email activity, 
      instant messaging, 
      phone logs \uncover<8->{\alert{(*cough*)}}.
    \end{itemize}
  \end{block}

\end{frame}

\begin{frame}
  \frametitle{Examples}

    \includegraphics[width=\textwidth]{bearman_sex_network.jpg}\\

\end{frame}

\begin{frame}
  \frametitle{Examples}

  \begin{block}{Relational networks}
    \begin{itemize}
    \item<1-> 
      Consumer purchases \\
      \visible<2->{(Wal-Mart, Target, Amazon, ...)}
%% : $\approx 1 \ \mbox{petabyte} \ = 10^{15} \ \mbox{bytes}$)}
    \item<3-> 
      Thesauri: Networks of words generated by meanings
    \item<4-> 
      Knowledge/Databases/Ideas
    \item<5-> 
      Metadata---Tagging:
      \wordwikilink{http://bit.ly}{bit.ly}
      \wordwikilink{http://www.flickr.com}{flickr}
    \end{itemize}
  \end{block}
  \begin{overprint}
    \onslide<5-| handout:1| trans:1>
    \includegraphics[width=0.8\textwidth]{delicious.pdf}
  \end{overprint}

\end{frame}


\begin{frame}
  \frametitle{Clickworthy Science:}

  \begin{center}
    \includegraphics[height=0.75\textheight]{bollen2009a_fig5.png}\\
    Bollen et al.\cite{bollen2009a};
    a higher resolution figure is
    \wordwikilink{http://www.plosone.org/article/slideshow.action?uri=info:doi/10.1371/journal.pone.0004803&imageURI=info:doi/10.1371/journal.pone.0004803.g005}{here}
  \end{center}

\end{frame}


\neuralreboot{7xEX-48RHCY}{0}{72}{Dog has fun.}

%%%%%%%%%%%%%%%%%%%%%%%%%%%%%%%%%%%%%
% properties
%%%%%%%%%%%%%%%%%%%%%%%%%%%%%%%%%%%%%

\section{Properties\ of\ Complex\ Networks}

\begin{frame}
  \frametitle{}

  \begin{block}<1->{A notable feature of large-scale networks:}
    \begin{itemize}
    \item<2->
      Graphical renderings are often just a big mess.
      \begin{overprint}
        \onslide<1-2 | handout: 0 | trans:0>
        \onslide<3->
        \begin{columns}
          \column{0.4\textwidth}
          \includegraphics[height=\textwidth]{nw_purerandom_graphviz01_10}
          \column{0.6\textwidth}
          \begin{itemize}
          \item[] 
            $\Leftarrow$ Typical hairball
          \item 
            number of nodes $N$ = 500
          \item 
            number of edges $m$ = 1000
          \item 
            average degree $\tavg{k}$ = 4
          \end{itemize}
        \end{columns}
      \end{overprint}
    \item<4->
      And even when renderings somehow look good:\\
      \visible<5->{
      \alertb{``That is a very graphic analogy which aids 
      understanding wonderfully while being,
      strictly speaking, wrong in every possible way''} \\
      {\small
      said Ponder [Stibbons]
      ---\textit{Making Money}, T. Pratchett.
      }
      }
    \item<6-> 
      We need to extract \alert{digestible, meaningful aspects}.
    \end{itemize}
  \end{block}

\end{frame}

\begin{frame}
%%  \frametitle{Properties}

  \begin{block}{Some key aspects of real complex networks:}
  \begin{columns}
    \column{0.05\textwidth}
    \column{0.45\textwidth}
    \begin{itemize}
    \item \alertb{degree distribution}$^\ast$
    \item assortativity
    \item homophily
    \item clustering
    \item motifs
    \item modularity
    \end{itemize}
    \column{0.45\textwidth}
    \begin{itemize}
    \item concurrency
    \item hierarchical scaling
    \item network distances
    \item centrality
    \item efficiency
    \item interconnectedness
    \item robustness
    \end{itemize}
    \column{0.1\textwidth}
  \end{columns}
  \end{block}

  \begin{itemize}
  \item<1-> 
    Plus coevolution of network structure 
    \\ and processes on networks.
  \item[$\ast$]
    Degree distribution is the elephant in the room that
    we are now all very aware of...
  \end{itemize}

\end{frame}

\begin{frame}
  \frametitle{Properties}

  \begin{block}<+->{1. degree distribution $P_k$}
    \begin{itemize}
    \item<+->
      $P_k$ is the probability that a randomly selected
      node has degree $k$.
    \item<+->
      $k$ = node degree = number of connections.
    \item<+->
      \alert{ex 1:}
      \erdosrenyi\ random networks have Poisson degree distributions: \\
      \insertassignmentquestionsoft{05}{5}
      $$ P_k = e^{-\tavg{k}} \frac{\tavg{k}^k}{k!} $$
    \item<+->
      \alert{ex 2:}
      \alert{``Scale-free'' networks:}
      $P_k \propto k^{-\gamma}$ $\Rightarrow$ `hubs'.
    \item<+->
      link cost controls skew.
    \item<+->
      hubs may facilitate or impede contagion.
    \end{itemize}
  \end{block}
 
\end{frame}

\begin{frame}
  \frametitle{Properties}

  \begin{block}{Note:}
    \begin{itemize}
    \item<1->
      \erdosrenyi\ random networks are a \alertb{\textit{mathematical construct}}.
    \item<2->
      `Scale-free' networks are \alert{growing networks} that form
      according to a \alert{plausible mechanism}.
    \item<3-> Randomness is out there, just not to the degree of
       a completely random network.
    \end{itemize}
  \end{block}

\end{frame}

\begin{frame}
  \frametitle{Properties}

  \begin{block}{2. Assortativity/3. Homophily:}
    \begin{itemize}
    \item<1-> Social networks: \wordwikilink{http://en.wikipedia.org/wiki/Homophily}{Homophily} = birds of a feather
    \item<2-> e.g., degree is standard property for sorting:\\
      measure degree-degree correlations.
    \item<3->
      \alert{Assortative} network:\cite{newman2002a} 
      similar degree nodes connecting to each other.\\
      \visible<5->{\textit{Often \alertb{social}: company directors, coauthors, actors.}}
    \item<4->
      \alert{Disassortative} network: high degree nodes connecting to low degree nodes.\\
      \visible<6->{\textit{Often \alertb{techological} or \alertb{biological}: 
        Internet, WWW, protein interactions, neural networks, food webs.}}
    \end{itemize}
  \end{block}

\end{frame}

\begin{frame}
  \frametitle{Local socialness:}

  \begin{block}{4. Clustering:}
    \begin{columns}
      \column{0.4\textwidth}
      \includegraphics[width=\textwidth]{clustering-sketch-C1-tp-3.pdf}
      \column{0.6\textwidth}
      \begin{itemize}
      \item<2-> Your friends tend to know each other.
      \item<3-> Two measures (explained on following slides):
        \begin{enumerate}
        \item<3-> Watts \& Strogatz\cite{watts1998a}
          $$ 
          C_1 
          = 
          \avg{
            \frac{\sum_{j_1 j_2 \in {\cal N}_i} a_{j_1 j_2}}
            {k_i(k_i-1)/2}}_{i}
          $$  
        \item<3-> Newman\cite{newman2003a}
          $$ 
          C_2 
          = 
          \frac{3 \times \textrm{\#triangles}}
          {\textrm{\#triples} }
          $$ 
        \end{enumerate}
      \end{itemize}
    \end{columns}  
  \end{block}

\end{frame}


\begin{frame}
%%  \frametitle{First clustering measure:}

  \begin{block}{}
  \begin{columns}
    \column{0.03\textwidth}
    \column{0.37\textwidth}
%%    \includegraphics[width=\textwidth]{clustering-sketch-C1-tp-3.pdf}\\
    Example network:\\
    \includegraphics[width=\textwidth]{clustering-sketch-example-network-tp-3.pdf}\\
    Calculation of $C_1$:\\
    \includegraphics[width=\textwidth]{clustering-sketch-example-network-C1-calculation-tp-3.pdf}
    \column{0.6\textwidth}
    \begin{itemize}
    \item<2-> $C_1$ is the \alert{average fraction of 
        pairs of neighbors who are connected}.
    \item<3-> Fraction of pairs of neighbors who are connected is
      $$ \frac{\sum_{j_1 j_2 \in {\cal N}_i} a_{j_1 j_2}}{k_i(k_i-1)/2} $$
      where
      $k_i$ is node $i$'s degree, and 
      ${\cal N}_i$ is the set of $i$'s neighbors.
    \item<4->
      Averaging over all nodes, we have:\\
      $ C_1 = \frac{1}{n}{\sum_{i=1}^{n}\frac{\sum_{j_1 j_2 \in {\cal N}_i} a_{j_1 j_2}}{k_i(k_i-1)/2}} 
      \visible<5->{ = \avg{\frac{\sum_{j_1 j_2 \in {\cal N}_i} a_{j_1 j_2}}{k_i(k_i-1)/2}}_{i} }$
    \end{itemize}
  \end{columns}
  \end{block}

\end{frame}

\begin{frame}
  \frametitle{Triples and triangles}

  \begin{block}{}
  \begin{columns}
    \column{0.35\textwidth}
    \centering
    Example network:\\
    \includegraphics[width=\textwidth]{clustering-sketch-example-network-tp-3.pdf}\\
    Triangles:\\
    \includegraphics[width=0.7\textwidth]{clustering-sketch-example-network-triangles-tp-3.pdf}\\
    Triples:\\
    \includegraphics[width=\textwidth]{clustering-sketch-example-network-triples-tp-3.pdf}
    \column{0.65\textwidth}
    \begin{itemize}
    \item<1->
      Nodes $i_1$, $i_2$, and $i_3$ form a \alert{triple}
      around $i_1$ if $i_1$ is connected to $i_2$ and $i_3$.
    \item<2->
      Nodes $i_1$, $i_2$, and $i_3$ form a \alert{triangle}
      if each pair of nodes is connected
    \item<3-> 
      The definition
      $ C_2 = \frac{3 \times \textrm{\#triangles}}{\textrm{\#triples}} $
      measures the fraction of \alert{closed triples}
    \item<4-> 
      The \alert{`3'} appears because for each
      triangle, we have 3 closed triples.
    \item<5-> 
      Social Network Analysis (SNA): fraction of
      \alert{transitive triples}.
    \end{itemize}
  \end{columns}
  \end{block}

\end{frame}

\begin{frame}
  \frametitle{Clustering:}

  \begin{block}<+->{Sneaky counting for undirected, unweighted networks:}
    \begin{itemize}
    \item<+->
      If the path $i$--$j$--$\ell$ exists 
      then $a_{ij} a_{j\ell} = 1$.
    \item<+->
      Otherwise, $a_{ij} a_{j\ell} = 0$.
    \item<+->
      We want $i \ne \ell$ for good triples.
    \item<+->
      In general, a path of $n$ edges between nodes $i_{1}$ and $i_{n}$ travelling
      through nodes $i_{2}$, $i_{3}$, \ldots $i_{n-1}$ exists $\iff$
      $a_{i_{1}i_{2}} a_{i_{2}i_{3}} a_{i_{3}i_{4}} \cdots a_{i_{n-2}i_{n-1}} a_{i_{n-1} i_{n}}$ = 1.
    \item<+->
      $$ 
      \#\textrm{triples}
      = 
      \frac{1}{2} 
      \left(
        \sum_{i=1}^{N}
        \sum_{\ell=1}^{N}
        \left[ 
          A^2 
        \right]_{i\ell}
        -
        \textrm{Tr}
        A^2
    \right)
      $$
    \item<+->
      $$ 
      \#\textrm{triangles} 
      = 
      \frac{1}{6} 
      \textrm{Tr} 
      A^3 
      $$
    \end{itemize}
  \end{block}

\end{frame}

\begin{frame}
  \frametitle{Properties}

  \begin{block}{}
  \begin{itemize}
  \item<1-> For sparse networks, $C_1$ tends to discount
    highly connected nodes.
  \item<2-> $C_2$ is a useful and often preferred variant
  \item<3-> In general, $C_1 \ne C_2$.
  \item<4-> $C_1$ is a global average of a local ratio.
  \item<5-> $C_2$ is a ratio of two global quantities.
  \end{itemize}
  \end{block}

\end{frame}

\begin{frame}
  \frametitle{Properties}

  \begin{block}{5. motifs:}
  \begin{itemize}
  \item<2->small, recurring functional subnetworks 
  \item<3->e.g., Feed Forward Loop:
    \begin{overprint}
      \onslide<3-| handout:1| trans:1>
      \begin{center}
        \includegraphics[width=0.45\textwidth]{feedforwardloop-tp.pdf}%
      \end{center}
    \end{overprint}
    Shen-Orr, Uri Alon, \etal\cite{shen-orr2002a}
  %% , Wiggins \etal
  \end{itemize}
  \end{block}

\end{frame}

\begin{frame}
  \frametitle{Properties}

  \begin{block}{6. modularity and structure/community detection:}
    \begin{center}
      \begin{tabular}{c}
        \includegraphics[height=0.6\textheight]{ncaa_annotated-tp-10}\\
        Clauset \etal, 2006\cite{clauset2006a}: NCAA football
      \end{tabular}
    \end{center}
  \end{block}

\end{frame}

\begin{frame}
  \frametitle{Properties}

  \begin{block}{7. concurrency:}
    \begin{itemize}
    \item<+-> 
      transmission of a contagious element
      only occurs during contact
    \item<+-> 
      rather obvious but easily missed in a simple model
    \item<+-> 
      dynamic property---static networks are not enough
    \item<+-> 
      knowledge of previous contacts crucial
    \item<+-> 
      beware cumulated network data
    \item<+-> 
      Kretzschmar and Morris, 1996\cite{kretzschmar1996a}
    \item<+-> 
      ``Temporal networks'' become a concrete area of study 
      for Piranha Physicus in 2013.
    \end{itemize}
  \end{block}

\end{frame}

\begin{frame}
  \frametitle{Properties}

  \begin{block}{8. Horton-Strahler ratios:}
    \begin{itemize}
    \item<1-> Metrics for branching networks:
      \begin{itemize}
      \item<2-> Method for ordering streams hierarchically
      \item<3->
        Number: $R_n = N_{\omega}/N_{\omega+1}$ 
      \item<4->
        Segment length: $R_l = \tavg{l_{\omega+1}}/\tavg{l_{\omega}}$ 
      \item<5->
        Area/Volume: $R_a = \tavg{a_{\omega+1}}/\tavg{a_{\omega}}$ 
      \end{itemize}
    \end{itemize}
    \begin{overprint}
      \onslide<1-| handout:1| trans:1>
      \begin{center}
        \includegraphics[height=0.4\textheight]{network1}%
        \includegraphics[height=0.4\textheight]{network2} 
        \includegraphics[height=0.3\textheight]{network3}%
      \end{center}
    \end{overprint}
  \end{block}

\end{frame}

\begin{frame}
  \frametitle{Properties}

  \begin{block}<1->{9. network distances:}
    \begin{block}<2->{\alert{(a) shortest path length $d_{ij}$:}}
      \begin{itemize}
      \item <3->Fewest number of steps between nodes $i$ and $j$.      
      \item <4->(Also called the chemical distance between $i$ and $j$.)
      \end{itemize}
    \end{block}
    \begin{block}<5->{\alert{(b) average path length $\tavg{d_{ij}}$:}}
      \begin{itemize}
      \item <6-> Average shortest path length in whole network.
      \item <7-> 
        Good algorithms exist for calculation.
      \item <8->
        Weighted links can be accommodated.
      \end{itemize}
    \end{block}

  \end{block}

\end{frame}

\begin{frame}
  \frametitle{Properties}

  \begin{block}{9. network distances:}
    \begin{itemize}
    \item<1->
      \alert{network diameter $d_{\textrm{max}}$:}\\
      Maximum shortest path length between any two nodes.
    \item<2->
      \alert{closeness $d_{\textrm{cl}} = [\sum_{ij} d_{ij}^{\ -1} / \binom{n}{2}]^{-1}$:}\\
      Average `distance' between any two nodes.
    \item<3->
      Closeness handles disconnected networks ($d_{ij}=\infty$)
    \item<4->
      $d_{\textrm{cl}} = \infty$ only when all nodes are isolated.
    \item<4->
      Closeness perhaps compresses too much into one number
    \end{itemize}
  \end{block}

\end{frame}

\begin{frame}
  \frametitle{Properties}

  \begin{block}{10. centrality:}
    \begin{itemize}
    \item<2-> Many such measures of a node's `importance.'  
    \item<3-> \alert{ex 1:} Degree centrality: $k_i$.
    \item<4-> \alert{ex 2:} Node $i$'s betweenness \\
      = fraction of shortest paths that pass through $i$.
    \item<5-> \alert{ex 3:} Edge $\ell$'s betweenness \\
      = fraction of shortest paths that travel along $\ell$.
    \item<6-> \alert{ex 4:} Recursive centrality: Hubs and Authorities
      (Jon Kleinberg\cite{kleinberg1998a})
    \end{itemize}
    
  \end{block}

\end{frame}


\begin{frame}
  \frametitle{Properties}

  \begin{block}{Interconnected networks and robustness (two for one deal):}
    ``Catastrophic cascade of failures in interdependent networks''\cite{buldyrev2010a}.
    Buldyrev et al., Nature 2010.
    
    \includegraphics[width=0.8\textwidth]{buldyrev2010a_fig1.pdf}
  \end{block}

\end{frame}



\section{Nutshell}

\begin{frame}[label=]
  \frametitle{Nutshell:}

  \begin{block}{Overview Key Points:}
    \begin{itemize}
    \item<1->
      The field of complex networks came into
      existence in the late 1990s.
    \item<2->
      Explosion of papers and interest since 1998/99.
    \item<3->
      Hardened up much thinking about complex systems.
    \item<4->
      Specific focus on networks that are 
      \alert{large-scale}, 
      \alertb{sparse}, 
      \alert{natural} or \alert{man-made}, 
      \alertb{evolving} and \alertb{dynamic}, 
      and 
      (crucially) \alert{measurable}.
    \item<5->
      Three main (blurred) categories: 
      \begin{enumerate}
      \item 
      \alert{Physical} (e.g., river networks),
      \item 
      \alert{Interactional} (e.g., social networks),
      \item 
      \alert{Abstract} (e.g., thesauri).
      \end{enumerate}
    \end{itemize}
    
  \end{block}

\end{frame}



