\section{Introduction}

\begin{frame}

  \frametitle{Generalized contagion model}

  \begin{block}<1->{Basic questions about contagion}
    \begin{itemize}
    \item<2-> How many types of contagion are there?
    \item<3-> How can we categorize real-world contagions?
    \item<4-> Can we connect models of disease-like and social contagion?
    \item<5-> \alert{Focus:} mean field models.
    \end{itemize}
  \end{block}

\end{frame}

\section{Independent\ Interaction\ models}

\begin{frame}
  \frametitle{Mathematical Epidemiology (recap)}

  \begin{block}<1->{The standard \alert{SIR model}\cite{murray2002a}}
    \begin{itemize}
    \item<2-> = basic model of disease contagion
    \item<3-> Three states:
      \begin{enumerate}
      \item<4-> \alertb{S = Susceptible}
      \item<5-> \alertb{I = Infective/Infectious}
      \item<6-> \alertb{R = Recovered}
        \uncover<7->{ or Removed or Refractory}
      \end{enumerate}
      \item<8-> $S(t) + I(t) + R(t) = 1$
      \item<9-> Presumes random interactions (mass-action principle)
      \item<10-> Interactions are independent (no memory)
      \item<11-> Discrete and continuous time versions
    \end{itemize}
  \end{block}

\end{frame}


\begin{frame}
  \frametitle{Independent Interaction Models}

  \begin{block}{Discrete time automata example:}
    \medskip
    \begin{columns}
      \begin{column}{0.4\textwidth}
        \includegraphics[width=\textwidth]{SIRtransitions_std}
      \end{column}
      \begin{column}{0.5\textwidth}
        \visible<2->{Transition Probabilities:}

        \bigskip
        \visible<3->{\alertb{$\beta$} for being infected given contact with infected}\\
        \visible<4->{\alertb{$r$} for recovery}\\
        \visible<5->{\alertb{$\rho$} for loss of immunity}\\
      \end{column}
    \end{columns}
  \end{block}

\end{frame}


\begin{frame}
  \frametitle{Independent Interaction Models}

  \begin{block}{Original models attributed to}
    \begin{itemize}
    \item<2-> 1920's: Reed and Frost
    \item<3-> 1920's/1930's: Kermack and McKendrick\cite{kermack1927a,kermack1932a,kermack1933a}
    \item<4-> Coupled differential equations with a mass-action principle
    \end{itemize}
  \end{block}

\end{frame}

\begin{frame}
  \frametitle{Independent Interaction models}

  \begin{block}{Differential equations for continuous model}
    $$
    \frac{\dee{}}{\dee{t}} S = - \beta  \alert{I S} + \rho R 
    $$
    $$
    \frac{\dee{}}{\dee{t}} I=  \beta \alert{I S} - r I 
    $$
    $$
    \frac{\dee{}}{\dee{t}} R=  r I - \rho R 
    $$
  \end{block}
  $\beta$, $r$, and $\rho$ are now \alert{rates}.
  
  \begin{block}<2->{Reproduction Number $R_0$:}
  \begin{itemize}
  \item<3-> $R_0$ = expected number of infected individuals resulting
    from a single initial infective
  \item<4-> Epidemic threshold: If $R_0 > 1$, `epidemic' occurs.
  \end{itemize}
  \end{block}

\end{frame}

\begin{frame}
  \frametitle{Reproduction Number $R_0$}

  \begin{block}{Discrete version:}
    \begin{itemize}
    \item<1-> Set up: One Infective in a randomly mixing population of Susceptibles
    \item<2-> At time $t=0$, single infective random bumps into a Susceptible
    \item<3-> Probability of transmission = $\beta$
    \item<4-> At time $t=1$, single Infective remains infected with probability $1-r$
    \item<5-> At time $t=k$, single Infective remains infected with probability $(1-r)^k$
    \end{itemize}
  \end{block}    

\end{frame}

\begin{frame}
  \frametitle{Reproduction Number $R_0$}

  \begin{block}{Discrete version:}
    \begin{itemize}
    \item<1-> Expected number infected by original Infective:
      $$
      R_0 = \beta + (1-r)\beta + (1-r)^2\beta + (1-r)^3\beta + \ldots
      $$
      \uncover<2->{
        $$
        = \beta\left( 1 + (1-r) + (1-r)^2 + (1-r)^3 + \ldots \right)
        $$
      }
      $$
      \uncover<3->{= \beta \frac{1}{1 - (1-r)}}
      \uncover<4->{\alert{= \beta/r}}
      $$
      \item<5->{Similar story for continuous model.}
    \end{itemize}
  \end{block}

\end{frame}

\begin{frame}
  \frametitle{Independent Interaction models}

  \centering 

  Example of epidemic threshold:

  % add figure here
  \includegraphics[width=0.6\textwidth]{figR0_noname}

  \begin{itemize}
  \item \visible<2->{Continuous phase transition.}
  \item \visible<3->{Fine idea from a simple model.}
  \end{itemize}

\end{frame}

\begin{frame}
  \frametitle{Simple disease spreading models}

  \begin{block}<1->{Valiant attempts to use SIR and co. elsewhere:}
    \begin{itemize}
    \item<2-> Adoption of ideas/beliefs (Goffman \& Newell, 1964)\cite{goffman1964a}
    \item<3-> Spread of rumors (Daley \& Kendall, 1964, 1965)\cite{daley1964a,daley1965a}
    \item<4-> Diffusion of innovations (Bass, 1969)\cite{bass1969a}
    \item<5-> Spread of fanatical behavior (Castillo-Ch\'{a}vez \& Song, 2003)%\cite{chastillo-chavez2003a}
    \end{itemize}
  \end{block}

  % (religion, beliefs, rumors)

\end{frame}

\section{Interdependent\ interaction\ models}

\begin{frame}
  \frametitle{Granovetter's model (recap of recap)}

  \begin{itemize}
  \item Action based on perceived behavior of others.
  \end{itemize}

  \includegraphics[width=1\textwidth]{figthreshold_noname}

  \begin{itemize}
  \item<1-> Two states: S and I.
  \item<1-> Recovery now possible (SIS).
  \item<1-> $\phi$ = fraction of contacts `on' (e.g., rioting).
  \item<1-> Discrete time, synchronous update.
  \item<1-> This is a \alert{Critical mass model}.
  \item<1-> \alert{Inter}dependent interaction model.
  \end{itemize}

\end{frame}

\begin{frame}
  \frametitle{Some (of many) issues}

  \begin{block}{}
    \begin{itemize}
    \item<1-> Disease models assume independence of infectious events.
    \item<2-> Threshold models only involve proportions:
      $3/10 \equiv 30/100$.
    \item<3-> Threshold models ignore exact sequence of influences
    \item<4-> Threshold models assume immediate polling.
    \item<5-> Mean-field models neglect network structure
    \item<6-> Network effects only part of story: \\
      media, advertising, direct marketing.
    \end{itemize}
  \end{block}

\end{frame}

%\begin{frame}
%  \frametitle{Overview of I. Generalized Contagion}

%  \begin{enumerate}
%  \item Independent Interaction models.
%  \item Threshold models.
%  \item Other models of contagion.
%  \item Generalized model.
%  \item Results: analysis, numerics, and model simulations.
%  \item Conclusions.
%  \end{enumerate}
%
%\end{frame}

\section{Generalized\ Model}


\begin{frame}
  \frametitle{Generalized model}

  \begin{block}{Basic ingredients:}
    \begin{itemize}
    \item<1-> Incorporate memory of a contagious element\cite{dodds2004a,dodds2005a}
    \item<2-> Population of $N$ individuals, each in state S, I, or R.
    \item<3-> Each individual randomly contacts another at each time step.
    \item<4-> $\phi_t$ = fraction infected at time $t$ \\
      \quad \quad \ = probability of \underline{\alertb{contact}} with infected individual
    \item<5-> With probability $p$, contact with infective\\
      \quad leads to an \underline{\alertb{exposure}}.
    \item<6-> If exposed, individual receives a dose of size $d$\\
      \quad  drawn from distribution \alertb{$f$}.  Otherwise $d=0$.
    \end{itemize}
  \end{block}

\end{frame}

%%%%%%%%%%%%%%%
% 2a. ingredients:
%     memory
%     prob of infection being transferred

\begin{frame}
  \frametitle{Generalized model---ingredients}

  \begin{block}{$\boxed{\mbox{S} \Rightarrow \mbox{I}}$}
    \begin{itemize}
    \item<2->
      Individuals `remember' last $T$ contacts:
      $$ D_{t,i} = \sum_{t'=t-T+1}^{t} d_i(t') $$
    \item<3-> 
      Infection occurs if individual $i$'s `threshold' is exceeded:
      $$ D_{t,i} \ge \dstari $$
    \item<4->
      Threshold $\dstari$ drawn from arbitrary distribution \alert{$g$} at $t=0$.
    \end{itemize}
  \end{block}

\end{frame}

\begin{frame}
  \frametitle{Generalized model---ingredients}

  \begin{block}<1->{$\boxed{\mbox{I} \Rightarrow \mbox{R}}$}
    When  $D_{t,i} < \dstari $,\\
    individual $i$ recovers to state R
    with probability $r$.
  \end{block}

  \bigskip
    
  \begin{block}<2->{$\boxed{\mbox{R} \Rightarrow \mbox{S}}$}
    Once in state R, individuals become susceptible again with
    probability $\rho$.
  \end{block}

\end{frame}

\begin{frame}
  \frametitle{A visual explanation}

   \includegraphics[width=1\textwidth]{SIRtransitions4}

\end{frame}


%%%%%%%%%%%%%% 5 mins
% 3b. our model---results

% homogeneous

% heterogeneous

\begin{frame}
  \frametitle{Generalized mean-field model}

  \begin{block}<1->{Study SIS-type contagion first:}
    \begin{itemize}
    \item<2->
      Recovered individuals are immediately susceptible again:
      $$
      \alert{r=\rho=1.}
      $$
    \item<3->
      Look for steady-state behavior 
      as a function of exposure probability $p$.
    \item<4->
      Denote fixed points by $\phifix$.
    \end{itemize}
  \end{block}
  
  \begin{block}<5->{Homogeneous version:}
    \begin{itemize}
    \item<6->
      All individuals have threshold $\dstar$
    \item<7->
      All dose sizes are equal: $d=1$
    \end{itemize}
  \end{block}

\end{frame}


\subsection{Homogeneous\ version}

\begin{frame}
  \frametitle{Homogeneous, one hit models:}

  \begin{block}<1->{Fixed points for \alertb{$r<1$}, $\dstar=1$, and $T=1$:}
    \begin{itemize}
    \item<2->
      \alert{$r<1$} means recovery is probabilistic.
    \item<3->
      \alert{$T=1$} means individuals forget past interactions.
    \item<4->
      \alert{$\dstar=1$} means one positive interaction 
      will infect an individual.
    \item<5-> 
      Evolution of infection level:
      $$
      \phi_{t+1} = 
      \uncover<6->{
        \underbrace{p \phi_{t}}_{\mbox{\alertb{a}}}
      }
      \uncover<7->{
        + 
        \underbrace{\phi_t(1 - p\phi_t)}_{\mbox{\alertb{b}}}
      }
      \uncover<8->{
        \underbrace{(1-r)}_{\mbox{\alertb{c}}}.
      }
      $$
      \begin{enumerate}
      \uncover<6->{
      \item[a:]
        Fraction infected between $t$ and $t+1$,
        independent of past state or recovery.
        }
      \uncover<7->{
      \item[b:]
         Probability of being infected and not being reinfected.
      }
      \uncover<8->{
      \item[c:]
        Probability of not recovering.
      }
      \end{enumerate}
    \end{itemize}
  \end{block}

\end{frame}

\begin{frame}
  \frametitle{Homogeneous, one hit models:}

  \begin{block}{Fixed points for \alertb{$r<1$}, $\dstar=1$, and $T=1$:}
    \begin{itemize}
    \item<1-> 
      Set $\phi_t = \phifix$:
      \uncover<2->{
        $$
        \phifix = p \phifix + ( 1 -p \phifix) \phifix (1-r)
        $$
      }
      \uncover<3->{
        $$
        \Rightarrow  1 = p + (1-p\phifix)(1-r), \quad \phifix \ne 0,
        $$
      }
      \uncover<4->{
        $$
        \Rightarrow 
        \alert{\phifix = \frac{1 - r/p}{1-r}} 
        \quad \mbox{and} \quad 
        \alert{\phifix = 0}.
        $$
      }
    \item<5-> 
      Critical point at $p = p_c = r$.
    \item<6-> 
      Spreading takes off if $p/r > 1$
    \item<7->
      Find continuous phase transition as for SIR model.
    \item<8->
      \alertb{Goodness:} Matches $R_o = \beta/\gamma > 1$ condition.
    \end{itemize}
  \end{block}

\end{frame}

\begin{frame}
  \frametitle{Simple homogeneous examples}

  \begin{block}<1->{Fixed points for $r=1$, $\dstar=1$, and $T>1$}
    \begin{itemize}
    \item<2->
      \alert{$r=1$} means recovery is immediate.
    \item<3->
      \alert{$T>1$} means individuals remember at least 2 interactions.
    \item<4->
      \alert{$\dstar=1$} means only one positive interaction in past $T$ 
      interactions will infect individual.
    \item<5->
      Effect of individual interactions is independent from effect of others.
    \item<6->
      Call \alert{$\phifix$} the steady state level of infection.
    \item<7->
      $\Pr$(infected) = 1 - $\Pr$(uninfected):
      \uncover<8->{
      $$
      \phifix = 1 - (1 - p\phifix)^T.
      $$
      }
    \end{itemize}
  \end{block}
  
\end{frame}

\begin{frame}
  \frametitle{Homogeneous, one hit models:}

  \begin{block}{Fixed points for $r=1$, $\dstar=1$, and $T>1$}
    \begin{itemize}
    \item<1-> Closed form expression for $\phifix$:
      $$
      \phifix = 1 - (1 - p\phifix)^T.
      $$
    \item<2->
      Look for critical infection probability $p_c$.
    \item<3->
      As $\phifix \rightarrow 0$, we see
      $$
      \phifix \simeq p T \phifix 
      \uncover<4->{\ \ \Rightarrow \alert{p_c = 1/T}.}
      $$
    \item<5->
      Again find continuous phase transition...
    \item<6->
      Note: we can solve for $p$ but not $\phifix$:
      $$
      p = (\phifix)^{-1} [ 1 - (1-\phifix)^{1/T} ].
      $$
    \end{itemize}
  \end{block}
  
\end{frame}

\begin{frame}
  \frametitle{Homogeneous, one hit models:}

  \begin{block}{Fixed points for \alertb{$r \le 1$}, $\dstar=1$, and \alertb{$T \ge 1$}}
    \begin{itemize}
    \item<1-> 
      Start with $r=1$, $\dstar=1$, and $T \ge 1$ case we have
      just examined:
      $$
      \phifix = 1 - (1 - p\phifix)^T.
      $$
    \item<2-> 
      For $r<1$, add to right hand side fraction who:
      \begin{enumerate}
      \item<3->
        Did not receive any infections in last T time steps,
      \item<4->
        And \alert{did not recover} from a previous infection.
      \end{enumerate}
    \item<5->
      Define corresponding dose histories.  Example:
      \uncover<6->{
      $$
      H_1 = \{ \ldots, d_{t-T-2}, d_{t-T-1}, 1, 
      \underbrace{0, 0, \ldots, 0, 0}_{\mbox{$T$ 0's}} \},
      $$
      }
    \item<7->
      With history $H_1$, probability of being infected
      (not recovering in one time step) is $1-r$.
    \end{itemize}
  \end{block}
 
\end{frame}

\begin{frame}
  \frametitle{Homogeneous, one hit models:}

  \begin{block}{Fixed points for \alertb{$r \le 1$}, $\dstar=1$, and \alertb{$T \ge 1$}}
    \begin{itemize}
    \item<1->
      In general, relevant dose histories are:
      $$
      H_{m+1} = \{ \ldots, d_{t-T-m-1}, 1, \underbrace{0, 0, \ldots, 0, 0}_{\mbox{$m$ 0's}}, \underbrace{0, 0, \ldots, 0, 0}_{\mbox{$T$ 0's}} \}.
      $$
    \item<2->
      Overall probabilities for dose histories occurring:
      \uncover<2->{
      $$
      P(H_1) = p\phifix (1 -p\phifix)^T (1-r), 
      $$
      }
      $$
      \uncover<3->{
        P(H_{m+1}) = 
        }
      \uncover<4->{
       \underbrace{p\phifix}_{\alertb{a}}
      }
      \uncover<5->{
        \underbrace{(1 -p\phifix)^{T+m}}_{\alertb{b}}
      }
      \uncover<6->{
        \underbrace{(1-r)^{m+1}}_{\alertb{c}}.
      }
      $$
      \begin{enumerate}
      \item<4->[a:] $\Pr$(infection $T+m+1$ time steps ago)
      \item<5->[b:] $\Pr$(no doses received in $T+m$ time steps since)
      \item<6->[c:] $\Pr$(no recovery in $m$ chances)
      \end{enumerate}
    \end{itemize}
  \end{block}


\end{frame}


\begin{frame}
  \frametitle{Homogeneous, one hit models:}

  \begin{block}{Fixed points for \alertb{$r \le 1$}, $\dstar=1$, and \alertb{$T \ge 1$}}
    \begin{itemize}
    \item<1->
      $\Pr$(recovery) = $\Pr$(seeing no doses for at least $T$ time steps
      and recovering) 
      $$
      \uncover<2->{
        = \alert{r} \sum_{m=0}^{\infty} P(H_{T+m})
      }
      \uncover<3->{
        = \alert{r} \sum_{m=0}^{\infty} 
        p\phifix
        (1 -p\phifix)^{T+m}
        (1-r)^{m}
      }
      $$
      $$
      \uncover<4->{
        = \alert{r} 
        \frac{p\phifix (1-p\phifix)^T}
        {1 - (1-p\phifix)(1-r)}.
      }
      $$
    \item<5->
      Fixed point equation:
      $$
      \phifix =
      1 - \frac{r (1-p\phifix)^T }
      {1 - (1-p\phifix)(1-r)}.
      $$
    \end{itemize}
  \end{block}
\end{frame}

\begin{frame}
  \frametitle{Homogeneous, one hit models:}

  \begin{block}{Fixed points for \alertb{$r \le 1$}, $\dstar=1$, and \alertb{$T \ge 1$}}
    \begin{itemize}
    \item<1-> 
      Fixed point equation (again):
      $$
      \phifix =
      1 - \frac{r (1-p\phifix)^T }
      {1 - (1-p\phifix)(1-r)}.
      $$
    \item<2-> 
      Find critical exposure probability by examining
      above as $\phifix \rightarrow 0$.
    \item<3-> 
      $$
      \Rightarrow \quad 
      \alert{p_c} = \frac{1}{T + 1/r - 1} \alert{= \frac{1}{\alert{T + \tau}}}.
      $$
      where \alertb{$\tau$ = mean recovery time} for simple relaxation
      process.
    \item<4->
      Decreasing $r$ keeps individuals infected for longer
      and decreases $p_c$.
    \end{itemize}
  \end{block}
\end{frame}

\begin{frame}
  \frametitle{Epidemic threshold:}

  \begin{block}{Fixed points for $\dstar=1$, \alertb{$r \le 1$}, and $T \ge 1$}
  \begin{columns}
    \column{0.5\textwidth}
    \begin{itemize}
    \item 
      $
      \phifix =
      1 - \frac{r (1-p\phifix)^T }
      {1 - (1-p\phifix)(1-r)}
      $
    \item
      $\phifix = 0$
    \item
      $p_c = 1/(T+\tau)$
    \end{itemize}
    \column{0.5\textwidth}
  \includegraphics[width=\textwidth]{figgc_r0p50_k1_T2_paper2_noname.pdf}
  \end{columns}
  \end{block}

  \begin{itemize}
  \item
    Example details:
    $T=2$ \& $r=1/2$ $\Rightarrow p_c = 1/3$.
  \item
    \alertb{Blue} = stable, \alert{red} = unstable, fixed points.
  \item
    $\tau = 1/r - 1$ = characteristic recovery time = 1.
  \item
    $T + \tau \simeq $  average memory in system = 3.
  \item<2->
    Phase transition can be seen as
    a \alertb{transcritical bifurcation}.\cite{strogatz1994a}
  \end{itemize}
\end{frame}

\begin{frame}
  \frametitle{Homogeneous, multi-hit models:}

  \begin{itemize}
  \item<1-> 
    All right: $\dstar=1$ models correspond
    to simple disease spreading models.
  \item<2-> 
    What if we allow $\dstar \ge 2$?
  \item<3->
    Again first consider SIS with immediate recovery ($r=1$)
  \item<4->
    Also continue to assume unit dose sizes ($f(d) = \delta(d-1)$).
  \item<5->
    To be infected, must have at least $\dstar$
    exposures in last $T$ time steps.
  \item<6->
    Fixed point equation:
    $$
    \phifix = 
    \sum_{i=\dstar}^{T}
    \binom{T}{i}
    (p\phifix)^{i} (1 - p\phifix)^{T-i}.
    $$
  \item<7->
    As always, $\phifix=0$ works too.
 \end{itemize}

\end{frame}

\begin{frame}
  \frametitle{Homogeneous, multi-hit models:}

  \begin{block}<1->{Fixed points for $r = 1$, $\dstar > 1$, and $T \ge 1$}
    \begin{itemize}
    \item<2-> 
      Exactly solvable for small $T$.
    \item<3-> 
      e.g., for $\dstar=2$, $T=3$:
    \end{itemize}
  \end{block}

  \begin{columns}
    \column{0.5\textwidth}
    \begin{overprint}
      \onslide<1-5 | handout:0 | trans:0>
      \onslide<6->
      \includegraphics[width=\textwidth]{figgc_r1_k2_T3_noname.pdf} 
    \end{overprint}
    \column{0.5\textwidth}
    \begin{itemize}
    \item<4->
      Fixed point equation:
      $\phifix = 3 p^2 \phifix^2 (1 - p \phifix) + p^3 \phifix^3$
    \item<5->
      See new structure: a \alert{saddle node bifurcation}\cite{strogatz1994a} 
      appear as $p$ increases.
    \item<6->
      $(p_b,\phifix)=(8/9,27/32)$.
    \end{itemize}
  \end{columns}

  \begin{itemize}
  \item<7->
    Behavior akin to output of Granovetter's threshold model.
  \end{itemize}

\end{frame}

\begin{frame}
  \frametitle{Homogeneous, multi-hit models:}

  \begin{itemize}
  \item Another example:
  \end{itemize}

  \begin{center}
  Critical Mass Models\\
  \includegraphics[width=0.6\textwidth]{figgc_r1_k3_T12_paper2_noname.pdf}\\
  \end{center}

  \begin{itemize}
  \item $r=1$, $\dstar=3$, $T=12$ \hfill Saddle-node bifurcation.
  \end{itemize}
  
\end{frame}


\begin{frame}
 \frametitle{Fixed points for $r = 1$, \alertb{$\dstar > 1$}, and \alertb{$T \ge 1$}}

 \begin{columns}
   \column{0.6\textwidth}
   \begin{itemize}
   \item<1->
     $T=24$, $\dstar$ = 1, 2, \ldots 23.
   \end{itemize}
   \includegraphics[width=\columnwidth]{figgc_T24_kvar_r1_noname.pdf} 
   \column{0.4\textwidth}
   \begin{itemize}
   \item 
     $\dstar=1 \rightarrow \dstar>1$: 
     jump between continuous phase transition
     and pure critical mass model.
   \item 
     Unstable curve for $\dstar=2$ \alert{does not} hit $\phifix=0$.
   \end{itemize}
 \end{columns}

 \begin{itemize}
 \item<2-> 
   See \alert{either} simple phase transition or saddle-node bifurcation,
   nothing in between.
 \end{itemize}

\end{frame}

\begin{frame}
 \frametitle{Fixed points for $r = 1$, \alertb{$\dstar > 1$}, and \alertb{$T \ge 1$}}

 \begin{itemize}
 \item 
   Bifurcation points for example fixed $T$, varying $\dstar$:
 \end{itemize}
 \begin{columns}
   \column{0.7\textwidth}
   \includegraphics[width=\textwidth]{figgc_bipts_r1_noname.pdf}
   \column{0.3\textwidth}
   \begin{itemize}
   \item 
     $T=96$ ($\vartriangle$).
   \item 
     $T=24$ ($\triangleright$),
   \item 
     $T=12$ ($\triangleleft$),
   \item 
     $T=6$ ($\Box$),
   \item 
     $T=3$ ($\bigcirc$), 
   \end{itemize}

 \end{columns}
\end{frame}




\begin{frame}
  \frametitle{Fixed points for \alertb{$r < 1$}, \alertb{$\dstar > 1$}, and \alertb{$T \ge 1$}}

  \begin{itemize}
  \item<1->
    For $r < 1$, need to determine probability of
    recovering as a function of time since 
    dose load last dropped below threshold.
  \item<2->
    Partially summed random walks:
    $$
    D_i(t) = \sum_{t'=t-T+1}^{t} d_i(t')
    $$
  \item<3->
    Example for $T=24$, $\dstar=14$:
    \begin{overprint}
      \onslide<1-3 | handout: 0 | trans: 0>
      \onslide<4->
      \includegraphics[width=0.9\columnwidth]{figrandomwalkcalc_noname.pdf}
    \end{overprint}
  \end{itemize}

\end{frame}

\begin{frame}
  \frametitle{Fixed points for \alertb{$r < 1$}, \alertb{$\dstar > 1$}, and \alertb{$T \ge 1$}}

  \begin{itemize}
  \item<1->
    Define $\gamma_m$ as fraction of individuals 
    for whom $D(t)$ last equaled, and his since been
    below, their threshold $m$ time steps ago,
  \item<2->
    Fraction of individuals below threshold but not recovered:
    $$
    \Gamma(p,\phifix;r) = \sum_{m=1}^{\infty} (1-r)^m \gamma_m(p,\phifix).
    $$
  \item<3->
    Fixed point equation:
    $$
    \phifix = \Gamma(p,\phifix;r) 
    + \sum_{i=\dstar}^{T}
    \binom{T}{i}
    (p\phifix)^{i} (1 - p\phifix)^{T-i}.
    $$
  \end{itemize}

\end{frame}


\begin{frame}
  \frametitle{Fixed points for \alertb{$r < 1$}, \alertb{$\dstar > 1$}, and \alertb{$T \ge 1$}}

  \begin{block}<1->{Example: $T=3$, $\dstar=2$}
    \begin{itemize}
    \item<2-> 
      Want to examine how dose load can drop below threshold of $\dstar=2$:
      $$D_n=2 \Rightarrow D_{n+1}=1$$
    \item<3-> 
      Two subsequences do this:
      \uncover<4->{
        $
        \qquad \alertb{\{d_{n-2},d_{n-1},d_{n},d_{n+1}\} = \{1,1,0,\alert{0}\}}
        $
      }
      \uncover<5->{
        $
        \mbox{and} \ \alertb{\{d_{n-2},d_{n-1},d_{n},d_{n+1},d_{n+2}\} 
          = \{1,0,1,\alert{0},\alert{0}\}}.
        $
      }
    \item<6->
      Note: second sequence includes an extra 0 since
      this is necessary to stay below $\dstar=2$.
    \item<7->
      To stay below threshold, observe
      acceptable following sequences may be composed of
      any combination of two subsequences:
      $$ 
      a=\{0\} \quad \mbox{and} \quad b=\{1,0,0\}. 
      $$
    \end{itemize}
  \end{block}

\end{frame}

\begin{frame}
  \frametitle{Fixed points for \alertb{$r < 1$}, \alertb{$\dstar > 1$}, and \alertb{$T \ge 1$}}

  \begin{itemize}
  \item<1-> Determine number of sequences of length $m$ that
    keep dose load below $\dstar=2$.
  \item<2->
    $N_a$ = number of $a = \{0\}$ subsequences.
  \item<3->
    $N_b$ = number of $b = \{1,0,0\}$ subsequences.
    \uncover<4->{
      $$
      m = N_a \cdot 1 + N_b \cdot 3
      $$
    }
    \uncover<5->{
      Possible values for $N_b$:
      $$
     0, 1, 2, \ldots, \left\lfloor\frac{m}{3}\right\rfloor.
      $$
      where $\lfloor\cdot \rfloor$ means floor.
    }
  \item<6->
    \uncover<6->{
      Corresponding possible values for $N_a$:
      $$
      m, m-3, m-6, \ldots, m-3\left\lfloor\frac{m}{3}\right\rfloor.
      $$
    }
  \end{itemize}


\end{frame}

\begin{frame}
  \frametitle{Fixed points for \alertb{$r < 1$}, \alertb{$\dstar > 1$}, and \alertb{$T \ge 1$}}

  \begin{itemize}
  \item<1->
    How many ways to arrange $N_a$ $a$'s and $N_b$ $b$'s?
  \item<2-> 
    Think of overall sequence in terms of subsequences:
    $$ \{Z_1, Z_2, \ldots, Z_{N_a + N_b} \} $$
  \item<3->
    $N_a + N_b$ slots for subsequences.
  \item<4-> 
    Choose positions of either $a$'s or $b$'s:
    $$
    \binom{N_a + N_b}{N_a} = \binom{N_a + N_b}{N_b}.
    $$
  \end{itemize}

\end{frame}

\begin{frame}
  \frametitle{Fixed points for \alertb{$r < 1$}, \alertb{$\dstar > 1$}, and \alertb{$T \ge 1$}}

  \begin{itemize}
  \item<1->
    Total number of allowable sequences of length $m$:
    $$
    \sum_{N_b=0}^{\lfloor m/3\rfloor}
    \binom{N_b+N_a}{N_b}
    = 
    \sum_{k=0}^{\lfloor m/3\rfloor}
    \binom{m-2k}{k}
    $$
    where $k=N_b$ and we have used $m = N_a + 3N_b$.
  \item<2->
    $P(a) = (1-p\phifix)$ and $P(b) = p\phifix(1-p\phifix)^2$
  \item<3->
    Total probability of allowable sequences of length $m$:
    $$
    \chi_m(p,\phifix)
    = 
    \sum_{k=0}^{\lfloor m/3\rfloor}
    \binom{m-2k}{k}
    (1-p\phifix)^{m-k}
    (p\phifix)^k.
    $$
  \item<4->
    Notation: Write a randomly chosen 
    sequence of $a$'s and $b$'s of length $m$
    as $D_{m}^{a,b}$.
  \end{itemize}

\end{frame}

\begin{frame}
  \frametitle{Fixed points for \alertb{$r < 1$}, \alertb{$\dstar > 1$}, and \alertb{$T \ge 1$}}

  \begin{itemize}
  \item<1->
    Nearly there...  must account for details of sequence endings.
  \item<1->
    Three endings $\Rightarrow$ Six possible sequences:
  \end{itemize}

    \bigskip

    {\small
    \begin{tabular}{l}
        $ D_1 = \{1, 1, 0, 0, D_{m-1}^{a,b}\} $ \\
        \qquad\qquad\qquad\qquad\qquad\qquad\qquad
        \uncover<3->{\alertb{$ P_1 = (p\phi)^2 (1-p\phi)^2 \chi_{m-1}(p,\phi) $}} \\
        $ D_2 = \{1, 1, 0, 0, D_{m-2}^{a,b}, 1\} $ \\
        \qquad\qquad\qquad\qquad\qquad\qquad\qquad
        \uncover<3->{\alertb{$ P_2 = (p\phi)^3 (1-p\phi)^2 \chi_{m-2}(p,\phi) $}} \\
        $ D_3 = \{1, 1, 0, 0, D_{m-3}^{a,b}, 1, 0\}$ \\
        \qquad\qquad\qquad\qquad\qquad\qquad\qquad
        \uncover<3->{\alertb{$ P_3 = (p\phi)^3 (1-p\phi)^3 \chi_{m-3}(p,\phi) $}} \\
        $ D_4 = \{1, 0, 1, 0, 0, D_{m-2}^{a,b}\} $ \\
        \qquad\qquad\qquad\qquad\qquad\qquad\qquad
        \uncover<3->{\alertb{$ P_4 = (p\phi)^2 (1-p\phi)^3 \chi_{m-2}(p,\phi) $}} \\
        $ D_5 = \{1, 0, 1, 0, 0, D_{m-3}^{a,b}, 1\} $ \\
        \qquad\qquad\qquad\qquad\qquad\qquad\qquad
        \uncover<3->{\alertb{$ P_5 = (p\phi)^3 (1-p\phi)^3 \chi_{m-3}(p,\phi) $}} \\
        $ D_6 = \{1, 0, 1, 0, 0, D_{m-4}^{a,b}, 1, 0\} $ \\
        \qquad\qquad\qquad\qquad\qquad\qquad\qquad
        \uncover<3->{\alertb{$ P_6 = (p\phi)^3 (1-p\phi)^4 \chi_{m-4}(p,\phi) $}}
    \end{tabular}
    }

\end{frame}

\begin{frame}
  \frametitle{Fixed points for \alertb{$r < 1$}, \alertb{$\dstar = 2$}, and \alertb{$T = 3$}}
  $$
  \mbox{F.P.\ Eq:} \ \
  \phifix = \Gamma(p,\phifix;r) 
  + \sum_{i=\dstar}^{T}
  \binom{T}{i}
  (p\phifix)^{i} (1 - p\phifix)^{T-i}.
  $$
  where
  $
  \Gamma(p,\phifix;r) = $
  $$
  \alert{(1-r)(p\phi)^2 (1-p\phi)^2}
   +  \sum_{m=1}^{\infty} (1-r)^m 
  (p\phi)^2 (1-p\phi)^2 \times
  $$
  $$
  \left[ 
    \chi_{m-1} + 
    \chi_{m-2} + 
    2p\phi (1-p\phi)\chi_{m-3} + 
    p\phi (1-p\phi)^2\chi_{m-4}
  \right]
  $$
  and
  $$
  \chi_m(p,\phifix)
  = 
  \sum_{k=0}^{\lfloor m/3\rfloor}
  \binom{m-2k}{k}
  (1-p\phifix)^{m-k}
  (p\phifix)^k.
  $$
  \small{
    Note: $\alert{(1-r)(p\phi)^2 (1-p\phi)^2}$ accounts
    for $\{1, 0, 1, 0\}$ sequence.
  }
\end{frame}


\begin{frame}
  \frametitle{Fixed points for \alertb{$r < 1$}, \alertb{$\dstar > 1$}, and \alertb{$T \ge 1$}}

  \begin{block}{$T=3$, $\dstar=2$}
  \begin{center}
      \includegraphics[width=0.625\textwidth]{figgc_T3_k2_bif_theorycomp2_noname.pdf}
      \begin{itemize}
      \item 
        $r=0.01, 0.05, 0.10, 0.15, \alertb{0.20},\ldots, 1.00$.
      \end{itemize}
  \end{center}
  \end{block}

\end{frame}

\begin{frame}
  \frametitle{Fixed points for \alertb{$r < 1$}, \alertb{$\dstar > 1$}, and \alertb{$T \ge 1$}}

  \begin{block}{$T=2$, $\dstar=2$}
  \begin{center}
      \includegraphics[width=0.625\textwidth]{figgc_T2_k2_bif_theorycomp_noname.pdf}
      \begin{itemize}
      \item 
        $r=0.01, 0.05, 0.10,\ldots, 0.3820 \pm 0.0001$.
      \item<2-> 
        No spreading for $r \gtrsim 0.382$.
      \end{itemize}
  \end{center}
  \end{block}

\end{frame}

\begin{frame}
  \frametitle{What we have now:}
  
  \begin{itemize}
  \item 
    Two kinds of contagion processes:
    \begin{enumerate}
    \item<2-> 
      Continuous phase transition: \alert{SIR-like}.
    \item<3-> 
      Saddle-node bifurcation: \alert{threshold model-like}.
    \end{enumerate}
  \item<4->
    \alert{$\dstar=1$:} spreading from small seeds possible.
  \item<5->
    \alert{$\dstar>1$:} critical mass model.
  \item<6->
    \alertb{Are other behaviors possible?}
  \end{itemize}

\end{frame}

\subsection{Heterogeneous\ version}

\begin{frame}
  \frametitle{Generalized model}

  \begin{itemize}
  \item<1-> 
    Now allow for general dose distributions ($f$)
    and threshold distributions ($g$).
  \item<2->
    Key quantities:
    $$
    P_k
    =
    \int_{0}^{\infty} \dee{\dstar}
    g(\dstar)
    P\left(
      %%    \textstyle{\sum_{i=1}^{k} d_i \ge \dstar}
      \sum_{j=1}^{k} d_j \ge \dstar
    \right)
    \  \mbox{where} \  1 \le k \le T.
    $$
  \item<3->
    $P_k$ = Probability that the threshold of \\
    \quad \quad \  a randomly selected individual\\
    \quad \quad \  will be exceeded by $k$ doses.
  \item<4->
    e.g., \\
    $P_1$ = Probability that \underline{\alertb{one dose}} will exceed\\
    \quad \quad \ the threshold of a random individual\\
    \ \quad  = Fraction of \underline{\alertb{most vulnerable}} individuals.
  \end{itemize}

\end{frame}

\begin{frame}
  \frametitle{Generalized model---\alertb{heterogeneity}, $r=1$}

  \begin{itemize}
  \item
    Fixed point equation:
    $$
    \phifix
    =
    \sum_{k=1}^{T}
    \binom{T}{k}
    (p\phifix)^{k}
    (1-p\phifix)^{T-k}
    \underline{\alertb{P_k}}
    $$
  \item<2-> 
    Expand around $\phifix=0$ to find
    when spread from single seed is possible:
    $$\alert{\boxed{p P_1 T \ge 1}}
    \uncover<3->{\qquad \mbox{or} \qquad
      \boxed{\alert{\Rightarrow p_c = 1/(TP_1)}}
    }
    $$
  \item<4->
    Very good:
    \begin{enumerate}
    \item<4-> 
      $P_1T$ is the expected number of vulnerables the
      initial infected individual meets before recovering.
    \item<5-> 
      $pP_1T$ is $\therefore$ the expected number of
      successful infections (equivalent to $R_0$).
    \end{enumerate}
  \item<6->
    Observe: $p_c$ may exceed 1 meaning no spreading from
    a small seed.
  \end{itemize}

\end{frame}

\begin{frame}
  \frametitle{Heterogeneous case}

  \begin{itemize}
  \item<1->
    \alertb{Next:} Determine slope of fixed point curve at 
    critical point $p_c$.
  \item<2->
    Expand fixed point equation around $(p,\phifix) = (p_c,0)$.
  \item<3->
    Find slope depends on $(P_1 - P_2/2)$\cite{dodds2005a}\\
    (see Appendix).
  \item<4->
    Behavior near fixed point depends on whether
    this slope is
    \begin{enumerate}
    \item<5->
      positive: $P_1>P_2/2$ (continuous phase transition)
    \item<6->
      negative: $P_1<P_2/2$ (discontinuous phase transition)
    \end{enumerate}
%%  \item<6->
%%    Interpretation 
  \item<7->
    Now find \underline{\alert{three}} 
    basic universal classes
    of contagion models...
  \end{itemize}

\end{frame}

\begin{frame}
  \frametitle{Heterogeneous case}

  \begin{block}{Example configuration:}
    \begin{itemize}
    \item<1-> Dose sizes are lognormally distributed with mean 1
      and variance 0.433.
    \item<2-> Memory span: $T = 10$.
    \item<3-> Thresholds are uniformly set at 
      \begin{enumerate}
      \item<3-> $d_\ast = 0.5$
      \item<3-> $d_\ast = 1.6$
      \item<3-> $d_\ast = 3$
      \end{enumerate}
    \item<4-> Spread of dose sizes matters, details are not important.
    \end{itemize}
  \end{block}

\end{frame} 

\begin{frame}
  \frametitle{Three universal classes}

  \includegraphics[width=1\textwidth]{figgc_paperfigs_classes6c_noname}

  \begin{itemize}
  \item<2-> 
    Epidemic threshold: \hfill \small{$P_1 > P_2/2$, $p_c = 1/(TP_1) < 1$}
  \item<3-> 
    Vanishing critical mass: \hfill \small{$P_1 < P_2/2$, $p_c = 1/(TP_1) < 1$}
  \item<4-> 
    Pure critical mass: \hfill \small{$P_1 < P_2/2$, $p_c = 1/(TP_1) > 1$}
  \end{itemize}

\end{frame}


\begin{frame}
  \frametitle{Heterogeneous case}

  \begin{block}{Now allow $r<1$:}
    \includegraphics[width=0.6\textwidth]{figgc_T3_k2k1_bif_theorycomp2b_noname}
  \end{block}
  
  \begin{itemize}
  \item 
    II-III transition generalizes: $p_c = 1/[P_1 (T + \tau)]$
    where $ \tau = 1/r - 1 = $ expected recovery time
  \item 
    I-II transition less pleasant analytically.
  \end{itemize}
  
\end{frame}


\begin{frame}
  \frametitle{More complicated models}

  \includegraphics[width=0.95\textwidth]{figgc_manybif_thy_comb03_noname}

  \begin{itemize}
  \item 
    Due to heterogeneity in individual thresholds.
  \item
    Three classes based on behavior for small seeds.
  \item
    Same model classification holds:  I, II, and III.
  \end{itemize}

\end{frame}

\begin{frame}
  \frametitle{Hysteresis in vanishing critical mass models}

  \centering
  \includegraphics[width=0.65\textwidth]{figgc_manybif_thy_comb03b_noname}

\end{frame}

\section{Nutshell}

\begin{frame}
  \frametitle{Nutshell (one half)}

  \begin{itemize}
  \item<1-> 
    Memory is a natural ingredient. % emulsifying
  \item<2->
    Three universal classes of contagion processes:
    \begin{enumerate}
    \item 
      \alertb{I. Epidemic Threshold}
    \item 
      \alertb{II. Vanishing Critical Mass}
    \item 
      \alertb{III. Critical Mass}
    \end{enumerate}
  \item<3->
    Dramatic changes in behavior possible.
  \item<4->
    To change kind of model: `adjust' memory, recovery, fraction
    of vulnerable individuals ($T$, $r$, $\rho$, $P_1$, and/or $P_2$).
  \item<5->
    To change behavior given model: `adjust' probability
    of exposure ($p$) and/or initial number infected ($\phi_0$).
  \end{itemize}
\end{frame}

\begin{frame}
  \frametitle{Nutshell (other half)}

  \begin{itemize}
  \item<1-> 
    Single seed infects others if $p P_1 (T + \tau) \ge 1$.
  \item<2-> 
    Key quantity: $\alertb{p_c = 1/[P_1 (T + \tau)]}$
  \item<3-> 
    If $p_c < 1$ $\Rightarrow$ contagion can spread from single seed.
  \item<4->
    Depends only on:
    \begin{enumerate}
    \item 
      \alertb{System Memory ($T+\tau$).}
    \item 
      \alertb{Fraction of highly vulnerable individuals ($P_1$).}
    \end{enumerate}
  \item<5->
    \alertb{Details unimportant:}
    Many threshold and dose distributions give same $P_k$.
  \item<6->
    Another example of a model where
    vulnerable/gullible population may be more important than
    a small group of super-spreaders or influentials.
  \end{itemize}

\end{frame}

\section{Appendix}

\begin{frame}
  \frametitle{Appendix: Details for Class I-II transition:}

  \begin{eqnarray*}
    \label{gcontlong.eq:transIcalc1}
    \lefteqn{\phifix 
      = 
      \sum_{k=1}^{T}
      \binom{T}{k}
      P_k
      (p\phifix)^{k}
      (1-p\phifix)^{T-k},
    }
    \nonumber \\
%%    \uncover<2->{
      & = & 
      \sum_{k=1}^{T}
      \binom{T}{k}
      P_k
      (p\phifix)^{k}
      \sum_{j=0}^{T-k}
      \binom{T-k}{j}
      (-p\phifix)^{j},
      \nonumber \\
%%    }
%%    \uncover<3->{
      & = & 
      \sum_{k=1}^{T}
      \sum_{j=0}^{T-k}
      \binom{T}{k}
      \binom{T-k}{j}
      P_k
      (-1)^{j}(p\phifix)^{k+j},
      \nonumber \\
%%    }
%%    \uncover<4->{
      & = &
      \sum_{m=1}^{T}
      \sum_{k=1}^{m}
      \binom{T}{k}
      \binom{T-k}{m-k}
      P_k
      (-1)^{m-k}
      (p\phifix)^{m},
      \nonumber \\
%%    }
%%    \uncover<5->{
      & = & 
      \sum_{m=1}^{T}
      C_m
      (p\phifix)^{m}
      \nonumber \\
%%    }
  \end{eqnarray*}

\end{frame}


\begin{frame}
  \frametitle{Appendix: Details for Class I-II transition:}

  $$
  C_m = 
  (-1)^{m}
  \binom{T}{m}
  \sum_{k=1}^{m}
  (-1)^{k}
  \binom{m}{k}
  P_k,
  $$
  since 
  \begin{eqnarray*}
    \label{gcontlong.eq:ckcalc}  
    \binom{T}{k} \binom{T-k}{m-k}
    & = &
    \frac{T!}{k!(T-k)!}
    \frac{(T-k)!}{(m-k)!(T-m)!}
    \nonumber  \\
    & = & 
    \frac{T!}{m!(T-m)!}
    \frac{m!}{k!(m-k)!}
    \nonumber \\
    & = &
    \binom{T}{m} \binom{m}{k}.
  \end{eqnarray*}

\end{frame}

\begin{frame}
  \frametitle{Appendix: Details for Class I-II transition:}

  \begin{itemize}
  \item<1->
    Linearization gives
    \begin{equation*}
      \label{gcontlong.eq:transIcalc2}
      \phifix
      \simeq
      C_1 p \phifix
      + C_2 p_c^2 \phifix^2.
    \end{equation*}
    where $C_1 = T P_1 (= 1/p_c)$ 
    and $C_2 = \binom{T}{2}(-2P_1 + P_2)$.
  \item<2->
    Using $p_c = 1/(TP_1)$:
    \begin{equation*}
      \label{gcontlong.eq:transIcalc3}
      \phifix 
      \simeq
      \frac{C_1}{C_2 p_c^2} (p-p_c)
      =
      \frac{T^2 P_1^3}{(T-1)(P_1 - P_2/2)} (p-p_c).
    \end{equation*}
  \item<3->
    Sign of derivative governed by $P_1 - P_2/2$.
  \end{itemize}
\end{frame}

\begin{comment}
  
\begin{frame}
  \frametitle{Generalized model---ingredients}
  Basic idea: Incorporate memory of contagious element.

  \ding{228} Population of $N$ individuals, each in state S, I, or R.

  Independent of state:\\
  \ding{228} Each individual randomly contacts another at each time step.

  \ding{228} $\phi_t$ = fraction infected at time $t$ \\
  \quad \quad \ = probability of \underline{\alertb{contact}} with infected individual

  \ding{228} With probability $p$, contact with infective\\
  \quad leads to an \underline{\alertb{exposure}}.

  \ding{228} If exposed, individual receives a dose of size $d$\\
  \quad  drawn from distribution \alert{$f(d)$}.  Otherwise $d=0$.
\end{frame}

%%%%%%%%%%%%%
% 2a. ingredients:
%     memory
%     prob of infection being transferred

\begin{frame}
  \frametitle{Generalized model---ingredients}

  \alertb{$\boxed{\mbox{S} \Rightarrow \mbox{I}}$}

  Individuals `remember' last $T$ contacts:
  $$ D_{t,i} = \sum_{t'=t-T+1}^{t} d_i(t') $$

  Infection occurs if individual $i$'s `threshold' is exceeded:
  $$ D_{t,i} \ge \dstari $$

  Threshold $\dstari$ drawn from arbitrary distribution \alert{$g$} at $t=0$.
\end{frame}

\begin{frame}
  \frametitle{Generalized model---ingredients}

  \alertb{$\boxed{\mbox{I} \Rightarrow \mbox{R}}$}

  When  $D_{t,i} < \dstari $,\\
  individual $i$ recovers to state R
  with probability $r$.

  \alertb{$\boxed{\mbox{R} \Rightarrow \mbox{S}}$}

  Once in state R, individuals become susceptible again with
  probability $\rho$.
\end{frame}

% \begin{frame}
%   \frametitle{Model `explained'}
%  \centering
%   % \epsfig{SIRtransitions4.ps,width=.97\textwidth}
% \end{frame}


% P_k
\begin{frame}
  \frametitle{Generalized model}

  Important quantities:
  $$  
  P_k
  =
  \int_{0}^{\infty} \dee{\dstar}
  g(\dstar)
  P\left(
%    \textstyle{\sum_{i=1}^{k} d_i \ge \dstar}
    \sum_{j=1}^{k} d_j \ge \dstar
    \right)
  $$
  where $1 \le k \le T$.

  $P_k$ = Probability that the threshold of \\
\quad \quad \  a randomly selected individual\\
\quad \quad \  will be exceeded by $k$ doses.

e.g., \\
$P_1$ = Probability that \underline{\alertb{one dose}} will exceed\\
\quad \quad \ the threshold of a random individual\\
\ \quad  = Fraction of \underline{\alertb{most vulnerable}} individuals.

\end{frame}

\begin{frame}
  \frametitle{Generalized model---dose response curves}

  Contact with $K$ infected individuals in last $T$ time steps:
  $$
  P_{\textnormal{inf}}
  = \sum_{k=1}^K
  \binom{K}{k} 
  p^k ( 1 -p)^{K-k} P_k.
  $$
  % \epsfig{file=figdoseresponse11b_noname.ps,width=0.9\textwidth}\\
% \textbf{A.} Independent Interaction model; \\
% \textbf{B.} Stochastic Threshold model;\\
% \textbf{C.} Deterministic Threshold model.
\textbf{A.} Independent Interaction model
\hfill 
{\small $p=0.3$, 
$d = 1$, 
$\dstar=1$}\\
\textbf{B.} Stochastic threshold model
\hfill
{\small $p=1$, 
$f(d) \sim$ logn,
$\dstar=5$}\\
\textbf{C.} Deterministic threshold model
\hfill
{\small $p=1$, 
$d=1$,
$\dstar=5$}

%\textbf{A.} I. I. model
%\textbf{B.} Stochastic T. model
%\textbf{C.} Deterministic threshold model: 

\end{frame}


%%%%%%%%%%%% 5 mins
% 3b. our model---results

% homogeneous

% heterogeneous

\begin{frame}
  \frametitle{Generalized mean-field model}

  SIS-type contagion.

  Recovered individuals are immediately susceptible again.\\
  ($r=\rho=1$)
  
  Look for steady-state behavior (fixed points $\phifix$)\\
  as a function of exposure probability $p$.

%  \alertb{Homogeneous version:}\\
%  \ding{228} All individuals have threshold $\bar{\dstar}$\\
%  \ding{228} All dose sizes are equal: $d=1$

\end{frame}


% \begin{frame}
%   \frametitle{Generalized model}
% 
%   \alertb{Model details:}
% 
%   Mean field.
% 
%   Population size $N = 10^5$
% 
%   Number of time steps $N_t = 10^4$.
% 
%   Initialize with fixed $\phi$.
% 
% \end{frame}

% \begin{frame}
%   \frametitle{Fixed points for \alertb{$r<1$}, $\dstar=1$, and $T=1$}
% 
% $$
%   \phi_{t+1} = \underbrace{p \phi_{t}}_{\mbox{a}} 
%   + \phi_t
%   \underbrace{(1 - p\phi_t)}_{\mbox{c}}
%   \underbrace{(1-r)}_{\mbox{d}}.
% $$
% 
% a: Fraction newly infected between $t$ and $t+1$.
% 
% b: Probability of not being infected.
% 
% c: Probability of not recovering.
% 
% \end{frame}
% 
% \begin{frame}
%   \frametitle{Fixed points for \alertb{$r<1$}, $\dstar=1$, and $T=1$}
% 
%   Set $\phi_t = \phifix$:
% $$
% \phifix = p \phifix + ( 1 -p \phifix) \phifix (1-r)
% $$
% %$$
% %\Rightarrow  1 = p + (1-p\phifix)(1-r), \quad \phifix \ne 0,
% %$$
% 
% 
% $$
%  \Rightarrow 
%  \alert{\phifix = \frac{1 - r/p}{1-r}} 
%  \quad \mbox{and} \quad 
%  \alert{\phifix = 0}.
% $$
% 
% 
% \end{frame}
% 
% 
% %%%%%%%%%%%%%%%
%3a.  SIS, r=1, dstar=1, T>1
% 
% 
% \begin{frame}
%   \frametitle{Fixed points for $r=1$, $\dstar=1$, and \alertb{$T>1$}}
% 
%   
%   Recovery is immediate.
% 
%   Probability of being uninfected = $(1 - p\phifix)^T$.
% 
% 
% \alert{  $$
%   \phifix = 1 - (1 - p\phifix)^T.
%   $$
%   }
%   
% \end{frame}
% 
% 
% \begin{frame}
%   \frametitle{Fixed points for $r=1$, $\dstar=1$, and \alertb{$T>1$}}
% 
%   $$
%   \phifix = 1 - (1 - p\phifix)^T.
%   $$
% 
%   $$
%   \phifix \rightarrow 0, \ \ \phifix \simeq p T \phifix \ \ \Rightarrow \alert{p_c = 1/T}.
%   $$
% 
%   Can solve for $p$ but not $\phifix$:
%   $$
%   p = \phifix^{-1} [ 1 - (1-\phifix)^{1/T} ].
%   $$
%   
%   
% \end{frame}
% 
% 
% %%%%%%%%%%%%%%%
% %3c.  SIS, r<=1, dstar=1, T>=1
% 
% \begin{frame}
%   \frametitle{Fixed points for \alertb{$r \le 1$}, $\dstar=1$, and \alertb{$T \ge 1$}}
% 
%   Add fraction who did not receive any infections in last T time steps
%   and have not recovered from a previous infection.
%   $$H_1 = \{ \ldots, d_{t-T-2}, d_{t-T-1}, 1, \underbrace{0, 0, \ldots, 0, 0}_{\mbox{$T$ 0's}} \},$$
%   $$H_{m+1} = \{ \ldots, d_{t-T-m-1}, 1, \underbrace{0, 0, \ldots, 0, 0}_{\mbox{$m$ 0's}}, \underbrace{0, 0, \ldots, 0, 0}_{\mbox{$T$ 0's}} \};$$
% 
%   $$P(H_1) = p\phifix (1 -p\phifix)^T (1-r), $$
%   $$P(H_{m+1}) = p\phifix (1 -p\phifix)^{T+m} (1-r)^{m+1}. $$
% 
% \end{frame}

% \begin{frame}
%   \frametitle{Fixed points for $\dstar=1$, \alertb{$r \le 1$}, and \alertb{$T \ge 1$}}
% 
%   Fixed point equation:
% 
%   $$
%   \phifix =
%  1 - \frac{r (1-p\phifix)^T }
%     {1 - (1-p\phifix)(1-r)}.
%   $$
% 
%   $$
%   \phifix \rightarrow 0
%   \quad \Rightarrow \quad 
%   \alert{p_c} = \frac{1}{T + 1/r - 1} \alert{= \frac{1}{T + \tau}}.
%   $$
% 
%   \hfill $\tau$ = mean recovery time.
% 
% \end{frame}


% \begin{frame}
%   \frametitle{Homogeneous Model}
% 
%   Fixed points for $\dstar=1$, \alertb{$r \le 1$},  and \alertb{$T \ge 1$}.
%   $$
%   \begin{array}{l}
%     \phifix =
%     1 - \frac{r (1-p\phifix)^T }
%     {1 - (1-p\phifix)(1-r)}\\
%     \\
%     p_c = 1/(T+\tau)\\
%     \\
%     \mbox{from} \ \ p (T+\tau) \ge 1
%   \end{array}
%   \ \ \raisebox{-4cm}{% \epsfig{file=figgc_r0p50_k1_T2_paper2_noname.ps,width=0.425\textwidth}}
%   $$
%   $\tau = 1/r - 1$ = characteristic recovery time\\
%   $T + \tau \simeq $  total memory in system\\
% %  \hfill $T=2=1/r \rightarrow p_c = 1/3$ \qquad \mbox{}
% \hfill $\Rightarrow$  \alertb{Epidemic Threshold Models}
% \end{frame}
% 
% 
% 
% %%%%%%%%%%%%%%%
% %3d.  SIS, r=1, dstar>1, T>=dstar
% 
% \begin{frame}
%   \frametitle{Homogeneous models}
% 
%   Fixed points for $\dstar > 1$, \alertb{$r = 1$},  and \alertb{$T \ge 1$}.\\
%   $$
%   \begin{array}{l}
%   \phifix = \\
%   \sum_{i=\dstar}^{T}
%   \binom{T}{i}
%   (p\phifix)^{i} (1 - p\phifix)^{T-i}\\
%   \\
%   p_c = \infty\\
%   \end{array}
%    \raisebox{-4cm}{
%     % \epsfig{file=figgc_r1_k3_T12_paper2_noname.ps,width=0.425\textwidth}}
%   $$
% %  $\dstar=3$, $T=12$ 
%   $d=1$, $\bar{\dstar}=3$, $T=12$ \hfill $\Rightarrow$ Critical mass models\\
%   \hfill \ldots are there any other types?
%   
% \end{frame}

% \begin{frame}
%   \frametitle{Homogeneous models}
% 
%   \ding{228} Nontrivial threshold ($\dstar > 1$) \\
%   \ding{228} Immediate recovery ($r=1$)
% 
%   To be infected,\\ must have at least $\dstar$
%   exposures in last $T$ time steps:
% 
%   $$
%   \phifix = 
%   \sum_{i=\dstar}^{T}
%   \binom{T}{i}
%   (p\phifix)^{i} (1 - p\phifix)^{T-i}.
%   $$
% 
% \end{frame}
% 
% \begin{frame}
%   \frametitle{Homogeneous models}
%   \centering
%   Critical Mass Models\\
%   % \epsfig{file=figgc_r1_k3_T12_paper2_noname.ps,width=0.6\textwidth}\\
% %  $\dstar=3$, $T=12$ 
%   $d=1$, $\bar{\dstar}=3$, $T=12$ \hfill Saddle-node bifurcation
%   
% \end{frame}
% 

% \begin{frame}
%   \frametitle{Fixed points for $r = 1$, \alertb{$\dstar > 1$}, and \alertb{$T \ge 1$}}
% 
%   Solvable for small $T$; e.g., for $\dstar=2$, $T=3$:
% 
%   \begin{center}
%   % \epsfig{file=figgc_r1_k2_T3_noname.ps,width=0.4\textwidth}
%   \end{center}
% 
%  $\phifix = 3 p^2 \phifix (1 - p \phifix) + p^3 \phifix^2$
%  $\Rightarrow$ $(p_b,\phifix)=(8/9,27/32)$.
% 
% \end{frame}

% \begin{frame}
%  \frametitle{Fixed points for $r = 1$, \alertb{$\dstar > 1$}, and \alertb{$T \ge 1$}}

% \begin{center}
%  % \epsfig{file=figgc_T24_kvar_r1_noname.ps,width=0.45\columnwidth} 
%  % \epsfig{file=figgc_bipts_r1_noname.ps,width=0.45\textwidth}
%  $T=96$ ($\vartriangle$).
%  $T=24$ ($\triangleright$),
%  $T=12$ ($\triangleleft$),
%  $T=6$ ($\Box$),
%  and
%  $T=3$ ($\bigcirc$), 
% $T=24$ \hfill Bifurcation points \\
% \mbox{} \hfill $T=3$, 6, 12, 24, 96; 
%  \end{center}
%\end{frame}


% \begin{frame}
%   \frametitle{Homogeneous models}
% 
%   \centering
%   
%   Two classes of contagion models:
% 
% % \epsfig{file=figgc_r0p50_k1_T2_paper2_noname,width=0.475\textwidth}
% \hfill      
% % \epsfig{file=figgc_r1_k3_T12_paper2_noname.ps,width=0.475\textwidth}\\
% 
%  % \epsfig{file=figgc_T24_kvar_r1_noname.ps,width=0.45\columnwidth}
%  $T=96$ ($\vartriangle$).
%  $T=24$ ($\triangleright$),
%  $T=12$ ($\triangleleft$),
%  $T=6$ ($\Box$),
%  and
%  $T=3$ ($\bigcirc$), 
%  $T=24$\\
% 
%   \qquad I. Epidemic threshold models \hfill
%   II. Critical mass models
% 
%   \ding{228} Shift from $\dstar=1$ to $\dstar>1$ causes fundamental change.
% 
% \end{frame}


%%%%%%%%%%%%%
%3e   SIS, some specific cases


% \begin{frame}
%   \frametitle{Fixed points for \alertb{$r < 1$}, \alertb{$\dstar > 1$}, and \alertb{$T \ge 1$}}
% 
%   $D_i(t) = \sum_{t'=t-T+1}^{t} d_i(t')$ \hfill 
%   Partially summed random walk.
% 
%   \centering
%   % \epsfig{file=figrandomwalkcalc_noname.ps,width=0.7\columnwidth}
%   
%   $T=24$, $\dstar=14$.
% \end{frame}

% \begin{frame}
%   \frametitle{Fixed points for \alertb{$r < 1$}, \alertb{$\dstar > 1$}, and \alertb{$T \ge 1$}}
% 
%   $\gamma_m$ = fraction of individuals for whom $D(t)$ last equaled the threshold $m$ time steps ago.
% 
%   Fraction not recovered:
%   $$
%   \Gamma(p,\phifix;r) = \sum_{m=1}^{\infty} (1-r)^m \gamma_m(p,\phifix).
%   $$
% 
%   Fixed point equation:
% $$
%   \phifix = \Gamma(p,\phifix;r) 
%   + \sum_{i=\dstar}^{T}
%   \binom{T}{i}
%   (p\phifix)^{i} (1 - p\phifix)^{T-i}.
% $$
% 
% \end{frame}


% \begin{frame}
%   \frametitle{Fixed points for \alertb{$r < 1$}, \alertb{$\dstar > 1$}, and \alertb{$T \ge 1$}}
% 
% $T=3$, $\dstar=2$:
% 
% Two sequences lead to being below threshold:
% $$D_n=2 \Rightarrow D_{n+1}=1$$
% \alertb{\qquad \qquad \qquad $\{d_{n-2},d_{n-1},d_{n},d_{n+1}\} = \{1,1,0,\alert{0}\}$} \hfill  \\
% \alertb{\qquad \qquad \qquad $\{d_{n-2},d_{n-1},d_{n},d_{n+1},d_{n+2}\} = \{1,0,1,\alert{0},\alert{0}\}$} 
% 
%   Acceptable following sequences composed of
%   $$ a=\{0\} \quad \mbox{and} \quad b=\{1,0,0\}. $$
% \end{frame}

% \begin{frame}
%   \frametitle{Fixed points for \alertb{$r < 1$}, \alertb{$\dstar > 1$}, and \alertb{$T \ge 1$}}
% 
%   $N_a$ = number of $a$ subsequences.\\
%   $N_b$ = number of $b$ subsequences.
% 
%   $$
%   m = N_a \cdot 1 + N_b \cdot 3
%   $$
% 
%   $$
%   N_b = 0, 1, \ldots, [m/3].
%   $$
% 
%   $$
%   N_a = m, m-3, m-6, \ldots, m-3[m/3].
%   $$
% 
% \end{frame}

% \begin{frame}
%   \frametitle{Fixed points for \alertb{$r < 1$}, \alertb{$\dstar > 1$}, and \alertb{$T \ge 1$}}
% 
%   How many ways to arrange $N_a$ $a$'s and $N_b$ $b$'s?
% 
%   Think of overall sequence in terms of subsequences:
%   $$ \{Z_1, Z_2, \ldots, Z_{N_a + N_b} \} $$
% 
%   $N_a + N_b$ slots for subsequences.
% 
%   Choose positions of $a$'s or $b$'s:
%   $$
%   \binom{N_a + N_b}{N_a} = \binom{N_a + N_b}{N_b}.
%   $$
% 
% \end{frame}

% \begin{frame}
%   \frametitle{Fixed points for \alertb{$r < 1$}, \alertb{$\dstar > 1$}, and \alertb{$T \ge 1$}}
%   Total number of allowable sequences of length $m$:
%   $$
%   \sum_{N_b=0}^{[m/3]}
%   \binom{N_b+N_a}{N_b}
%   = 
%   \sum_{k=0}^{[m/3]}
%   \binom{m-2k}{k}
%   $$
% 
%   $P(a) = (1-p\phifix)$ and $P(b) = p\phifix(1-p\phifix)^2$
%   
%   Total probability of allowable sequences of length $m$:
%   $$
%   \chi_m(p,\phifix)
%   = 
%   \sum_{k=0}^{[m/3]}
%   \binom{m-2k}{k}
%   (1-p\phifix)^{m-k}
%   (p\phifix)^k.
%   $$
% 
% \end{frame}

% \begin{frame}
%   \frametitle{Fixed points for \alertb{$r < 1$}, \alertb{$\dstar > 1$}, and \alertb{$T \ge 1$}}
% 
%   Three endings $\Rightarrow$ Six possible sequences:
%   $$ D_1 = \{1, 1, 0, 0, D_{m-1}^{a,b}\} $$
%   $$ D_2 = \{1, 1, 0, 0, D_{m-2}^{a,b}, 1\} $$
%   $$ D_3 = \{1, 1, 0, 0, D_{m-3}^{a,b}, 1, 0\}$$
%   $$ D_4 = \{1, 0, 1, 0, 0, D_{m-2}^{a,b}\} $$
%   $$ D_5 = \{1, 0, 1, 0, 0, D_{m-3}^{a,b}, 1\} $$
%   $$ D_6 = \{1, 0, 1, 0, 0, D_{m-4}^{a,b}, 1, 0\} $$
% 
% 
% \end{frame}

% \begin{frame}
%   \frametitle{Fixed points for \alertb{$r < 1$}, \alertb{$\dstar > 1$}, and \alertb{$T \ge 1$}}
% 
%   Their corresponding probabilities:
%   $$ P_1 = (p\phi)^2 (1-p\phi)^2 \chi_{m-1}(p,\phi) $$
n%   $$ P_2 = (p\phi)^3 (1-p\phi)^2 \chi_{m-2}(p,\phi) $$
%   $$ P_3 = (p\phi)^3 (1-p\phi)^3 \chi_{m-3}(p,\phi) $$
%   $$ P_4 = (p\phi)^2 (1-p\phi)^3 \chi_{m-2}(p,\phi) $$
%   $$ P_5 = (p\phi)^3 (1-p\phi)^3 \chi_{m-3}(p,\phi) $$
%   $$ P_6 = (p\phi)^3 (1-p\phi)^4 \chi_{m-4}(p,\phi) $$
% \end{frame}


% \begin{frame}
%   \frametitle{Fixed points for \alertb{$r < 1$}, \alertb{$\dstar = 2$}, and \alertb{$T = 3$}}
%   $$
%   \mbox{F.P.\ Eq:} \ \
%   \phifix = \Gamma(p,\phifix;r) 
%   + \sum_{i=\dstar}^{T}
%   \binom{T}{i}
%   (p\phifix)^{i} (1 - p\phifix)^{T-i}.
%   $$
%   \begin{eqnarray*}
%   \Gamma(p,\phifix;r) & = & (1-r)(p\phi)^2 (1-p\phi)^2 
%    +  \sum_{m=1}^{\infty} (1-r)^m 
%   (p\phi)^2 (1-p\phi)^2 \times
%   \\ \nonumber
%   \\ \nonumber
n%   &  & 
%   \left[ 
%     \chi_{m-1} + 
%     \chi_{m-2} + 
%     2p\phi (1-p\phi)\chi_{m-3} + 
%     p\phi (1-p\phi)^2\chi_{m-4}
%   \right]
%   \\ \nonumber
%   \end{eqnarray*}
%   $$
%   \mbox{where} \ \  \chi_m(p,\phifix)
%   = 
%   \sum_{k=0}^{[m/3]}
%   \binom{m-2k}{k}
%   (1-p\phifix)^{m-k}
%   (p\phifix)^k.
%   $$

% \end{frame}

% \begin{frame}
%   \frametitle{Fixed points for \alertb{$r < 1$}, \alertb{$\dstar > 1$}, and \alertb{$T \ge 1$}}
% 
n%   \begin{center}
%       % \epsfig{file=figgc_T3_k2_bif_theorycomp2_noname.ps,width=0.625\textwidth}\\
%       $T=3$, $\dstar=2$;
%       \hfill
%       $r=0.01, 0.05, 0.10, 0.15, \alertb{0.20},\ldots, 1.00$.
%   \end{center}
% \end{frame}

% \begin{frame}
%   \frametitle{Fixed points for \alertb{$r < 1$}, \alertb{$\dstar > 1$}, and \alertb{$T \ge 1$}}
% 
%   \begin{center}
%       % \epsfig{file=figgc_T2_k2_bif_theorycomp_noname.ps,width=0.625\textwidth}\\
%       $T=2$, $\dstar=2$;
%       \hfill
%       $r=0.01, 0.05, 0.10,\ldots, 0.3820 \pm 0.0001$.
%   \end{center}
% \end{frame}


\begin{frame}
  \frametitle{Generalized model---heterogeneity, $r=1$}

%  Generalize to \alert{heterogeneous case} ($r=1$).

%  \ding{228} doses and thresholds arbitrarily distributed.

  Fixed point equation:
  $$
  \phifix
  =
  \sum_{k=1}^{T}
  \binom{T}{k}
  (p\phifix)^{k}
  (1-p\phifix)^{T-k}
  \underline{\alertb{P_k}}
  $$
  $$ \hfill \ \mbox{where} \
  P_k
  =
  \int_{0}^{\infty} \dee{\dstar}
  g(\dstar)
  P\left(
%    \textstyle{\sum_{i=1}^{k} d_i \ge \dstar}
    \sum_{j=1}^{k} d_j \ge \dstar
    \right)
  $$

  Spread from single seed if $p P_1 T \ge 1 \Rightarrow p_c = 1/(TP_1)$.

  Find \underline{\alertb{three}} universal classes
  of contagion models...

\end{frame}

\begin{frame}
  \frametitle{Heterogeneous case---Three universal classes}

  \begin{center}
    % \epsfig{file=figgc_paperfigs_classes6c_noname.ps,width=0.95\textwidth}
  \end{center}

  Epidemic threshold: \hfill $P_1 > P_2/2$, $p_c = 1/(TP_1) < 1$

  Vanishing critical mass: \hfill $P_1 < P_2/2$, $p_c = 1/(TP_1) < 1$

  Pure critical mass: \hfill $p_c = 1/(TP_1) > 1$

\end{frame}

\begin{frame}
  \frametitle{Heterogeneous case---Transitions between classes}

  \begin{center}
    % \epsfig{file=figgc_paperfigs_classes7a_noname.ps,width=0.9\textwidth}
  \end{center}

  I--II: $P_1 = P_2/2$, $p_c = 1/(TP_1) < 1$

  \hfill II---III: $p_c = 1/(TP_1) = 1$
\end{frame}


\begin{frame}
  \frametitle{SIS model}

%  Now allow $r<1$: \hfill
%  \raisebox{-4cm}{\epsfig{file=figgc_T3_k2k1_bif_theorycomp2b_noname.ps,width=0.49\textwidth}}

  II-III transition generalizes: $p_c = 1/[P_1 (T + \tau)]$ 
  
  (I-II transition less pleasant analytically)

\end{frame}

\begin{frame}
  \frametitle{More complicated models}

  % \epsfig{file=figgc_manybif_thy_comb03_noname.ps,width=0.95\textwidth}

  \ding{228} Due to heterogeneity in individual thresholds.

  \ding{228} Same model classification holds:  I, II, and III.
\end{frame}

\begin{frame}
  \frametitle{Hysteresis in vanishing critical mass models}

  \centering
  % \epsfig{file=figgc_manybif_thy_comb03b_noname.ps,width=0.65\textwidth}

\end{frame}

 

\begin{frame}
  \frametitle{SIRS model}
 % And $\rho<1$\ldots \hfill
%  \raisebox{-4cm}{% \epsfig{file=figgcSIR_logn_070b_noname.ps,width=0.49\textwidth}}

  Type II models become type I as $\rho$ decreases.

  (Type I and III models stay in same class)

\end{frame}

\begin{frame}
  \frametitle{SIR model}
  
%  Once immune state R is reached, no return to S: $\rho=0$.
  All individuals end in state R: $\rho=0$.

  But: for $T>1$ individuals remain infected\\ 
  for relatively long times:

  $$ \tmax \propto (1-p)^{-(T-1)} $$

  $\Rightarrow$ More `sick days.'

  $\Rightarrow$ More chance of spreading to
  other populations.
\end{frame}

% MUST DO BETTER THAN THIS!
%\begin{frame}
%  \frametitle{SIRS indicates SIR behavior}
%
%  \centering
%  % \epsfig{file=figgc_manybif_thy_comb03c_noname.ps,width=0.6\textwidth}\\
%  
%  Total fraction infected $\nearrow$ faster in SIS `growth' region.
% \end{frame}

%\begin{frame}
%  \frametitle{SIR model, $\rho=0$}
%  
%  $\tmax \propto (1-p)^{-(T-1)}$ \hfill
%  \raisebox{-4cm}{% \epsfig{file=figgcSIR_logn_050comb3_noname.ps,width=0.65\textwidth}}
%\end{frame}

 
%%%%%%%%%%%% 2 mins
% 4. conclusions
%    future work
%     SIR etc.
%     networks
%     memory may be important in disease



\begin{frame}
  \frametitle{Discussion}

  \begin{itemize}
  \item <1->
   Memory is crucial ingredient. % emulsifying
  \item <2->
   Three universal classes of contagion processes:\\
  \qquad \alertb{I. Epidemic Threshold}\\
  \qquad \alertb{II. Vanishing Critical Mass} \\ %\hfill $\Rightarrow P_1$ vs.\ $P_2/2$\\
  \qquad \alertb{III. Critical Mass}
  \item <3->
   Dramatic changes in behavior possible.
  \item <4->
   To change kind of model: `adjust' memory, recovery, fraction
  of vulnerable individuals ($T$, $r$, $\rho$, $P_1$, and/or $P_2$).
  \item <5->
   To change behavior given model: `adjust' probability
  of exposure ($p$) and/or initial number infected ($\phi_0$).
  \end{itemize}


\end{frame}

\begin{frame}
  \frametitle{Discussion}

  \begin{itemize}
  \item<1->
    If $p P_1 (T + \tau) \ge 1$, contagion can spread from single seed.
  \item<2->
    Key quantity: $\alertb{p_c = 1/[P_1 (T + \tau)]}$\\
  \item<3->
    Depends only on:\\
    1. \alertb{System Memory ($T+\tau$).}\\
    2. \alertb{Fraction of highly vulnerable individuals ($P_1$).}
  \item<4-> 
    \alert{Details unimportant} (Universality):\\
    Many threshold and dose distributions give same $P_k$.
  \item<5->
    Most vulnerable/gullible population may be more important than
    small group of super-spreaders or influentials.
  \end{itemize}

\end{frame}

\begin{frame}
  \frametitle{Future work/questions}

  \begin{itemize}
  \item <1->
   Do any real diseases work like this?\\
   \visible<2->{Cholera...}
  \item <3->
   Examine model's behavior on networks
  \item <4->
   Media/advertising + social networks model
  \item <5->
   Classify real-world contagions
  \end{itemize}

\end{frame}

\end{comment}

\begin{comment}


\begin{frame}
  \frametitle{Social Sciences---Threshold models}

  Action based on perceived behavior of others.

  \raisebox{3.5cm}{
    \begin{tabular}{l}
      \includegraphics[width=0.16\textwidth]{figthreshold_eg1_noname}\\
      \includegraphics[width=0.16\textwidth]{figthreshold_eg2_noname}
    \end{tabular}
  }
  \includegraphics[width=0.32\textwidth]{figthreshold3_noname}
  \includegraphics[width=0.32\textwidth]{figthresholdF3b_noname}

  $\phi$ = fraction of contacts `on' (e.g., rioting)\\
  Two states: S and I.  \hfill $\Rightarrow$ Critical mass model



%
%  \end{frame}
%%
%% \begin{frame}
%%    \frametitle{Fixed points for $r=1/3$, $\dstar=1$, and $T=1$}

%  \centering
%
%

% %%%%%%%%%%%%%%%
% %3a.  SIS, r=1, dstar=1, T>1
% 
% 
% 
%   \end{frame}
%%

\begin{frame}
  \frametitle{Generalized model}

  \alertb{Model details:}

  Mean field.

  Population size $N = 10^5$

  Number of time steps $N_t = 10^4$.

  Initialize with fixed $\phi$.





\end{frame}



%%%%%%%%%%%%
% outtakes %
%%%%%%%%%%%%

% SIR comparison
%
%  \end{frame}
%%
%% \begin{frame}
%%    \frametitle{}
%  
%\includegraphics{figtest_nw_SIR_01f2_noname.ps,width=0.45\textwidth}
%
%


%
%  \end{frame}
%%
%% \begin{frame}
%%    \frametitle{}
%  
%  \includegraphics{figtest_nw_thr2_06fcost1_noname.ps,width=0.45\textwidth}
%  \includegraphics{figtest_nw_thr2_06fcost2_noname.ps,width=0.45\textwidth}
%
%


\begin{frame}
  \frametitle{Discussion}

  \ding{228} Memory is a natural ingredient. % emulsifying

  \ding{228} Three universal classes of contagion processes:\\
  \qquad \alertb{I. Epidemic Threshold}\\
  \qquad \alertb{II. Vanishing Critical Mass} \\ %\hfill $\Rightarrow P_1$ vs.\ $P_2/2$\\
  \qquad \alertb{III. Critical Mass}

  \ding{228} Dramatic changes in behavior possible.

  \ding{228} To change kind of model: `adjust' memory, recovery, fraction
  of vulnerable individuals ($T$, $r$, $\rho$, $P_1$, and/or $P_2$).

  \ding{228} To change behavior given model: `adjust' probability
  of exposure ($p$) and/or initial number infected ($\phi_0$).



\end{frame}

\begin{frame}
  \frametitle{Discussion}

  Single seed infects others if $p P_1 (T + \tau) \ge 1$.

  Key quantity: $\alertb{p_c = 1/[P_1 (T + \tau)]}$\\

  If $p_c < 1$ $\Rightarrow$ contagion can spread from single seed.
  
  Depends only on:\\
  1. \alertb{System Memory ($T+\tau$).}\\
  2. \alertb{Fraction of highly vulnerable individuals ($P_1$).}

  Details unimportant:\\
  Many threshold and dose distributions give same $P_k$.

  \ding{228} Most vulnerable/gullible population may be more important than
  small group of super-spreaders or influentials.




\end{frame}



\begin{frame}
  \frametitle{Overview}

  \alertb{\ding{202}} The SIR model and $R_0$.

  \alertb{\ding{203}} Some important aspects of real epidemics:\\
  \qquad \alert{\ding{42}} peculiar distributions of epidemic size\\
  \qquad \alert{\ding{42}} resurgence

  \alertb{\ding{204}} Usual simple models do not fare well regarding \alertb{\ding{203}}
%  (\alertb{\ding{204}} does not yield \alertb{\ding{203}})

  \alertb{\ding{205}} A simple model incorporating movement does fare well.

  \alertb{\ding{206}} Conclusion.

\end{frame}


\begin{frame}
  \frametitle{Independent Interaction models}

  Infectious diseases---Tuberculosis, HIV, Ebola, SARS,...

  SIR model of infectious disease dynamics:

  Population of individuals, each of which are either S, I, or R.

  \alertb{S = Susceptible}
 
  \alertb{I = Infective/Infectious}

  \alertb{R = Recovered/Removed/Refractory} \hfill $[S(t) + I(t) + R(t) = 1]$

  Reed and Frost, 20's\\
  Kermack and McKendrick, 1927. \hfill many, many variations.



\end{frame}


  
\end{comment}





%     Examples of time series for single initial seeds and deterministic on/off response functions
%     averaging to a tent map distribution
%     for $\tavg{k}=5$.
%    Here, $N=10^4$ and the histograms of cascade size $s$ are derived from $s(t)$ for $t \ge 100$.







