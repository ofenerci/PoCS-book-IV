\section{Random}

\changelecturelogo{.18}{2011-02-07no-mouse-click-tp-10}

\begin{frame}

  \begin{block}{What's this?}
    \includegraphics[width=\textwidth]{figtallestbuildings100_noname.pdf}
  \end{block}

\end{frame}

\changelecturelogo{.18}{tarot-card-unknown-mechanism}

\begin{frame}
  \frametitle{Advances in sociotechnical algorithms:}

  \displaypaper{silver2016a}{1}

  \begin{center}
    \includegraphics[width=\textwidth,height=0.65\textheight,keepaspectratio]{silver2016a_fig3}
  \end{center}

  \begin{itemize}
  \item 
    Nature News (2016):
    \wordwikilink{http://www.nature.com/news/digital-intuition-1.19230}{Digital Intuition}
  \item 
    Wired (2012): \wordwikilink{http://www.wired.com/wiredscience/2012/04/network-science-of-the-game-of-go/}{Network
    Science of the game of Go}
  \end{itemize}

\end{frame}

\begin{frame}
  \displaypaper{tero2010a}{2}

  \begin{center}
    \includegraphics[height=0.5\textheight]{slime_mold_1-660x501.jpg}
  \end{center}

  Urban deslime in action:  
  \wordwikilink{https://www.youtube.com/watch?v=GwKuFREOgmo}{https://www.youtube.com/watch?v=GwKuFREOgmo}

\end{frame}

\begin{frame}

  \begin{block}{The Teletherm is nigh ... }
    \begin{center}
    \includegraphics[width=0.5\textwidth]{figburlington_teletherm001_noname.pdf}
    \includegraphics[width=0.5\textwidth]{figcentralpark_teletherm001_noname.pdf}
    \end{center}
    \begin{itemize}
    \item 
      Hibernal Teletherm $\approx$ February 4.
    \item 
      Halfway between Winter Solstice and Spring Equinox
    \item 
      Bonus: \wordwikilink{http://en.wikipedia.org/wiki/Groundhog_Day}{Groundhog Day}, 
      \wordwikilink{http://en.wikipedia.org/wiki/Imbolc}{Imbolc}, \ldots
    \item 
      Aesteval Teletherm $\approx$ July 19 (164 days later).
    \item
      To enjoy: 
      \wordwikilink{http://panometer.org/instruments/teletherms/}{site}
      and 
      \wordwikilink{http://arxiv.org/abs/1508.05938}{paper}.
    \end{itemize}
  \end{block}

\end{frame}

%%%%%%%%%%%%%%%%%%%%%%%%%%%%%%%%%%%
%% benford's law: ouroboros action

\changelecturelogo{.18}{tarot-card-law-of-first-digits}

\framedisplaypaper{mir2016a}{1}{fig1}


\changelecturelogo{.18}{tarot-card-random-walks}

\begin{frame}
  \frametitle{Applied knot theory:}

  \smallskip

  \displaypaper{fink1999a}{1}

  \smallskip

  \begin{center}
    \includegraphics[width=\textwidth,height=0.60\textheight,keepaspectratio]{fink1999a_fig1.pdf}
  \end{center}

\end{frame}

\begin{frame}
  \small
  \frametitle{Applied knot theory:}

  \begin{center}
    \includegraphics[width=\textwidth]{fink1999a_tab1.pdf}
  \end{center}

  \begin{columns}
    \column{0.025\textwidth}
    \column{0.40\textwidth}
    \begin{itemize}
    \item 
      $h$ = number of moves
    \item 
      $\gamma$ = number of center moves
    \item 
      $
      K(h,\gamma)
      =
      2^{\gamma-1}
      \binom{h-\gamma-2}
      {\gamma-1}
      $
    \end{itemize}
    \column{0.05\textwidth}
    \column{0.50\textwidth}
    \begin{itemize}
    \item 
      $
      s 
      = 
      \sum_{i=1}^{h} 
      x_{i}
      $
      where $x$ = -1 for $L$ and +1 for $R$.
    \item 
      $
      b 
      = 
      \frac{1}{2}
      \sum_{i=2}^{h-1}
      \left|
        \omega_{i} \alertr{+} \omega_{i-1}
      \right|
      $
      where $\omega = \pm 1$ represents winding direction.
    \end{itemize}
    \column{0.025\textwidth}
  \end{columns}

\end{frame}


%% Irregular verbs

\changelecturelogo{.18}{tarot-card-emergence-of-stories}

\begin{frame}
  \frametitle{Irregular verbs}

  \begin{block}{Cleaning up the code that is English:}
    \displaypaper{lieberman2007a}{1}
  \end{block}

  \begin{columns}
    \column{0.4\textwidth}
    \includegraphics[width=\textwidth]{NatureEvolutionofLanguageCover.pdf}
    \column{0.6\textwidth}
    \begin{block}{}
      \begin{itemize}
      \item<1-> 
        Exploration of how verbs with irregular 
        conjugation gradually become regular over time.
      \item<1-> 
        Comparison of verb behavior in Old, Middle, and Modern English.
      \end{itemize}
    \end{block}
  \end{columns}

\end{frame}
  
\begin{frame}
  \frametitle{Irregular verbs}


  \begin{block}{}
    \includegraphics[width=0.9\textwidth]{lieberman2007a_fig1a}

    \begin{itemize}
    \item 
      Universal tendency towards regular conjugation
    \item 
      Rare verbs tend to be regular in the first place
    \end{itemize}
  \end{block}

\end{frame}

\begin{frame}
  \frametitle{Irregular verbs}

  \begin{block}{}
    \includegraphics[width=\textwidth]{lieberman2007a_fig1b}

    \begin{itemize}
    \item<+->
      Rates are relative.
    \item<+->
      The \alertb{more common} a verb is, the \alertb{more resilient}
      it is to change.
    \end{itemize}
  \end{block}

\end{frame}

\begin{frame}[plain]
  \frametitle{Irregular verbs}

    \includegraphics[width=1.2\textwidth]{lieberman2007a_tab1}

    \begin{itemize}
    \item 
      \textcolor{red}{Red} = regularized
    \item 
      Estimates of half-life for regularization ($\propto f^{1/2}$)
    \end{itemize}

\end{frame}

\begin{frame}
%%  \frametitle{Irregular verbs}

  \begin{block}{}
    \begin{center}
      \includegraphics[width=0.8\textwidth]{lieberman2007a_fig2a}
    \end{center}

  \begin{itemize}
  \item 
    `Wed' is next to go.
  \item 
    -ed is the winning rule...
  \item<+->
    But `snuck' is 
    \wordwikilink{http://books.google.com/ngrams/graph?content=snuck\%2Csneaked\&year_start=1800\&year_end=2000\&corpus=0\&smoothing=3}{sneaking up on sneaked.}\cite{michel2010a}
  \end{itemize}
  \end{block}

\end{frame}

%% \begin{frame}
%%   \frametitle{Irregular verbs}
%% 
%%   \begin{block}{}
%%   \includegraphics[width=\textwidth]{lieberman2007a_fig2b}
%% 
%%   \begin{itemize}
%%   \item Regularization rate $\propto$ word frequency$^{-1/2}$
%%   \item Half life $\propto$ word frequency$^{1/2}$
%%   \end{itemize}
%%   \end{block}
%% 
%% \end{frame}

\begin{frame}
%%  \frametitle{Irregular verbs}

  \begin{block}{}
    \begin{center}
      \includegraphics[width=.8\textwidth]{lieberman2007a_fig3}
    \end{center}

  \begin{itemize}
  \item Projecting back in time to proto-Zipf story of many tools.
  \end{itemize}
  \end{block}

\end{frame}


%%% personality

\changelecturelogo{.18}{tarot-card-unknown-mechanism}

\begin{frame}
  \small
  \frametitle{Personality distributions:}

  \begin{block}{}
    \displaypaper{rentfrow2008a}{1}
  \end{block}

  \bigskip

  \begin{columns}
    \column{0.025\textwidth}
    \column{0.47\textwidth}
    Five Factor Model (FFM):
    \begin{itemize}
    \item 
      Extraversion [E]
    \item 
      Agreeableness [A]
    \item 
      Conscientiousness [C]
    \item 
      Neuroticism [N]
    \item 
      Openness [O]
    \end{itemize}
    \column{0.025\textwidth}
    \column{0.47\textwidth}
    ``...a robust and widely accepted framework for 
    conceptualizing the structure of personality...
    Although the FFM is not universally accepted in the field...''\cite{rentfrow2008a}

    \medskip

    \visible<2->{
      \alert{A concern:} self-reported data.
    }
    \column{0.025\textwidth}
  \end{columns}
\end{frame}

\begin{frame}
  \frametitle{Agreeableness:}

  \includegraphics[width=\textwidth]{rentfrow2008a_agreeableness.pdf}

\end{frame}

\begin{frame}
  \frametitle{Conscientiousness:}

  \includegraphics[width=\textwidth]{rentfrow2008a_conscientiousness.pdf}

\end{frame}

\begin{frame}
  \frametitle{Extraversion:}

  \includegraphics[width=\textwidth]{rentfrow2008a_extraversion.pdf}

\end{frame}

\begin{frame}
  \frametitle{Openness}

  \includegraphics[width=\textwidth]{rentfrow2008a_openness.pdf}

\end{frame}

\begin{frame}
  \frametitle{Neuroticism:}

  \includegraphics[width=\textwidth]{rentfrow2008a_neuroticism.pdf}

\end{frame}









%%%%%%%%%%%%%%%%%%%%%%%%%%%%%%%%%%%%%%%%%%%%%%%%%%%%%%%%
%% for references
\changelecturelogo{.18}{2011-02-07no-mouse-click-tp-10}




\begin{comment}



\begin{frame}

  \displaypaper{blasius2009a}{2}

    Zipf chess

    \url{http://physics.aps.org/articles/v2/97}

    \url{http://iopscience.iop.org/0295-5075/97/6/68002/article}


    Add Chess 

  \end{frame}

  \begin{frame}
    \url{http://www.wired.com/wiredscience/2012/04/network-science-of-the-game-of-go/}
  \end{frame}


  \begin{frame}

    Mastering the game of Go with deep neural networks and tree search : Nature : Nature Publishing Group from Chris Danforth’s Tweet


  \end{frame}






  %%%%%%%%%%%%%%%%%%%%%%%%%%%%%%%%%%%%%%%%%%%%%%%%%%%%%%%%%% 

    

    \begin{frame}

      Ants!
    \end{frame}


    \section{People}

    \begin{frame}
      \frametitle{Selflessness}

      Radiolab Podcast: 
      \wordwikilink{http://www.radiolab.org/2010/dec/14/equation-good/}{``An Equation for Good''}
      \bigskip
      \begin{columns}
    \column{0.6\textwidth}
    \includegraphics[width=\textwidth]{radiolab-2.pdf}
    \column{0.4\textwidth}
    \begin{itemize}
    \item 
      Natural selection, 
    \item 
      the `mystery of altrusim', 
    \item 
      George Price, 
    \item 
      madness.
    \end{itemize}
  \end{columns}

\end{frame}

\begin{frame}
  \frametitle{The Invention of Money}

  \begin{columns}
    \column{0.4\textwidth}
    \includegraphics[width=\textwidth]{2011-01-15thisamericanlife-inventionofmoney.jpg}
    \column{0.6\textwidth}
    \begin{itemize}
    \item 
      ``This American Life''
      Podcast on 
      \wordwikilink{http://www.thisamericanlife.org/radio-archives/episode/423/the-invention-of-money}{money and belief}
    \item 
      (1) Brazil and (2) the Fed...
    \end{itemize}
  \end{columns}

\end{frame}



\begin{frame}
  \frametitle{}
  
  \wordwikilink{http://en.wikipedia.org/wiki/Dunbar\'s\_number}{Dunbar's number} 
  and scaling.

  \cite{hill2008a}


\end{frame}

\begin{frame}
  \frametitle{Social networks}

  What's the average number of friends?

\end{frame}

\begin{frame}
  \frametitle{Prediction:}

  \includegraphics[width=\textwidth]{extrapolating-tp-10.pdf}\\
  \wordwikilink{http://xkcd.com/605/}{http://xkcd.com/605/}

  \visible<2->{By the third trimester, there will be hundreds of babies inside you...}
\end{frame}

\section{Random}

\changelecturelogo{.18}{2011-02-07no-mouse-click-tp-10}

\begin{frame}
  \frametitle{What's this?}
  
  \includegraphics[width=\textwidth]{figtallestbuildings100_noname.pdf}

\end{frame}

\changelecturelogo{.18}{2011-02-07do-not-open-this-box-tp-10}


\section{Videos}

\begin{frame}<1 | handout=0 | trans=1>
  \frametitle{Mimicry---the lying lyrebird}

  \begin{center}
    \includemovie[
    controls=true,
    toolbar=true,
    poster=lyrebird.jpg,
    ]{100mm}{75mm}{videos/2010/lyrebird.mp4}
  \end{center}

%    text=(tap, tap, tap, ...)

% \movie[borderwidth=5pt,%
% width=3cm,%
% height=2cm,%
% poster,%
% showcontrols=true%
% ]%
% {}%
% {videos/lyrebird.mp4} 

\end{frame}


\section{Universality}

\begin{frame}
  \frametitle{Universal numbers}

  \begin{columns}
    \column{0.3\textwidth}
    %% picture of a hand
    %% picture of a Simpson's hand
    \column{0.7\textwidth}
    \begin{itemize}
    \item<1-> 
      Accidents of evolution give us 5 + 5 = 10 fingers
      and hence base 10.
    \item<2->
      We could be happy about 6, 8, or 12.
    \end{itemize}
  \end{columns}
  

\end{frame}

\begin{frame}
  \frametitle{}


  \begin{columns}
    \column{0.4\textwidth}
    \includegraphics[width=\textwidth]{470px-Babylonian_numerals.png}
    \column{0.6\textwidth}
    \begin{itemize}
    \item<1->
      Beep.
    \end{itemize}
  \end{columns}
\end{frame}

\begin{frame}
  \frametitle{Great moments in Universality}

  Phyllotaxis

  Add pictures
\end{frame}


\section{Quirkology}

\begin{frame}
  \frametitle{Richard Wiseman's research:}
  
  \begin{columns}
    \column{0.4\textwidth}
    \includegraphics[width=\textwidth]{quirkology-bookcover.pdf}\\
    \column{0.6\textwidth}
    \begin{itemize}
    \item<1-> 
      \wordwikilink{http://www.quirkology.com}{http://www.quirkology.com}
    \item<2-> 
      Letter writing exercise...
    \end{itemize}
  \end{columns}
  
\end{frame}

\begin{frame}
  \frametitle{Quirkology}

  \begin{block}<1->{People who draw letters so others can read them tend to:}
    \begin{itemize}
    \item<2-> 
      be high `self-monitors'
    \item<3-> 
      be concerned with how others see them
    \item<4-> 
      adapt better to social situations
    \item<5-> 
      skilled at altering how others see them
    \item<6-> 
      be more adept at lying...
    \end{itemize}    
  \end{block}

  \begin{block}<7->{People how draw letters so they can read them tend to:}
    \begin{itemize}
    \item<8->
      be low `self-monitors'
    \item<9->
      follow their inner values
    \item<10->
      remain the same across social settings
    \item<11->
      be less adept at lying...
    \end{itemize}
  \end{block}

\end{frame}


\begin{frame}
  \frametitle{Quirkology}

  \begin{itemize}
  \item<1->
    And those who convince themselves
    they drew their letters the opposite way
    to what they really did\ldots

    \bigskip

    \visible<2->{\alertb{are good at deceiving themselves.}}
  \end{itemize}

\end{frame}


\begin{frame}
  \frametitle{Slide mold and optimal networks}

\end{frame}


\begin{frame}
  \frametitle{}

\end{frame}

\begin{frame}<handout: 0 | trans: 0>

  \wordwikilink{http://www.bigpumpkins.com/}{Big pumpkins}

\end{frame}







\end{comment}
