First survive, then get better.


Folkloristics
\cite{abello2012a}

Distant reading

Add papers by guy at Stanford
Franco Moretti!


\begin{frame}
  \frametitle{Scalable stories:}

  \begin{block}{The \alertb{narrative hierarchy}---explaining things on many scales:}
    \begin{itemize}
    \item 
      1 to 3 word encapsulation, a soundbite, a buzzframe,
    \item 
      a sentence/title,
    \item 
      a few sentences (logline),
    \item 
      a paragraph,
    \item 
      a short paper,
    \item 
      a long paper,
    \item 
      a chapter,
    \item 
      a book,
    \item 
      \ldots
    \end{itemize}
  \end{block}

\end{frame}

\begin{frame}
  \frametitle{The Great Man Theory, etc.:}

  \amazonbook{campbell1991a}
  \bigskip
  \amazonbook{campbell2008a}

\end{frame}


\begin{frame}
  \frametitle{Seven ``good'' stories?:}

  \amazonbook{booker2005a}
\end{frame}

\begin{frame}
  \frametitle{How to write a screenplay:}

  \amazonbook{snyder2005a}

  \begin{itemize}
  \item 
    Logline = one or two sentence summary.
  \item 
    Logline fails to be a summary of logline.
  \item 
    Believes irony is key.
  \end{itemize}
\end{frame}


\begin{frame}
  
  \wordwikilink{http://exp.lore.com/post/40411963108/kurt-vonneguts-classic-lecture-on-the-shapes-of-stories}{Vonnegut's lecture on the shapes of stories:}
  \begin{center}
    \includegraphics[height=0.94\textheight]{tumblr_mft5lpRiy01r2qa6go1_1280.jpg}
  \end{center}

\end{frame}
