\section{Scaling-at-large}

\begin{frame}
  \includegraphics[width=\textwidth]{2015-09-02pocs-postcard-allometry_polaroid-1200px.png}
\end{frame}

\begin{frame}
  \frametitle{Scalingarama}

  \begin{block}{General observation:}
    Systems (complex or not) 
    that cross many spatial and temporal scales
    often exhibit some form of \alertg{scaling}.
  \end{block}

  \begin{block}{Outline---All about scaling:}
    \begin{itemize}
    \item<1->
      Basic definitions.
    \item<2->
      Examples.
    \end{itemize}
  \end{block}

  \begin{block}{Later:}
    \begin{itemize}
    \item<3->
      How to measure your power-law relationship.
    \item<4->
      Scaling in blood and river networks.
    \item<5->
      The Unsolved Allometry Theoricides.
    \end{itemize}
  \end{block}

\end{frame}

%%%%%%%%%%%%%%%%%%%%%%%%%%%%%%%%%%%%% 
%% definitions and examples
%%%%%%%%%%%%%%%%%%%%%%%%%%%%%%%%%%%%% 

\begin{frame}
  \frametitle{Definitions}

  \begin{block}{}
    A \alertb{power law} relates two
    variables $x$ and $y$ as follows:
    {\Large
      $$ y = c x^\alpha $$
    }
  \end{block}

  \begin{block}{}
    \begin{itemize}
    \item 
      $\alpha$ is the \alertg{scaling exponent} (or just exponent)
    \item 
      ($\alpha$ can be any number in principle but we will
      find various restrictions.)
    \item 
      $c$ is the \alertg{prefactor} (which can be important!)
    \end{itemize}
  \end{block}

\end{frame}

\begin{frame}
  \frametitle{Definitions}

  \begin{block}{}
    \begin{itemize}
    \item<1->
      The \alertb{prefactor} $c$ must \alertb{balance dimensions}.
    \item<2->
      Imagine the height $\ell$ and volume $v$ of 
      a family of shapes
      are related as:
      $$\ell = c v^{1/4}$$
    \item<3->
      Using $[ \cdot ]$ to indicate dimension, then 
      $$[c] = [l]/[V^{1/4}] = L/L^{3/4} = L^{1/4}.$$
    \end{itemize}
  \end{block}

\end{frame}



\begin{frame}
  \frametitle{Looking at data}

  \begin{block}{}
    \begin{itemize}
    \item<1->
      Power-law relationships
      are linear in log-log space:
      $$y = c x^\alpha $$
      $$ \Rightarrow \alertb{\log_{b} y = \alpha \log_{b} x + \log_{b} c} $$
      with slope equal to $\alpha$, the scaling exponent.
    \item<2-> 
      Much searching for straight lines on 
      \alertb{log-log} or \alertb{double-logarithmic plots}.
    \item<3-> 
      Good practice: \alertg{Always, always, always use base 10}.
    \item<4-> 
      Talk only about orders of magnitude (powers of 10).
    \end{itemize}
  \end{block}

\end{frame}

\begin{frame}[plain]
  \frametitle{A beautiful, heart-warming example:}

  \begin{block}{}
    \begin{columns}
      \column{0.03\textwidth}
      \column{0.35\textwidth}
      \includegraphics[width=.8\textwidth]{zhang2000a_fig1-tp-5.pdf} 
      \begin{itemize}
      \item 
        $G$ = volume of gray matter:\\ \alertg{`computing elements'}
      \item 
        $W$ = volume of white matter:\\ \alertg{`wiring'}
      \item 
        \alertb{$W \sim c G^{1.23}$}
      \end{itemize}
      \column{0.9\textwidth}
      \includegraphics[width=\textwidth]{zhang2000a_fig2.jpg}  
      \begin{itemize}
      \item 
        \small from Zhang \& Sejnowski, PNAS (2000)\cite{zhang2000a}
      \end{itemize}
    \end{columns}
    

  \end{block}

\end{frame}

\begin{frame}
  \frametitle{Why is $\alpha \simeq 1.23$?}

  \begin{block}<2->{Quantities (following Zhang and Sejnowski):}
    \begin{itemize}
    \item $G = $ Volume of gray matter (cortex/processors)
    \item $W = $ Volume of white matter (wiring)
    \item $T = $ Cortical thickness (wiring)
    \item $S = $ Cortical surface area
    \item $L = $ Average length of white matter fibers
    \item $p$ = density of axons on white matter/cortex interface
    \end{itemize}
  \end{block}

  \begin{block}<3->{A rough understanding:}
    \begin{itemize}
    \item<4-> \alertb{$G \sim S T$} (convolutions are okay)
    \item<5-> \alertg{$W \sim \frac{1}{2} p S L$}
    \item<6-> \alertb{$G \sim L^3$} \visible<8>{$\leftarrow$ this is a little sketchy...}
    \item<7-> Eliminate $S$ and $L$ to find \alertg{$W \propto G^{\, 4/3}/T$}
    \end{itemize}
  \end{block}
  

\end{frame}

\begin{frame}
  \frametitle{Why is $\alpha \simeq 1.23$?}

  \begin{block}<1->{A rough understanding:}
    \begin{itemize}
    \item<1-> 
      We are here: \alertg{$W \propto G^{\, 4/3}/T$}
    \item<2->
      Observe weak scaling $T \propto G^{\, 0.10 \pm 0.02}$.
    \item<3->
      (Implies $S \propto G^{\, 0.9}$ $\rightarrow$ convolutions fill space.)
    \item<4->
      $\Rightarrow W \propto G^{\, 4/3}/T \propto G^{\, 1.23 \pm 0.02}$
    \end{itemize}
  \end{block}

\end{frame}

\begin{frame}
  \frametitle{Trickiness:}

  \begin{block}{}
    \begin{center}
      \includegraphics[width=0.6\textwidth]{zhang2000a_fig5.jpg}
    \end{center}
  \end{block}

  \begin{block}{}
    \begin{itemize}
    \item With $V = G + W$, some power laws must be approximations.
    \item<2-> Measuring exponents is a hairy business...
    \end{itemize}
  \end{block}


\end{frame}

\begin{frame}
  \frametitle{Good scaling:}

  \begin{block}{General rules of thumb:}
    \begin{itemize}
    \item<1->
      \textit{High quality:} scaling persists over\\
      \alertb{three or more orders of magnitude}\\
      for \alertg{each variable}.
    \item[]
    \item<2->
      \textit{Medium quality:} scaling persists over \\
      \alertb{three or more orders of magnitude}\\
      for \alertg{only one variable} and 
      \alertb{at least one} for \alertg{the other}.
    \item[]
    \item<3->
      \textit{Very dubious:} scaling `persists' over\\
      \alertb{less than an order of magnitude}\\
      for \alertg{both variables}.
    \end{itemize}
  \end{block}

\end{frame}

\begin{frame}
  \frametitle{Unconvincing scaling:}

  \begin{block}{Average walking speed as a function of city population:}
    \begin{columns}
      \column{0.5\textwidth}
      \includegraphics[width=\textwidth]{bettencourt2007a_fig.pdf}
      \column{0.5\textwidth}
      Two problems:
      \begin{enumerate}
      \item 
        use of natural log, and
      \item 
        minute varation in dependent variable.
      \end{enumerate}
    \end{columns}
    \begin{itemize}
    \item 
      {\small from Bettencourt et al. (2007)\cite{bettencourt2007a};
        otherwise totally great---see later.}
    \end{itemize}
  \end{block}

\end{frame}

\begin{frame}
  \frametitle{Definitions}

  \begin{columns}
    \column{0.15\textwidth}
    \column{0.6\textwidth}
    \begin{block}<1->{Power laws are the signature of \alertg{scale invariance}:}
      \bigskip

      Scale invariant \alertb{`objects'}\\
      look the \alertb{`same'}\\ 
      when they are 
      appropriately \alertb{rescaled.}
    \end{block}
    \column{0.25\textwidth}
  \end{columns}

  \bigskip

  \begin{block}{}
    \begin{itemize}
    \item<2-> 
      \alertb{Objects} = geometric shapes, time series, functions, relationships, distributions,...
    \item<3-> 
      \alertb{`Same'} might be \alertb{`statistically the same'}
    \item<4-> 
      To \alertb{rescale} means to change the units
      of measurement for the relevant variables
    \end{itemize}
  \end{block}

\end{frame}

\begin{frame}
  \frametitle{Scale invariance}

  \begin{block}{Our friend \alertb{$y=cx^{\alpha}$}:}
    \begin{itemize}
    \item<1->
      If we rescale $x$ as $x = rx'$ and $y$ as $y = r^\alpha y'$,
    \item<2->
      then
      $$r^\alpha y' = c (rx')^{\alpha}$$
    \item<3->
      $$\Rightarrow y' = c r^{\alpha} {x'}^{\alpha}r^{-\alpha}$$
    \item<4->
      $$\Rightarrow y' = c {x'}^{\alpha}$$
    \end{itemize}
  \end{block}

\end{frame}

\begin{frame}
  \frametitle{Scale invariance}

  \begin{block}{Compare with \alertb{$y=c e^{-\lambda x}$}:}
    \begin{itemize}
    \item<+->
      If we rescale $x$ as $x = rx'$, then
      $$ y = c e^{-\lambda rx'} $$
    \item<+-> 
      Original form cannot be recovered.
    \item<+-> 
      \alertg{Scale matters} for the exponential.
    \end{itemize}
  \end{block}

  \begin{block}<+->{More on \alertb{$y=c e^{-\lambda x}$}:}
    \begin{itemize}
    \item<+-> 
      Say $x_0 = 1/\lambda$ is the \alertb{characteristic scale}.
    \item<+-> 
      For $x \gg x_0$, $y$ is small,\\
      while for $x \ll x_0$, $y$ is large.
    \end{itemize}
  \end{block}

\end{frame}


%%%%%%%%%%%%%%%%%%%%%%%%%%%%%%%%%%%%% 
%% allometry
%%%%%%%%%%%%%%%%%%%%%%%%%%%%%%%%%%%%% 

\section{Allometry}      

\begin{frame}
  \small
  %% \frametitle{Definitions:}

  \begin{block}{Isometry:}
    \begin{columns}
      \column{0.1\textwidth}
      \column{0.35\textwidth}
      \includegraphics[width=\textwidth]{iso_tree.jpg}
      \column{0.55\textwidth}
      \begin{itemize}
      \item 
        Dimensions scale linearly with each other.
      \end{itemize}
    \end{columns}
  \end{block}

  \begin{block}{Allometry:}
    \begin{columns}
      \column{0.1\textwidth}
      \column{0.35\textwidth}
      \includegraphics[width=\textwidth]{allo_tree.jpg}\\
      \column{0.55\textwidth}
      \begin{itemize}
      \item 
        Dimensions scale nonlinearly.
      \end{itemize}
    \end{columns}
  \end{block}

  \begin{block}<+->{\wordwikilink{http://en.wikipedia.org/wiki/Allometry}{Allometry:}}
    \begin{itemize}
    \item<+-> 
      Refers to differential growth rates of the parts
      of a living organism's body part or process.
    \item<+-> 
      First proposed by Huxley and Teissier, Nature, 1936\\
      ``Terminology of relative growth''\cite{huxley1936a,shingleton2010a}
    \end{itemize}
  \end{block}

\end{frame}

\begin{frame}
  \frametitle{Definitions}

  \begin{block}<1->{Isometry versus Allometry:}
    \begin{itemize}
    \item 
      Iso-metry = `same measure'
    \item 
      Allo-metry = `other measure'
    \end{itemize}
  \end{block}

  \bigskip

  \begin{block}<2->{Confusingly, we use allometric scaling to refer to both:}
    \begin{enumerate}
    \item<3-> 
      Nonlinear scaling of a dependent variable
      on an independent one (e.g., $y \propto x^{1/3}$)
    \item<4->
      The relative scaling of correlated measures\\
      (e.g., white and gray matter).
    \end{enumerate}
  \end{block}

\end{frame}

\changelecturelogo{.18}{mcmahon1983a_p2}

\section{Examples\ in\ Biology}

\begin{frame}
  \frametitle{An interesting, earlier treatise on scaling:}

  \begin{columns}
    \column{0.35\textwidth}
    McMahon and Bonner, 1983\cite{mcmahon1983a}
    \column{0.6\textwidth}
    \includegraphics[height=0.85\textheight]{mcmahon1983a_cover.pdf}
    \column{0.05\textwidth}
  \end{columns}

\end{frame}

%% \begin{frame}
%%   \frametitle{For the following slide:}
%%   
%%   \includegraphics[angle=0,height=0.85\textheight]{mcmahon1983a_p3caption1}
%% %   \includegraphics[angle=-1,height=0.85\textheight]{mcmahon1983a_p3caption2}
%% %   \includegraphics[angle=-1,oheight=0.85\textheight]{mcmahon1983a_p3caption3}
%%   
%%   \small{p.\ 2, McMahon and Bonner\cite{mcmahon1983a}}
%%   
%% \end{frame}


\begin{frame}[plain]
  \frametitle{The many scales of life:}

  \begin{block}{}
    \begin{columns}
      \column{0.025\textwidth}
      \column{0.4\textwidth}
      \includegraphics[angle=0,width=\textwidth]{mcmahon1983a_p3caption1}\\
      {\small p.\ 2, McMahon and Bonner\cite{mcmahon1983a}}
      \column{0.8\textwidth}
      \includegraphics[angle=-1,height=0.85\textheight]{mcmahon1983a_p2}
    \end{columns}
  \end{block}

\end{frame}

%% \begin{frame}
%%   \frametitle{For the following slide:}
%%   
%% %   \includegraphics[angle=-1,height=0.85\textheight]{mcmahon1983a_p3caption1}
%%   \includegraphics[angle=0,height=0.85\textheight]{mcmahon1983a_p3caption2}
%% %   \includegraphics[angle=-1,height=0.85\textheight]{mcmahon1983a_p3caption3}
%%   
%%   \small{p.\ 2, McMahon and Bonner\cite{mcmahon1983a}}
%%   
%% \end{frame}

\begin{frame}[plain]
  \frametitle{The many scales of life:}

  \begin{block}{}
    \begin{columns}
      \column{0.025\textwidth}
      \column{0.35\textwidth}
      \includegraphics[angle=0,width=\textwidth]{mcmahon1983a_p3caption2}\\
      \small p.\ 3, McMahon and Bonner\cite{mcmahon1983a}
      \\
      \small
      More on the Elephant Bird \wordwikilink{https://en.wikipedia.org/wiki/Aepyornis}{here}.
      \column{0.8\textwidth}
      \includegraphics[angle=0,height=0.85\textheight]{mcmahon1983a_p3fig1}
    \end{columns}
  \end{block}

\end{frame}

%% \begin{frame}
%%   \frametitle{For the following slide:}
%%   
%% %   \includegraphics[angle=-1,height=0.85\textheight]{mcmahon1983a_p3caption1}
%% %   \includegraphics[angle=-1,height=0.85\textheight]{mcmahon1983a_p3caption2}
%%   \includegraphics[angle=0,height=0.85\textheight]{mcmahon1983a_p3caption3}
%%   
%%   \small{p.\ 2, McMahon and Bonner\cite{mcmahon1983a}}
%%   
%% \end{frame}

\begin{frame}[plain]
  \frametitle{The many scales of life:}

  \begin{block}{}
    \begin{columns}
      \column{0.025\textwidth}
      \column{0.3\textwidth}
      \includegraphics[angle=0,width=\textwidth]{mcmahon1983a_p3caption3}
      {\small p.\ 3, McMahon and Bonner\cite{mcmahon1983a}}
      \column{0.8\textwidth}
      \includegraphics[angle=0,height=0.85\textheight]{mcmahon1983a_p3fig2}    
    \end{columns}
  \end{block}

\end{frame}

\begin{frame}

  \begin{block}{Size range (in grams) and cell differentiation:}
    \begin{columns}
      \column{0.14\textwidth}
      \column{0.28\textwidth}
      \includegraphics[angle=0,width=\textwidth]{mcmahon1983a_p4sizerange.pdf}\\
      {\tiny $10^{-13}$ to $10^{8}$, p.\ 3, McMahon and Bonner\cite{mcmahon1983a}}
      \column{0.1\textwidth}
      \column{0.38\textwidth}
      \includegraphics[angle=0,width=\textwidth]{mcmahon1983a_p22cells}
      \column{0.1\textwidth}
    \end{columns}
  \end{block}

  

\end{frame}


\begin{frame}
  \frametitle{Non-uniform growth:}

  \begin{block}{}
    \begin{center}
      \includegraphics[angle=0,width=0.85\textwidth]{mcmahon1983a_p32humangrowth.pdf}    
    \end{center}
  \end{block}
  
  \small{p.\ 32, McMahon and Bonner\cite{mcmahon1983a}}

\end{frame}

\begin{frame}
  \frametitle{Non-uniform growth---arm length versus height:}

  \begin{block}{Good example of a \alertg{break in scaling}:}
    \includegraphics[angle=0,width=0.85\textwidth]{mcmahon1983a_p32armgrowth.pdf}

    A \alert{crossover} in scaling occurs around a height of 1 metre.
  \end{block}
  
  \small{p.\ 32, McMahon and Bonner\cite{mcmahon1983a}}
\end{frame}

\begin{frame}
  %% \frametitle

  \begin{block}{Weightlifting: $M_{\textrm{world record}} \propto M_{\textrm{lifter}}^{\, 2/3}$}
    \includegraphics[angle=1.5,width=0.75\textwidth]{mcmahon1983a_p56weightlifting.pdf}
    
    Idea: Power $\sim$ cross-sectional area of isometric lifters.
  \end{block}
  
  \small{p.\ 53, McMahon and Bonner\cite{mcmahon1983a}}

\end{frame}

\begin{frame}
  %% \frametitle

  \begin{block}{Titanothere horns: $L_{\textrm{horn}} \sim L_{\textrm{skull}^4}$}
    \begin{center}
      \includegraphics[angle=0,height=0.82\textheight]{mcmahon1983a_p36horns.pdf}    
    \end{center}
  \end{block}
  
  \small{p.\ 36, McMahon and Bonner\cite{mcmahon1983a}; a bit dubious.}

\end{frame}

%% \begin{frame}
%%   \frametitle{The allometry of trees:}
%%   
%%   Prothero...
%%   
%% \end{frame}


\begin{frame}
  \frametitle{Animal power}

  \begin{block}{Fundamental biological and ecological constraint:}
    $$
    \alertb{\boxed{P = c\, M^{\, \alpha}}}
    $$
    $$P = \mbox{basal metabolic rate}$$
    $$M = \mbox{organismal body mass}$$

    %% Mammals, poikilotherms, birds, trees,\\
    %% bacteria, rocks,\ldots\\
    \begin{columns}
      \column{0.025\textwidth}
      \column{0.3\textwidth}
      \includegraphics[width=\textwidth]{shrew.jpg}
      \column{0.025\textwidth}
      \column{0.3\textwidth}
      \begin{overprint}
        \onslide<1| handout:0| trans:0>
        \onslide<2-| handout:1| trans:1>
        \includegraphics[width=\textwidth]{shrew-elephant-tp-5.pdf}
      \end{overprint}
      \column{0.025\textwidth}
      \column{0.3\textwidth}
      \includegraphics[width=\textwidth]{250px-Re-exposure_of_elephant_-_lahugala_park1.jpg}
      \column{0.025\textwidth}
    \end{columns}
  \end{block}

\end{frame}
\changelecturelogo{.18}{sliderule_small}

\begin{frame}
  \frametitle{Stories---The Fraction Assassin:}

  \includegraphics[width=0.9\textwidth]{2014-08-26pocs-sketch-metabolism-assassin_shadow.png}
\end{frame}

\changelecturelogo{.18}{mcmahon1983a_p2}

\begin{frame}
  \frametitle{Ecology---\wordwikilink{http://en.wikipedia.org/wiki/Species-area\_curve}{Species-area law:}}

  \begin{block}{Allegedly (data is messy):\cite{macarthur1963a,levin1992a}}
    \displaypaper{macarthur1963a}{2}
    \begin{itemize}
    \item 
      $$
      N_{\textrm{species}} \propto A^{\, \beta}
      $$
    \item 
      According to physicists---on islands: $\beta \approx 1/4$.
    \item 
      Also---on continuous land: $\beta \approx 1/8$.
    \end{itemize}
  \end{block}

  
  %% \begin{block}{A focus:}
  %%   \begin{itemize}
  %%   \item<1-> How much energy do organisms need to live?
  %%   \item<2-> And how does this scale with organismal size?
  %%   \end{itemize}
  %% \end{block}

\end{frame}

\begin{frame}
  \frametitle{Cancer:}

  \displaypaper{tomasetti2015a}{2}

  \begin{center}
    \includegraphics[width=0.6\textwidth]{tomasetti2015a_fig1.pdf}
  \end{center}

  Roughly:
  $
  p 
  \sim
  r^{2/3}
  $
  where
  $p$ = life time probability 
  and
  $r$ = rate of stem cell replication.

\end{frame}

\begin{frame}
  
  \begin{block}{}
  \displaypaper{meyer-vernet2015a}{4}

  \includegraphics[width=\textwidth]{meyer-vernet2015a_fig1.pdf}\\
  \includegraphics[width=\textwidth]{meyer-vernet2015a_fig1_caption.pdf}
  \end{block}

  \insertassignmentquestionsoft{01}{1}
\end{frame}

\changelecturelogo{.18}{2015-09-02pocs-postcard-allometry_polaroid-1200px.png}

\section{Physics}

\begin{frame}
  \frametitle{Engines:}

  \includegraphics[height=0.9\textheight]{mcmahon1983a_p61engines.pdf}
  
  \tiny{BHP = brake horse power}
\end{frame}


\begin{frame}
  \small
  \frametitle{The allometry of nails:}

  \begin{block}{Observed: Diameter $\propto$ Length$^{2/3}$ or $d \propto \ell^{2/3}$.}
    \includegraphics[angle=0,width=0.45\textwidth]{mcmahon1983a_p58nails.pdf}
    \
    \includegraphics[angle=0,width=0.45\textwidth]{mcmahon1983a_p59nails.pdf}
  \end{block}

  \begin{block}<+->{Since $\ell d^2 \propto$ Volume $v$:}
    \begin{itemize}
    \item<+-> 
      Diameter $\propto$ \uncover<+->{Mass$^{2/7}$ or $d \propto v^{2/7}$.}
    \item<+-> 
      Length $\propto$ \uncover<+->{Mass$^{3/7}$ or $\ell \propto v^{3/7}$.}
    \item<+-> 
      Nails lengthen faster than they broaden (c.f. trees).
    \end{itemize}
  \end{block}

  \small{p.\ 58--59, McMahon and Bonner\cite{mcmahon1983a}}

\end{frame}

\begin{frame}
  \frametitle{The allometry of nails:}

  \begin{block}<1->{A buckling instability?:}
    \begin{itemize}
    \item<2-> 
      \wordwikilink{http://en.wikipedia.org/wiki/Buckling}{Physics/Engineering result}: 
      Columns buckle under a load which depends on $d^4/\ell^2$.
    \item<3->
      To drive nails in, posit resistive force $\propto$ nail circumference = $\pi d$.
    \item<4->
      Match forces independent of nail size: $\alertg{d^4/\ell^2 \propto d}$.
    \item<5->
      Leads to \alertg{$d \propto \ell^{2/3}$}.
    \item<6->
      Argument made by Galileo\cite{galilei1638a} in 1638 in
      \wordwikilink{http://www.liberliber.it/biblioteca/g/galilei/discorsi_e_dimostrazioni_matematiche_intorno_a_due_nuove_etc/pdf/discor_p.pdf}{``Discourses on Two New Sciences.''} 
      Also, see \wordwikilink{http://en.wikipedia.org/wiki/Two_New_Sciences}{here.}
    \item<7->
      Another smart person's contribution: \wordwikilink{http://en.wikipedia.org/wiki/Buckling}{Euler, 1757}
    \item<8->
      Also see McMahon, ``Size and Shape in Biology,'' Science, 1973.\cite{mcmahon1973a}
    \end{itemize}
  \end{block}

\end{frame}



\begin{frame}
  %% \frametitle

  \begin{block}{Rowing: Speed $\propto $ (number of rowers)$^{1/9}$}
    \begin{center}
      \includegraphics[angle=0.8,width=0.8\textwidth]{mcmahon1983a_p46rowing.pdf} \\
      \includegraphics[width=0.8\textwidth]{mcmahon1983a_p47rowing.pdf}
    \end{center}
  \end{block}

\end{frame}


\begin{frame}
  \frametitle{Physics:}

  \begin{block}{Scaling in elementary laws of physics:}
    \begin{itemize}
    \item 
      Inverse-square law of gravity
      and Coulomb's law: 
      $$
      F 
      \propto 
      \frac{m_1 m_2}{r^{2}}
      \quad 
      \mbox{and} 
      \quad
      F 
      \propto 
      \frac{q_1 q_2}{r^{2}}.
      $$
    \item 
      Force is diminished by expansion of space away from source.  
    \item 
      The square is $d-1=3-1=2$, the dimension of
      a sphere's surface.
    \end{itemize}
  \end{block}

\end{frame}

%%%%%%%%%%%%%%%%%%%%%%%%% 
%% dimensional analysis
%%%%%%%%%%%%%%%%%%%%%%%%% 

\begin{frame}
  \frametitle{Dimensional Analysis:}

  The 
  \wordwikilink{https://en.wikipedia.org/wiki/Buckingham\_π\_theorem}{Buckingham
    $\pi$ theorem}:\footnote{\wordwikilink{http://en.wikipedia.org/wiki/Stigler's\_law\_of\_eponymy}{Stigler's
      Law of Eponymy} applies.
    See \wordwikilink{https://en.wikipedia.org/wiki/Dimensional\_analysis\#History}{here}.
  }

  \medskip

  \displaypaper{buckingham1914a}{1}

  \bigskip

  As captured in the 1990s in the MIT physics library:

  \medskip
  
  \includegraphics[width=0.15\textwidth,page=1]{buckingham1914a_scanned}
  \
  \includegraphics[width=0.15\textwidth,page=2]{buckingham1914a_scanned}
  \
  \includegraphics[width=0.15\textwidth,page=3]{buckingham1914a_scanned}
  \
  \includegraphics[width=0.15\textwidth,page=4]{buckingham1914a_scanned}
  \
  \includegraphics[width=0.15\textwidth,page=5]{buckingham1914a_scanned}
  \
  \includegraphics[width=0.15\textwidth,page=6]{buckingham1914a_scanned}

\end{frame}

\begin{frame}
  \small
  \frametitle{Dimensional Analysis:\footnote{Length is a dimension,
      furlongs and \wordwikilink{https://en.wikipedia.org/wiki/Smoot}{smoots} are units}}

  \begin{block}<+->{Fundamental equations cannot depend on units:}
    \begin{itemize}
    \item<+-> 
      System involves $n$ related quantities with some unknown equation $f(q_1, q_2, \ldots, q_n) = 0$.
    \item<+-> 
      Geometric ex.: area of a square, side length $\ell$:\\ 
      $A = \ell^2$ where $[A] = L^2$ and $[\ell] = L$.
    \item<+-> 
      Rewrite as a relation of $p \le n$ independent
      \wordwikilink{https://en.wikipedia.org/wiki/Dimensionless\_quantity}{dimensionless parameters}
      where $p$ is the number of independent dimensions (mass, length,
      time, luminous intensity \ldots):
      $$
      F(\pi_1,\pi_2, \ldots, \pi_p) = 0
      $$
    \item<+-> 
      e.g., $A/\ell^2 - 1 = 0$ where $\pi_1 = A/\ell^2$.
    \item<+-> 
      Another example: $F = ma$ $\Rightarrow$ $F/ma - 1 = 0$.
    \item<+-> 
      Plan: solve problems using only backs of envelopes.
      %% \item<+-> 
      %%   Aside: Good for physics.
      %%   What about biological, ecological, social, algorithmic systems?
    \end{itemize}

  \end{block}
\end{frame}


\begin{frame}
  \frametitle{Example:}

  \begin{block}{Simple pendulum:}
    \begin{columns}
      \column{0.02\textwidth}
      \column{0.42\textwidth}
      \includegraphics[width=\textwidth]{2015-09-03pocs-platypus-pendulum_1200px-mod-tp-10.png}
      \column{0.02\textwidth}
      \column{0.54\textwidth}
      \begin{itemize}
      \item<+->
        Idealized mass/platypus swinging forever.
      \item<+->
        Four quantities: 
        \begin{enumerate}
        \item<+-> 
          Length 
          $\ell$, 
        \item<+-> 
          mass $m$, 
        \item<+-> 
          gravitational
          acceleration $g$, 
          and 
        \item<+-> 
          pendulum's period $\tau$.
        \end{enumerate}
      \end{itemize}
      \column{0.03\textwidth}
    \end{columns}
  \end{block}

  \begin{block}<+->{}
    \begin{itemize}
    \item<+-> 
      Variable dimensions:
      $[\ell] = L$, 
      $[m] = M$, 
      $[g] = L T^{-2}$,
      and $[\tau] = T$.
    \item<+-> 
      Turn over your envelopes and find some $\pi$'s.
    \end{itemize}
  \end{block}

  %% \includegraphics[width=\textwidth]{2015-09-02pocs-homer-pendulum_1200px-tp-10.png}

\end{frame}

\changelecturelogo{0.18}{2015-09-02pocs-homer-pendulum_1200px-tp-10.png}

\begin{frame}
  \small

  \begin{block}<+->{A little formalism:}
    \begin{itemize}
    \item<+-> 
      Game: find all possible independent combinations of the
      $\{q_1, q_2, \ldots, q_n\}$,
      that form dimensionless quantities
      $\{\pi_1, \pi_2, \ldots, \pi_p\}$,
      where we need to figure out $p \le n$.
    \item<+-> 
      Consider 
      $
      \pi_i 
      = 
      q_1^{x_{1}} 
      q_2^{x_{2}} 
      \cdots
      q_n^{x_{n}} 
      $.
    \item<+->
      We (desperately) want to find all sets of powers $x_{j}$ that create dimensionless quantities.
    \item<+-> 
      Dimensions:
      want
      $
      [\pi_i]
      = 
      [q_1]^{x_{1}} 
      [q_2]^{x_{2}} 
      \cdots
      [q_n]^{x_{n}} 
      = 
      1
      $.
    \item<+->
      For the platypus pendulum we have \\
      $[q_1] = L$, $[q_2] = M$, $[q_3] = L T^{-2}$, and $[q_4] = T$,\\
    \item<+->[]
      with dimensions
      $d_1=L$, $d_2=M$, and $d_3=T$.
    \item<+->
      So:
      $
      [\pi_i]
      =
      L^{x_{1}}
      M^{x_{2}}
      (LT^{-2})^{x_{3}}
      T^{x_{4}}
      $.
    \item<+->
      We regroup:
      $
      [\pi_i]
      =
      L^{x_{1} + x_{3}}
      M^{x_{2}}
      T^{-2x_{3} + x_{4}}
      $.
    \item<+->
      We now need:
      $x_{1} + x_{3} = 0$,
      $x_{2} =  0$,
      and
      $-2x_{3} + x_{4}$.
    \item<+->
      Time for \visible<+->{\alertg{matrixology} \ldots}
    \end{itemize}
  \end{block}

\end{frame}

\begin{frame}
  \frametitle{\small Well, of course there are matrices:}

  \small
  \begin{block}{}
    \begin{itemize}
    \item<+->
      Thrillingly, we have:
      $$
      \m{A}\vec{x}
      =
      \mymatrix{cccc}{
        1 & 0 & 1 & 0 \\
        0 & 1 &  0 & 0 \\
        0 & 0 & -2 & 1 \\
      }
      \colvec{ x_{1} \\ x_{2} \\ x_{3} \\ x_{4} }
      = 
      \colvec{ 0 \\ 0 \\ 0}
      $$
    \item<+->
      A nullspace equation: $ \m{A} \vec{x} = \vec{0}$.
    \item<+->
      \alertg{Number of dimensionless parameters = Dimension of null
        space = $n - r$}
      where $n$ is the number of columns of $\m{A}$ and $r$ is the rank
      of $\m{A}$.
    \item<+->
      Here: $n=4$ and $r=3$ 
      \visible<+->{$\rightarrow$ $F(\pi_1) = 0$}
      \visible<+->{$\rightarrow$ $\pi_1$ = const.}
    \item<+->
      \alertdg{In general: Create a matrix $\m{A}$
        where $ij$th entry is the power of dimension $i$ in the $j$th
        variable,
        and solve by row reduction to find basis null vectors.}
    \item<+->
      We (you) find:
      \alertg{
        $
        \pi_1 = \ell / g \tau^2 = \mbox{const}.
        $
      }
      \visible<+->{
        Upshot: $\tau \propto \sqrt{\ell}$.
      }
    \item<+->[]
      \insertassignmentquestionsoft{01}{1}
    \end{itemize}
  \end{block}

\end{frame}


\changelecturelogo{.18}{2015-09-02pocs-postcard-allometry_polaroid-1200px.png}


\begin{frame}
  \small

  \displayamazonbook{barenblatt1996a}

  \begin{block}<2->{G.\ I. Taylor, magazines, and classified secrets:}
    \begin{columns}<3->
      \column{0.02\textwidth}
      \column{0.30\textwidth}
      1945 \\  New Mexico \\ Trinity test:
      \includegraphics[width=\textwidth]{trinities3-600x471.jpg}
      \column<+->{0.68\textwidth}
      Self-similar blast wave:
      \begin{itemize}
      \item<+-> 
        Radius: $[R] = L$, \\
        Time: $[t] = T$,\\
        Density of air: $[\rho]=M/L^{3}$,\\
        Energy: $[E] = ML^{2}/T^{2}$.
      \item<+-> 
        Four variables, three dimensions.
      \item<+-> 
        One dimensionless variable:\\
        $
        E
        = 
        \mbox{constant} \times
        \rho R^5 / t^2.
        $
        %% http://www.math.utah.edu/~zajac/Math1170/BlastRadius.pdf
        %% 1 \times 10^{21} ergs
      \item<+-> 
        Scaling: Speed decays as $1/R^{3/2}$.
      \end{itemize}
    \end{columns}
  \end{block}

  \uncover<+->{Related: Radiolab's 
  \wordwikilink{http://www.radiolab.org/story/elements/}{Elements}
  on the Cold War, the Bomb Pulse, and the dating of cell age (33:30).}

\end{frame}

\begin{frame}
  \frametitle{We're still sorting out units:}

  \begin{block}<1->{\wordwikilink{https://en.wikipedia.org/wiki/Proposed\_redefinition\_of\_SI\_base\_units}{Proposed
        2018 revision of SI base units:}}
    \begin{columns}
      \column{0.02\textwidth}
      \column{0.28\textwidth}
      \includegraphics[width=\textwidth]{SI_base_unit.pdf}\\
      \attribution{by Dono/Wikipedia}
      \includegraphics[width=\textwidth]{Relations_between_New_SI_units_definitions.pdf}\\
      \attribution{by Wikipetzi/Wikipedia}
      \column{0.02\textwidth}
      \column{0.68\textwidth}
      \begin{itemize}
      \item<2->
        Now: kilogram is an 
        \wordwikilink{https://en.wikipedia.org/wiki/Kilogram\#International\_prototype\_kilogram}{artifact} 
        in S\`{e}vres, France.
      \item<3->
        Future: 
        Defined by fixing Planck's constant 
        as $6.62606X \times 10^{−34}$
        s$^{-1}\cdot$m$^{2}\cdot$kg.\footnotemark[3]
      \item<5->
        Metre chosen to fix speed of light at 299792458 m$\cdot$s$^{-1}$.        
      \item<6->
        %% \begin{overprint}
        %%   \onslide<1-5| handout:0| trans:0>
        %%   \onslide<6-| handout:1| trans:1>
        \wordwikilink{http://www.radiolab.org/story/kg/}{Radiolab
          piece: $\le$ kg}\\
        \medskip
        \begin{center}
          \includegraphics[width=0.4\textwidth]{MassStandards_024.jpg}  
        \end{center}
        %% \end{overprint}
      \end{itemize}
    \end{columns}

  \end{block}

  \footnotetext[3]<4->{$X$ = still arguing \ldots}

\end{frame}


\begin{frame}
  \frametitle{Turbulence:}

  %% 

  \begin{block}{}
    \begin{columns}
      \column{0.02\textwidth}
      \column{0.40\textwidth}
      \includegraphics[width=\textwidth]{jet_cfd.jpg}    
      \column{0.58\textwidth}
      Big whirls have little whirls\\
      That heed on their velocity, \\
      And little whirls have littler whirls \\
      And so on to viscosity.
      
      \hfill---
      \wordwikilink{https://en.wikipedia.org/wiki/Lewis\_Fry\_Richardson}{Lewis
        Fry Richardson}
    \end{columns}
  \end{block}

  \small

  \begin{itemize}
  \item 
    Image from \wordwikilink{http://www.efluids.com/efluids/gallery/gallery_pages/jet\_cfd\_page.jsp}{here}.
  \item 
    Jonathan Swift (1733): \alertdg{``Big fleas have little fleas
      upon their backs to bite 'em, And little fleas have lesser
      fleas, and so, ad infinitum.''}
    \wordwikilink{https://en.wikipedia.org/wiki/The\_Siphonaptera}{The Siphonaptera.}
  \end{itemize}

\end{frame}


\begin{frame}

  \displaypaper{aragon2008a}{4}

  \begin{block}{}
    \begin{itemize}
    \item 
      Examined the probability pixels a distance $R$ apart share the same luminance.
    \item 
      \wordwikilink{http://www.nature.com/news/2006/060703/full/news060703-17.html}{``Van
        Gogh painted perfect turbulence''}
      by Phillip Ball, July 2006.
    \item 
      Apparently not observed in other famous painter's works
      or when van Gogh was settled.
    \item 
      Oops: Small ranges and natural log used.
    \end{itemize}
  \end{block}

  %% The disturbed artist intuited the deep forms of fluid flow.
  %% http://www.nature.com/news/2006/060703/full/060703-17.html

\end{frame}


\begin{frame}
  \frametitle{Advances in turbulence:}
  
  \begin{block}<+->{
      Kolmogorov, armed only with dimensional analysis and an envelope
      figures this out in 1941:
    }
    $$
    E(k)
    =
    C
    \epsilon^{2/3}
    k^{-5/3}
    $$
    \begin{itemize}
    \item 
      $E(k)$ = energy spectrum function.
    \item 
      $\epsilon$ = rate of energy dissipation.
    \item
      $k = 2\pi/\lambda$ = wavenumber.
    \end{itemize}
  \end{block}

  \begin{block}<+->{}
    \begin{itemize}
    \item<+-> 
      Energy is distributed across all modes,
      decaying with wave number.
    \item<+-> 
      No internal characteristic scale to turbulence.
    \item<+-> 
      Stands up well experimentally and there has been no other advance of similar magnitude.
    \end{itemize}
  \end{block}

\end{frame}

\begin{frame}
  %% \frametitle{\small Okay, okay, okay, ...}
  
  \small
  \begin{block}<+->{ ``The Geometry of Nature'': \wordwikilink{https://en.wikipedia.org/wiki/Fractal}{Fractals}}
    \begin{columns}
      \column{0.06\textwidth}
      \column{0.38\textwidth}
      \includegraphics[width=\textwidth]{Romanesco_Broccoli_detail.jpg}\footnotemark[4]
      \column{0.54\textwidth}
      \begin{itemize}
      \item 
        ``Anomalous'' scaling of lengths, areas, volumes relative to each other.
      \item<+-> 
        The enduring question: how do self-similar geometries form?
      \end{itemize}
      \column{0.02\textwidth}
    \end{columns}
    \begin{itemize}
    \item<+-> 
      \wordwikilink{https://en.wikipedia.org/wiki/Robert\_E.\_Horton}
      {Robert E. Horton}: Self-similarity of river (branching) networks (1945).\cite{horton1945a}
    \item<+->
      \wordwikilink{https://en.wikipedia.org/wiki/Hurst\_exponent}{Harold
        Hurst}---Roughness of time series (1951).\cite{hurst1951a}
    \item<+-> 
      \wordwikilink{https://en.wikipedia.org/wiki/Lewis\_Fry\_Richardson}{Lewis
        Fry Richardson}---Coastlines (1961).
    \item<+->
      \wordwikilink{https://en.wikipedia.org/wiki/Benoit\_Mandelbrot}{Beno\^{i}t
        Mandelbrot}---Introduced the term ``Fractals'' and explored
      them everywhere, 1960s on.\cite{mandelbrot1967a,mandelbrot1977a,mandelbrot1983a}
    \end{itemize}

    \footnotetext[4]<+->{Note to self: Make millions with the ``Fractal Diet''}

  \end{block}
\end{frame}


\section{Cities}

\begin{frame}
  \frametitle{Scaling in Cities:}

  \begin{block}{}
    \displaypaper{bettencourt2007a}{2}
    \begin{itemize}
    \item  Quantified levels of 
      \begin{itemize}
      \item Infrastructure
      \item Wealth
      \item Crime levels
      \item Disease
      \item Energy consumption
      \end{itemize}
      as a function of city size $N$ (population).
    \end{itemize}
  \end{block}

\end{frame}

\begin{frame}
  %% \frametitle{Scaling in Cities:}

  \begin{block}{}
    \includegraphics[width=\textwidth]{bettencourt2007a_fig1and2.pdf}
  \end{block}

\end{frame}

\begin{frame}
  \frametitle{Scaling in Cities:}

  \begin{block}{}
    \includegraphics[width=\textwidth]{bettencourt2007a_tab1.pdf}
  \end{block}

\end{frame}

\begin{frame}
  \frametitle{Scaling in Cities:}

  \begin{block}{Intriguing findings:}
    \begin{itemize}
    \item<1-> 
      Global supply costs scale \alertg{sublinearly} with $N$ ($\beta<1$).
      \begin{itemize}
      \item Returns to scale for infrastructure.
      \end{itemize}
    \item<2->
      Total individual costs scale \alertg{linearly} with $N$ ($\beta=1$)
      \begin{itemize}
      \item Individuals consume similar amounts independent of city size.
      \end{itemize}
    \item<3->
      Social quantities scale \alertg{superlinearly} with $N$ ($\beta>1$)
      \begin{itemize}
      \item Creativity (\# patents), wealth, disease, crime, ...
      \end{itemize}
    \end{itemize}
  \end{block}

  \medskip

  \begin{block}<4->{Density doesn't seem to matter...}
    \begin{itemize}
    \item Surprising given that across the world,
      we observe two orders of magnitude variation
      in area covered by
      \wordwikilink{http://en.wikipedia.org/wiki/Urban\_agglomeration}{agglomerations} 
      of fixed populations.
    \end{itemize}
  \end{block}

\end{frame}

\begin{frame}

  \begin{block}{A possible theoretical explanation?}
    \displaypaper{bettencourt2013a}{4}
    \bigskip
    \#sixthology
  \end{block}

\end{frame}


\begin{frame}

  \begin{block}{Density of public and private facilities:}
    \includegraphics[width=0.49\textwidth]{um2009a_fig1A.pdf}
    \includegraphics[width=0.49\textwidth]{um2009a_fig1B.pdf}
    $$
    \rhofac \propto \rhopop^{\alpha} 
    $$    
    \begin{itemize}
    \item<1->
      \alert{Left plot:} ambulatory hospitals in the U.S.
    \item<1-> 
      \alert{Right plot:} public schools in the U.S.
      %% \item<2->
      %%   Note: possible break in scaling for public schools.
      %%   Transition from $\alpha \simeq 2/3$ to 
      %%   $\alpha = 1$ around $\rhopop \simeq 100$.
    \end{itemize}
  \end{block}

\end{frame}

\section{Money}

\begin{frame}
  
  \href{http://xkcd.com/980/}{
    \includegraphics[width=\textwidth]{money.png}
  }\\
  Explore the original zoomable and  interactive version here: 
  \wordwikilink{http://xkcd.com/980/}{http://xkcd.com/980/}.

\end{frame}

\begin{frame}
  \frametitle{\wordwikilink{http://en.wikipedia.org/wiki/Moore's_law}{Moore's Law:}}

  \begin{block}{}
    \includegraphics[height=0.85\textheight]{transistor_Count_and_Moore_Law.png}
  \end{block}

\end{frame}

\section{Technology}

\begin{frame}
  \small

  \begin{block}{Scaling laws for technology production:}
    \begin{itemize}
    \item<+->
      ``Statistical Basis for Predicting Technological Progress\cite{nagy2013a}''
      Nagy et al., PLoS ONE, 2013.
    \item<+->
      $y_t$ = stuff unit cost;
      $x_t$ = total amount of stuff made.
    \item<+->
      Wright's Law, cost decreases as a power of total stuff made:\cite{wright1936a}
      $$ 
      y_t \propto x_t^{-w}.
      $$
    \item<+-> 
      \wordwikilink{http://en.wikipedia.org/wiki/Moore's_law}{Moore's Law},
      framed as cost decrease 
      connected with doubling of transistor density every two years:\cite{moore1965a}
      $$
      y_t \propto e^{- m t}. 
      $$
    \item<+-> 
      Sahal's observation that Moore's law gives rise to Wright's law if
      stuff production grows exponentially:\cite{sahal1979a}
      $$
      x_t \propto e^{g t}.
      $$
    \item<+-> 
      Sahal + Moore gives Wright with $w = m/g$.
    \end{itemize}
  \end{block}

\end{frame}

\begin{frame}

  \begin{block}{}
    \begin{center}
      \includegraphics[height=0.9\textheight]{nagy2013a_fig3.pdf}
    \end{center}
  \end{block}
  
\end{frame}

\begin{frame}[plain]

  \begin{block}{}
    \includegraphics[height=0.9\textheight]{nagy2013a_fig4.pdf}
  \end{block}
  
\end{frame}

\section{Specialization}

\begin{frame}

  \begin{block}{Scaling of Specialization:}
    ``Scaling of Differentiation in Networks: Nervous Systems, Organisms,
    Ant Colonies, Ecosystems, Businesses, Universities, Cities, Electronic
    Circuits, and \alertr{Legos}''\\
    M. A. Changizi, M. A. McDannald and D. Widders\cite{changizi2002a}\\
    J. Theor. Biol., 2002.\\
    \includegraphics[width=\textwidth]{changizi2002a_fig3}
    \begin{itemize}
    \item 
      \wordwikilink{http://www.wired.com/wiredscience/2012/01/the-mathematics-of-lego/}{Nice 2012 wired.com write-up}
    \end{itemize}

    %% \includegraphics[width=\textwidth]{lego-cities-andrew-sullivan.jpg}
    %% \url{http://andrewsullivan.thedailybeast.com/2012/01/how-lego-cities-are-like-real-cities.html}

  \end{block}

\end{frame}

\begin{frame}

  \begin{block}{$C \sim N^{1/d}$, $d \ge 1$:}
    \begin{itemize}
    \item 
      $C$ = network differentiation = \# node types.
    \item 
      $N$ = network size = \# nodes.
    \item<+->
      $d$ = combinatorial degree.
    \item<+->
      Low $d$: strongly specialized parts.
    \item<+->
      High $d$: strongly combinatorial in nature, parts are reused. 
    \item<+->
      Claim: Natural selection produces high $d$ systems.
    \item<+->
      Claim: Engineering/brains produces low $d$ systems.
    \end{itemize}

  \end{block}

\end{frame}

\begin{frame}

  \begin{block}{}
    \includegraphics[width=\textwidth]{changizi2002a_tab1}
  \end{block}

\end{frame}

\begin{frame}

  \begin{block}{Shell of the nut:}
    \begin{itemize}
    \item<+->
      Scaling is a fundamental feature of complex systems.
    \item<+->
      Basic distinction between isometric and allometric scaling.
    \item<+->
      Powerful envelope-based approach: Dimensional analysis.
    \item<+->
      ``Oh yeah, well that's just dimensional analysis'' said the [insert
      your own adjective] physicist.
    \item<+->
      \alertg{Tricksiness:}
      A wide variety of mechanisms give rise to scalings, 
      both normal and unusual.
    \end{itemize}
  \end{block}

\end{frame}

