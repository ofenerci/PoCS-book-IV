\section{Cascade size}

%% idea

\subsection{Distributed seeds}

  
  

\subsection{Theory}

  
  


\subsection{Comparison to simulations}

  \textbf{Comparison to data}

      
    \includegraphics[width=\textwidth]{gleeson2007a_fig1.pdf}
    
  
  \textbf{Comparison to data}

      
    \includegraphics[width=\textwidth]{gleeson2007a_fig2.pdf}
    
  
\section{Groups}

  \textbf{Extensions}

  
   Assumption of sparse interactions is good
   Degree distribution is (generally) key to a network's function
   Still, random networks don't represent all networks
   Major element missing: \alert{group structure}
  


  \textbf{Group structure---Ramified random networks}
 
   \centering
   \includegraphics[width=0.48\textwidth]{ramifiednetwork}
 
   $p$ = intergroup connection probability\\
   $q$ = intragroup connection probability.
 

  \textbf{Bipartite networks}
 
   \centering
   \includegraphics[height=0.75\textheight]{bipartite}
 
   % boards of directors
   % movies
   % transportation
 
 
 
 

  \textbf{Context distance}
 
   \centering
   \includegraphics[width=1\textwidth]{bipartite2}
  

  \textbf{Generalized affiliation model}
 
   \centering
   \includegraphics[width=1\textwidth]{generalcontext2}
 
   (Blau \& Schwartz, Simmel, Breiger)
 
 
 % \includegraphics[width=0.6\textwidth]{figgroupcascade_good}
 
 %% cascade windows for group based networks


  \textbf{Generalized affiliation model networks with triadic closure}

  
   Connect nodes with probability $\propto \exp^{-\alpha d}$\\
  where\\
  $\alpha$ = homophily parameter\\
  and \\
  $d$ = distance between nodes (height of lowest common ancestor)
  
  $\tau_1$ = intergroup probability of friend-of-friend connection
  
  $\tau_2$ = intragroup probability of friend-of-friend connection
  
 

  \textbf{Cascade windows for group-based networks}

  \includegraphics[width=1\textwidth]{figgroupcascade3}\\


%%  
%% %%   
%%
%% %%   \textbf{Starting cascades with seed groups---group-based networks}
%% 
%%   \centering
%%   \includegraphics[width=0.4\textwidth]{fignetwork_ramify_cw21_3b_noname}\\
%%   \includegraphics[width=0.4\textwidth]{fignetwork_multibip_cw03_3_noname}
%% 
%%   
%% 
%%   
%% %% %%
%% %%   \textbf{Starting cascades with seed groups---group-based networks}
%%   
%%   \begin{tabular}{cc}
%%     \includegraphics[width=0.4\textwidth]{fignetwork_ramify_cw21_5b_noname} &
%%     \includegraphics[width=0.4\textwidth]{fignetwork_ramify_cw22_4b_noname} \\
%%     \includegraphics[width=0.4\textwidth]{fignetwork_multibip_cw03_5_noname} & 
%%     \includegraphics[width=0.4\textwidth]{fignetwork_multibip_cw03_4_noname}
%%   \end{tabular}
%% 

  \textbf{Assortativity in group-based networks}

  \centering

  \includegraphics[width=0.6\textwidth]{fignw_thr_ramify_startprob211_mod2_1_noname}

  
   Very surprising: the most connected nodes aren't always the most influential
   \alert{Assortativity} is the reason
  
  

  \textbf{Social contagion}

  \textbf{Summary}
    

     \alert{`Influential vulnerables'} are key to spread.
     Early adopters are mostly vulnerables.
     Vulnerable nodes important but not necessary.
     Groups may greatly facilitate spread.
     Seems that cascade condition is a global one.
     Most extreme/unexpected cascades occur in highly connected networks 
     `Influentials' are posterior constructs.\\
     Many potential influentials exist.
    
  
  

%% %%   \textbf{Summary}
%% 
%%   
%% %   Cascade initiators not greatly above average.
%% %   Average initiators easier to find and influence.
%% %   Early adopters may be above or below average.
%%   
%% 
%% 
  \textbf{Social contagion}

  \textbf{Implications}
  
  
    Focus on \tc{blue}{the influential vulnerables}.
  
    Create entities that can be transmitted successfully
    through many individuals rather than broadcast from one `influential.'
  
    Only \tc{blue}{simple ideas} can spread by word-of-mouth.\\
    \qquad (Idea of opinion leaders spreads well...)
  
    Want enough individuals who will adopt and display.
  
    Displaying can be \tc{blue}{passive} = free (yo-yo's, fashion),\\
    or \tc{blue}{active} = harder to achieve (political messages).
  
    Entities can be novel or designed to combine with others,
    e.g. block another one.
  
  



%% %%    \textbf{Social Sciences: Threshold models}
                                %
                                %  At time $t+1$, fraction rioting
                                %  = fraction with $\gamma \le \phi_t$.
                                %
                                %  \[ \phi_{t+1} = \int_{0}^{\phi_t} f(\gamma) \dee{\gamma}
                                %  = \left. F(\gamma) \right|_{0}^{\phi_t} = F(\phi_t) \]
                                %
                                %  $\Rightarrow$ Iterative maps of the interval.
                                %

                                %
                                %  %%
%% %%    \textbf{Social Sciences: Threshold models}
                                %
                                %  Distribution of individual thresholds, $f(\gamma)$ $\Rightarrow$ interval maps
                                %  \includegraphics{[],figthreshold2_noname.ps,width=0.45\textwidth}
                                %  \includegraphics{[],figthresholdF2b_noname.ps,width=0.45\textwidth}\\
                                %  $\Rightarrow$ Single stable state model
                                %


                                % 
%% %%   \textbf{Social Sciences---Threshold models}
%% 
%%   \includegraphics[width=0.45\textwidth]{figthreshold3_noname}
%%   \includegraphics[width=0.45\textwidth]{figthresholdF3b_noname}
%% 
%%   Critical mass model
%% 
%% %% 




                                %
  %%
  %%   %%    \textbf{Cascade windows for group-based networks}
                                %
                                %  \centering
                                %  \includegraphics[width=0.65\textwidth]{fignetwork_ud_meanfield03_pdd_noname}
                                %
                                %


                                %  
%% %%   \textbf{Starting cascades with seed groups---random networks}
%% 
%%   \centering
%%   \includegraphics[width=0.6\textwidth]{fignw_thr_ramify_startprob211_mod2_1_noname}
%% 
%% 
%% 
%% 
%% 
%% 




                                % definition of model
                                % figure



%%   \caption{
%%     Examples of networks where (A) cascades are possible
%%     even when no vulnerable cluster of nonzero fractional
%%     size exists, and (C) activation spreads at an
%%     exponential rate also in the absence of a vulnerable
%%     cluster.  In both networks, all nodes have the
%%     same threshold $\phi=1/3$.  
%%     In plot A, the initiator $i_0$ and $i_0$'s 
%%     two neighbors are the only vulnerable nodes
%%     in the network.  The ladder then allows
%%     activation to propagate from left to right, and the cascade
%%     grows at a linear rate.
%%     The network shown in plot C is a renormalized
%%     version of a trivalent Bethe Lattice, which is shown in plot B.
%%     Each node of the Bethe lattice is replaced by two nodes and additional edges
%%     as shown.  Activation in the resulting network will spread providing
%%     both nodes in a single group are initially activated together,
%%     and the rate of spread will be exponential.


