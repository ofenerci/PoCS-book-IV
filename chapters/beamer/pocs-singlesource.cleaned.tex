\section{Single Source}

%% \subsection{History}
%% 
%% %%   \textbf{History in brief}
%%   
%%   \textbf{Seeking optimal universality:}
%%     
%%      Two major real-world branching networks
%%       
%%        \alert{Blood networks}
%%        \alertb{River networks}
%%       
%%      Blood networks argued
%%       to lead to 
%%       $$B \propto M^{\alpha}$$
%%       where
%%       
%%        
%%         $B$ = basal metabolic rate and $M$ = body mass
%%        
%%       $\alpha=2/3$ or $3/4$ or something else...\cite{kleiber1961a,west1997a,banavar1999a,dodds2001d}
%%       
%%      River basins may or may not scale allometrically.
%%     
%%       Recall: Hack's law\cite{hack1957a}
%%       $$\ell \propto a^h$$
%%       If $h>1/2$ then basins elongate.
%%     
%%     
%%   
%%   
%% %% 
%% \subsection{Minimal volume calculation}
%% 
%% %%   \textbf{Geometric argument}
%% 
%%   
%%    
%%     Consider \alert{one source supplying many sinks} in a $d$ dimensional volume
%%   
%%     Material draw by sinks is invariant.
%%    
%%     See network as a bundle of virtual vessels:
%%     %%               
%%               
%%       \begin{center}
%%         \includegraphics[angle=-90,width=0.8\textwidth]{virtualvessels4.pdf}
%%       \end{center}
%%       %%    
%%     \alert{The simplest question}: how does number  of sustainable
%%     sinks $N_{\textnormal{sinks}}$
%%     scale with volume $V$ for the most efficient network design?
%%    
%%     Or: what is highest $\alpha$ for $N_{\textnormal{sinks}} \propto V^{\alpha}$?
%%    
%%     Covered in PoCS CSYS 300: we will recap and refine here.
%%   
%% 
%% %   
%% %    Assume some cap on flow speed of material, $v_{\textnormal{max}}$
%% 
%% %% 
%% %%   \textbf{Geometric argument}
%% 
%%   
%%    Consider families of systems that grow allometrically.
%%    Family = a basic shape $\Omega$ indexed by volume $V$.
%%   \begin{center}
%%     \includegraphics[angle=-90,width=0.8\textwidth]{shapescaling}    
%%   \end{center}
%%   \bigskip
%%    Orient shape to have dimensions $L_1 \times L_2 \times  ... \times L_d$
%%    In 2-d,
%%     $L_1 \propto A^{\gamma_1}$ and $L_2 \propto A^{\gamma_2}$
%%     where $A$ = area.
%%    In general, have $d$ lengths which scale
%%     as $L_i \propto V^{\gamma_i}$.
%%    For above example, width grows faster than
%%     height: $\gamma_1 > \gamma_2$.
%%   
%% 
%% %% 
%% %%   \textbf{Geometric argument}
%% 
%%   \textbf{Some generality:}
%%     
%%      Consider $d$ dimensional spatial regions living in 
%%       $D$ dimensional ambient spaces.  {Notation: \alert{$\volume{V}$}.}
%%      River networks: \alertb{$d=2$ and $D=3$}
%%      Cardiovascular networks: \alertb{$d=3$ and $D=3$}
%%     
%%       \alert{Star-convexity of $\volume{V}$:} A spatial
%%       region is star-convex if from at least one point, all other
%%       points in the region can be reached by travelling along straight lines
%%       while remaining within the region.
%%     
%%       Assume source can be located at a point which has direct line of
%%       sight to all sources.
%%     
%%       We can generalize to a much broader class of shapes...
%%     
%%     
%%   
%% 
%% %% 
%% 
%% %%   \textbf{Geometric argument}
%% 
%%   
%%    Reminder of best and worst configurations
%%     \begin{center}
%%       \includegraphics[angle=-90,width=0.8\textwidth]{efficientnetworks5.pdf}
%%     \end{center}
%%     \bigskip
%%    \alert{Basic idea:}
%%     Minimum volume of material in system $V_{\textnormal{net}} \propto$ sum of distance
%%     from the source to the sinks.
%%    See what this means for sink density $\rho$ if sinks do not
%%     change their feeding habits with overall size.
%%   
%% 
%% 

  \textbf{Geometric argument}

  
   
    Consider \alert{one source} supplying \alert{many sinks} in a \alertb{$d$-dim.} volume
    in a \alertb{$D$-dim.} ambient space.
  
    Assume \alertb{sinks are invariant}.
  
    Assume \alert{$\rho = \rho(V)$}.
   
    See network as a bundle of virtual vessels:
    \begin{center}
              
        
        \includegraphics[angle=-90,width=0.8\textwidth]{virtualvessels4.pdf}
          \end{center}
   
    \alert{Q:} how does the number of sustainable
    sinks $N_{\textnormal{sinks}}$
    scale with volume $V$ for the most efficient network design?
   
    \alert{Or:} what is the highest $\alpha$ for $N_{\textnormal{sinks}} \propto V^{\alpha}$?
  


  \textbf{Geometric argument}

  
   Allometrically growing regions:
%   Family = a basic shape $\Omega$ indexed by volume $V$.
  \begin{center}
    \includegraphics[angle=-90,width=0.8\textwidth]{shapescaling}    
  \end{center}
  \bigskip
%   Orient shape to have dimensions $L_1 \times L_2 \times  ... \times L_d$
%   In 2-d,
%    $L_1 \propto A^{\gamma_1}$ and $L_2 \propto A^{\gamma_2}$
%    where $A$ = area.
   Have $d$ length scales which scale
    as 
    {
      $$
      \alertb{L_i} \propto \alertb{V}^{\alertb{\gamma_i}}
      \mbox{\ where $\gamma_1 + \gamma_2 + \ldots + \gamma_d = 1$.}
      $$
    }
   
    For \alert{isometric} growth, $\gamma_i = 1/d$.
  
    For \alert{allometric} growth, 
    we must have at least two of the $\{\gamma_i\}$ being different
%   For above example, width grows faster than
%    height: $\gamma_1 > \gamma_2$.
  



  \textbf{Geometric argument}

  
   Best and worst configurations (Banavar et al.)
    \begin{center}
      \includegraphics[angle=-90,width=0.8\textwidth]{efficientnetworks5.pdf}
    \end{center}
    \bigskip
   \alert{Rather obviously:}\\
    $\min V_{\textnormal{net}} \propto \sum$
    distances
    from source to sinks.

%   See what this means for sink density $\rho$ if sinks do not
%    change their feeding habits with overall size.
  


  \textbf{Minimal network volume:}

  Real supply networks are close to optimal:

  \includegraphics[width=\textwidth]{gastner2006a_fig1.pdf}

  \bigskip

  {\small (2006)
    Gastner and Newman\cite{gastner2006a}:
    ``Shape and efficiency in spatial distribution networks'' }


  \textbf{Minimal network volume:}

  \textbf{Add one more element:}
    
     Vessel cross-sectional area
      may vary with distance from the source.
    
      Flow rate increases as cross-sectional area decreases.
     e.g., a collection network may
      have vessels tapering as they approach
      the central sink.
    
      Find that vessel volume $v$ must scale
      with vessel length $\ell$ to affect overall
      system scalings.
    
      Consider vessel radius $r \propto (\ell+1)^{-\epsilon}$,
      tapering from $r=r_{\max}$ where $\epsilon \ge 0$.
    
      Gives
      $
      v \propto \ell^{1-2\epsilon}
      $ if $\epsilon < 1/2$
    
      Gives
      $
      v \propto 1 - \ell^{-(2\epsilon-1)} \rightarrow 1$ for large $\ell$
      if $\epsilon > 1/2$
    
      Previously, we looked at $\epsilon=0$ only.
    
  

  \textbf{Minimal network volume:}

  For $0 \le \epsilon < 1/2$, approximate network volume by integral over region:
  $$ 
  \alertb{\min V_{\textnormal{net}}}  \propto 
  \int_{\volume{V}} \alertb{\rho} \, ||\vec{x}||^{1-2\epsilon} \, \dee{\vec{x}} 
  $$
  %%   {
  %%     $$
  %%     \rightarrow 
  %%     \rho V^{1+\gamma_{\max}}
  %%     \int_{\volume{c}} (c_1^{2} u_1^2 + \ldots + c_k^{2} u_k^2 )^{(1-2\epsilon)/2}
  %%     \dee{\vec{u}}
  %%     $$
  %%   }
  {\insertassignmentquestionsoft{02}{2}}
  {
    $$
    \propto
    \alert{ \rho V^{1+\gamma_{\max}(1-2\epsilon)} } 
    $$
  }
  {
    For $\epsilon > 1/2$, find simply that 
    $$
    \alertb{\min V_{\textnormal{net}}}  
    \propto 
    \rho V
    $$
  }
  
  
    So if supply lines can taper fast enough and without
    limit, minimum network volume can be made negligible.
%  
%    \alert{The problem:} must eventually reach a limiting speed
%    or size (e.g., blood velocity and cells).
  

  \textbf{Geometric argument}

  \textbf{For $0 \le \epsilon < 1/2$:}
    
     
      $
      \boxed{\alert{
          \min V_{\textnormal{net}} 
          \propto
          \rho V^{1+\gamma_{\max}(1-2\epsilon)} 
        }}
      $
     
      If scaling is \alertb{isometric}, we have $\gamma_{\max} = 1/d$:
      $$
      \min V_{\textnormal{net/iso}} 
      \propto
      \rho V^{1+(1-2\epsilon)/d}
      $$
     
      If scaling is \alertb{allometric}, we have
      $\gamma_{\max} = \gamma_{\textnormal{allo}} > 1/d$:
      and 
      $$
      \min V_{\textnormal{net/allo}} 
      \propto
      \rho V^{1+(1-2\epsilon)\gamma_{\textnormal{allo}}}
      $$
     
      Isometrically growing volumes 
      \alert{require less network volume} 
      than allometrically growing volumes:
      $$
      \frac{\min V_{\textnormal{net/iso}}}{\min V_{\textnormal{net/allo}}} \rightarrow 0 
      \mbox{\ as $V \rightarrow \infty$}
      $$
        
  

  \textbf{Geometric argument}

  \textbf{For $0 \le \epsilon < 1/2$:}
    
     
      $
      \boxed{\alert{
          \min V_{\textnormal{net}} 
          \propto
          \rho V
        }}
      $
     
      Network volume scaling is now independent 
      of overall shape scaling.
    
  

  \medskip

  \textbf{Limits to scaling}
    
     
      Can argue that $\epsilon$ must effectively be 0
      for real networks over large enough scales.
     
      Limit to how fast material can move,
      and how small material packages can be.
     
      e.g., blood velocity and blood cell size.
    
  


\subsection{Blood networks}

  \textbf{Blood networks}

  
   Velocity at capillaries and 
    aorta approximately constant across body size\cite{weinstein2006a}: 
    $\epsilon = 0$.
   \alert{Material costly} $\Rightarrow$ expect lower optimal bound of 
    $V_{\textnormal{net}} \propto \rho V^{(d+1)/d}$ to be followed closely.
  
    For cardiovascular networks, \alert{$d=D=3$}.
  
    Blood volume scales linearly with blood 
    volume\cite{stahl1967a}, $V_{\textnormal{net}} \propto V$.
  
    Sink density must $\therefore$ decrease as volume increases:
    $$
    \alertb{\rho \propto V^{-1/d}}.
    $$
  
    Density of suppliable sinks \alert{decreases} with organism size.
        



  \textbf{Blood networks}

  
   Then $P$, the rate of overall energy 
    use in $\Omega$, can at most scale with volume as
    $$
    P \propto \rho V 
    {
      \propto \rho \, M
    }
    {
      \propto M^{\, (d-1)/d}
    }
    $$
   
    For $d=3$ dimensional organisms, we have 
    $$\alertb{\boxed{ P \propto M^{\, 2/3}}}$$
   
    Including other constraints may raise scaling exponent
    to a higher, less efficient value.
      


  \textbf{Recap:}

  
   
    The exponent $\alpha = 2/3$ works for all birds and
    mammals up to 10--30 kg
   For mammals $>$ 10--30 kg, maybe we have a new scaling regime
   Economos: limb length break in scaling around 20 kg
   White and Seymour, 2005: unhappy with large herbivore measurements.
Find $\alpha \simeq 0.686 \pm 0.014$
  


%% %%   \textbf{Prefactor:}
%% 
%%   \textbf{Stefan-Boltzmann law:}
%%     
%%     
%%       $$\diff{E}{t} = \sigma S T^4$$
%%       where $S$ is surface and $T$ is temperature.
%%      
%%       Very rough estimate of prefactor based on scaling
%%       of normal mammalian body temperature and surface
%%       area $S$:
%%       $$B \simeq 10^5M^{2/3} \mbox{erg/sec}.$$
%%     
%%       Measured for $M \leq 10$ kg:
%%       $$B=2.57\times 10^5M^{2/3} \mbox{erg/sec}.$$
%%     
%%   
%% 
%% 
\subsection{River networks}

  \textbf{River networks}

  
   View river networks as collection networks.
   Many sources and one sink.
   $\epsilon$?
   Assume $\rho$ is constant over time and $\epsilon=0$:
    $$V_{\textnormal{net}} \propto \rho V^{(d+1)/d} = \mbox{constant} \times V^{\, 3/2} $$
   Network volume grows faster than
    basin `volume' (really area).
   \alert{It's all okay:}\\ 
    Landscapes are $d$=2 surfaces living in $D$=3 dimension.
  
    Streams can grow not just in width but in depth...
  
    If $\epsilon > 0$, $V_{\textnormal{net}}$ will grow more slowly
    but 3/2 appears to be confirmed from real data.
  


%% %%   \textbf{Hack's law}
%% 
%%   
%%    Volume of water in river network can be calculated 
%%     by adding up basin areas
%%    Flows sum in such a way that 
%%     $$ V_{\textnormal{net}} = \sum_{\mbox{\scriptsize all pixels}} a_{\mbox{\scriptsize pixel $i$}} $$
%%    Hack's law again:
%%     $$
%%     \ell \sim a^{\, h}
%%     $$
%%    
%%     Can argue     
%%     $$ V_{\textnormal{net}} \propto V_{\textnormal{basin}}^{1+h} = a_{\textnormal{basin}}^{1+h}$$
%%     where 
%%     $h$ is Hack's exponent.
%%    
%%     $\therefore$ minimal volume calculations gives 
%%     $$
%%     \boxed{
%%       h=1/2
%%     }
%%     $$
%%   
%% 
%% %% 
%% %%   \textbf{Real data:}
%% 
%%   %%     
%%     
%%      Banavar et al.'s approach\cite{banavar1999a} is okay 
%%       because $\rho$ \alertb{really is constant}.
%%      \alert{The irony:} shows optimal basins are isometric
%%      Optimal Hack's law: $\msl \sim a^{h}$ with
%%       $h=1/2$ 
%%      {(Zzzzz)}
%%     
%%     
%%     %%       
%%       \includegraphics[width=\textwidth]{banavar1999fig2.png}\\
%%       {\small From Banavar et al. (1999)\cite{banavar1999a}}
%%     %%   %% %% 
%% %%   \textbf{Even better---prefactors match up:}
%% 
%%   \begin{center}
%%     \includegraphics[width=0.8\textwidth]{figwatervolume02_noname.pdf}
%%   \end{center}
%% 
%% 
