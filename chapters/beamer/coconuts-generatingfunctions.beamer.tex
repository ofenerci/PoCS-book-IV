
\section{Generating\ Functions}

\subsection{Definitions}

\begin{frame}[label=]
  \frametitle{Generatingfunctionology\cite{wilf2006a}}
  
%   \begin{block}{The idea:}
    \begin{itemize}
    \item<1->
      \alert{Idea:} Given a sequence 
      \alert{$a_0, a_1, a_2, \ldots,$}
      \alertb{associate} each element with a 
      distinct function or other mathematical object.
    \item<2->
      Well-chosen functions allow us 
      to manipulate sequences and retrieve
      sequence elements.
    \end{itemize}
%  \end{block}

  \begin{block}<3->{Definition:}
    \begin{itemize}
    \item<3->
      The \alert{generating function} \alertb{(g.f.)} for a sequence 
      $\{a_n\}$
      is
      $$
      \alertb{
      F(x) = \sum_{n=0}^{\infty} a_n x^n.
      }
      $$
    \item<4->
      Roughly: transforms a vector in $R^\infty$
      into a function defined on $R^{1}$.
    \item<5->
      Related to Fourier, Laplace, Mellin, \ldots
    \end{itemize}
  \end{block}

\end{frame}

\begin{frame}[label=]
  \frametitle{Simple example}
  
  \begin{block}{Rolling dice and flipping coins:}
    \begin{itemize}
    \item<1-> 
      $p_k^{(\Box)} = \Prob(\mbox{throwing a} \ k) = 1/6$ where $k = 1, 2, \ldots, 6$.
      $$
      F^{(\Box)}(x) 
      = 
      \sum_{k=1}^{6} p_k^{(\Box)} x^k 
      = 
      \frac{1}{6} (x + x^2 + x^3 + x^4 + x^5 + x^6).
      $$
    \item<2-> 
      $p_0^{\textnormal{(coin)}} = \Prob({\textnormal{head}}) = 1/2$, 
      $p_1^{\textnormal{(coin)}} = \Prob({\textnormal{tail}}) = 1/2$.
      $$
      F^{\textnormal{(coin)}}(x) 
      = 
      p_0^{\textnormal{(coin)}} x^0 + p_1^{\textnormal{(coin)}} x^1
      = 
      \frac{1}{2} (1 + x).
      $$
     \item<3-> 
     A generating function for a probability distribution is
      called a \alertb{Probability Generating Function (p.g.f.)}.
    \item<4-> 
      We'll come back to these simple examples
      as we derive various delicious properties 
      of generating functions.
    \end{itemize}
  \end{block}

\end{frame}

\begin{frame}[label=]
  \frametitle{Example}

    \begin{itemize}
    \item<1->
      Take a degree distribution with exponential decay:
      $$ \alert{P_k = c e^{-\lambda k}} $$
      where geometricsumfully, we have $c = 1 - e^{-\lambda}$
    \item<2->
      The generating function for this distribution is
      $$ 
      F(x)
      =
      \sum_{k=0}^\infty
      P_k x^k
      \uncover<3->{
      =
      \sum_{k=0}^\infty
      c e^{-\lambda k} x^k
      }
      \uncover<4->{
      =
      \frac{c}{1-xe^{-\lambda}}.
      }
      $$
    \item<5->
      Notice that $F(1) = c/(1-e^{-\lambda}) = 1$.
    \item<6->
      For probability distributions, we must
      always have \alert{$F(1)=1$} since
      $$
      F(1) 
      = 
      \sum_{k=0}^\infty P_k 1^k 
      \uncover<7->{
        =
        \sum_{k=0}^\infty P_k
      }
      \uncover<8->{
        =
        1.
      }
      $$
    \item<8->
      Check die and coin p.g.f.'s.
    \end{itemize}
  
\end{frame}

\subsection{Basic\ Properties}

\begin{frame}[label=]
  \frametitle{Properties of probability generating functions}

  \small

  \begin{itemize}
  \item<1->
    Average degree:
    $$
    \tavg{k} 
    = 
    \sum_{k=0}^\infty k P_k
    \uncover<2->{
      =
      \left. 
        \sum_{k=0}^\infty k P_k x^{k-1} 
      \right|_{x=1}
    }
    $$
    $$
    \uncover<3->{
      =
      \left. 
        \diff{}{x} F(x)
      \right|_{x=1}
    }
    \uncover<4->{
      \alert{
        =
        F'(1)
      }
    }
    $$
  \item<5-> In general, many calculations become simple, if a little abstract.
  \item<6-> For our exponential example:
    $$
    F'(x) = 
    \frac{(1-e^{-\lambda}) e^{-\lambda}}
    {(1-xe^{-\lambda})^2}.
    $$
  \item<7->
    $$
    \mbox{So:} \ \
    \tavg{k} = F'(1) =     
    \frac{e^{-\lambda}}
    {(1-e^{-\lambda})}.
    $$
    \item<8->
      Check for die and coin p.g.f.'s.
  \end{itemize}

\end{frame}

\begin{frame}[label=]
  \frametitle{Properties of generating functions}

  \small
  \begin{block}{Useful pieces for probability distributions:}
  \begin{itemize}
  \item<2->
    Normalization:
    $$ \alert{F(1) = 1} $$
  \item<3->
    First moment:
    $$ \alertb{\tavg{k} = F'(1)} $$
  \item<4->
    Higher moments:
    $$ 
    \alert{
    \tavg{k^n} = 
    \left.
    \left(
      x \diff{}{x}
    \right)^n
    F(x)
    \right|_{x=1}
    }
    $$
  \item<5->
    $k$th element of sequence (general):
    $$
    \alertb{
    \left.
      P_k
      =
      \frac{1}{k!}
      \diff{^{k}}{x^k} F(x)
    \right|_{x=0}
    }
    $$
  \end{itemize}
  \end{block}

\end{frame}

% up to page R-e

\subsection{Giant\ Component\ Condition}

\begin{frame}[label=]
  \frametitle{Edge-degree distribution}

  \begin{itemize}
  \item<1-> Recall our condition for
    a giant component:
    $$
    \alert{
      \tavg{k}_R = 
      \frac{\tavg{k^2} - \tavg{k}}
      {\tavg{k}}
      > 1.
    }
    $$
  \item<2-> Let's re-express our condition in
    terms of generating functions.
  \item<3-> We first need the g.f.\ for \alert{$R_k$}.
  \item<4-> We'll now use this notation:
    \begin{itemize}
    \item[]<5->  \alert{$F_P (x)$} is the g.f.\ for \alert{$P_k$}.
    \item[]<6->  \alertb{$F_R (x)$} is the g.f.\ for \alertb{$R_k$}.
    \end{itemize}
  \item<7->
    Giant component condition in terms of g.f. is:
    $$
    \alert{\tavg{k}_R = F'_{R}(1) > 1}.
    $$
  \item<8->
    Now find how $F_R$ is related to $F_P$\ldots
  \end{itemize}
\end{frame}

\begin{frame}[label=]
  \frametitle{Edge-degree distribution}

  \begin{itemize}
  \item<1->
    We have 
    $$
    F_R(x)
    =
    \sum_{k=0}^{\infty}
    \alert{R_k}
    x^k
    \uncover<2->{
      =
      \sum_{k=0}^{\infty}
      \alert{
      \frac{(k+1)P_{k+1}}
      {\tavg{k}}
      }
      x^k.
    }
    $$
    \uncover<3->{
      Shift index to $j=k+1$ and pull out $\frac{1}{\tavg{k}}$:
    }
    $$
    \uncover<4->{
      F_R(x)
      =
      \frac{1}{\tavg{k}}
      \sum_{j=1}^{\infty}
      j P_{j} x^{j-1}
    }
    \uncover<5->{
      =
      \frac{1}{\tavg{k}}
      \sum_{j=1}^{\infty}
      P_{j} 
      \alertb{\diff{}{x}}
      \alertb{x^{j}}
    }
    $$
    $$
    \uncover<6->{
      = 
    \frac{1}{\tavg{k}}
    \alertb{\diff{}{x}}
    \sum_{j=1}^{\infty}
    P_{j} \alertb{x^{j}}
    }
    \uncover<7->{
      = 
    \frac{1}{\tavg{k}}
    \alertb{\diff{}{x}}
    \left(
    \alertb{F_P}(x) - \alert{P_0}
    \right)
      }
    \uncover<8->{
      = 
    \frac{1}{\tavg{k}}
    \alertb{F'_P}(x).
      }
    $$
    \uncover<9->{
    Finally, since $\tavg{k} = F'_P(1)$,
    $$
    \boxed{
      \alert{
        F_R(x) = \frac{F'_P(x)}{F'_P(1)}
      }
    }
    $$
    }
  \end{itemize}

\end{frame}


%% \begin{frame}[label=]
%%   \frametitle{Edge-degree distribution}
%% 
%%   \begin{itemize}
%%   \item<1->
%%     We have 
%%     $$
%%     F_R(x)
%%     =
%%     \sum_{k=0}^{\infty}
%%     \alert{R_k}
%%     x^k
%%     \uncover<2->{
%%       =
%%       \sum_{k=0}^{\infty}
%%       \alert{
%%       \frac{(k+1)P_{k+1}}
%%       {\tavg{k}}
%%       }
%%       x^k.
%%     }
%%     $$
%%     \uncover<3->{
%%       Shift index to $j=k+1$ and pull out $\frac{1}{\tavg{k}}$:
%%     }
%%     $$
%%     \uncover<4->{
%%       F_R(x)
%%       =
%%       \frac{1}{\tavg{k}}
%%       \sum_{j=1}^{\infty}
%%       j P_{j} x^{j-1}
%%     }
%%     \uncover<5->{
%%       =
%%       \frac{1}{\tavg{k}}
%%       \sum_{\alert{j=0}}^{\infty}
%%       P_{j} \alertb{\, j \, x^{j-1}}
%%     }
%%     $$
%%     $$
%%     \uncover<6->{
%%     =
%%     \frac{1}{\tavg{k}}
%%     \sum_{j=0}^{\infty}
%%     \alertb{\diff{}{x}}
%%     P_{j} \alertb{x^{j}}
%%   }
%%     \uncover<7->{
%%       = 
%%     \frac{1}{\tavg{k}}
%%     \alertb{\diff{}{x}}
%%     \sum_{j=0}^{\infty}
%%     P_{j} \alertb{x^{j}}
%%     }
%%     \uncover<8->{
%%       = 
%%     \frac{1}{\tavg{k}}
%%     \alertb{F'_P}(x).
%%       }
%%     $$
%%     \uncover<9->{
%%     Finally, since $\tavg{k} = F'_P(1)$,
%%     $$
%%     \boxed{
%%       \alert{
%%         F_R(x) = \frac{F'_P(x)}{F'_P(1)}
%%       }
%%     }
%%     $$
%%     }
%% %    \begin{overprint}
%% %      \onslide<1-3| handout:0 | trans: 0>
%%  %   F_R(x)
%% %    =
%% %    \frac{1}{\tavg{k}}
%% %%    \sum_{j=1}^{\infty}
%% %    \sum_{\alert{j=}\alertb{\cancel{1}} \raisebox{-2pt}{\scriptsize\alert{0}}}^{\infty}
%% %    j P_{j}
%% %    $$
%% %    \end{overprint}
%%   \end{itemize}
%% 
%% \end{frame}

\begin{frame}[label=]
  \frametitle{Edge-degree distribution}

  \begin{itemize}
  \item<1-> Recall giant component condition is
    $\tavg{k}_R = F'_R(1) > 1$.
  \item<2->
    Since we have $F_R(x) = F'_P(x)/F'_P(1)$,
    \uncover<3->{
    $$
    F'_R(x)
    =
    \frac{
      F''_P(x)
    }
    {
      F'_P(1).
    }
    $$
  }
  \item<4->
    Setting $x=1$, our condition becomes
    $$
    \boxed{
      \alert{
        \frac{
          F''_P(1)
        }
        {
          F'_P(1)
        }
        > 1
      }}
    $$
  \end{itemize}

\end{frame}

\subsection{Component\ sizes}

\begin{frame}[label=]
  \frametitle{Size distributions}
  
    To figure out the \alert{size of the largest component}
    ($S_1$), we need
    more resolution on component sizes.

    \begin{block}<2->{Definitions:}
      \begin{itemize}
      \item<2-> 
        \alert{$\pi_n$} = probability that a random node
        belongs to a finite component of size $n < \infty$.
      \item<3-> 
        \alert{$\rho_n$} = probability 
        that a random end of a random link leads to 
        a finite subcomponent of size $n < \infty$.
      \end{itemize}
    \end{block}
    \begin{block}<4->{Local-global connection:}
      \alert{
        $$
        P_k, R_k \Leftrightarrow \pi_n, \rho_n
        $$
        $$
        \mbox{neighbors}
        \Leftrightarrow
        \mbox{components}
        $$
      }
    \end{block}
  
\end{frame}

\begin{frame}[label=]
  \frametitle{Connecting probabilities:}

  \includegraphics[width=\textwidth]{randomnetworks-pi-rho-P-tp-5.pdf}

  \begin{itemize}
  \item 
    Markov property of random networks connects 
    $\pi_n$, $\rho_n$, and $P_k$.
  \end{itemize}

\end{frame}

\begin{frame}[label=]
  \frametitle{Connecting probabilities:}

  \includegraphics[width=\textwidth]{randomnetworks-rho-R-tp-5.pdf}

  \begin{itemize}
  \item 
    Markov property of random networks connects $\rho_n$ and $R_k$.
  \end{itemize}

\end{frame}

\begin{frame}
  \frametitle{Size distributions}
  
  \begin{block}<1->{G.f.'s for component size distributions:}
    \begin{itemize}
    \item<2->
      $$ 
      F_{\pi}(x) = \sum_{n=0}^{\infty} \pi_n x^n 
      \mbox{\ and \ }
      F_{\rho}(x) = \sum_{n=0}^{\infty} \rho_n x^n $$
    \end{itemize}
  \end{block}

  \begin{block}<3->{The largest component:}
    \begin{itemize}
    \item<3->
      \alertb{Subtle key:} $F_\pi(1)$ is the probability
      that a node belongs to a \alert{finite} component.
    \item<4->
      Therefore: $S_1 = 1 - F_\pi(1)$.
    \end{itemize}
  \end{block}
  \begin{block}<5->{Our mission, which we accept:}
    \begin{itemize}
    \item<5->
      Determine and connect the four generating functions
      $$
      F_P,
      F_R,
      F_\pi,
      \mbox{\, and\, }
      F_\rho.
      $$
    \end{itemize}
  \end{block}
\end{frame}

\subsection{Useful\ results}

\begin{frame}[label=rn-sneakyresult1]
  \frametitle{Useful results we'll need for g.f.'s}
  
  \begin{block}<1->{Sneaky Result 1:}
    \begin{itemize}
    \item<2-> 
      Consider two random variables 
      \alert{$U$} and \alert{$V$} whose
      values may be $0,1,2,\ldots$
    \item<3-> 
      Write probability
      distributions as \alert{$U_k$ and $V_k$}
      and g.f.'s as \alertb{$F_{U}$ and $F_{V}$}.
    \item<4->
      \alertb{SR1}: If a third random variable is defined as
      $$
      \alert{W = \sum_{i=1}^{U} V^{(i)}}
      \
      \mbox{with each}
      \
      V^{(i)}
      \stackrel{d}{=}
      V
      $$
      \uncover<5->{
      then
      $$
      \alertb{
        \boxed{
          F_{W}(x) 
          = 
          F_{U} 
          \left(
            F_{V}(x)
          \right)
        }
      }
      $$
    }
    \end{itemize}
    
  \end{block}
\end{frame}

\begin{frame}[label=]
  \frametitle{Proof of SR1:}

  \uncover<1->{
  Write probability that variable $W$ has value $k$ as \alert{$W_k$}.
}
  \uncover<2->{
  $$
  W_k
  =
  \sum_{j=0}^{\infty} U_j \times%
  \mbox{Pr(sum of $j$ draws of variable $V$ = $k$)}
  $$
  }
  \uncover<3->{
    $$
    =\sum_{j=0}^{\infty} U_j 
    \sum_{\stackrel{\{i_1, i_2, \ldots, i_j \} |\ }{\alert{i_1+i_2+\ldots+i_j = k}}} 
    V_{i_1} V_{i_2} \cdots V_{i_j}
    $$
  }
  $$
  \uncover<4->{
    \therefore
    F_W(x) = \sum_{k=0}^\infty W_k x^k
  }
  \uncover<5->{
    =
    \sum_{k=0}^\infty
    \sum_{j=0}^{\infty} U_j 
    \sum_{\stackrel{\{i_1, i_2, \ldots, i_j \} |\ }{{i_1+i_2+\ldots+i_j = k}}} 
    V_{i_1} V_{i_2} \cdots V_{i_j}
    x^k
  }
  $$  
  $$
  \uncover<6->{
    =
    \sum_{\alert{j}=0}^{\infty} 
    \alert{U_j}
    \sum_{\alert{k}=0}^\infty
  }
  \uncover<7->{
    \sum_{\stackrel{\{i_1, i_2, \ldots, i_j \} |\ }{{i_1+i_2+\ldots+i_j = k}}} 
    V_{i_1} \alertb{x^{i_1}}
    V_{i_2} \alertb{x^{i_2}}
    \cdots 
    V_{i_j} \alertb{x^{i_j}}
  }
  $$  
\end{frame}

%% insert piece to help with understanding next step

\begin{frame}[label=]
  \frametitle{Proof of SR1:}

  With some concentration, observe:
  \begin{overprint}
  \onslide<1| handout: 0 | trans: 0>
  $$
  F_W(x) = 
  \sum_{\alert{j}=0}^{\infty} 
  \alert{U_j}
    \sum_{\alert{k}=0}^\infty
    \underbrace{
    \sum_{\stackrel{\{i_1, i_2, \ldots, i_j \} |\ }{{i_1+i_2+\ldots+i_j = k}}} 
    V_{i_1} \alertb{x^{i_1}}
    V_{i_2} \alertb{x^{i_2}}
    \cdots 
    V_{i_j} \alertb{x^{i_j}}
  }_{
    \mbox{$x^k$ piece of $\left( \sum_{i'=0}^\infty V_{i'}x^{i'}\right)^j$}
  }
  $$
  \onslide<2-| handout: 1 | trans: 1>
  $$
  F_W(x) = 
  \sum_{\alert{j}=0}^{\infty} 
  \alert{U_j}
  \underbrace{
    \sum_{\alert{k}=0}^\infty
    \underbrace{
      \sum_{\stackrel{\{i_1, i_2, \ldots, i_j \} |\ }{{i_1+i_2+\ldots+i_j = k}}} 
      V_{i_1} \alertb{x^{i_1}}
      V_{i_2} \alertb{x^{i_2}}
      \cdots 
      V_{i_j} \alertb{x^{i_j}}
    }_{ % first underbrace
      \mbox{$x^k$ piece of $\left( \sum_{i'=0}^\infty V_{i'}x^{i'}\right)^j$}
    }
  }_{ % second underbrace
    \mbox{$\left( \sum_{i'=0}^{\infty} V_{i'}x^{i'}\right)^j = \left(F_V(x) \right)^j$ }
  } 
  $$
  \end{overprint}
  \uncover<3->{
    $$
    =
  \sum_{\alert{j}=0}^{\infty} 
  \alert{U_j}
  \left( F_V(x) \right)^j
    $$
  }
  $$
  \uncover<4->{
    =
    F_U \left( F_V(x) \right)
  }
  \uncover<5->{
  \alert{\faCheck}
  }
  $$

\end{frame}

\begin{frame}[label=rn-sneakyresult2]
  \frametitle{Useful results we'll need for g.f.'s}

  \begin{block}<1->{Sneaky Result 2:}
    \begin{itemize}
    \item<2->
      Start with a random variable $\alert{U}$ with
      distribution $\alert{U_k}$ ($k=0,1,2,\dots$)
    \item<3-> 
      \alertb{SR2}: If a second random variable
      is defined as 
      $$
      \alert{ V = U + 1}
      \uncover<4->{
        \mbox{\ \ then \ }
        \alertb{
        \boxed{F_{V}(x) = xF_{U}(x)}
        }
      }
      $$
    \item<5->
      \alert{Reason:} $V_{k} = U_{k-1}$ for $k \ge 1$ and $V_{0}=0$.
    \item<6->
      $$
      \therefore 
      F_{V}(x)
      = 
      \sum_{k=0}^{\infty} V_k x^k
      \uncover<7->{
        =
        \sum_{\alert{k=1}}^{\infty} \alert{U_{k-1}} x^k
      }
      $$
      $$
      \uncover<8->{
        =
        x \sum_{\alert{j=0}}^{\infty} \alert{U_{j}} x^j
      }
      \uncover<9->{
        =
        x F_U(x).
      }
      \uncover<10->{
        \alert{\faCheck}
      }
      $$
    \end{itemize}
  \end{block}

\end{frame}

\begin{frame}
  \frametitle{Useful results we'll need for g.f.'s}

  \begin{block}<1->{Generalization of SR2:}
    \begin{itemize}
    \item<2->
    (1) If $V=U+i$ then
    $$
    \alert{F_V(x) = x^i F_U(x).}
    $$
    \item<3->
    (2) If $V=U-i$ then
     $$
%%     \alert{
%%       F_V(x) = x^{-i} 
%%       \left(
%%         F_U(x) 
%%         - U_0 - U_1 x - \ldots - U_{i-1} x^{i-1}
%%       \right)
%%     }
     \alert{
       F_V(x) = x^{-i} F_U(x)
     }
     $$
    \uncover<4->{
      $$
      =
      x^{-i} 
      \sum_{{k=0}}^{\infty} {U_{k}} x^k
      $$
     }
    \end{itemize}
  \end{block}

\end{frame}

\subsection{Size\ of\ the\ Giant\ Component}

\begin{frame}[label=]
  \frametitle{Connecting generating functions:}

  \begin{itemize}
  \item<1->
    \alert{Goal:} figure out forms of 
    the component generating functions, $F_\pi$ and $F_\rho$.
  \end{itemize}

  \includegraphics[width=.9\textwidth]{randomnetworks-pi-rho-P-tp-5.pdf}

  \begin{itemize}
  \item 
    Relate $\pi_n$ to $P_k$ and $\rho_n$ through
    one step of recursion.
  \end{itemize}

\end{frame}

\begin{frame}[label=]
  \frametitle{Connecting generating functions:}

  \begin{itemize}
  \item<1->
    \alert{$\pi_n$} = probability that a random node
    belongs to a finite component of size $n$
    \uncover<2->{
      $$
      = \sum_{k=0}^{\infty}
      P_k \times
      \mbox{Pr}
      \left(
        \begin{array}{l}
          \mbox{sum of sizes of subcomponents} \\
          \mbox{at end of $k$ random links} = n-1 \\
        \end{array}
      \right)
      $$
    }
  \item<3->
    $$
    \mbox{Therefore: \ \ }
    \boxed{
      F_{\pi}(x)
      =
      \uncover<5->{
        \underbrace{x}_{
          %%          \hyperlink{rn-sneakyresult2\ifslides{<4>}}{\beamerbutton{SR2}}
          \ifthenelse{\boolean{flatswitch}}
          {\protect\hyperlink{rn-sneakyresult2}{\beamerbutton{SR2}}} %% flat slides
          {\protect\hyperlink{rn-sneakyresult2<4>}{\beamerbutton{SR2}}} %% normal slides
        }
      }
      \uncover<4->{
        \underbrace{
          F_{P}
          \left(
            F_{\rho} (x)
          \right)
        }_{
          %%          \hyperlink{rn-sneakyresult1\ifslides{<5>}}{\beamerbutton{SR1}}
          \ifthenelse{\boolean{flatswitch}}
          {\protect\hyperlink{rn-sneakyresult1}{\beamerbutton{SR1}}} %% flat slides
          {\protect\hyperlink{rn-sneakyresult1<5>}{\beamerbutton{SR1}}} %% normal slides
        }
      }
    }
    $$
  \item<6->
    Extra factor of $x$ accounts for random node itself.
  \end{itemize}

\end{frame}

\begin{frame}[label=]
  \frametitle{Connecting generating functions:}

  \includegraphics[width=\textwidth]{randomnetworks-rho-R-tp-5.pdf}

  \begin{itemize}
  \item 
    Relate $\rho_n$ to $R_k$ and $\rho_n$ through
    one step of recursion.
  \end{itemize}

\end{frame}

\begin{frame}[label=]
  \frametitle{Connecting generating functions:}
  
  \begin{itemize}
  \item<1-> \alert{$\rho_n$}
    = probability that 
    a random link leads to 
    a finite subcomponent of size $n$.
  \item<2-> Invoke one step of recursion: \\
    \alert{$\rho_n$}
    = probability that in following
    a random edge, the outgoing
    edges of the node reached
    lead to finite subcomponents of combined size $n-1$,
    \uncover<3->{
      $$
      =
      \sum_{k=0}^\infty R_k \times
      \mbox{Pr}
      \left(
        \begin{array}{l}
          \mbox{sum of sizes of subcomponents} \\
          \mbox{at end of $k$ random links} = n-1 \\
        \end{array}
      \right)
      $$
    }
  \item<4->
    $$
    \mbox{Therefore: \ \ }
    \boxed{
      F_{\rho}(x)
      =
      \uncover<6->{
        \underbrace{x}_{
%%          \hyperlink{rn-sneakyresult2\ifslides{<4>}}{\beamerbutton{SR2}}
          \ifthenelse{\boolean{flatswitch}}
          {\protect\hyperlink{rn-sneakyresult2}{\beamerbutton{SR2}}} %% flat slides
          {\protect\hyperlink{rn-sneakyresult2<4>}{\beamerbutton{SR2}}} %% normal slides
        }
        }
      \uncover<5->{
        \underbrace{
          F_{R}
          \left(
            F_{\rho} (x)
          \right)
        }_{
%%          \hyperlink{rn-sneakyresult1\ifslides{<5>}}{\beamerbutton{SR1}}
          \ifthenelse{\boolean{flatswitch}}
          {\protect\hyperlink{rn-sneakyresult1}{\beamerbutton{SR1}}} %% flat slides
          {\protect\hyperlink{rn-sneakyresult1<5>}{\beamerbutton{SR1}}} %% normal slides
        }
      }
    }
    $$
  \item<7->
    Again, extra factor of $x$ accounts for random node itself.
  \end{itemize}

\end{frame}

\begin{frame}[label=]
  \frametitle{Connecting generating functions:}

  \begin{itemize}
  \item We now have two functional equations
    connecting our generating functions:
    $$
    \alertb{
      F_{\pi}(x)
      =
      x F_{P}
      \left(
        F_{\rho} (x)
      \right)
    }
    \mbox{\ \  and \ \ }
    \alertb{
    F_{\rho}(x)
    =
    x F_{R}
    \left(
      F_{\rho} (x)
    \right)
    }
    $$
  \item<2->
    Taking stock: We know $F_P(x)$ and $F_R(x)=F'_P(x)/F'_P(1)$.
  \item<3->
    We first untangle the \alertb{second equation} to find $F_\rho$
  \item<4->
    We can do this because it \alert{only involves} $F_\rho$ and $F_R$.
  \item<5->
    The first equation then immediately gives us $F_\pi$ in
    terms of $F_\rho$ and $F_R$.
  \end{itemize}

\end{frame}

\begin{frame}[label=]
  \frametitle{Component sizes}
  
  \begin{itemize}
  \item<1-> 
    Remembering vaguely what we are doing:\\
    \medskip
    \uncover<2->{
    Finding $F_\pi$ to obtain the \alert{fractional size of the largest component} $S_1 = 1 - F_\pi(1)$.
    }
  \item<3-> Set $x=1$ in our two equations:
    \uncover<4->{
    $$
    \alertb{
      F_{\pi}(1)
      =
      F_{P}
      \left(
        F_{\rho} (1)
      \right)
    }
    \mbox{\ \  and \ \ }
    \alertb{
    F_{\rho}(1)
    =
    F_{R}
    \left(
      F_{\rho} (1)
    \right)
    }
    $$
    }
  \item<5->
    Solve second equation numerically for $F_{\rho}(1)$.
  \item<6->
    Plug $F_{\rho}(1)$ into first equation to
    obtain $F_{\pi}(1)$.
  \end{itemize}

\end{frame}

\begin{frame}[label=]
  \frametitle{Component sizes}

  \alert{Example}: \alertb{Standard random graphs}.
  \begin{itemize}
  \item<1->
    $
    \mbox{We can show \ }
    F_P(x) = e^{-\avg{k}(1-x)}
    $
        \uncover<2->{
    $$
    \therefore
    F_R(x) = F'_P(x)/F'_P(1) 
    \uncover<3->{
      = e^{-\tavg{k}(1-x)}/e^{-\tavg{k}(1-x')}|_{x'=1}
    }
    $$
    }
    $$
    \uncover<4->{
      = e^{-\tavg{k}(1-x)}
    }
    \uncover<5->{
      \alert{= F_P(x)} \qquad \alertb{\mbox{...aha!}}
    }
    $$
    \item<6->
      RHS's of our two equations are the same.
    \item<7-> 
      So $\alert{F_\pi(x) = F_\rho(x)} = x F_R(F_\rho(x)) = x F_R(F_\pi(x))$
    \item<8->
      Consistent with how our dirty (but wrong) trick worked earlier...
    \item<9->
      $\pi_n = \rho_n$ just as $P_k = R_k$.
  \end{itemize}

\end{frame}


\begin{frame}[label=]
  \frametitle{Component sizes}

  \begin{itemize}
  \item<1-> 
    We are down to\\
    $
    F_\pi(x) = x F_R(F_\pi(x))
    $
    and
    $
    F_R(x) = e^{-\avg{k}(1-x)}
    $.
  \item<2->
    $$
    \therefore
    F_\pi(x) = x e^{-\avg{k}(1-F_\pi(x))}
    $$
  \item<3-> We're first after $S_1 = 1 - F_\pi(1)$ so set $x=1$
    and replace $F_\pi(1)$ by $1-S_1$:
    \begin{columns}
      \column{0.5\textwidth}
      \uncover<4->{
        $$ 1 - S_1  = e^{-\avg{k}S_1} $$
      }
      \uncover<4->{
        $$ \mbox{Or:\ } \tavg{k} = \frac{1}{S_1} \ln \frac{1}{1-S_1} $$
      }
      \column{0.4\textwidth}
      \begin{overprint}
        \onslide<1-3 | handout: 0 | trans: 0>
        \onslide<4- | handout: 1 | trans: 1>
        \includegraphics[width=\textwidth,angle=-90]{figphastrans3}
      \end{overprint}
      \column{0.1\textwidth}
    \end{columns}
  \item<5-> Just as we found with our dirty trick \ldots
  \item<6-> Again, we (usually) have to resort to numerics \ldots
  \end{itemize}

\end{frame}

\subsection{Average\ Component\ Size}

\begin{frame}[label=]
  \frametitle{Average component size}

  \begin{itemize}
  \item<1-> Next: find \alert{average size} of finite components $\tavg{n}$.
  \item<2-> 
    Using standard G.F. result:
    $
    \alertb{\tavg{n} = F'_\pi(1)}.
    $
  \item<3-> 
    Try to avoid finding $F_\pi(x)$...
  \item<4-> 
    Starting from 
    $
    F_{\pi}(x)
    =
    x
    F_{P}
    \left(
      F_{\rho} (x)
    \right)
    $, 
    we differentiate:
    $$
    F'_{\pi}(x)
    =
    F_{P}
    \left(
      F_{\rho} (x)
    \right)
    +    
    x
    F'_{\rho} (x)
    F'_{P}
    \left(
      F_{\rho} (x)
    \right)
    $$
  \item<5-> 
    While
    $
    F_{\rho}(x)
    =
    x
    F_{R}
    \left(
      F_{\rho} (x)
    \right)
    $
    gives
    $$
    F'_{\rho}(x)
    =
    F_{R}
    \left(
      F_{\rho} (x)
    \right)
    +   
    x
    F'_{\rho} (x)
    F'_{R}
    \left(
      F_{\rho} (x)
    \right)
    $$
  \item<6->
    Now set $x=1$ in both equations.
  \item<7->
    We solve the second equation for $F'_\rho(1)$
    (we must already have $F_\rho(1)$).
  \item<8->
    Plug $F'_\rho(1)$ and $F_\rho(1)$ into first equation to find $F'_\pi(1)$.
  \end{itemize}

\end{frame}
  
\begin{frame}[label=]
  \frametitle{Average component size}

    \alert{Example}: \alertb{Standard random graphs}.
    \begin{itemize}
    \item<2->
      Use fact that $F_P=F_R$ and $F_\pi=F_\rho$.
    \item<3->
      Two differentiated equations reduce to only one:
    $$
    F'_{\pi}(x)
    =
    F_{P}
    \left(
      F_{\pi} (x)
    \right)
    +
    x
    F'_{\pi} (x)
    F'_{P}
    \left(
      F_{\pi} (x)
    \right)
    $$
  \uncover<4->{
    $$
    \mbox{Rearrange: \ \ }
    F'_{\pi}(x)
    =
    \frac{
      F_{P}
      \left(
        F_{\pi} (x)
      \right)
    }
    {
      1 -
      x
      F'_{P}
      \left(
        F_{\pi} (x)
      \right)
    }
    $$
    }
    \item<5->
      Simplify denominator using $F'_P(x) = \tavg{k} F_P(x)$
    \item<6->
      Replace $F_P(F_\pi(x))$ using $F_\pi(x) = x F_P(F_\pi(x))$.
    \item<7->
      Set $x=1$ and replace $F_\pi(1)$ with $1-S_1$.
    \uncover<8->{
      $$
      \mbox{End result:\ \ }
      \tavg{n} 
      = 
      F'_{\pi}(1)
      =
      \frac{
        \left(
          1-S_1
        \right)
      }
      {
        1 -
        \tavg{k}(1-S_1)
      }
      $$
    }
    \end{itemize}
    
\end{frame}
  
\begin{frame}[label=]
  \frametitle{Average component size}

  \begin{itemize}
  \item<1->
    Our result for standard random networks:
    $$
    \tavg{n} 
    = 
    F'_{\pi}(1)
    =
    \frac{
      \left(
        1-S_1
      \right)
    }
    {
      1 -
      \tavg{k}(1-S_1)
    }
    $$
  \item<2->
    Recall that $\tavg{k}=1$ is the critical value 
    of average degree for standard random networks.
  \item<3->
    Look at what happens when we increase $\tavg{k}$ 
    to 1 from below.
  \item<4->
    We have $S_1=0$ for all $\tavg{k}<1$ 
    \uncover<5->{
      so
      $$
      \avg{n} =
      \frac{
          1
      }
      {
        1 -
        \tavg{k}
      }
      $$
    }
  \item<5->
    This blows up as $\tavg{k} \rightarrow 1$.
  \item<6-> 
    \alert{Reason:} we have a power law distribution of component
    sizes at $\tavg{k}=1$.
  \item<7->
    Typical critical point behavior....
  \end{itemize}
\end{frame}

%%%
%%%
% fix up (1-S_1) factor discussion
%%%
%%% add a graph

\begin{frame}[label=]
  \frametitle{Average component size}
  \begin{itemize}
  \item<1->
    Limits of $\tavg{k}=0$ and $\infty$ make sense for
    $$
    \tavg{n} 
    = 
    F'_{\pi}(1)
    =
    \frac{
      \left(
        1-S_1
      \right)
    }
    {
      1 -
      \tavg{k}(1-S_1)
    }
    $$
  \item<2->
    As $\tavg{k} \rightarrow 0$, $S_1=0$, and $\tavg{n} \rightarrow 1$.
  \item<3->
    All nodes are isolated.
  \item<4->
    As $\tavg{k} \rightarrow \infty$, $S_1\rightarrow 1$
    and $\tavg{n} \rightarrow 0$.
  \item<5->
    No nodes are outside of the giant component.
  \end{itemize}

  \begin{block}<6->{Extra on largest component size:}
    \begin{itemize}
    \item<6->
      For $\tavg{k}=1$, $S_1 \sim N^{2/3}/N$.
    \item<7->
      For $\tavg{k}< 1$, $S_1 \sim (\log N)/N$.
    \end{itemize}
  \end{block}
  
\end{frame}

% TODO
%\begin{frame}[label=]
%  \frametitle{}
%
% finding higher moments
%
%\end{frame}

