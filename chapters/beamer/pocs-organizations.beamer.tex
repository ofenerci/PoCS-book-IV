\section{Overview}

\begin{frame}
  \frametitle{Overview}

  \begin{block}{The basic idea/problem/motivation/history:}
    \begin{itemize}
    \item<1-> 
      Organizations as information exchange entities.
    \item<2-> 
      Catastrophe recovery.
    \item<3-> 
      Solving ambiguous, ill-defined problems.
    \item<4->
      Robustness as `optimal' design feature.
    \end{itemize}
  \end{block}

  \begin{block}<5->{A model of organizational networks:}
    \begin{itemize}
    \item<5->
      Network construction algorithm.
    \item<6->
      Task specification.
    \item<7->
      Message routing algorithm.
    \end{itemize}
  \end{block}

  \begin{block}<8->{Results:}
    \begin{itemize}
    \item<8->
      Performance measures.
    \end{itemize}
  \end{block}

\end{frame}

\subsection{Toyota}

%%   Toyota-Aisin example.
%%  eye-sheen
%% 
%%  1997
%%  brake valve part, only supplier
%%  4 hours supply (just in time)
%%  
%%  14000 cars per day
%%  
%%  months of delay predicted
%%  6 months before new machines would arrive
%%  
%%  recovered in about 5 days
%%  
%%  sewing machine maker with no experience in car parts
%%  spent about 500 man hours refitting a milling machine
%%  to produce 40 valves a day
%%  36 suppliers
%%  150 subcontractors
%%  50 supply lines
%%  
%%  result of strengths of ties across a hierarchy
%% 
%%  wrote manuals on how to deal with such a crisis
%%  
%% scary for other companies

\begin{frame}
  \frametitle{February, 1997:}

  \begin{block}<1->{Aisin (eye-sheen), maker of brake valve parts for Toyota, burns to ground.\cite{nishiguchi2000a}}
    \begin{itemize}
    \item<2->
      4 hours supply (``just in time'').
    \item<3->
      14,000 cars per day $\rightarrow$ 0 cars per day.
    \item<4->
      6 months before new machines would arrive.
    \item<5-> 
      \uncover<5->{
        Recovered in 5 days.}
    \end{itemize}
  \end{block}

  \begin{itemize}
  \item 
    For more see Nishiguchi and Beaudet\cite{nishiguchi2000a}\\
    ``Fractal Design: Self-organizing Links in Supply Chain''\\
    in ``Knowledge Creation: A New Source of Value''
  \end{itemize}
  

\end{frame}

\begin{frame}
  \frametitle{February, 1997:}

  \begin{block}{Some details:}
    \begin{itemize}
    \item<+->
      36 suppliers, 150 subcontractors
    \item<+-> 
      50 supply lines
    \item<+-> 
      Sewing machine maker with no experience in car parts
      spent about 500 man hours refitting a milling machine
      to produce 40 valves a day.
    \item<+->
      Recovery depended on horizontal links which
      arguably provided:
      \begin{enumerate}
      \item<+-> 
        robustness
      \item<+->
        searchability
      \end{enumerate}
    \end{itemize}
  \end{block}

\end{frame}

\begin{frame}
  \frametitle{Some things fall apart:}
  \includegraphics[width=\textwidth]{lehmann-brothers-bag.jpg}
\end{frame}

\begin{frame}
  \includegraphics[width=\textwidth]{lehmann-brothers-bag-zoom.jpg}
\end{frame}

\begin{frame}
  \frametitle{Rebirth:}
  \includegraphics[width=\textwidth]{despicable_me_bank_of_evil_formerly_lehman_brothers.png}
\end{frame}

\subsection{Ambiguous\ problems}

\begin{frame}
  \frametitle{Motivation}

  \begin{block}{Recovery from catastrophe involves solving problems that are:}
    \begin{itemize}
    \item<2-> 
      Unanticipated,
    \item<3-> 
      Unprecedented,
    \item<4->
      Ambiguous (nothing is obvious),
    \item<5-> 
      Distributed (knowledge/people/resources),
    \item<6-> 
      Limited by existing resources,
    \item<7-> 
      Critical for survival.
    \end{itemize}
  \end{block}

  \begin{block}<8->{Frame:}
    \begin{itemize}
    \item<8-> 
      Collective solving of ambiguous problems
    \end{itemize}
  \end{block}

\end{frame}

\begin{frame}
  \frametitle{Motivation}

  \begin{block}{Ambiguity:}
    \begin{itemize}
    \item<1->
      Question much less answer is not well understood.
    \item<2->
      Back and forth search process rephrases question.
    \item<3->
      Leads to iterative process of query reformulation.
    \item<4->
      Ambiguous tasks are inherently not decomposable.
    \item<5->
      How do individuals collectively work
      on an ambiguous organization-scale problem?
    \item<6->
      How do we define ambiguity?
    \end{itemize}
\end{block}

\end{frame}

\begin{frame}
  \frametitle{Let's modelify:}

  \begin{block}{Modeling ambiguous problems is hard\ldots}
    \begin{itemize}
    \item<2->
      Model response instead\ldots
    \item<3-> 
      Individuals need novel information and must communicate with others
      outside of their usual contacts.
    \item<4->  
      Creative search is intrinsically inefficient.
    \end{itemize}
  \end{block}

  \begin{block}<5->{Focus on robustness:}
    \begin{enumerate}
    \item<6-> 
      Avoidance of individual failures.
    \item<7->
      Survival of organization even when failures do occur.
    \end{enumerate}
   \end{block}

\end{frame}

\subsection{Models\ of\ organizations:}

\begin{frame}
  \frametitle{Why organizations exist:}

  \begin{block}{\wordwikilink{http://en.wikipedia.org/wiki/Ronald\_Coase}{Ronald Coase}, 1937, ``The Nature of the Firm''\cite{coase1937a}}
    \begin{itemize}
    \item<1-> 
      Notion of \wordwikilink{http://en.wikipedia.org/wiki/Transaction\_cost}{Transaction Costs}.
    \item<1-> 
      More efficient for individuals to cooperate outside of the market.
    \end{itemize}
  \end{block}

  \begin{center}
    \includegraphics[width=0.8\textwidth]{coase}
  \end{center}

\end{frame}

\begin{frame}
  \frametitle{Real organizations---Extremes}

  \begin{block}<1->{Hierarchy:}
    \begin{itemize}
    \item 
      Maximum efficiency,
    \item 
      Suited to static environment,
    \item 
      Brittle.
    \end{itemize}
  \end{block}

  \begin{block}<2->{Market:}
    \begin{itemize}
    \item
      Resilient,
    \item
      Suited to rapidly changing environment,
    \item
      Requires costless interactions.
    \end{itemize}
  \end{block}

\end{frame}

\begin{frame}
  \frametitle{Real organizations\ldots}

  \begin{block}{But real, complex organizations are in the middle\ldots}
    \begin{center}
      \includegraphics[width=\textwidth]{orgnetworkrange}
    \end{center}
    \begin{itemize}
    \item
      \alertb{``Heterarchies''} (D. Stark, 1999)\cite{stark1999a}
    \end{itemize}
  \end{block}

\end{frame}


\begin{frame}
  \frametitle{Organizations as efficient hierarchies}

  \begin{block}{}
    \begin{itemize}
    \item<1->
      Economics: \alertb{Organizations $\equiv$ Hierarchies.}
    \item<1->
      e.g., Radner (1993)\cite{radner1993a}, Van Zandt (1998)\cite{vanzandt1998a}
    \item<1->
      Hierarchies performing associative operations:
      \begin{center}
        \includegraphics[width=0.8\textwidth]{associativenet}
      \end{center}
    \end{itemize}
  \end{block}

\end{frame}

\begin{frame}
  \frametitle{Optimal network topologies for local search}

  %% queueing, point of collapse,
  %% average search time + congestion,
  %% simulated annealing

  \begin{block}{Guimer\`{a} et al., 2002\cite{guimera2002b}}
  \begin{center}
    \includegraphics[width=0.45\textwidth]{mb}
    \includegraphics[width=0.45\textwidth]{md}
    \begin{itemize}
    \item<1->
      Parallel search and congestion.
    \item<1->
      Queueing and network collapse.
    \item<1-> 
      Exploration of random search mechanisms.
    \end{itemize}
  \end{center}
  \end{block}

\end{frame}

\begin{frame}
  \frametitle{Optimal network topologies for local search}

  \begin{columns}
    \column{0.6\textwidth}
    \includegraphics[width=\textwidth]{r1}
    \column{0.4\textwidth}
    \begin{itemize}
    \item<1->
      Betweenness: $\beta$.
    \item<1->
      Polarization: 
      $$\pi = \frac{\max \beta}{\tavg{\beta}}-1.$$
    \item<1->
      $L$ = number of links.
    \end{itemize}
  \end{columns}

  \bigskip


  \begin{itemize}
  \item<1-> 
    Goal: minimize average search time.
  \item<1-> 
    Few searches $\Rightarrow$ hub-and-spoke network.
  \item<1-> 
    Many searches $\Rightarrow$ decentralized network.
  \end{itemize}

\end{frame}

%% \begin{frame}
%%  \frametitle{Organizations as efficient hierarchies}
%%
%%  Goh?
%%
%%\end{frame}

\section{Modelification}

\subsection{Goals}

\begin{frame}
  \frametitle{Desirable organizational qualities:}

  \begin{block}{}
    \begin{enumerate}
    \item<1->
      Low cost (requiring few links).
    \item<2->
      Scalability.
    \item<3->
      Ease of construction---existence is plausible.
    \item<4->
      Searchability.
    \item<5-> 
      \alert{`Ultra-robustness'}:
      \begin{enumerate}
      \item<6->[I]
        \alertb{Congestion robustness}\\
        (Resilience to failure due to information exchange);
      \item<7->[II] 
        \alertb{Connectivity robustness}\\
        (Recoverability in the event of failure).
      \end{enumerate}
    \end{enumerate}
  \end{block}

\end{frame}


%%\begin{frame}
%%  \frametitle{Searchability}
%%
%%  Guimer\`{a} \textit{et al.}, 2002
%%
%%  Optimal network topologies for searching
%%  using only local information.
%%
%%  \hspace{3ex}
%%  Low cost searches $\Rightarrow$ hub-based networks.
%%
%%  \hspace{3ex}
%%  High cost searches $\Rightarrow$ featureless networks.
%%  
%%\end{frame}

\begin{frame}
  \frametitle{Searchability}

  \begin{block}{Small world problem:}
    \begin{itemize}
    \item<1->
      Can individuals pass a message
      to a target individual using only personal connections?
    \item<1->
      Yes, large scale networks searchable 
      if nodes have \alertb{identities}.
    \item<1->
      ``Identity and Search in Social Networks,''
      Watts, Dodds, \& Newman, 2002.\cite{watts2002b}
    \end{itemize}
  \end{block}


\end{frame}

%%%%%%%%%%%
%% model
%%%%%%%%%%%

\subsection{Model}

\begin{frame}
  \frametitle{Model}

  \begin{block}{Organizational network robustness:}
    ``Information exchange and the robustness of organizational networks,''\\
    Dodds, Watts, and Sabel, 2003.\cite{dodds2003c}\\
    Proc. Natl. Acad. Sci., edited by 
    \wordwikilink{http://en.wikipedia.org/wiki/Harrison\_White}{Harrison White}
  \end{block}

  \begin{block}<2->{Formal organizational structure:}
    \begin{itemize}
    \item<2->
      \alertb{Underlying hierarchy:}
      \begin{itemize}
      \item
        branching ratio $b$
      \item
        depth $L$
      \item
        $N = (b^L-1)/(b-1)$ nodes
      \item
        $N-1$ links
      \end{itemize}
    \item<3->
      \alertb{Additional informal ties:}
      \begin{itemize}
      \item 
        Choose $m$ links according to a
        two parameter probability distribution
      \item 
        $ 0 \le m \le (N-1)(N-2)/2 $
      \end{itemize}
    \end{itemize}
    
  \end{block}



\end{frame}


\begin{frame}
  \frametitle{Model---underlying hierarchy}

  \begin{block}{Model---formal structure:}
    \begin{center}
      \includegraphics[width=0.75\textwidth]{networkvariation8}
    \end{center}
    $$ 
    b=3, \quad  L=3, \quad N=13
    $$
  \end{block}

\end{frame}

%%\begin{frame}
%%  \frametitle{Model---underlying hierarchy}
%%  \begin{center}
%%    \includegraphics[width=0.75\textwidth]{networkvar1}
%%  \end{center}
%%  \Large \[ b=3, \quad  L=4, \quad N=40\]
%% \end{frame}

\begin{frame}
  \frametitle{Model---addition of links}

  \begin{block}{Team-based networks ($m=12$):}
    \begin{center}
      \includegraphics[width=0.75\textwidth]{networkvariation10}
    \end{center}
  \end{block}
    
\end{frame}

\begin{frame}
  \frametitle{Model---addition of links}

  \begin{block}{Random networks ($m=12$):}
    \begin{center}
      \includegraphics[width=0.75\textwidth]{networkvariation11}
    \end{center}
  \end{block}

\end{frame}

\begin{frame}
  \frametitle{Model---addition of links}

  \begin{block}{Random interdivisional networks ($m=6$):}
    \begin{center}
      \includegraphics[width=0.75\textwidth]{networkvariation12}
    \end{center}
  \end{block}

\end{frame}

\begin{frame}
  \frametitle{Model---addition of links}

  \begin{block}{Core-periphery networks ($m=6$):}
    \begin{center}
      \includegraphics[width=0.75\textwidth]{networkvariation13}
    \end{center}
  \end{block}

\end{frame}

\begin{frame}
  \frametitle{Model---addition of links}

  \begin{block}{Multiscale networks $(m=12)$:}
    \begin{center}
      \includegraphics[width=0.75\textwidth]{networkvariation14}
    \end{center}
  \end{block}

\end{frame}


\begin{frame}
  \frametitle{Model---construction}

  \begin{center}
    \includegraphics[width=\textwidth]{linkaddition3}
  \end{center}

\end{frame}

\begin{frame}
  \frametitle{Model---construction}

  \begin{block}{}
    \begin{itemize}
    \item<1-> 
      Link addition probability:
      $$
      P(D,d_1,d_2) 
      \propto 
      e^{-D/\lambda} e^{-f(d_1,d_2)/\zeta}
      $$
    \item<1-> 
      First choose $(D,d_1,d_2)$.
    \item<1-> 
      Randomly choose $(y,x_1,x_2)$ given $(D,d_1,d_2)$.
    \item<1-> 
      Choose links without replacement.
    \end{itemize}
  \end{block}

\end{frame}

\begin{frame}
  \frametitle{Model---construction}

  \begin{block}{Requirements for $f(d_1,d_2)$:}
    \begin{enumerate}
    \item<2-> 
      $f \geq 0$ for $d_1+d_2 \geq 2$
    \item<3->
      $f$ increases monotonically with $d_1$, $d_2$.
    \item<4->
      $f(d_1,d_2) = f(d_2,d_1)$.
    \item<5->
      $f$ is maximized when $d_1=d_2$.
    \end{enumerate}
  \end{block}

  \begin{block}<6->{Simple function satisfying 1--4:}
    $$
    f(d_1,d_2) = (d_1^2 + d_2^2-2)^{1/2}
    $$
    $$
    \Rightarrow
    P(y,x_1,x_2) \propto e^{-D/\lambda} e^{-(d_1^2 + d_2^2-2)^{1/2}/\zeta} 
    $$  
  \end{block}

\end{frame}

%%\begin{frame}
%%  \frametitle{Model---limiting cases}
%%
%%\vfill
%%
%%  \hspace{3ex}
%%  $\lambda$ large, $\zeta=0$: team-based networks
%%
%%  \hspace{3ex}
%%  $\lambda$ large, $\zeta$ large: random networks
%%
%%  \hspace{3ex}
%%  $\lambda=0$, $\zeta$ large: random interdivisional networks
%%
%%  \hspace{3ex}
%%  $\lambda$ small, $\zeta$ small: core-periphery networks
%%
%%  \hspace{3ex}
%%  $\lambda$, $\zeta$ intermediate: multiscale networks
%%
%%\vfill
%%\end{frame}

\begin{frame}
  \frametitle{Model---limiting cases}

  \vfill
  \begin{center}
    \includegraphics[width=\textwidth]{networkspace}
  \end{center}
  \vfill
\end{frame}



%%%%%%%%%%%%%
%% testing %%
%%%%%%%%%%%%%

\subsection{Testing}

\begin{frame}
  \frametitle{Message passing pattern}

  \begin{itemize}
  \item<1->
    Each of $T$ time steps,  each node generates a message with probability $\mu$.
  \item<2-> 
    Recipient of message chosen based on distance from sender.
  \item<3-> 
    $$
    P(\mbox{recipient at distance}\ d) \propto e^{-d/\xi}.
    $$
  \end{itemize}
  \begin{enumerate}
  \item<3-> 
    $\xi$ = measure of uncertainty;
  \item<3-> 
    $\xi=0$: local message passing;
  \item<3->  
    $\xi=\infty$: random message passing.
  \end{enumerate}
\end{frame}

\begin{frame}
  \frametitle{Message passing pattern:}

  \begin{block}{Distance $d_{12}$ between two nodes $x_1$ and $x_2$:}
    \bigskip
    \begin{columns}
      \column{0.6\textwidth} 
      \includegraphics[width=\textwidth]{linkaddition3}
      \column{0.4\textwidth} 
      $$
      d_{12} = \max(d_1,d_2) =3
      $$
    \end{columns}
    \bigskip
    \begin{itemize}    
    \item 
      Measure unchanged with presence of informal ties.
    \end{itemize}
  \end{block}
\end{frame}

\begin{frame}
  \frametitle{Message passing pattern}

  \begin{block}{Simple message routing algorithm:} 
    \begin{itemize}
    \item<1->
      Look ahead one step:
      always choose neighbor closest to recipient node.
    \item<2->
      \alertb{Pseudo-global knowledge:}
      \begin{enumerate}
      \item 
        Nodes understand hierarchy.
      \item 
        Nodes know only local informal ties.
      \end{enumerate}
    \end{itemize}
  \end{block}

\end{frame}


\begin{frame}
  \frametitle{Message passing pattern}

  \begin{block}{Interpretations:}
    \begin{enumerate}
    \item<1->
      Sender knows specific recipient.
    \item<2->
      Sender requires certain kind of recipient.
    \item<3->
      Sender seeks specific information but recipient unknown.
    \item<4->
      Sender has a problem but information/recipient unknown.
    \end{enumerate}

  \end{block}

\end{frame}

\begin{frame}
  \frametitle{Message passing pattern}

  \begin{block}{Performance:}
  \begin{itemize}
  \item<1->
    Measure Congestion Centrality $\rho_i$,
    fraction of messages passing through node $i$.
  \item<2->
    Similar to betweenness centrality.
  \item<3->  
    However: depends on 
    \begin{enumerate}
    \item<3->
      Search algorithm;
    \item<4->
      Task specification ($\mu$, $\xi$).
    \end{enumerate}
  \item<5->  
    Congestion robustness comes from
    minimizing $\rho_{\textnormal{max}}$.
  \end{itemize}
  \end{block}

\end{frame}


%% \begin{frame}
%%   \frametitle{Message passing pattern}
%% 
%%   Two message routing algorithms: $C_1$ and $C_2$ search.
%% 
%%   $C_1$: Look ahead one step---choose neighbor closest to recipient node.
%% 
%%   $C_2$: Look ahead two steps---choose neighbor with neighbor closest to recipient node.
%% 
%%   Nodes understand hierarchy but know only local informal ties.
%% 
%%   Examine $C_1$ only.
%% 
%% \end{frame}

\subsection{Results}

%%%%%%%%%%%%%
%% results %%
%%%%%%%%%%%%%

\begin{frame}
  \frametitle{Performance testing:}

  \begin{block}{Parameter settings (unless varying):}
    \begin{itemize}
    \item<1->
      Underlying hierarchy:
      $b=5$, $L=6$, $N=3096$;
    \item<2->
      Number of informal ties:
      $m=N$.
    \item<3->
      Link addition algorithm:
      $\lambda=\zeta=0.5$.
    \item<4->
      Message passing:
      $\xi=1$, $\mu=10/N$, $T=1000$.
    \end{itemize}
  \end{block}

\end{frame}

\begin{frame}
  \frametitle{Results---congestion robustness}
  \begin{center}
    \includegraphics[width=0.75\textwidth]{figmsdomain_rhomax30_1234_5C_noname}
    %%    \includegraphics[width=0.65\textwidth]{figmsdomain_rhomax30_1234_5C_noname}
  \end{center}
\end{frame}

%% \begin{frame}
%%   \frametitle{Results}
%%   \begin{center}
%%     \includegraphics[width=0.5\textwidth]{figmsdomain_rhomax30_1234_3_3d_noname}
%%   \end{center}
%% \end{frame}

\begin{frame}
  \frametitle{Results---varying number of links added:}

  \begin{columns}
    \column{0.75\textwidth}
    \includegraphics[width=\textwidth]{figzl_linkadd_rhomax2_32_noname}
    \column{0.25\textwidth}
    \begin{itemize}
    \item[] 
        {\Large$\diamond$}=TB
    \item[]
        $\bigtriangledown$=R
    \item[]
        $\bigtriangleup$=RID 
    \item[]
        {\small$\bigcirc$}=CP
    \item[]
        $\Box$=MS 
    \end{itemize}
  \end{columns}

\end{frame}

\begin{frame}
  \frametitle{Results---varying message passing pattern}

  \begin{columns}
    \column{0.75\textwidth}
    \includegraphics[width=\textwidth]{figzl_rhomax_xi_31_noname}
    \column{0.25\textwidth}
    \begin{itemize}
    \item[] 
        {\Large$\diamond$}=TB
    \item[]
        $\bigtriangledown$=R
    \item[]
        $\bigtriangleup$=RID 
    \item[]
        {\small$\bigcirc$}=CP
    \item[]
        $\Box$=MS 
    \end{itemize}
  \end{columns}

\end{frame}


\begin{frame}
  \frametitle{Results---Maximum firm size}

  \begin{itemize}
  \item<1->
    Congestion may increase with size
    of network.
  \item<2-> 
    Fix rate of message passing ($\mu$)
    and
    Message pattern ($\xi$).
  \item<3->
    Fix branching ratio of hierarchy  and add more levels.
  \item<4-> 
    Individuals have limited capacity 
    $\Rightarrow$ limit to firm size.
  \end{itemize}

\end{frame}


\begin{frame}
  \frametitle{Results---Scalability}

  \begin{columns}
    \column{0.75\textwidth}
    \includegraphics[width=\textwidth]{figrhomaxNscaling33_6lin_noname}
    \column{0.25\textwidth}
    \begin{itemize}
    \item[] 
        {\Large$\diamond$}=TB
    \item[]
        $\bigtriangledown$=R
    \item[]
        $\bigtriangleup$=RID 
    \item[]
        {\small$\bigcirc$}=CP
    \item[]
        $\Box$=MS 
    \end{itemize}
  \end{columns}

\end{frame}

\begin{frame}
  \frametitle{Results---Scalability}

  \begin{columns}
    \column{0.75\textwidth}
    \includegraphics[width=\textwidth]{figrhomaxNscaling33_10log2_noname}
    \column{0.25\textwidth}
    \begin{itemize}
    \item[] 
        {\Large$\diamond$}=TB
    \item[]
        $\bigtriangledown$=R
    \item[]
        $\bigtriangleup$=RID 
    \item[]
        {\small$\bigcirc$}=CP
    \item[]
        $\Box$=MS 
    \end{itemize}
  \end{columns}

\end{frame}


\begin{frame}
  \frametitle{Connectivity Robustness}

  \begin{block}<1->{Inducing catastrophic failure:}
    \begin{itemize}
    \item<1->
      Remove $N_r$ nodes and measure relative size
      of largest component $C = S/(N-N_r)$.
    \item<2-> 
      Four deletion sequences:
      \begin{enumerate}
      \item 
        Top-down;
      \item 
        Random;
      \item 
        Hub;
      \item 
        Cascading failure.
      \end{enumerate}
    \item<3->
      Results largely independent of sequence.
    \end{itemize}
  \end{block}
  
\end{frame}

\begin{frame}
  \frametitle{Results---Connectivity Robustness}

  \begin{columns}
    \column{0.75\textwidth}
    \includegraphics[width=\textwidth]{figrobustness51_mod}
    \column{0.25\textwidth}
    \begin{itemize}
    \item[] 
        {\Large$\diamond$}=TB
    \item[]
        $\bigtriangledown$=R
    \item[]
        $\bigtriangleup$=RID 
    \item[]
        {\small$\bigcirc$}=CP
    \item[]
        $\Box$=MS 
    \end{itemize}
  \end{columns}

    

\end{frame}

\begin{frame}
  \frametitle{Summary of results}

  \small
  \begin{tabular}{l|lll}
    Feature & Congestion  & Connectivity & Scalability \\ 
    & Robustness & Robustness &  \\\hline
    \\
    Core-periphery & good & average &average \\
    \\
    Random & poor & good & poor \\
    \\
    Rand. Interdivisional & poor & good & poor \\
    \\
    Team-based & poor & poor & poor\\
    \\
    \alertb{Multiscale} & \alertb{good} & \alertb{good} &\alertb{good} \\
  \end{tabular}
  
\end{frame}

\section{Conclusion}

\begin{frame}
  \frametitle{Conclusary moments}

  \begin{block}{Multi-scale networks:}
    \begin{enumerate}
    \item<1->
      Possess good Congestion Robustness and
      Connectivity Robustness $\Rightarrow$ Ultra-robust;
    \item<2->
      Scalable;
    \item<3->
      Relatively insensitive to parameter choice;
    \end{enumerate}
    \begin{itemize}
    \item<4->
      Above suggests existence of multi-scale structure is plausible.
    \end{itemize}
  \end{block}
  
\end{frame}

\begin{frame}
  \frametitle{Conclusary moments}

  \begin{block}{}
    \begin{itemize}
    \item<1->
      Foregoing is an attempt to model what organizations
      might look like beyond simple hierarchies (2003).
    \item<2->
      Possible work: develop `bottom up' model of organizational
      networks based on social search, identity 
      (emergent searchability).
    \item<3->
      Balance of \alertb{generalists versus specialists}---how many
      middle managers does an organization need?
    \item<4->
      Still a need for data on real organizations\ldots
    \end{itemize}
  \end{block}


\end{frame}




