

  \showtarotcards{0.20}{
    overview,
    golden-age-of-reductionism,
    manifesto,
    scaling,
    power-law-size-distributions,
    variable-transformation,
    random-walks,
    rich-get-richer,
    efficient-language,
    law-of-first-digits,
    surprise-of-being-robust-yet-fragile,
    misreading-of-the-book-of-sand,
    tales-of-tails,
    data-poor,
    pouring-data,
    emergence-of-structure,
    emergence-of-destruction,
    emergence-of-thinking,
    emergence-of-stories,
    mechanics-statistical,
    thresholds-of-percolation,
    complex-networks,
    web-elements,
    random-networks,
    strangeness-of-friends,
    small-world-networks,
    theory-six-degrees,
    scale-free-networks,
    confusion-of-contagions,
    disease-of-random-mixing,
    unending-pandemics,
    thresholds-of-the-mind,
    social-wild,
    social-contagion,
    fame,
    contagious-stories,
    paths-of-universality,
%    dependent-paths,
%    end,
}                                                
\section{Universality}
  \small

  \textbf{Limits to what's possible:}

  \textbf{\wordwikilink{http://en.wikipedia.org/wiki/Universality\_(dynamical\_systems)}{Universality}:}
      The property that the macroscopic aspects of a system do not depend
      sensitively on the system's details.
    
      Key figure:
      \wordwikilink{http://en.wikipedia.org/wiki/Leo\_Kadanoff}{Leo Kadanoff}
    
      Kadanoff's retrospective: ``Innovations in Statistics Physics''~\cite{kadanoff2014a}
  \textbf{Examples:}
      The Central Limit Theorem:
      $$
      P(x; \mu, \sigma) \dee{x} = 
      \frac{1}{\sqrt{2\pi}\sigma}
      e^{-(x-\mu)^2/2\sigma^2} \dee{x}.
      $$
     
      Navier Stokes equation for fluids.
     
      Nature of phase transitions in statistical mechanics.
  \textbf{Universality}
    
     
      Sometimes \alert{details don't matter too much}.
     
      \alertb{Many-to-one mapping} from micro to macro
     
      Suggests not all possible behaviors are available 
      \qquad at higher levels of complexity.
     
      Universality means some things are fated.
  \textbf{Large questions:}
      How universal is universality?
    
      What are the possible long-time states (attractors) for a universe?
    
  \textbf{Fluid mechanics}
  
   
    Fluid mechanics = One of the great successes of 
    understanding complex systems.
   
    Navier-Stokes equations: micro-macro system evolution.
   
    The big three: Experiment + Theory + Simulations.
   
    Works for many very different `fluids':
      the atmosphere,
     
      oceans,
     
      blood,
     
      galaxies,
     
      the earth's mantle \ldots
    
      \alert{and ball bearings on lattices \ldots ?}
  
  \textbf{Lattice gas models}

  \textbf{Collision rules in 2-d on a hexagonal lattice:}
  \begin{center}
    \includegraphics[width=0.7\textwidth]{lattice_gas_rules_example.jpg}
  \end{center}
  
    Lattice matters \ldots
   
    No `good' lattice in 3-d.
  
    Upshot: play with `particles' of a system to obtain new or specific macro behaviours.
  \textbf{Hexagons---\wordwikilink{http://en.wikipedia.org/wiki/Honeycomb}{Honeycomb:}}

  \begin{center}
    \includegraphics[height=0.65\textheight]{Honey_comb.jpg}
  \end{center}
  
    Orchestrated?  Or an accident of bees working hard?
   
    See ``On Growth and Form'' by 
    \wordwikilink{http://en.wikipedia.org/wiki/D'Arcy\_Wentworth\_Thompson}{D'Arcy Wentworth Thompson}.\cite{thompson1952a,thompson1961a}
  \textbf{Hexagons---\wordwikilink{http://en.wikipedia.org/wiki/Giant's\_Causeway}{Giant's Causeway:}}

  \includegraphics[width=\textwidth]{giantscauseway_mist_medium.jpg}

  {\tiny \url{http://newdesktopwallpapers.info}}

%% from http://newdesktopwallpapers.info/Ireland%20Wallpapers/slides/Giant's%20Causeway,%20County%20Antrim,%20Ireland.html
  \textbf{Hexagons---\wordwikilink{http://en.wikipedia.org/wiki/Giant's\_Causeway}{Giant's Causeway:}}

  \includegraphics[width=\textwidth]{giantscauseway_hex_medium.jpg}

  {\tiny \url{http://www.physics.utoronto.ca/}}

%% from http://www.physics.utoronto.ca/news_repository/u-of-t-scientists-solve-mystery-of-giants-causeway-with-kitchen-materials/image/image_view_fullscreen
  \textbf{Saturn has a hexagon:}

  \begin{center}
    \includegraphics[width=0.7\textwidth]{Saturn_N_polar_hexagon_W00077335.jpg}
  %% source: http://en.wikipedia.org/wiki/Saturn's_hexagon
  \end{center}
   
    \wordwikilink{http://en.wikipedia.org/wiki/Saturn's\_hexagon}{One side is longer than Earth's diameter}
  \textbf{Hexagons run amok:}
    
    \includegraphics[width=\textwidth]{340px-Graphen.jpg}\\
    \bigskip
    \includegraphics[width=\textwidth]{screenie-bluesbrothers-fence.jpg}
      
       
        \wordwikilink{http://en.wikipedia.org/wiki/Graphene}{Graphene}:
        single layer of carbon molecules
        in a perfect hexagonal lattice (super strong).
       
        \wordwikilink{http://en.wikipedia.org/wiki/Chicken\_wire}{Chicken wire} \ldots
  
  \textbf{Triumph of the Hexagon}
    \youtubevideo{xyY0ymMYXPo}{}{}
    
    From the remarkable 
    \wordwikilink{http://hexnet.org}{Hexnet.org}, 
    the Global Hexagonal Awareness Resource Center.
  \showtarotcards{0.20}{
    overview,
    golden-age-of-reductionism,
    manifesto,
    scaling,
    power-law-size-distributions,
    variable-transformation,
    random-walks,
    rich-get-richer,
    efficient-language,
    law-of-first-digits,
    surprise-of-being-robust-yet-fragile,
    misreading-of-the-book-of-sand,
    tales-of-tails,
    data-poor,
    pouring-data,
    emergence-of-structure,
    emergence-of-destruction,
    emergence-of-thinking,
    emergence-of-stories,
    mechanics-statistical,
    thresholds-of-percolation,
    complex-networks,
    web-elements,
    random-networks,
    strangeness-of-friends,
    small-world-networks,
    theory-six-degrees,
    scale-free-networks,
    confusion-of-contagions,
    disease-of-random-mixing,
    unending-pandemics,
    thresholds-of-the-mind,
    social-wild,
    social-contagion,
    fame,
    contagious-stories,
    paths-of-universality,
    dependent-paths,
%    end,
}                                                
\section{Symmetry\ Breaking}
  \textbf{Symmetry Breaking}
    \displaypaper{anderson1972a}{2}
    
    \bigskip
      
      
      \includegraphics[width=\textwidth]{501px-Andersonphoto.jpg}
       
        \wordwikilink{http://en.wikipedia.org/wiki/Philip_Warren_Anderson}{Anderson}
        argues against idea that the only real scientists are
        those working on the fundamental laws.
      
        Symmetry breaking $\rightarrow$ different laws/rules at
        different scales \ldots
    \bigskip

    {2006 study: \wordwikilink{http://physicsworld.com/cws/article/news/25623}{``most creative physicist in the world''}}
  \textbf{Symmetry Breaking}

  \textbf{\alertg{``Elementary entities of science X obey the laws of science Y''}}
        
         X 
         solid state or many-body physics 
         chemistry \\ \mbox{}
         molecular biology 
         cell biology
        [$\vdots$]
         psychology
         social sciences
      
        
         Y
         elementary particle physics 
         solid state many-body physics
         chemistry 
         molecular biology 
        [$\vdots$]
         physiology 
         psychology
    
  \textbf{Symmetry Breaking}

  \textbf{Anderson:}
      \ [the more we know about] ``fundamental laws, the less
      relevance they seem to have to the very real problems
      of the rest of science.''
    
      \alertb{Scale} and \alertb{complexity}
      thwart the constructionist hypothesis.
    
      Accidents of history 
      and \wordwikilink{http://en.wikipedia.org/wiki/Path_dependence}{path dependence}
      matter.
  \textbf{Symmetry Breaking}
    \displayamazonbook{sornette2006a}
      Page 291--292 of Sornette\cite{sornette2003a}:\\
      Renormalization $\equiv$ Anderson's hierarchy.
    
      But Anderson's hierarchy is not a simple one: the rules change.
    
      Crucial dichotomy between evolving systems
      following stochastic paths that lead
      to \\
      (a) \alertb{inevitable} 
      \alertb{or} 
      (b) \alertb{particular}
      destinations (states).
%% more ????

  \textbf{More is different:}

  \includegraphics[width=\textwidth]{xkcd435-complexity.png}\\
  \wordwikilink{http://xkcd.com/435/}{http://xkcd.com/435/}
\section{The\ Big\ Theory}
  \textbf{A real science of complexity:}

  \textbf{A real theory of \sout{everything} anything:}
      Is not just about the ridiculously small stuff \ldots
    
      It's about the increase of complexity
  \medskip
  {
      
      Symmetry breaking/
      Accidents of history
      
      vs.
      Universality
    
  }
  \medskip
  
   
    Second law of thermodynamics: \alertb{we're toast in the long run}.
   
    So how likely is the local complexification of structure we enjoy?
   
    How likely are the Big Transitions?
  \textbf{Why complexify?}
    \displaypaper{arthur1993b}{1}
      Argues that evolution toward increased performance brings a
      ratcheting cycle
      of complexification and simplification.
     
      Jet engine replaced the complex piston engine and then itself
      became more complex.
     
      Complexification $\equiv$ evolution of algorithms?
      
      Differential equations and stories $\subset$ Algorithms.
      
      Life is a loaded word: The Search for Extraterrestrial Algorithms (SETA)?
  
  \textbf{Why complexify?}

  \textbf{Driving complexity's trajectory:}
      Big Bang
     
      Randomness leads to replicating structures;
     
      Biological evolution;
     
      Sociocultural evolution;
     
      Technological evolution;
     
      Sociotechnological evolution.
  \textbf{Complexification---the Big Transitions:}
  
    
      Big Bang.
     
      Big Randomness.
     
      %% galactic networks
      %% stars and planets
      Big Structure.
     
      Big Replicate.
     
      Big Life.
     
      Big Evolve.
    
     
      Big Word.
     
      Big Story.
     
      Big Number.
     
      Big Farm.
     
      Big God.
     
      Big Make.
     
      Big City.
     
      Big Culture.
    
    
      Big Science.
     
      Big Data.
     
      Big Information.
     
      Big Algorithm.
     
      Big Connection.
     
      Big Social.
     
      Big Awareness.
     
      Big Spread.
    
      Big \ldots ?
  
%% three levels of complexity
%%  

%%  \includegraphics[width=\textwidth]{2011-07-15complexity-framing-cropped.jpg}
  \includegraphics[width=\textwidth]{2011-07-15complexity-framing-cropped-sharp-tp-3.pdf}
%%  \includegraphics[width=\textwidth]{2011-01-18complexity-framing-sketch-tp-cropped.pdf}
  \small
    
      \includegraphics[width=\textwidth]{xkcd-904-sports.png}\\
      {\tiny
      \wordwikilink{http://xkcd.com/904/}{http://xkcd.com/904/}}
      \includegraphics[width=\textwidth]{2014-11-15narrative-hierarchy-sketches-stories_001_base_1200px-tp-5.png}\\
      \includegraphics[width=\textwidth]{2014-11-15narrative-hierarchy-sketches-stories_001_broken-story_1200px-tp-5.png}\\
      \includegraphics[width=\textwidth]{2014-11-15narrative-hierarchy-sketches-stories_001_broken_1200px-tp-5.png}\\
%%       
%%        Sociotechnical algorithms for measuring/predicting decisions, contagion, demographics, weather, \ldots
      
        \wordwikilink{http://nautil.us/issue/5/fame/homo-narrativus-and-the-trouble-with-fame}{Homo narrativus}---we run on stories.
      
        Extraction of metaphors, frames, narratives, and stories from large-scale text.
       
        \wordwikilink{http://www.uvm.edu/~pdodds/fama/2015/06/04/the-narrative-hierarchystories-and-storytelling-on-all-scales/}{The narrative hierarchy: Scalability of stories}.
       
        Adjacent narratives, mistruths, and conspiracy theories.
       
        The taxonomy of human stories.
      
      \centering
      \includegraphics[width=0.2\textwidth]{2015-09-28adjacent-stories001-tp-5.png}
      \includegraphics[width=0.2\textwidth]{2015-09-28adjacent-stories003-tp-5.png}
      \includegraphics[width=0.2\textwidth]{2015-09-28adjacent-stories004-tp-5.png}
      \includegraphics[width=0.2\textwidth]{2015-09-28adjacent-stories002-tp-5.png}\\
      \includegraphics[width=0.2\textwidth]{2015-09-28adjacent-stories006-tp-5.png}
      \includegraphics[width=0.2\textwidth]{2015-09-28adjacent-stories008-tp-5.png}
      \includegraphics[width=0.2\textwidth]{2015-09-28adjacent-stories010-tp-5.png}
      \includegraphics[width=0.2\textwidth]{2015-09-28adjacent-stories011-tp-5.png}
  \small
  %% \textbf{\wordwikilink{http://en.wikipedia.org/wiki/Terry\_Pratchett}{Pratchett} on stories:}
  \textbf{\wordwikilink{http://en.wikipedia.org/wiki/Terry\_Pratchett}{(Sir Terry) Pratchett's}
    \wordwikilink{http://wiki.lspace.org/wiki/Narrativium}{Narrativium}:
    %% and 
    %% \wordwikilink{http://wiki.lspace.org/wiki/Narrative\_Causality}{Narrative Causality}:
  }
    
    \includegraphics[width=\textwidth]{4463841109_f3dbc05754_pratchett.jpg}
      
       
        ``The most common element on the disc, although not
        included in the list of the standard five: earth, fire, air,
        water and surprise. It ensures that everything runs properly
        as a story.''
      
        ``A little narrativium goes a long way: the simpler the story,
        the better you understand it. Storytelling is the opposite of
        reductionism: 26 letters and some rules of grammar are no story
        at all.''
  
    
    
      ``Heroes only win when outnumbered, and things which have a
      one-in-a-million chance of succeeding often do so.''
  \textbf{\wordwikilink{http://exp.lore.com/post/40411963108/kurt-vonneguts-classic-lecture-on-the-shapes-of-stories}{Kurt Vonnegut on the shapes of stories:}}

  \includegraphics[height=0.9\textheight]{tumblr_mft5lpRiy01r2qa6go1_1280_1.jpg}
  \textbf{\wordwikilink{http://exp.lore.com/post/40411963108/kurt-vonneguts-classic-lecture-on-the-shapes-of-stories}{Kurt Vonnegut on the shapes of stories:}}

  \includegraphics[height=0.9\textheight]{tumblr_mft5lpRiy01r2qa6go1_1280_2.jpg}
\insertvideo{oP3c1h8v2ZQ}{}{}{Kurt Vonnegut on the shapes of stories:}
  \textbf{
      \wordwikilink{http://hedonometer.org/books.html}{Online, interactive
        Emotional Shapes of Stories} for 10,000+ books:
    }
    \centering
    \includegraphics[width=\textwidth]{2014-09-15frankenstein.png}
%% 
%%   \textbf{Ian Stewart and Jack Cohen:}
%% 
%%   
%%     Ian Stewart and Jack Cohen are the source of the coinage Pan
%%     narrans, of which they say We are not Homo sapiens, Wise Man. We are
%%     the third chimpanzee. What distinguishes us from the ordinary
%%     chimpanzee Pan troglodytes and the bonobo chimpanzee Pan paniscus, is
%%     something far more subtle than our enormous brain, three times as
%%     large as theirs in proportion to body weight. It is what that brain
%%     makes possible. And the most significant contribution that our large
%%     brain made to our approach to the universe was to endow us with the
%%     power of story. 
%%     {We are \alertb{Pan narrans}, the storytelling ape.}
%%   
%%   
%% 
  \showtarotcards{0.20}{
    overview,
    golden-age-of-reductionism,
    manifesto,
    scaling,
    power-law-size-distributions,
    variable-transformation,
    random-walks,
    rich-get-richer,
    efficient-language,
    law-of-first-digits,
    surprise-of-being-robust-yet-fragile,
    misreading-of-the-book-of-sand,
    tales-of-tails,
    data-poor,
    pouring-data,
    emergence-of-structure,
    emergence-of-destruction,
    emergence-of-thinking,
    emergence-of-stories,
    mechanics-statistical,
    thresholds-of-percolation,
    complex-networks,
    web-elements,
    random-networks,
    strangeness-of-friends,
    small-world-networks,
    theory-six-degrees,
    scale-free-networks,
    confusion-of-contagions,
    disease-of-random-mixing,
    unending-pandemics,
    thresholds-of-the-mind,
    social-wild,
    social-contagion,
    fame,
    contagious-stories,
    paths-of-universality,
    dependent-paths,
    end,
}                                                
\section{Final\ words}
  \small

  \textbf{The absolute basics:}

  \textbf{Modern basic science in three steps:}
      Find interesting/meaningful/important phenomena,
      optionally involving spectacular amounts of data.
    
      Describe what you see.
    
      Explain it.
  
    \alertg{Unlocks our (limited) ability to:}
    Create, predict, and control.
  
    And be good people: \alertg{Share.}
  
    \alertg{Beware your assumptions:}
    Don't use tools/models because they're there,
    or because everyone else does \ldots
\section{For\ your\ consideration}
  \small
  \textbf{This is a thing that could be next:}
    
      CoNKs: The PoCS\\ strikes back:
      \smallskip
      \includegraphics[width=\textwidth]{networksvox-icon.png}\\
      \smallskip
      CSYS/MATH 303: \\
      \wordwikilink{http://www.uvm.edu/~pdodds/teaching/courses/303/}{Complex
        Networks}\\
      \wordwikilink{https://twitter.com/@networksvox)}{@networksvox}
       
        Branching networks (rivers, cardiovascular systems).
       
        Optimal (re)distribution networks (hospitals, coffee shops, airlines, post, Internet).
       
        Structure detection for complex systems.
       
        Moar Contagion.
       
        Random networks-arama.
       
        Distributed Search.
       
        Organizational networks.
       
        Deeper investigations of scale-free networks.
       
        and more \ldots
  
\begin{comment}
  
\section{Flotsam}
  \textbf{Homo narrativus---What's the Story?:}
    
    \includegraphics[width=\textwidth]{xkcd-904-sports.png}\\
    \wordwikilink{http://xkcd.com/904/}{http://xkcd.com/904/}
    
      Mechanisms = 
      Evolution equations, 
      algorithms,
      stories, \ldots
    
      Rollover zing:
      ``Also, all financial analysis.
      And, more directly, D\&D.''
\end{comment}
