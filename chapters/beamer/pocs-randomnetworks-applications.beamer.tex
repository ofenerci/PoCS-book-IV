\section{Analysis\ of\ real\ networks}

\begin{frame}[label=]
  \frametitle{More on building random networks}

  \begin{itemize}
  \item<1-> 
    \alert{Problem:} 
    How much of a real network's structure is non-random?
  \item<2-> 
    Key elephant in the room: the \alert{degree distribution} $P_k$.
  \item<3-> 
    First observe \alert{departure} of $P_k$ from a Poisson distribution.
  \item<4->
    \alertb{Next:}
    measure the departure of a real network 
    with a \alertb{degree frequency $N_k$}
    from a random
    network with the same degree frequency.
  \item<5->
    Degree frequency $N_k$ = observed frequency of degrees for a real network.
  \item<6-> 
    \alert{What we now need to do}: 
    Create an ensemble of random networks
    with degree frequency $N_k$ and then compare.
  \end{itemize}

\end{frame}

\subsection{How\ to\ build\ revisited}

\begin{frame}[label=]
  \frametitle{Building random networks: Stubs}

  \begin{block}{Phase 1:}
    \begin{itemize}
    \item<1-> \alert{Idea:} start with a soup of unconnected nodes
      with \alertb{stubs} (half-edges):
      \begin{center}
        \includegraphics[angle=-90,width=0.7\textwidth]{rn_stubs02_examples}
      \end{center}
    \end{itemize}
    \begin{columns}<2->
      \column{0.5\textwidth}
      \includegraphics[angle=-90,width=\textwidth]{rn_stubs03_examples}
      \column{0.5\textwidth}
      \begin{itemize}
      \item<3-> Randomly select stubs (not nodes!) and connect them.
      \item<4-> Must have an even number of stubs.
      \item<5-> Initially allow \alert{self-} and \alert{repeat} connections.
      \end{itemize}
    \end{columns}

  \end{block}

\end{frame}



\begin{frame}[label=]
  \frametitle{Building random networks: First rewiring}

  \begin{block}{Phase 2:}
    \begin{itemize}
    \item<1-> Now find any (A) self-loops and (B) repeat edges 
      and \alert{randomly rewire} them.
      \begin{center}
        (A)
        \includegraphics[height=0.125\textheight]{rn_stubs04_examples}
        \qquad
        (B)
        \includegraphics[height=0.1\textheight]{rn_stubs05_examples}
      \end{center}
    \item<2-> \alert{Being careful:} we can't change
      the degree of any node, so we can't simply move
      links around.
    \item<3-> \alert{Simplest solution:} 
      randomly rewire \alertb{two edges} at a time.
    \end{itemize}
  \end{block}

\end{frame}

\begin{frame}[label=]
  \frametitle{General random rewiring algorithm}

  \begin{columns}<1->
    \column{0.4\textwidth}
    \includegraphics[width=\textwidth]{rn_stubs06_examples}
    \column{0.6\textwidth}
    \begin{itemize}
    \item<1-> Randomly choose \alert{two edges}.\\
      (Or choose problem edge and a random edge)
    \item<2-> Check to make sure edges are \alertb{disjoint}.
    \end{itemize}
  \end{columns}
  \begin{columns}<3->
    \column{0.4\textwidth}
    \begin{overprint}
      \onslide<1-3 | handout: 0 | trans:0>
      \onslide<4->
      \includegraphics[width=\textwidth]{rn_stubs08_examples}
    \end{overprint}
    \column{0.6\textwidth}
    \begin{itemize}
    \item<3-> Rewire one end of each edge.
    \item<4-> Node degrees \alert{do not change}.
    \item<5-> Works if $e_1$ is a self-loop or repeated edge.
    \item<6-> Same as finding on/off/on/off 4-cycles.
      and rotating them.
    \end{itemize}
  \end{columns}
  
\end{frame}

\begin{frame}[label=]
  \frametitle{Sampling random networks}

  \begin{block}<1->{Phase 2:}
    \begin{itemize}
    \item<1-> Use rewiring algorithm to remove
      all self and repeat loops.
    \end{itemize}
  \end{block}  

  \begin{block}<2->{Phase 3:}
    \begin{itemize}
    \item<2-> \alert{Randomize network} wiring by applying
      rewiring algorithm liberally.
    \item<3-> \alertb{Rule of thumb}:
      \# Rewirings $\simeq$
      10 $\times$ \# edges\cite{milo2003a}.
    \end{itemize}
  \end{block}  
  
\end{frame}

\begin{frame}[label=]
  \frametitle{Random sampling}

  \begin{itemize}
  \item<1-> \alertb{Problem} with only joining up stubs
    is \alert{failure} to randomly sample from all possible networks.
  \item<2-> Example from Milo et al. (2003)\cite{milo2003a}:
    \begin{overprint}
      \onslide<1 | handout: 0 | trans:0>
      \onslide<2>
      \includegraphics[width=0.6\textwidth]{milo2003a_fig2panel1}
      \includegraphics[width=0.3\textwidth]{milo2003a_fig2panel2}
    \end{overprint}
  \end{itemize}

\end{frame}


\begin{frame}[label=]
  \frametitle{Sampling random networks}
  \begin{itemize}
  \item<1->
    What if we have \alert{$P_k$} instead of \alertb{$N_k$}?
  \item<2->
    Must now create nodes before start of the construction algorithm.
  \item<3->
    Generate $N$ nodes by sampling from degree distribution $P_k$.
  \item<4-> 
    Easy to do exactly numerically since $k$ is discrete.
  \item<5-> 
    \alert{Note:} not all $P_k$ will always give nodes
    that can be wired together.
  \end{itemize}
\end{frame}

% an example?

\subsection{Motifs}

\begin{frame}[label=]
  \frametitle{Network motifs}

  \begin{itemize}
  \item<1-> 
    Idea of \alert{motifs}\cite{shen-orr2002a} 
    introduced by Shen-Orr, Alon et al. in 2002.
  \item<2-> 
    Looked at gene expression within full context of
    \alertb{transcriptional regulation networks}.
  \item<3-> 
    Specific example of Escherichia coli.
  \item<4-> 
    Directed network with 577 interactions (edges) and 424 operons (nodes).
  \item<5-> 
    Used network randomization to produce ensemble of alternate networks
    with same degree frequency $N_k$.
  \item<6-> 
    Looked for \alert{certain subnetworks} (\alertb{motifs}) that appeared more or less often than expected
  \end{itemize}
  
\end{frame}

\begin{frame}[label=]
  \frametitle{Network motifs}

    \begin{columns}
      \column{0.5\textwidth}
      \includegraphics[width=\textwidth]{shen-orr2002a_fig1ab}
      \column{0.5\textwidth}
      \includegraphics[width=\textwidth]{shen-orr2002a_fig2a}
    \end{columns}

    \begin{itemize}
    \item<1-> $Z$ only turns on in response to sustained activity in $X$.
    \item<2-> Turning off $X$ rapidly turns off $Z$.
    \item<3-> Analogy to elevator doors.
    \end{itemize}
  
\end{frame}

\begin{frame}[label=]
  \frametitle{Network motifs}

  \includegraphics[width=0.7\textwidth]{shen-orr2002a_fig1cd}

  \begin{itemize}
  \item Master switch.
  \end{itemize}
  
\end{frame}

\begin{frame}[label=]
  \frametitle{Network motifs}

  \includegraphics[width=0.7\textwidth]{shen-orr2002a_fig1ef}
  
\end{frame}

\begin{frame}[label=]
  \frametitle{Network motifs}

  \begin{itemize}
  \item<1-> Note: selection of motifs to test is reasonable
    but nevertheless ad-hoc.
  \item<2-> For more, see work carried out
    by Wiggins et al. at Columbia.
  \end{itemize}
 
\end{frame}


%% \begin{frame}[label=]
%%   \frametitle{}
%% 
%%   Random walks on networks.
%% 
%%   More on random walks \wordwikilink{http://en.wikipedia.org/wiki/Random_walk}{here}.
%% 
%% \end{frame}


