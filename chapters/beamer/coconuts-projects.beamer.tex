

\begin{frame}
  Better background
\end{frame}

\begin{frame}
  Process sandbox from PoCS
\end{frame}

\begin{frame}
  Multilayer networks

\end{frame}

\begin{frame}
  Paul Lessard's project
\end{frame}

\begin{frame}
  Use Github for all projects

  Build projects that are reusable and
  could form part of a bigger educational project
  next year.

  
\end{frame}


\begin{frame}
  Tweets for political debates.
  
  
\end{frame}

\begin{frame}

  \begin{itemize}
  \item 
    ``Accounting for Taste.''
  \item 
    Natural numbers as a bipartite network, implications for studying primes.
  \end{itemize}
  
\end{frame}

%% \begin{frame}
%%   
%%   \begin{block}{All projects will be about stories:}
%%     \begin{itemize}
%%     \item 
%%       Works of literature.
%%     \item 
%%       Fairy tales.
%%     \item 
%%       Frames and micro-narratives in social and traditional media.
%%     \item 
%%       What is the story of stories?
%%       Taxonomy.
%%     \end{itemize}
%%   \end{block}
%%   
%%   
%%   
%%   
%% 
%% \end{frame}




%% add sugarscape


\section{The\ Plan}

\begin{frame}
  \frametitle{Semester projects}


  \begin{block}{Requirements:}
    \begin{enumerate}
    \item<1-> $\approx$ 3--5 minute introduction to project (fifth week)
    \item<2-> 10--15 minute final presentation
    \item<3-> Report: $\ge$ 4--5 pages (single space), journal-style
    \end{enumerate}
  \end{block}

\end{frame}

\section{Narrative\ hierarchy}

\begin{frame}
  \frametitle{Narrative hierarchy}

  \begin{block}{Presenting at many scales:}
    \begin{itemize}
    \item 
      1 to 3 word encapsulation, a sound bite,
    \item 
      a sentence/title,
    \item 
      a few sentences,
    \item 
      a paragraph,
    \item 
      a short paper,
    \item 
      a long paper,
    \item 
      $\ldots$
    \end{itemize}
  \end{block}

\end{frame}

\section{Suggestions\ for\ Projects}


\begin{frame}
  \frametitle{topics}

  \begin{itemize}
  \item<1->    
    Develop and elaborate an \alert{online experiment}
    to study some aspect of \alert{social phenomena}.
  \item<2-> e.g., 
    collective search, 
    cooperation, 
    cheating, 
    influence, 
    creation,
    decision-making, 
    etc.
  \item<3-> Part of the PLAY project.
  \end{itemize}

\end{frame}

\begin{frame}
  \frametitle{topics}

  \begin{block}{Explore and critique Fowler and Christakis et al. work on
    social contagion of:}
  \begin{columns}
    \column{0.7\textwidth}
    \includegraphics[width=\textwidth]{cacioppo2009a_fig1}
    \column{0.3\textwidth}
  \begin{itemize}
  \item 
    Obesity\cite{christakis2007a}
  \item
    Smoking cessation\cite{christakis2008a}
  \item
    Happiness\cite{fowler2008a}
  \item
    Loneliness\cite{cacioppo2009a}
  \end{itemize}
  \end{columns}
  \end{block}
  
  One question: how does the (very) sparse sampling
  of a real social network affect their findings?

\end{frame}

\begin{frame}
  \frametitle{topics}

  \begin{itemize}
  \item<1->    
    Explore ``self-similarity of complex networks''\cite{song2005a,song2006a}\\
    First work by Song \etal, Nature, 2005.
  \item<1->
    See accompanying comment by Strogatz\cite{strogatz2005a}
  \end{itemize}
    \includegraphics[width=0.49\textwidth]{song2005a_fig1a}
    \includegraphics[width=0.49\textwidth]{song2005a_fig1b}

\end{frame}

\begin{frame}
  \frametitle{topics}

  \begin{columns}
    \column{0.6\textwidth}
      \includegraphics[width=\textwidth]{spacelabelslegends.pdf}
    \column{0.4\textwidth}
      \begin{itemize}
      \item 
        Study Hidalgo et al.'s ``The Product Space Conditions the Development of Nations''\cite{hidalgo2007a}
      \item 
        How do products depend on each other, and how does this network evolve?
      \end{itemize}      
  \end{columns}

\end{frame}

\begin{frame}
  \frametitle{topics}

  \begin{columns}
    \column{0.4\textwidth}  
    \includegraphics[width=\textwidth]{gonzalez2008a_fig1ab.pdf}
    \column{0.6\textwidth}  
    \includegraphics[width=\textwidth]{brockmann2006a_fig1b.pdf}
    \begin{itemize}
    \item<1-> 
      Study movement and interactions of people.
    \item<1-> 
      Brockmann \etal\cite{brockmann2006a} ``Where's George'' study.
    \item<1-> 
      Barabasi's group: tracking movement
      via cell phones\cite{gonzalez2008a}.
    \end{itemize}
  \end{columns}

\end{frame}

%% \begin{frame}
%%   \frametitle{topics:}
%% 
%%   Explore Sugarscape.
%% 
%% \end{frame}

\begin{frame}
  \frametitle{topics:}

  \begin{itemize}
  \item 
    Explore ``Catastrophic cascade of failures in interdependent networks''
    Buldyrev et al., Nature 2010\cite{buldyrev2010a}.
  \end{itemize}

  \begin{center}
    \includegraphics[width=\textwidth]{buldyrev2010a_fig1.pdf}
  \end{center}

\end{frame}


\begin{frame}
  \frametitle{topics}

  \begin{columns}
    \column{0.5\textwidth}
    \includegraphics[width=\textwidth]{bohorquez2009a_figS2}
    \column{0.5\textwidth}
    \begin{itemize}
    \item<1->
      Physics/Society---\alert{Wars:} Study work that
      started with Lewis Richardson's ``Variation of the frequency of
      fatal quarrels with magnitude'' in 1949.
    \item<2->
      Specifically explore Clauset et al. 
      and Johnson et al.'s work\cite{clauset2007b,johnson2006a,bohorquez2009a}
      on terrorist attacks and civil wars
    \end{itemize}
  \end{columns}

\end{frame}

\begin{frame}
  \frametitle{Culturomics---explore `book networks'}

  \small{``Quantitative analysis of culture using millions of
    digitized books'' by Michel et al., Science, 2011\cite{michel2011a}}

  \includegraphics[width=0.45\textwidth]{michel2011a_fig3a.pdf} 
  \includegraphics[width=0.45\textwidth]{michel2011a_fig3e.pdf} \\
  \includegraphics[width=0.45\textwidth]{michel2011a_fig3f.pdf}
  \includegraphics[width=0.35\textwidth]{michel2011a_fig4f.pdf}

  {\small
    \wordwikilink{http://www.culturomics.org/}{http://www.culturomics.org/}\\
    \wordwikilink{http://ngrams.googlelabs.com/}{Google Books ngram viewer}
  }

\end{frame}

\begin{frame}
  \frametitle{Study networks and creativity:}

  \begin{columns}
    \column{0.5\textwidth}
    \includegraphics[width=\textwidth]{guimera2005b_fig2}
    \column{0.5\textwidth}
    \begin{itemize}
    \item
      Guimer\`{a} et al., Science 2005:\cite{guimera2005b}
      ``Team Assembly Mechanisms Determine Collaboration Network Structure and Team Performance''
    \item 
      Broadway musical industry
    \item 
      Scientific collaboration in Social Psychology, Economics, Ecology, and Astronomy.
    \end{itemize}
  \end{columns}

\end{frame}

\begin{frame}
  \frametitle{topics}

  \begin{itemize}
  \item<1->
    \alertb{Semantic networks}: explore word-word
    connection networks generated by linking semantically related words.
  \item<2->
    Also: Networks based on morphological or phonetic similarity.
  \item<3-> 
    More general: Explore \alertb{language evolution}
  \item<4->
    One paper to start with: ``The small world of human language''
    by Ferrer i Cancho and Sol\'{e}\cite{ferrericancho2001a}
  \item<5-> 
    Study spreading of
    neologisms.
  \item<6-> 
    Examine new words relative to existing words---is there 
    a pattern?  Phonetic and morphological similarities.
  \item<7-> 
    \alert{Crazy:} Can new words be predicted?
  \item<8-> 
    Use Google Books n-grams as a data source.
  \end{itemize}

\end{frame}



\begin{frame}
  \frametitle{topics}

  \begin{itemize}
  \item<1-> 
    Study the human disease and disease gene networks (Goh \etal, 2007):
  \end{itemize}
  \includegraphics[width=\textwidth]{goh2007a_fig2a}

\end{frame}


\begin{frame}
  \frametitle{topics}

  \begin{itemize}
  \item<1-> 
    Study \alert{collective tagging} (or folksonomy)
  \item<2-> 
    e.g., \href{http://del.icio.us}{del.icio.us}, \href{http://www.flickr.com}{flickr}
  \item<3-> 
    See work by Bernardo Huberman et al. at HP labs.
  \end{itemize}

\end{frame}

\begin{frame}
  \frametitle{topics}

  \begin{itemize}
  \item<1->
    Study games (as in game theory) on
    networks.  
  \item<2->
    For cooperation: Review Martin Nowak's 2006 piece in Science:
    ``Five rules for the evolution of cooperation.''\cite{nowak2006a}
  \item<3-> Much work to explore: voter models, contagion-type models, etc.
  \end{itemize}

\end{frame}


%% \begin{frame}
%%   \frametitle{topics}
%% 
%%   \begin{itemize}
%%   \item<1->
%%     Investigate \alert{safety codes} (building, fire,
%%     etc.).  
%%   \item<2->
%%     What kind of relational networks do safety codes form?  How have they
%%     evolved?
%%   \end{itemize}
%% 
%% \end{frame}

%% \begin{frame}
%%   \frametitle{topics}
%% 
%%   \begin{itemize}
%%   \item<1->
%%     Statistics: Study Peter Hoff's (and
%%     others') work on \alert{latent variables}.  
%%   \item<2-> Idea: explain connection pattern in
%%      a network through \alertb{hidden} individual or dyadic variables
%%   \item<3->
%%     This method has been
%%     applied to the study of international relations networks.
%%   \item<4->
%%     Related and large: explore work on p* networks.
%%   \end{itemize}
%% 
%% \end{frame}

\begin{frame}
  \frametitle{topics:}

  \begin{itemize}
  \item
    Study social networks as revealed
    by email patterns, Facebook connections, tweets, etc.
  \item
    ``Empirical analysis of evolving social networks''
    Kossinets and Watts, Science, Vol 311, 88-90, 2006.\cite{kossinets2006a}
  \item
    ``Inferring friendship network structure by using mobile phone data''
    Eagle, et al., PNAS, 2009.
  \item
    ``Community Structure in Online Collegiate Social Networks''\\
    Traud et al., 2008.\\
    \wordwikilink{http://arxiv.org/abs/0809.0690}{http://arxiv.org/abs/0809.0690}
  \end{itemize}

\end{frame}


%% \begin{frame}
%%   \frametitle{topics}
%% 
%%   \begin{itemize}
%%   \item
%%     Engineering: Read and critically explore
%%     Bejan's book ``Shape and Structure, from Engineering to Nature.''\cite{bejan2000a}
%%   \item
%%     Bejan asks why we see branching network flow structures so often in
%%     Nature---trees, rivers, etc.
%%   \end{itemize}
%% 
%% \end{frame}

\begin{frame}
  \frametitle{topics}

  \begin{itemize}
  \item<1->
    Study Stuart Kauffman's \alertb{$nk$ boolean
    networks} which model regulatory gene networks\cite{kauffman1993a}
  \item<2->
    Explore work by Doyle, Alderson, et al. 
    as well as Pastor-Satorras et al. on the structure 
    of the \alertb{Internet(s)}.
  \item<3->
    \alertb{Review:} Study  work on massive multiplayer online games.  
    How do social networks
    form in these games?\cite{castronova2005a}
  \end{itemize}

\end{frame}

%% \begin{frame}
%%   \frametitle{topics}
%% 
%%   \begin{itemize}
%%   \item
%%     Explore work by Doyle, Alderson, et al. 
%%     as well as Pastor-Satorras et al. on the structure 
%%     of the \alertb{Internet(s)}.
%%   \end{itemize}
%% 
%% \end{frame}


%% \begin{frame}
%%   \frametitle{topics}
%% 
%%   \begin{itemize}
%%   \item
%%     Investigate and review Cybernetics, a
%%     forerunner to Complex Systems.
%%   \end{itemize}
%% 
%% \end{frame}

%% \begin{frame}
%%   \frametitle{topics}
%% 
%%   \begin{itemize}
%%   \item
%%     Read and review Herbert Simon's ``Sciences
%%     of the Artificial'' (or more Simon's work more generally).
%%   \end{itemize}
%% 
%% \end{frame}

%% \begin{frame}
%%   \frametitle{topics}
%% 
%%   \begin{itemize}
%%   \item
%%     Investigate and report on General Systems
%%     Theory.
%%   \end{itemize}
%% 
%% \end{frame}


%% \begin{frame}
%%   \frametitle{topics}
%% 
%%   \begin{itemize}
%%   \item
%%     \alertb{Review:} Study  work on massive multiplayer online games.  
%%     How do social networks
%%     form in these games?\cite{castronova2005a}
%%   \end{itemize}
%% 
%% \end{frame}

%% \begin{frame}
%%   \frametitle{topics}
%% 
%%   \begin{itemize}
%%   \item Study \alert{bipartite networks}: structure and dynamics
%%   \item Rich and interesting both mathematically
%%     and practically speaking.
%%   \end{itemize}
%% 
%% \end{frame}

\begin{frame}
  \frametitle{topics}

  \begin{itemize}
  \item Study scientific collaboration networks.
  \item Mounds of data + good models.
  \item See seminal work by De Solla Price\cite{price1965a}
    plus modern work by Redner, Newman, \etal
  \end{itemize}

\end{frame}

\begin{frame}
  \frametitle{topics}

  \begin{itemize}
  \item <1->
    Study Kearns et al.'s experimental studies
    of people solving classical graph theory problems\cite{kearns2006a}
  \item <1->
    ``An Experimental Study of the Coloring Problem on Human Subject Networks''
  \item <2-> (Possibly) Run some of these experiments for our class.
  \end{itemize}

\end{frame}


%% \begin{frame}
%%   \frametitle{topics}
%% 
%%   \begin{itemize}
%%   \item
%%     Biology: Study leaf network patterns (taken).
%%   \item    
%%     Key on very interesting work by Xia.
%%   \item
%%     Classic Monge problem: how to move stuff
%%     from one place to another.
%%   \item 
%%     Bulk flow versus network flow.
%%   \end{itemize}
%% 
%% \end{frame}



\begin{frame}
  \frametitle{topics}

  \begin{itemize}
  \item
    Vague/Large:

    Study amazon's recommender
    networks.
  \item
    See work by Sornette et al., Huberman et al.
  \end{itemize}

  \includegraphics[width=\textwidth]{beedlebard.pdf}

\end{frame}

%% \begin{frame}
%%   \frametitle{topics}
%% 
%%   \begin{itemize}
%%   \item
%%     Vague/Large:
%% 
%%     Study Netflix's open data\\
%%     (movies and people form a bipartite graph).
%%   \end{itemize}
%% 
%% \end{frame}

\begin{frame}
  \frametitle{topics}

  \begin{itemize}
  \item
    Vague/Large:

    Study network evolution 
    of the Wikipedia's
    content.

    \bigskip
    \includegraphics[width=0.3\textwidth]{wikipedia-tp.pdf}
  \end{itemize}

\end{frame}

\begin{frame}
  \frametitle{topics}

  \begin{itemize}
  \item
    Vague/Large:
    How is the media connected?
    Who copies whom?
  \item
    Possibly use NY Times API.
  \item 
    \url{http://memetracker.org/}
  \item 
    Problem: Need to be able to measure interactions.
  \end{itemize}

\end{frame}


%% \begin{frame}
%%   \frametitle{topics}
%% 
%%   \begin{itemize}
%%   \item
%%     Vague/Large:
%% 
%%  Study social network
%%     evolution in Second Life.
%%   \end{itemize}
%% 
%% \end{frame}

\begin{frame}
  \frametitle{topics}

  \begin{itemize}
  \item
    Vague/Large:

    Anything interesting to do
    with large-scale networks
    in evolution, biology, ethics, religion, history, influence, food,
    international relations, \ldots
  \item
    Data is key.
  \end{itemize}

\end{frame}


% web stuff:
% amazon book linkages
% del.icio.us

% percolation on networks?

% cooperation

% any data set where influence is clearly measured

% games on networks

% ------


%+ rinaldo's paper on impedance, whether or not networks
%will have flow in loops

%+ european paper on search in networks (star versus distributed)
%
%+ newman's work:
%  good delivery
%  friends of friends
%  random networks

%+ uri alon: motifs

%+ small worlds

%+ barabasi---scale free networks

%+ river networks

%+ cardiovascular networks---3/4 stuff

%+ pstar stuff

%+ kleinberg
%+ search in networks



