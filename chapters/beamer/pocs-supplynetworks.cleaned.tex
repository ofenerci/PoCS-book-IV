%%\section{Introduction}

  \textbf{Optimal supply networks}

  \textbf{What's the best way to distribute stuff?}
    
     Stuff = medical services, energy, people, 
     \alert{Some} fundamental network problems:
      
       Distribute stuff from a \alert{single source} to
        \alert{many sinks}
       Distribute stuff from \alert{many sources} to
        many sinks
       \alert{Redistribute} stuff between nodes that
        are both sources and sinks
      
     Supply and Collection are equivalent problems
    
  
  


  \textbf{Single source optimal supply}

  \textbf{Basic question for distribution/supply networks:}
    
    
      How does flow behave given cost:
      $$
      C 
      = 
      \sum_{j} I_j^{\, \gamma} Z_j
      $$
      where \\
      \alertb{$
      I_j 
      $
      = current on link $j$}\\
      and\\
      \alertb{$Z_j$ = link $j$'s impedance}?
    
      Example: $\gamma=2$ for electrical networks.
    
  
  


  \textbf{Single source optimal supply}

  \includegraphics[width=\textwidth]{bohn2007a_fig2}
  
  
  [(a)]
    $\gamma > 1$: \alert{Braided} (bulk) flow 
  [(b)]
    $\gamma < 1$:
    Local minimum: \alert{Branching} flow
  [(c)] 
    $\gamma < 1$:
    Global minimum: \alert{Branching} flow
    
  

  \medskip
  
  {\small From Bohn and Magnasco\cite{bohn2007a}}

  {\small See also Banavar et al.\cite{banavar2000a}}


  \textbf{Single source optimal supply}

  Optimal paths related to transport (Monge) problems:

  \includegraphics[width=0.48\textwidth]{xia2003a_fig1.pdf}
  \includegraphics[width=0.48\textwidth]{xia2003a_fig6.pdf}

  Xia (2003)\cite{xia2003a}


  \textbf{Growing networks:}

  \begin{center}
    \includegraphics[width=0.75\textwidth]{xia2007a_fig1.pdf}
  \end{center}

%%  \includegraphics[width=0.49\textwidth]{xia2007a_fig2}

  Xia (2007)\cite{xia2007a}


  \textbf{Growing networks:}

  \begin{center}
    \includegraphics[width=0.75\textwidth]{xia2007a_fig3.pdf}
  \end{center}

%%  \includegraphics[width=0.49\textwidth]{xia2007a_fig4}

  Xia (2007)\cite{xia2007a}


  \textbf{Single source optimal supply}

  \textbf{An immensely controversial issue...}
    
    
      The form of river networks and blood networks:
      optimal or not?\cite{west1997a,banavar1999a,dodds2001a,dodds2010a}
    
  

  \textbf{Two observations:}
    
    
      Self-similar networks appear everywhere in nature
      for single source supply/single sink collection.
    
      Real networks \alertb{differ} in \alert{details of scaling}
      but reasonably \alertb{agree} in \alert{scaling relations}.
    
  



  \textbf{River network models}

  \textbf{Optimality:}
    
    Optimal channel networks\cite{rodriguez-iturbe1997a}
    Thermodynamic analogy\cite{scheidegger1991a}
    
  
  {versus...}
  \textbf{Randomness:}
    
    Scheidegger's directed random networks
    Undirected random networks
    
  
  

\section{Optimal\ branching}

\subsection{Murray's\ law}

  \textbf{Optimization---Murray's law}

      
    \includegraphics[width=\textwidth]{murrays-law-tp-10}
    
    
     Murray's law (1926) connects branch radii at forks:\cite{murray1926a,murray1926b,murray1927a,labarbera1990a,thompson1961a}
      $$ \boxed{ \alert{r_0^{3} = r_1^{3} + r_2^{3}}} $$
      where $r_0$ = radius of main branch,
      and $r_1$ and $r_2$ are radii of sub-branches.
    
      
     Holds up well for outer branchings of blood networks.
     Also found to hold for trees\cite{murray1927a,mcculloh2003a,mcculloh2004a}.
     See D'Arcy Thompson's ``On Growth and Form'' for background inspiration\cite{thompson1952a,thompson1961a}.
    



%% %%   \textbf{Optimization approaches}
%% 
%%   \textbf{Cardiovascular networks:}
%%     
%%      Murray's law (1926) connects branch radii at forks:\cite{murray1926a,murray1926b,murray1927a,labarbera1990a,thompson1961a}
%%       $$ \boxed{ \alert{r_0^{3} = r_1^{3} + r_2^{3}}} $$
%%       where $r_0$ = radius of main branch\\
%%       and $r_1$ and $r_2$ are radii of sub-branches.
%%      See D'Arcy Thompson's ``On Growth and Form'' for background inspiration\cite{thompson1952a,thompson1961a}.
%%      Calculation assumes 
%%       \wordwikilink{http://en.wikipedia.org/wiki/Hagen-Poiseuille_equation}{Poiseuille flow}.
%%      Holds up well for outer branchings of blood networks.
%%      Also found to hold for trees\cite{murray1927a,mcculloh2003a,mcculloh2004a}.
%%      Use hydraulic equivalent of Ohm's law:
%%       $$
%%       \Delta p = \Phi Z  \Leftrightarrow V = IR
%%       $$
%%       where $\Delta p$ = pressure difference, $\Phi$ = flux.
%%     
%%   
%% 
%%  \textbf{Optimization---Murray's law}

    
     Use hydraulic equivalent of Ohm's law:
      $$
      \Delta p = \Phi Z  \Leftrightarrow V = IR
      $$
      where $\Delta p$ = pressure difference, $\Phi$ = flux.
              
        \includegraphics[width=\textwidth]{poiseuille-flow-tp-10}
        
        
        
          Fluid mechanics: 
          \wordwikilink{http://en.wikipedia.org/wiki/Hagen-Poiseuille_equation}{Poiseuille impedance}
          for smooth 
          \wordwikilink{http://en.wikipedia.org/wiki/Hagen-Poiseuille_equation}{Poiseuille flow}
          in a tube
          of radius $r$ and length $\ell$:
          $$ Z = \frac{8\eta \ell}{\pi r^4} $$
        
          
      $\eta$ = 
      \wordwikilink{http://en.wikipedia.org/wiki/Dynamic_viscosity}{dynamic viscosity}
      (units: $ML^{-1}T^{-1}$).
     
      Power required to overcome impedance: 
      $$ P_{\textnormal{drag}} = \Phi \Delta p  = \Phi^2 Z. $$
     
      Also have rate of energy expenditure in maintaining blood
      given metabolic constant $c$:
      $$ P_{\textnormal{metabolic}} = c r^2 \ell  $$
    


  \textbf{Optimization---Murray's law}

  \textbf{Aside on $P_{\textnormal{drag}}$}
  
   
    Work done = $F \cdot d$ = energy transferred by force $F$
   
    Power = $P$ = rate work is done = $F \cdot v$
   $\Delta p$ = Force per unit area
   $\Phi$ = Volume per unit time \\ = cross-sectional area $\cdot$ velocity
   So $\Phi \Delta p$ = Force $\cdot$ velocity
  
  



  \textbf{Optimization---Murray's law}

  \textbf{Murray's law:}
    
     Total power (cost):
      $$ 
      P = P_{\textnormal{drag}} + P_{\textnormal{metabolic}}
      {=
      \Phi^2 \frac{8\eta \alert{\ell}}{\pi \alert{r^4}}
      + c \alert{r^2 \ell}}
      $$
     Observe power increases linearly with $\ell$
     But $r$'s effect is nonlinear: 
      
        
        increasing $r$
        makes flow easier \alert{but increases metabolic cost} (as $r^2$)
      
        decreasing $r$
        decrease metabolic cost \alert{but impedance goes up} (as $r^{-4}$)
      
    
  



  \textbf{Optimization---Murray's law}

  \textbf{Murray's law:}
    
     Minimize $P$ with respect to $r$:
      $$
      \partialdiff{P}{r}
      = 
      \partialdiff{}{r} 
      \left( 
        \Phi^2 \frac{8\eta {\ell}}{\pi {r^4}}
      + c {r^2 \ell}
      \right)
    $$
      {
        $$
        = 
        -4 \Phi^2 \frac{8\eta {\ell}}{\pi {r^5}}
        + c {2r \ell}
        {\alert{=0}}
      $$
       Rearrange/cancel/slap:
        $$
        \alert{\Phi^2} = \frac{c \pi r^6}{16 \eta} {= k^2 \alert{r^6}}
        $$
        {where $k$ = constant.}
        
      }
    
  



  \textbf{Optimization---Murray's law}

  \textbf{Murray's law:}
    
     So we now have:
      $$
      \Phi = k r^3 
      $$
     
      Flow rates at each branching have to add up
      (else our organism is in serious trouble...):
      $$
      \Phi_0 = \Phi_1 + \Phi_2
      $$
      where again 0 refers to the  main branch and 1 and 2 refers
      to the offspring branches
    
      All of this means we have a groovy cube-law:
      $$ 
      \boxed{\alert{r_0^3 = r_1^3 + r_2^3}}
      $$
      
    
  


\subsection{Murray\ meets\ Tokunaga}


  \textbf{Optimization}

  \textbf{Murray meets Tokunaga:}
    
     
      $\Phi_\om$ = volume rate of flow into an order
      $\om$ vessel segment
     
      Tokunaga picture:
      $$ 
      \Phi_\om
      = 
      2 \Phi_{\om - 1}
      +
      \sum_{k=1}^{\om-1}
      T_k
      \Phi_{\om-k}
      $$
    
      Using $\phi_\om = k r_{\om}^{3}$
      $$
      r_\om^3
      = 
      2 r_{\om - 1}^{3}
      +
      \sum_{k=1}^{\om-1}
      T_k
      r_{\om-k}^{3}
      $$
    
      Find Horton ratio for vessel radius $R_r = r_{\om}/r_{\om-1}$...
    
  



  \textbf{Optimization}

  \textbf{Murray meets Tokunaga:}
    
     
      Find $R_r^{\, 3}$ satisfies same equation as $R_n$ and $R_v$\\
      ($v$ is for volume):
      $$
      \boxed{ \alert{R_r^3 = R_n = R_v} }
      $$
    
      Is there more we could do here to constrain the Horton
      ratios and Tokunaga constants?
    
  



  \textbf{Optimization}

  \textbf{Murray meets Tokunaga:}
    
     
      Isometry: $V_\om \propto \ell_\om^{\, 3}$
    
      Gives 
      $$\boxed{\alert{R_\ell^3 = R_v = R_n}}$$
     
      We need one more constraint...
    
      West et al (1997)\cite{west1997a} achieve similar
      results following Horton's laws.
    
      So does Turcotte et al. (1998)\cite{turcotte1998a}
      using Tokunaga (sort of).
    
  


%% \changelogo{.2}{virtualvessels4.pdf}

\section{Single\ Source}

\subsection{Geometric\ argument}

%% \subsection{History}
%% 
%% %%   \textbf{History in brief}
%%   
%%   \textbf{Seeking optimal universality:}
%%     
%%      Two major real-world branching networks
%%       
%%        \alert{Blood networks}
%%        \alertb{River networks}
%%       
%%      Blood networks argued
%%       to lead to 
%%       $$B \propto M^{\alpha}$$
%%       where
%%       
%%        
%%         $B$ = basal metabolic rate and $M$ = body mass
%%        
%%       $\alpha=2/3$ or $3/4$ or something else...\cite{kleiber1961a,west1997a,banavar1999a,dodds2001d}
%%       
%%      River basins may or may not scale allometrically.
%%     
%%       Recall: Hack's law\cite{hack1957a}
%%       $$\ell \propto a^h$$
%%       If $h>1/2$ then basins elongate.
%%     
%%     
%%   
%%   
%% %% 
%% \subsection{Minimal volume calculation}
%% 
%% %%   \textbf{Geometric argument}
%% 
%%   
%%    
%%     Consider \alert{one source supplying many sinks} in a $d$ dimensional volume
%%   
%%     Material draw by sinks is invariant.
%%    
%%     See network as a bundle of virtual vessels:
%%     %%               
%%               
%%       \begin{center}
%%         \includegraphics[angle=-90,width=0.8\textwidth]{virtualvessels4.pdf}
%%       \end{center}
%%       %%    
%%     \alert{The simplest question}: how does number  of sustainable
%%     sinks $N_{\textnormal{sinks}}$
%%     scale with volume $V$ for the most efficient network design?
%%    
%%     Or: what is highest $\alpha$ for $N_{\textnormal{sinks}} \propto V^{\alpha}$?
%%    
%%     Covered in PoCS CSYS 300: we will recap and refine here.
%%   
%% 
%% %   
%% %    Assume some cap on flow speed of material, $v_{\textnormal{max}}$
%% 
%% %% 
%% %%   \textbf{Geometric argument}
%% 
%%   
%%    Consider families of systems that grow allometrically.
%%    Family = a basic shape $\Omega$ indexed by volume $V$.
%%   \begin{center}
%%     \includegraphics[angle=-90,width=0.8\textwidth]{shapescaling}    
%%   \end{center}
%%   \bigskip
%%    Orient shape to have dimensions $L_1 \times L_2 \times  ... \times L_d$
%%    In 2-d,
%%     $L_1 \propto A^{\gamma_1}$ and $L_2 \propto A^{\gamma_2}$
%%     where $A$ = area.
%%    In general, have $d$ lengths which scale
%%     as $L_i \propto V^{\gamma_i}$.
%%    For above example, width grows faster than
%%     height: $\gamma_1 > \gamma_2$.
%%   
%% 
%% %% 
%% %%   \textbf{Geometric argument}
%% 
%%   \textbf{Some generality:}
%%     
%%      Consider $d$ dimensional spatial regions living in 
%%       $D$ dimensional ambient spaces.  {Notation: \alert{$\volume{V}$}.}
%%      River networks: \alertb{$d=2$ and $D=3$}
%%      Cardiovascular networks: \alertb{$d=3$ and $D=3$}
%%     
%%       \alert{Star-convexity of $\volume{V}$:} A spatial
%%       region is star-convex if from at least one point, all other
%%       points in the region can be reached by travelling along straight lines
%%       while remaining within the region.
%%     
%%       Assume source can be located at a point which has direct line of
%%       sight to all sources.
%%     
%%       We can generalize to a much broader class of shapes...
%%     
%%     
%%   
%% 
%% %% 
%% 
%% %%   \textbf{Geometric argument}
%% 
%%   
%%    Reminder of best and worst configurations
%%     \begin{center}
%%       \includegraphics[angle=-90,width=0.8\textwidth]{efficientnetworks5.pdf}
%%     \end{center}
%%     \bigskip
%%    \alert{Basic idea:}
%%     Minimum volume of material in system $V_{\textnormal{net}} \propto$ sum of distance
%%     from the source to the sinks.
%%    See what this means for sink density $\rho$ if sinks do not
%%     change their feeding habits with overall size.
%%   
%% 
%% 
%%  \textbf{Geometric argument}

  \textbf{
      ``Optimal Form of Branching Supply and Collection Networks''\cite{dodds2010a}
    }
    P.\ S.\ Dodds, Phys. Rev. Lett., \textbf{104}, 048702, 2010.
    
     
      Consider \alert{one source} supplying \alert{many sinks} in a 
      volume $V$ \alertb{$d$-dim.} region
      in a \alertb{$D$-dim.} ambient space.
    
      Assume \alertb{sinks are invariant}.
    
      Assume \alert{$\rho = \rho(V)$}, i.e., $\rho$ may vary with volume $V$.
     
      See network as a bundle of virtual vessels:
      \begin{center}
                  
          
          \includegraphics[angle=-90,width=0.8\textwidth]{virtualvessels4.pdf}
              \end{center}
     
      \alert{Q:} how does the number of sustainable
      sinks $N_{\textnormal{sinks}}$
      scale with volume $V$ for the most efficient network design?
     
      \alert{Or:} what is the highest $\alpha$ for $N_{\textnormal{sinks}} \propto V^{\alpha}$?
    
  


  \textbf{Geometric argument}

  
   Allometrically growing regions:
%   Family = a basic shape $\Omega$ indexed by volume $V$.
  \begin{center}
    \includegraphics[width=0.8\textwidth]{shapescaling-unrotated}    
  \end{center}
  \bigskip
%   Orient shape to have dimensions $L_1 \times L_2 \times  ... \times L_d$
%   In 2-d,
%    $L_1 \propto A^{\gamma_1}$ and $L_2 \propto A^{\gamma_2}$
%    where $A$ = area.
   Have $d$ length scales which scale
    as 
    {
      $$
      \alertb{L_i} \propto \alertb{V}^{\alertb{\gamma_i}}
      \mbox{\ where $\gamma_1 + \gamma_2 + \ldots + \gamma_d = 1$.}
      $$
    }
   
    For \alert{isometric} growth, $\gamma_i = 1/d$.
  
    For \alert{allometric} growth, 
    we must have at least two of the $\{\gamma_i\}$ being different
%   For above example, width grows faster than
%    height: $\gamma_1 > \gamma_2$.
  



  \textbf{Geometric argument}

  
   Best and worst configurations (Banavar et al.)
    \begin{center}
      \includegraphics[angle=-90,width=0.8\textwidth]{efficientnetworks5.pdf}
    \end{center}
    \bigskip
   \alert{Rather obviously:}\\
    $\min V_{\textnormal{net}} \propto \sum$
    distances
    from source to sinks.

%   See what this means for sink density $\rho$ if sinks do not
%    change their feeding habits with overall size.
  


  \textbf{Minimal network volume:}

  Real supply networks are close to optimal:

  \includegraphics[width=\textwidth]{gastner2006a_fig1.pdf}

  \bigskip

  {\small (2006)
    Gastner and Newman\cite{gastner2006a}:
    ``Shape and efficiency in spatial distribution networks'' }


  \textbf{Minimal network volume:}

  \textbf{We add one more element:}
    \includegraphics[width=\textwidth]{shapes-virtualv-4c.pdf}
    
     Vessel cross-sectional area
      may vary with distance from the source.
    
      Flow rate increases as cross-sectional area decreases.
     e.g., a collection network may
      have vessels tapering as they approach
      the central sink.
    
      Find that vessel volume $v$ must scale
      with vessel length $\ell$ to affect overall
      system scalings.
    
  

  \textbf{Minimal network volume:}

  \textbf{Effecting scaling:}
    \includegraphics[width=\textwidth]{shapes-virtualv-4c.pdf}
    
    
      Consider vessel radius $r \propto (\ell+1)^{-\epsilon}$,
      tapering from $r=r_{\max}$ where $\epsilon \ge 0$.
    
      Gives
      $
      v \propto \ell^{1-2\epsilon}
      $ if $\epsilon < 1/2$
    
      Gives
      $
      v \propto 1 - \ell^{-(2\epsilon-1)} \rightarrow 1$ for large $\ell$
      if $\epsilon > 1/2$
    
      Previously, we looked at $\epsilon=0$ only.
    
  

  \textbf{Minimal network volume:}

  For $0 \le \epsilon < 1/2$, approximate network volume by integral over region:
  $$ 
  \alertb{\min V_{\textnormal{net}}}  \propto 
  \int_{\volume{V}} \alertb{\rho} \, ||\vec{x}||^{1-2\epsilon} \, \dee{\vec{x}} 
  $$
  %%   {
  %%     $$
  %%     \rightarrow 
  %%     \rho V^{1+\gamma_{\max}}
  %%     \int_{\volume{c}} (c_1^{2} u_1^2 + \ldots + c_k^{2} u_k^2 )^{(1-2\epsilon)/2}
  %%     \dee{\vec{u}}
  %%     $$
  %%   }
  {\insertassignmentquestion{03}{3}{1}}
  {
    $$
    \propto
    \alert{ \rho V^{1+\gamma_{\max}(1-2\epsilon)} } 
    \
    \mbox{where}
    \
    \gamma_{\max} = \max_{i} \gamma_i.
    $$
  }
  {
    For $\epsilon > 1/2$, find simply that 
    $$
    \alertb{\min V_{\textnormal{net}}}  
    \propto 
    \rho V
    $$
  }
  
  
    So if supply lines can taper fast enough and without
    limit, minimum network volume can be made negligible.
%  
%    \alert{The problem:} must eventually reach a limiting speed
%    or size (e.g., blood velocity and cells).
  

  \textbf{Geometric argument}

  \textbf{For $0 \le \epsilon < 1/2$:}
    
     
      $
      \boxed{\alert{
          \min V_{\textnormal{net}} 
          \propto
          \rho V^{1+\gamma_{\max}(1-2\epsilon)} 
        }}
      $
     
      If scaling is \alertb{isometric}, we have $\gamma_{\max} = 1/d$:
      $$
      \min V_{\textnormal{net/iso}} 
      \propto
      \rho V^{1+(1-2\epsilon)/d}
      $$
     
      If scaling is \alertb{allometric}, we have
      $\gamma_{\max} = \gamma_{\textnormal{allo}} > 1/d$:
      and 
      $$
      \min V_{\textnormal{net/allo}} 
      \propto
      \rho V^{1+(1-2\epsilon)\gamma_{\textnormal{allo}}}
      $$
     
      Isometrically growing volumes 
      \alert{require less network volume} 
      than allometrically growing volumes:
      $$
      \frac{\min V_{\textnormal{net/iso}}}{\min V_{\textnormal{net/allo}}} \rightarrow 0 
      \mbox{\ as $V \rightarrow \infty$}
      $$
        
    
  

  \textbf{Geometric argument}

  \textbf{For $\epsilon > 1/2$:}
    
     
      $
      \boxed{\alert{
          \min V_{\textnormal{net}} 
          \propto
          \rho V
        }}
      $
     
      Network volume scaling is now independent 
      of overall shape scaling.
    
  

  \medskip

  \textbf{Limits to scaling}
    
     
      Can argue that $\epsilon$ must effectively be 0
      for real networks over large enough scales.
     
      Limit to how fast material can move,
      and how small material packages can be.
     
      e.g., blood velocity and blood cell size.
    
  


\subsection{Real\ networks}
%% \subsection{Blood\ networks}

  \textbf{Blood networks}

  
   Velocity at capillaries and 
    aorta approximately constant across body size\cite{weinberg2006a}: 
    $\epsilon = 0$.
   \alert{Material costly} $\Rightarrow$ expect lower optimal bound of 
    $V_{\textnormal{net}} \propto \rho V^{(d+1)/d}$ to be followed closely.
  
    For cardiovascular networks, \alert{$d=D=3$}.
  
    Blood volume scales linearly with body
    volume\cite{stahl1967a}, $V_{\textnormal{net}} \propto V$.
  
    Sink density must $\therefore$ decrease as volume increases:
    $$
    \alertb{\rho \propto V^{-1/d}}.
    $$
  
    Density of suppliable sinks \alert{decreases} with organism size.
        



  \textbf{Blood networks}

  
   Then $P$, the rate of overall energy 
    use in $\Omega$, can at most scale with volume as
    $$
    P \propto \rho V 
    {
      \propto \rho \, M
    }
    {
      \propto M^{\, (d-1)/d}
    }
    $$
   
    For $d=3$ dimensional organisms, we have 
    $$\alertb{\boxed{ P \propto M^{\, 2/3}}}$$
   
    Including other constraints may raise scaling exponent
    to a higher, less efficient value.
  
    \alertb{Exciting bonus:} 
    Scaling obtained by the supply network story and the surface-area law
    \alert{only match} for isometrically growing shapes.\\
    \insertassignmentquestion{03}{3}{3}
      


  \textbf{Recap:}

  
  
   
    The exponent $\alpha = 2/3$ works for all birds and
    mammals up to 10--30 kg
   
    For mammals $>$ 10--30 kg, maybe we have a new scaling regime
   
    Economos: limb length break in scaling around 20 kg
   
    White and Seymour, 2005: unhappy with large herbivore measurements.
    Find $\alpha \simeq 0.686 \pm 0.014$
  
  


  \textbf{General unhappiness:}

  \textbf{Everyone is confused:}
    
    
      White et al., Ecology, 2007:
      ``Allometric exponents do not support a universal metabolic allometry''\cite{white2007a}
    
      Savage et al., PLoS Computational Biology:
      ``Sizing Up Allometric Scaling Theory.''\cite{savage2008a}
     
      Banavar et al., PNAS, 2010:
      ``A general basis for quarter-power scaling in animals.''\cite{banavar2010a}
  
  



%% %%   \textbf{Prefactor:}
%% 
%%   \textbf{Stefan-Boltzmann law:}
%%     
%%     
%%       $$\diff{E}{t} = \sigma S T^4$$
%%       where $S$ is surface and $T$ is temperature.
%%      
%%       Very rough estimate of prefactor based on scaling
%%       of normal mammalian body temperature and surface
%%       area $S$:
%%       $$B \simeq 10^5M^{2/3} \mbox{erg/sec}.$$
%%     
%%       Measured for $M \leq 10$ kg:
%%       $$B=2.57\times 10^5M^{2/3} \mbox{erg/sec}.$$
%%     
%%   
%% 
%% 
%% \subsection{River\ networks}

  \textbf{River networks}

  
   View river networks as collection networks.
   Many sources and one sink.
   $\epsilon$?
   Assume $\rho$ is constant over time and $\epsilon=0$:
    $$V_{\textnormal{net}} \propto \rho V^{(d+1)/d} = \mbox{constant} \times V^{\, 3/2} $$
   Network volume grows faster than
    basin `volume' (really area).
   \alert{It's all okay:}\\ 
    Landscapes are $d$=2 surfaces living in $D$=3 dimensions.
  
    Streams can grow not just in width but in depth...
  
    If $\epsilon > 0$, $V_{\textnormal{net}}$ will grow more slowly
    but 3/2 appears to be confirmed from real data.
  


%% %%   \textbf{Hack's law}
%% 
%%   
%%    Volume of water in river network can be calculated 
%%     by adding up basin areas
%%    Flows sum in such a way that 
%%     $$ V_{\textnormal{net}} = \sum_{\mbox{\scriptsize all pixels}} a_{\mbox{\scriptsize pixel $i$}} $$
%%    Hack's law again:
%%     $$
%%     \ell \sim a^{\, h}
%%     $$
%%    
%%     Can argue     
%%     $$ V_{\textnormal{net}} \propto V_{\textnormal{basin}}^{1+h} = a_{\textnormal{basin}}^{1+h}$$
%%     where 
%%     $h$ is Hack's exponent.
%%    
%%     $\therefore$ minimal volume calculations gives 
%%     $$
%%     \boxed{
%%       h=1/2
%%     }
%%     $$
%%   
%% 
%% %% 
%% %%   \textbf{Real data:}
%% 
%%   %%     
%%     
%%      Banavar et al.'s approach\cite{banavar1999a} is okay 
%%       because $\rho$ \alertb{really is constant}.
%%      \alert{The irony:} shows optimal basins are isometric
%%      Optimal Hack's law: $\msl \sim a^{h}$ with
%%       $h=1/2$ 
%%      {(Zzzzz)}
%%     
%%     
%%     %%       
%%       \includegraphics[width=\textwidth]{banavar1999fig2.png}\\
%%       {\small From Banavar et al. (1999)\cite{banavar1999a}}
%%     %%   %% %% 
%% %%   \textbf{Even better---prefactors match up:}
%% 
%%   \begin{center}
%%     \includegraphics[width=0.8\textwidth]{figwatervolume02_noname.pdf}
%%   \end{center}
%% 
%% 

\changelogo{.18}{gastner2006c_fig6b-tp-1}

\section{Distributed\ Sources}

%% \subsection{Facility location}

  \textbf{Many sources, many sinks}
  
  \textbf{How do we distribute sources?}
    
     Focus on 2-d (results generalize to higher dimensions).
     Sources = hospitals, post offices, pubs, ...
     \alert{Key problem:} How do we cope with uneven population densities?
     Obvious: if density is uniform then sources are best distributed
      \alert{uniformly}.
     Which lattice is optimal? {The \alert{hexagonal lattice}}
     \alert{Q2:} Given population density is uneven, what do we do?
     We'll follow work by Stephan (1977, 1984)\cite{stephan1977a,stephan1984a},
      Gastner and Newman (2006)\cite{gastner2006c}, 
      Um \etal (2009)\cite{um2009a} and work cited by them.
    
  



  \textbf{Optimal source allocation}

  \textbf{Solidifying the basic problem}
    
     Given a region with some population distribution $\rho$, most likely uneven.
     Given resources to build and maintain $N$ facilities.
     \alert{Q:} How do we locate these $N$ facilities so as to
      \alert{minimize the average distance} between an \alertb{individual's residence} and 
      the \alertb{nearest facility}?
    
  

  \textbf{Optimal source allocation}

  {\centering
    \includegraphics[width=0.9\textwidth]{gastner2006c_fig1}
  }

  {\small 
    From Gastner and Newman (2006)\cite{gastner2006c}\\
    
     Approximately optimal location of 5000 facilities.
     Based on 2000 Census data.
     Simulated annealing + Voronoi tessellation.
    
  }


  \textbf{Optimal source allocation}

  \begin{center}
    \includegraphics[width=0.7\textwidth]{gastner2006c_fig2}
  \end{center}
  {\small
    From Gastner and Newman (2006)\cite{gastner2006c}
  }
  
   Optimal facility density $\rhofac$ vs.\ population density $\rhopop$.
   Fit is $\rhofac \propto \rhopop^{0.66}$ with $r^2 = 0.94$.
   Looking good for a 2/3 power...
  
  


\subsection{Size-density\ law}

  \textbf{Optimal source allocation}

  \textbf{Size-density law:}
    
    
    $$
    \boxed{\alert{\rhofac \propto \rhopop^{2/3}}}
    $$
    
      Why?
    
      Again: Different story to branching networks where
      there was either one source or one sink.
    
      Now sources \& sinks are distributed 
      throughout region...
    
  


  \textbf{Optimal source allocation}

    
     
      We first examine Stephan's treatment (1977)\cite{stephan1977a,stephan1984a}
    
      \alertb{``Territorial Division: The Least-Time Constraint
        Behind the Formation of Subnational Boundaries''} (Science, 1977)
    
      Zipf-like approach: invokes \alert{principle of minimal effort}.
    
      Also known as the Homer principle.
    


  \textbf{Optimal source allocation}

  
   
    Consider a region of area $A$ and population $P$ with
    a single functional center that everyone needs to access
    every day.
  
    Build up a general cost function based on time expended
    to \alert{access and maintain center}.
  
    Write \alert{average travel distance} to center as $\bar{d}$ and 
    assume \alert{average speed of travel} is $\bar{v}$.
  
    Assume \alertb{isometry}: average travel distance $\bar{d}$ will be on the length
    scale of the region which is $\sim$ \alertb{$A^{1/2}$}
  
    Average time expended per person in accessing facility
    is therefore
    \alertb{
      $$
      \bar{d}/\bar{v} = c A^{1/2} / \bar{v}
      $$
    }
    where $c$ is an unimportant shape factor.
  


  \textbf{Optimal source allocation}

  
   Next assume facility requires regular maintenance (person-hours per day)
   Call this quantity $\tau$
   If burden of mainenance is shared then average cost per person
    is \alert{$\tau/P$} where $P$ = population.
   Replace $P$ by $\rhopop A$ where $\rhopop$ is density.
   Total average time cost per person:
    $$
    T = \bar{d}/\bar{v} + \tau/(\rhopop A) 
    {= c \alert{A^{1/2}}/\bar{v} + \tau/(\rhopop \alert{A}).}
    $$
   Now Minimize with respect to $A$...
  


  \textbf{Optimal source allocation}

  
   Differentiating...
    $$
    \partialdiff{T}{A} = 
    \partialdiff{}{A} \left( c A^{1/2}/\bar{v} + \tau/(\rhopop A) \right)
    $$
    $$
    {
      =
      \frac{c}{2\bar{v} A^{1/2}}
      -\frac{\tau}{\rhopop A^2}
    }
    {
      \alert{ = 0 }
    }
    $$
   Rearrange:
    $$
    A = 
    \left(
      \frac{2 \bar{v} \tau}
      {c \rhopop}
    \right)^{2/3}
    {
      \propto \rhopop^{-2/3}
    }
    $$
   \# facilities per unit area $\rhofac$:
    $$ 
    \rhofac
    \propto
    \alert{A^{-1}  \propto \rhopop^{2/3}}
    $$
   Groovy...
    
  


  \textbf{Optimal source allocation}

  \textbf{An issue:}
    
     Maintenance ($\tau$) is assumed to be 
      \alert{independent} of population
      and area ($P$ and $A$)
    
  
  

  \textbf{Optimal source allocation}

  
   
    Stephan's online book\\
    \alert{``The Division of Territory in Society''}
    is
    \wordwikilink{http://www.edstephan.org/Book/contents.html}{here}.
   
    (It used to be
    \wordwikilink{http://www.ac.wwu.edu/~stephan/Book/contents.html}{here}.)
   
    The 
    \wordwikilink{http://www.edstephan.org/Book/chap0/0.html}{Readme} 
    is well worth reading (1995).
  

  %% winner of the first Zipf award!
  %% George Kingsley Zipf 
  %% Memorial Award
  %% 1984 Population Association of America

\subsection{Cartograms}

  \textbf{Cartograms}

  Standard world map:
  \includegraphics[width=\textwidth]{newman_world1024x512.png}


  \textbf{Cartograms}

  Cartogram of countries `rescaled' by population:
  \includegraphics[width=\textwidth]{newman_population1024x512.png}\\
  \includegraphics[width=0.25\textwidth]{newman_world1024x512.png}

  \textbf{Cartograms}

  \textbf{Diffusion-based cartograms:}
    
     Idea of cartograms is to \alert{distort areas} to 
      more accurately represent
      some local density $\rhopop$ (e.g. population).
     Many methods put forward---typically involve
      some kind of physical analogy to \alert{spreading or repulsion}.
     Algorithm due to Gastner and Newman (2004)\cite{gastner2004a}
      is based on \alertb{standard diffusion}:
      $$ 
      \nabla^2 \rhopop - \partialdiff{\rhopop}{t} = 0. 
      $$
     Allow density to diffuse and trace the 
      movement of individual elements and boundaries.
     Diffusion is constrained by boundary condition
      of surrounding area having density $\bar{\rhopop}$.
    
  


  \textbf{Cartograms}

  Child mortality:
  \includegraphics[width=\textwidth]{newman_childmort1024x512.png}


  \textbf{Cartograms}

  Energy consumption:
  \includegraphics[width=\textwidth]{newman_energyconsump1024x512.png}

  \textbf{Cartograms}

  Gross domestic product:
  \includegraphics[width=\textwidth]{newman_gdp1024x512.png}

  \textbf{Cartograms}

  Greenhouse gas emissions:
  \includegraphics[width=\textwidth]{newman_greenhouse1024x512.png}

  \textbf{Cartograms}

  Spending on healthcare:
  \includegraphics[width=\textwidth]{newman_healthcare1024x512.png}

  \textbf{Cartograms}
  
  People living with HIV:
  \includegraphics[width=\textwidth]{newman_hiv1024x512.png}


  \textbf{Cartograms}

  
   The preceding sampling of Gastner \& Newman's cartograms
    lives \wordwikilink{http://www-personal.umich.edu/~mejn/cartograms/}{here}.
  
    A larger collection can be found
    at \wordwikilink{http://www.worldmapper.org/}{worldmapper.org}.

    \bigskip

    \includegraphics[width=0.5\textwidth]{worldmapper.png}
  


  \textbf{Size-density law}

  \includegraphics[width=\textwidth]{gastner2006c_fig3}

  
    \alert{Left:} population density-equalized cartogram.
    \alert{Right:} (population density)$^{2/3}$-equalized cartogram.
    Facility density is uniform for $\rhopop^{2/3}$ cartogram.
  
  {\small
    From Gastner and Newman (2006)\cite{gastner2006c}
  }

%% %%   
%% 
%%   \includegraphics[width=\textwidth]{gastner2006c_fig4}
%% 
%%   From Gastner and Newman (2006)\cite{gastner2006c}
%% 
  \textbf{Size-density law}

  \includegraphics[width=\textwidth]{gastner2006c_fig5}

  {\small
    From Gastner and Newman (2006)\cite{gastner2006c}
  }
  
   Cartogram's Voronoi cells are somewhat hexagonal.
  
  

\subsection{A\ reasonable\ derivation}

  \textbf{Size-density law}

  \textbf{Deriving the optimal source distribution:}
    
     \alert{Basic idea:} Minimize the average distance
      from a random individual to the nearest facility.\cite{gastner2006c}
     Assume given a fixed population density $\rhopop$ defined
      on a spatial region $\Om$.
     Formally, we want to find the locations of 
      \alert{$n$ sources} $\{\vec{x}_1,\ldots,\vec{x}_n\}$
      that minimizes the \alert{cost function}
      $$
      F(\{\vec{x}_1,\ldots,\vec{x}_n\})
      =
      \int_{\Om}
      \alert{\rhopop(\vec{x})}
      \,
      \alertb{\min_{i}
      || \vec{x} - \vec{x}_i ||}
      \dee{\vec{x}}.
      $$
     Also known as the p-median problem.
     Not easy...  {in fact this one is an NP-hard problem.\cite{gastner2006c}}
     Approximate solution originally due to
      Gusein-Zade\cite{gusein-zade1982a}.
    
  


  \textbf{Size-density law}

  \textbf{Approximations:}
    
     For a given set of source placements $\{\vec{x}_1,\ldots,\vec{x}_n\}$,
      the region $\Om$ is divided up into 
      \wordwikilink{http://en.wikipedia.org/wiki/Voronoi_diagram}{Voronoi cells},
      one per source.
    
      Define \alert{$A(\vec{x})$} as the \alert{area} of the 
      Voronoi cell containing $\vec{x}$.
     As per Stephan's calculation, estimate
      typical distance from $\vec{x}$ to the nearest source (say $i$)
      as 
      $$
      \alertb{c_i A(\vec{x})^{1/2}}
      $$
      where $c_i$ is a shape factor for the $i$th Voronoi cell.
     
      Approximate $c_i$ as a constant $c$.
    
  


  \textbf{Size-density law}

  \textbf{Carrying on:}
    
     The cost function is now
      $$
      F
      =
      c \int_{\Om}
      \alertb{\rhopop(\vec{x})}
      \alertb{ A(\vec{x})^{1/2}}
      \dee{\vec{x}}.
      $$
     We also have that the \alert{constraint} that
      Voronoi cells divide up the overall area
      of $\Om$:
      $
      \sum_{i=1}^{n} A(\vec{x}_i) = A_\Om.
      $
     Sneakily turn this into an integral constraint:
      $$
      \int_\Om
      \frac{\dee{\vec{x}}}
      {A(\vec{x})}
      = n.
      $$
    
      Within each cell, $A(\vec{x})$ is constant.
    
      So... integral over each of the $n$ cells equals 1.
    
  


  \textbf{Size-density law}

  \textbf{Now a Lagrange multiplier story:}
    
     By varying $\{\vec{x}_1,...,\vec{x}_n\}$, minimize
      $$
      G(A) = 
      c \int_{\Om}
      \alertb{\rhopop(\vec{x})}
      \alertb{ A(\vec{x})^{1/2}}
      \dee{\vec{x}}
      -
      \lambda
      \left(n -
        \int_\Om
        \left[A(\vec{x})\right]^{-1}
        \dee{\vec{x}}
      \right)
      $$
    
      I Can Haz
      \wordwikilink{http://en.wikipedia.org/wiki/Calculus\_of\_variations}{Calculus of Variations}?
    
      Compute
      $\delta G / \delta A$,
      the \wordwikilink{http://en.wikipedia.org/wiki/Functional_derivative}{functional derivative}
      of the functional $G(A)$.
     This gives
      $$
      \int_{\Om}
      \left[
        \frac{c}{2} \alertb{\rhopop(\vec{x})}
        \alertb{ A(\vec{x})^{-1/2}}
        -
        \lambda
        \left[A(\vec{x})\right]^{-2}
      \right]
      \dee{\vec{x}} = 0.
      $$
     Setting the integrand to be zilch, we have:
      $$
      \rhopop(\vec{x})
      =
      2\lambda
      c^{-1}
      A(\vec{x})^{-3/2}.
      $$
    
  



  \textbf{Size-density law}

  \textbf{Now a Lagrange multiplier story:}
    
     Rearranging, we have
      $$
      A(\vec{x}) = (2{\lambda} c^{-1})^{2/3} \rhopop^{-2/3}.
      $$
    
      Finally, we indentify $1/A(\vec{x})$ as $\rhofac(\vec{x})$,
      an approximation of the local source density.
     Substituting $\rhofac=1/A$, we have
      $$
      \alert{\rhofac(\vec{x})
      = \left( 
        \frac{c}{2{\lambda}}
        \rhopop
    \right)^{2/3}}.
      $$
     Normalizing (or solving for $\lambda$):
      $$
      \alert{\rhofac(\vec{x})}
      =  n 
      \frac{[\rhopop(\vec{x})]^{2/3}}
      {\int_{\Om} [\rhopop(\vec{x})]^{2/3} \dee{\vec{x}}}
      \alert{\propto [\rhopop(\vec{x})]^{2/3}}.
      $$
    
  


\subsection{Global\ redistribution}

  \textbf{Global redistribution networks}

  \textbf{One more thing:}
    
     How do we supply these facilities?
     How do we best redistribute mail?  People?
     How do we get beer to the pubs?
     Gaster and Newman model: cost is 
      a function of basic maintenance and travel time:
      $$
      C_{\textnormal{maint}} + \gamma C_{\textnormal{travel}}.
      $$
     Travel time is more complicated:
      Take `distance' between nodes to be a composite
      of shortest path distance $\ell_{ij}$ and 
      number of legs to journey:
      $$
      (1-\delta) \ell_{ij} + \delta (\# \mbox{hops}).
      $$
     When $\delta=1$, only number of hops matters.
      
      
    
  


  \textbf{Global redistribution networks}

  \includegraphics[width=\textwidth]{gastner2006c_fig6}

  From Gastner and Newman (2006)\cite{gastner2006c}

\subsection{Public\ versus\ Private}

  \textbf{Public versus private facilities}

  \textbf{Beyond minimizing distances:}
    
    
      ``Scaling laws between population and facility densities'' by
      Um et al., Proc. Natl. Acad. Sci., 2009.\cite{um2009a}
    
      Um et al.\ find empirically and argue theoretically that the connection
      between facility and population density
      $$
      \rhofac \propto \rhopop^{\alpha}
      $$
      \alertb{does not universally hold} with $\alpha=2/3$.
    
      \alert{Two idealized limiting classes}:
      
      
        For-profit, commercial facilities: \alertb{$\alpha = 1$};
      
        Pro-social, public facilities: \alertb{$\alpha = 2/3$}.
      
    
      Um et al.\ investigate facility locations in the United States
      and South Korea.
    
  
  


  \textbf{Public versus private facilities: evidence}

  \includegraphics[width=0.49\textwidth]{um2009a_fig1A.pdf}
  \includegraphics[width=0.49\textwidth]{um2009a_fig1B.pdf}

  
  
    \alert{Left plot:} ambulatory hospitals in the U.S.
   
    \alert{Right plot:} public schools in the U.S.
  
    Note: break in scaling for public schools.
    Transition from $\alpha \simeq 2/3$ to 
    $\alpha = 1$ around $\rhopop \simeq 100$.
  


  \textbf{Public versus private facilities: evidence}

      
    \includegraphics[width=\textwidth]{um2009a_tab1A.pdf}\\
    \includegraphics[width=\textwidth]{um2009a_tab1B.pdf}
    
      {\small
        Rough \alertb{transition} between public and private at $\alpha \simeq 0.8$.

        \medskip

        Note: * indicates analysis is at state/province level; otherwise county level.}
  

  \textbf{Public versus private facilities: evidence}

  \includegraphics[width=\textwidth]{um2009a_fig2.pdf}

  \alert{A, C:} ambulatory hospitals in the U.S.;
  \alert{B, D:} public schools in the U.S.;
  \alert{A, B:} data; 
  \alert{C, D:} Voronoi diagram from model simulation.


  \textbf{Public versus private facilities: the story}

  \textbf{So what's going on?}
    
    
      Social institutions seek to \alertb{minimize distance of travel}.
    
      Commercial institutions seek to \alertb{maximize the
      number of visitors}.
    
      \alertb{Defns:} For the $i$th facility and its Voronoi cell $V_i$, define
      
      
        $n_i$ = population of the $i$th cell;
      
        $\tavg{r_i}$ = the average travel distance
        to the $i$th facility.
      
        $s_i$ = area of $i$th cell.
      
    
      Objective function to maximize for a facility (highly constructed):
      $$ 
      \alertb{v_i = n_i \tavg{r_i}^\beta}
      \
      \mbox{with}
      \
      0 \le \beta \le 1.
      $$
    
      Limits:
      
       $\beta = 0$: purely commercial.
       $\beta = 1$: purely social.
      
    
  


  \textbf{Public versus private facilities: the story}

  
   
    Proceeding as per the Gastner-Newman-Gusein-Zade calculation,
    Um et al.\ obtain:
    $$
    \alert{\rhofac(\vec{x})}
    =  n
    \frac{[\rhopop(\vec{x})]^{2/(\beta+2)}}
    {\int_{\Om} [\rhopop(\vec{x})]^{2/(\beta+2)} \dee{\vec{x}}}
    \alert{\propto [\rhopop(\vec{x})]^{2/(\beta+2)}}.
    $$
   
    For $\beta=0$, $\alpha=1$: commercial scaling is linear.
   
    For $\beta=1$, $\alpha=2/3$: social scaling is sublinear.
   
    You can try this too: \insertassignmentquestion{04}{4}{3}.
  


