%% fix up HOT forests---real data section

%% add some of the continuum calculation, preparing for the assignment

%% improve the discrete calculation ...

%% add notes on moritz2005a.pdf
%% forest fires


\section{Robustness}

%% \begin{frame}
%% 
%% Add Sornette's endogenous, exogenous work
%%   
%% Add results from a coded version for forest fires
%% 
%% Make the lectures about the continuous version (carlson1999a)
%% 
%% Give students a problem involving solving the 2-d discrete
%% model; theory plus simulations.
%% 
%% Give students questions for different distributions:
%% 
%% \includegraphics[width=0.8\textwidth]{carlson1999a_tab1.pdf}
%% 
%% Add data on real forest fires, etc.
%% 
%% Add more recent work on HOT networks
%% 
%% Add simulations!!!
%% \end{frame}

\subsection{HOT\ theory}


\begin{frame}
  \frametitle{Robustness}

  \begin{block}{}
  \begin{itemize}
  \item<1-> Many complex systems are prone to cascading catastrophic failure: \only<6->{\alert{exciting!!!}}
    \begin{itemize}
    \item<2-> 
      Blackouts
    \item<3-> 
      Disease outbreaks
    \item<4->
      Wildfires
    \item<5-> 
      Earthquakes
    \end{itemize}
  \item<7-> But complex systems also show persistent \alert{robustness}
    \visible<8->{(not as exciting but important...)}
  \item<9-> Robustness and Failure may be a power-law story...
  \end{itemize}
  \end{block}

\end{frame}

\changelecturelogo{.18}{DeathStarIDiagram-EGVV-tp-1.pdf}

\begin{frame}
  \frametitle{Our emblem of Robust-Yet-Fragile:}

  \includegraphics[width=\textwidth]{death-star-1200.jpg}

\end{frame}

\insertvideo{EVekNsgUqn4}{}{}{``That's no moon ...''}

\begin{frame}
  \frametitle{Robustness}
    
  \begin{block}{}
    \begin{itemize}
    \item<1-> 
      System robustness may result from 
      \begin{enumerate}
      \item<2-> 
        \alert{Evolutionary processes}
      \item<3-> 
        \alert{Engineering/Design}
      \end{enumerate}
    \item <4-> Idea: Explore systems \alertb{optimized to perform}
      under \alert{uncertain conditions}.
    \item <5-> 
      The handle:\\
      `Highly Optimized Tolerance' (HOT)\cite{carlson1999a,carlson2000a,carlson2002a,sornette2003a}
    \item <6-> 
      The catchphrase: \alertb{Robust yet Fragile}
    \item <7-> 
      The people: Jean Carlson and \wordwikilink{http://www.cds.caltech.edu/~doyle/}{John Doyle}
    \item <8-> 
      Great abstracts of the world \#73: ``There aren't any.''\cite{doyle1978a}
    \end{itemize}
  \end{block}

\end{frame}

\begin{frame}
  \frametitle{Robustness}

  \begin{block}{Features of HOT systems:\cite{carlson2000a,carlson2002a}}
    \begin{itemize}
    \item<2-> 
      High performance and robustness
    \item<3-> 
      Designed/evolved to handle known 
      stochastic environmental variability
    \item<4-> 
      \alert{Fragile} in the face of unpredicted environmental signals
    \item<5-> 
      Highly specialized, low entropy configurations
    \item<6-> 
      Power-law distributions appear (of course...)
    \end{itemize}
    
  \end{block}
\end{frame}

\begin{frame}
  \frametitle{Robustness}

  \begin{block}{HOT combines things we've seen:}
    \begin{itemize}
    \item<1-> 
      Variable transformation
    \item<2-> 
      Constrained optimization
    \end{itemize}
  \end{block}

  \begin{block}{}
  \begin{itemize}
  \item<3-> 
    Need power law transformation
    between variables: \alert{($Y = X^{-\alpha}$)}
  \item<4-> 
    Recall PLIPLO is bad...
  \item<5-> 
    \alertb{MIWO} is good\visible<6->{: Mild In, Wild Out}
  \item<7-> 
    $X$ has a characteristic size but $Y$ does not
  \end{itemize}
  \end{block}

\end{frame}


\begin{frame}
  \frametitle{Robustness}

  \begin{block}<1->{Forest fire example:\cite{carlson2000a}}
    \begin{itemize}
    \item<2-> Square $N\times{}N$ grid
    \item<3-> Sites contain a tree with probability $\rho$ = density
    \item<4-> Sites are empty with probability $1-\rho$
    \item<5-> Fires start at location $(i,j)$ according to some distribution $P_{ij}$
    \item<6-> Fires spread from tree to tree (nearest neighbor only)
    \item<7-> Connected clusters of trees burn completely
    \item<8-> Empty sites block fire
    \item<9-> \alert{Best case scenario:}\\
      \alertb{Build firebreaks to maximize average \# trees left intact given one spark}
    \end{itemize}
  \end{block}

\end{frame}

\begin{frame}
  \frametitle{Robustness}

  \begin{block}<1->{Forest fire example:\cite{carlson2000a}}
    \begin{itemize}
    \item<2-> Build a forest by adding one tree at a time
    \item<3-> Test $D$ ways of adding one tree
    \item<4-> $D$ = \alert{design parameter}
    \item<5-> Average over $P_{ij}$ = spark probability
    \item<6-> $D=1$: random addition
    \item<7-> $D=N^{\, 2}$: test all possibilities
    \end{itemize}
  \end{block}

  \begin{block}<8->{Measure average area of forest left untouched}
    \begin{itemize}
    \item<9-> 
      $f(c)$ = distribution of fire sizes $c$ (= cost)
    \item<10-> 
      Yield  = $Y = \rho - \tavg{c}$
    \end{itemize}
  \end{block}

\end{frame}

\begin{frame}
  \frametitle{Robustness}

  \begin{block}{Specifics:}
    \begin{itemize}
    \item 
      $$ P_{ij} = P_{i;a_x,b_x} P_{j;a_y,b_y}$$
      where
      $$ P_{i;a,b}
      \propto
      e^{-[(i+a)/b]^2}
      $$
    \item
      In the original work, $b_y > b_x$
    \item
      Distribution has more width in $y$ direction.
    \end{itemize}
        
  \end{block}

\end{frame}
  
\begin{frame}
%%  \frametitle{}
  
  \begin{block}{HOT Forests}
  \begin{columns} 
    \begin{column}{0.6\textwidth} 
      \includegraphics[width=0.95\textwidth]{carlson2000a_fig1.pdf}\cite{carlson2000a}\\
    \end{column}
    \begin{column}{0.3\textwidth}
      $N=64$\\
      \mbox{}\\
      \alert{(a)} $D=1$\\
      \alert{(b)} $D=2$\\ 
      \alert{(c)} $D=N$\\
      \alert{(d)} $D = N^2$\\
      \mbox{}\\
      $P_{ij}$ has a Gaussian decay
    \end{column}
  \end{columns}

  \begin{itemize}
  \item<2->
    Optimized forests do well on average \visible<4->{\alert{(robustness)}}
  \item<3->
    But rare extreme events occur \visible<5->{\alertb{(fragility)}}
  \end{itemize}
  \end{block}

% Sample configurations at peak yield for $N = 64$
% and varying values of the design parameter $D$, 


% FIG. 1. 
% Sample configurations at peak yield for $N = 64$
% and varying values of the design parameter $D$, 
% (a) the random case $D=1$, (b) $D=2$, (c) $D=N$,
% and $D = N^2$. Black sites are unoccupied.

\end{frame}

\begin{frame}
  \frametitle{HOT Forests}

  \begin{block}{}
      \includegraphics[width=0.95\textwidth]{carlson2000a_fig2.pdf}\cite{carlson2000a}
  \end{block}

\end{frame}

\begin{frame}
  \frametitle{HOT Forests:}

  \begin{block}{}
    \begin{itemize}
    \item 
      $Y$ =  `the average density of trees left unburned in a configuration after a single spark hits.'\cite{carlson2000a}
    \end{itemize}
      \includegraphics[width=0.95\textwidth]{carlson2000a_fig3.pdf}
  \end{block}

\end{frame}

\subsection{Narrative\ causality}

\insertvideo{DOFgFAcGHQc}{120}{242}{Narrative causality: ``One in a million ...''}

\begin{frame}
  \frametitle{Random Forests}

  \begin{block}{$D=1$: Random forests = Percolation\cite{stauffer1992a}}
    \begin{itemize}
    \item<1-> Randomly add trees
    \item<2-> Below critical density $\rho_c$, no fires take off
    \item<3-> Above critical density $\rho_c$, percolating
      cluster of trees burns
    \item<4-> Only at $\rho_c$, the critical density,
      is there a power-law distribution of tree cluster sizes
    \item<5-> Forest is random and featureless
    \end{itemize}
  \end{block}

\end{frame}

%% explain optimization


\begin{frame}
  \frametitle{HOT forests nutshell:}

  \begin{block}{}
    \begin{itemize}
    \item<1-> Highly structured
    \item<2-> Power law distribution of tree cluster sizes 
      for $\rho > \rho_c$
    \item<3-> No specialness of $\rho_c$ 
    \item<4-> Forest states are \alert{tolerant}
    \item<5-> Uncertainty is okay if well characterized
    \item<6-> \alertb{If $P_{ij}$ is characterized poorly}, failure
      becomes \alert{highly likely}
    \end{itemize}
  \end{block}

\end{frame}


\begin{frame}
  \frametitle{HOT forests---Real data:}

  \begin{block}{``Complexity and Robustness,'' Carlson \& Dolye\cite{carlson2002a}}
    \begin{columns}
      \column{0.03\textwidth}
      \column{0.43\textwidth}
      \includegraphics[width=\textwidth]{carlson2002a_fig1}
      \column{0.54\textwidth}
      \begin{itemize}
      \item 
        PLR = probability-loss-resource.
      \item 
        Minimize cost subject to resource (barrier) constraints: \\
        %% \left\{
        $
        C = \sum_{i} p_i l_i 
        $\\
        given\\
        $
        l_i = f(r_i)
        $
        and
        $
        \sum r_i \le R.
        $
        %% \right\}
      \end{itemize}
    \end{columns}
  \end{block}


  %% \begin{columns}
  %%   \column{0.3\textwidth}
  %%   
  %%   \column{0.7\textwidth}
  %%   \includegraphics[width=\textwidth]{carlson2002a_fig1}
  %% \end{columns}

\end{frame}

\begin{frame}
  \frametitle{HOT theory:}

  \begin{block}<+->{The abstract story, using figurative forest fires:}
    \begin{itemize}
    \item<+-> 
      Given some measure of failure size $y_i$ and
      correlated resource size $x_i$.
      with relationship $y_i = x_i^{-\alpha}$, $i=1,\ldots,N_{\textrm{sites}}$.
    \item<+-> 
      Design system to minimize $\avg{y}$\\
      subject to a constraint on the $x_i$.
    \item<3-> Minimize cost:
      $$ 
      C = \sum_{i=1}^{N_{\textrm{sites}}} Pr(y_i) y_i 
      $$
      \uncover<4->{
        Subject to $\sum_{i=1}^{N_{\textrm{sites}}} x_i = \mbox{constant}$.
      }
    \end{itemize}  
  \end{block}

\end{frame}

%% give sornette's argument
%% (i have seen this in doyle's papers or talks)
%% for how dimensionality enters


%% \begin{frame}
%%   \frametitle{HOT theory}
%%   %% present continuum version
%% 
%%   
%% 
%% \end{frame}


\begin{frame}
  %% \frametitle{HOT: Optimal fire walls in $d$ dimensions}
  %% \frametitle{HOT Theory---Two costs:}
  \small
  \begin{block}{}
    \begin{enumerate}
    \item<+->
      \alertg{Cost: Expected size of fire:}
      $$
      C_{\textrm{fire}}
      \propto \sum_{i=1}^{N_{\textrm{sites}}} (p_i a_i) a_i
      \uncover<3->{= \sum_{i=1}^{N_{\textrm{sites}}} p_i a_i^{\, 2}}
      $$ 
      \begin{itemize}
      \item<2-> 
        $a_i$ = area of $i$th site's region
      \item<2->
        $p_i$ = avg.\ prob.\ of fire at site in $i$th site's region
      \end{itemize}
    \item<4->
      \alertg{Constraint: building and maintaining firewalls}
      $$
      C_{\textrm{firewalls}} \propto \sum_{i=1}^{N_{\textrm{sites}}} a_i^{1/2} a_i^{-1}
      $$
      \begin{itemize}
      \item<5->
        We are assuming \alertb{isometry}.
      \item<6->
        In $d$ dimensions, 1/2 is replaced by $(d-1)/d$
      \end{itemize}
    \item<7->
      \insertassignmentquestionsoft{04}{4} to find:\\
      %% Use Lagrange Multipliers with tongs to find:
      $$ 
      \boxed{
        p_i \propto a_i^{-\gamma} 
        = a_i^{-(2 + 1/d)}.
      }
      $$
    \end{enumerate}
  \end{block}

\end{frame}

%% \begin{frame}
%%   \frametitle{HOT theory}
%%   
%%   \begin{block}{Extra constraint:}
%%     \begin{itemize}
%%     \item<1->
%%       Total area is constrained:
%%       $$
%%       \sum_{i=1}^{N_{\textrm{sites}} 1  = N^2.
%%       $$
%% 
%%       $$
%%       \sum_{i=1}^{N_{\textrm{sites}} \frac{1}{a_i} = N_{\textrm{regions}
%%       $$
%%       where
%%       $N_{\textrm{regions}$ = number of cells.
%%     \item<2->
%%       Can ignore in calculation...
%%     \end{itemize}
%%   \end{block}
%%   
%%   
%% 
%% \end{frame}
%% 
%% \begin{frame}
%%   \frametitle{HOT theory}
%%   
%%   \begin{block}{}
%%   \begin{itemize}
%%   \item<1-> Minimize \alertb{$C_{\textrm{fire}$} 
%%     given \alertb{$C_{\textrm{firewalls} = {\textrm{constant}$}.
%%   \item<2->
%%     $$
%%     0 =
%%     \partialdiff{}{a_j} 
%%     \left(
%%       C_{\textrm{fire} -\lambda C_{\textrm{firewalls}
%%     \right)
%%     $$
%%     \uncover<3->{
%%     $$
%%     \propto
%%     \partialdiff{}{a_j }
%%     \left(
%%       \sum_{i=1}^{N} p_i a_i^{\, 2} - \lambda' a_i^{(d-1)/d} a_i^{-1}
%%     \right)
%%     $$
%%     }
%%   \item<4-> 
%%     $$ 
%%     p_i \propto a_i^{-\gamma} = a_i^{-(2 + 1/d)} 
%%     $$
%%   \item<5-> 
%%     $$ 
%%     \mbox{For} \  d=2, \gamma = 5/2
%%     $$
%%   \end{itemize}
%%   \end{block}
%% 
%% \end{frame}



%% \begin{frame}
%%   \frametitle{HOT theory nutshell:}
%% 
%%   \begin{block}<+->{Summary of designed tolerance\cite{carlson2002a}}
%%     \begin{itemize}
%%     \item<+-> 
%%       Build more firewalls in areas where sparks are likely
%%     \item<+-> 
%%       Small connected regions in high-danger areas
%%     \item<+-> 
%%       Large connected regions in low-danger areas
%%     \item<+->
%%       Routinely see many small outbreaks \alert{(robust)}
%%     \item<+-> 
%%       Rarely see large outbreaks \alert{(fragile)}
%%     \item<+-> 
%%       Sensitive to changes in the environment ($P_{ij}$)
%%     \end{itemize}
%%   \end{block}
%% 
%% \end{frame}

%% 

\insertvideo{znWt_512GIo}{}{}{The Emperor's Robust-Yet-Fragileness:}


\subsection{Self-Organized\ Criticality}

\begin{frame}
  
  \frametitle{Avalanches of Sand and Rice...}

  \begin{center}
    \includegraphics[height=0.8\textheight]{rice-SOC.jpg}
  \end{center}

\end{frame}

\begin{frame}
  \frametitle{SOC theory}

  \begin{block}<1->{SOC = Self-Organized Criticality}
    \begin{itemize}
    \item<1-> Idea: natural dissipative systems exist at `critical states';
    \item<2-> Analogy: Ising model with temperature somehow self-tuning;
    \item<3-> Power-law distributions of sizes and frequencies arise `for free';
    \item<4-> Introduced in 1987 by Bak, Tang, and Weisenfeld\cite{bak1987a,bak1997a,jensen1998a}:\\
      ``Self-organized criticality - an explanation of 1/f noise'' (PRL, 1987);
    \item<5-> \alert{Problem:} Critical state is a very specific point;
    \item<6-> Self-tuning not always possible;
    \item<7-> Much criticism and arguing...
    \end{itemize}
  \end{block}

\end{frame}


\begin{frame}
  \frametitle{Per Bak's Magnum Opus:}

  \displayamazonbook{bak1997a}

\end{frame}

\begin{frame}
  \frametitle{Robustness}

  \begin{block}{HOT versus SOC}

    \begin{itemize}
    \item<1-> Both produce power laws
    \item<2-> Optimization versus self-tuning
    \item<3-> HOT systems viable over a wide range of high densities
    \item<4-> SOC systems have one special density
    \item<5-> HOT systems produce specialized structures
    \item<6-> SOC systems produce generic structures
    \end{itemize}
    
  \end{block}

\end{frame}

\begin{frame}
  \frametitle{HOT theory---Summary of designed tolerance\cite{carlson2002a}}

  \begin{block}{}
  \begin{center}
    \includegraphics[height=0.8\textheight]{carlson2002a_tab1}
  \end{center}
  \end{block}

\end{frame}


\begin{frame}
  \begin{block}{To read: `Complexity and Robustness'\cite{carlson2002a}}
      \includegraphics[width=0.48\textwidth,page=1]{carlson2002a}
      \includegraphics[width=0.48\textwidth,page=2]{carlson2002a}
  \end{block}
\end{frame}


\subsection{COLD\ theory}

\begin{frame}
  \frametitle{COLD forests}

  \begin{block}{Avoidance of large-scale failures}
      \begin{itemize}
      \item<1-> Constrained Optimization with Limited Deviations\cite{newman2002f}
      \item<2-> Weight cost of larges losses more strongly
      \item<3-> Increases average cluster size of burned trees...
      \item<4-> ... but reduces chances of catastrophe
      \item<5-> Power law distribution of fire sizes is truncated
      \end{itemize}
  \end{block}
  
\end{frame}


\begin{frame}
  \frametitle{Cutoffs}

  \begin{block}{Observed:}
    \begin{itemize}
    \item<1->
      Power law distributions often have an exponential cutoff
      $$ 
      P(x) \sim x^{-\gamma} e^{-x/x_c}
      $$
      where $x_c$ is the approximate cutoff scale.
    \item<2->
      May be Weibull distributions:
      $$ 
      P(x) \sim x^{-\gamma} e^{-a x^{-\gamma+1}}
      $$
    \end{itemize}
  \end{block}

\end{frame}

%% %% now for later on:
\subsection{Network\ robustness}

\begin{frame}
  \frametitle{Robustness}

  %% link to networks pdf

  \begin{block}{We'll return to this later on:}
    \begin{itemize}
    \item<1->
      \alertb{Network robustness}.
    \item<1->
      Albert et al., Nature, 2000:\\
      \alertb{``Error and attack tolerance of complex networks''}\cite{albert2000a}
    \item<1->
      General contagion processes acting on complex networks.\cite{watts2002b,watts2007a}
    \item<1->
      Similar robust-yet-fragile stories ...
      %% \item<1->
      %%   \lecturelink{networks-overview-slides}{Networks Overview}{67ish}
    \end{itemize}
  \end{block}

\end{frame}

\insertvideo{LowVhCfLm68}{}{}{The Emperor's Robust-Yet-Fragileness:}
