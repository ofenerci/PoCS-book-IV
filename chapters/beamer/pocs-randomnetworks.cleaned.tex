\section{Basics}

\subsection{Definitions}

\begin{frame}[label=]
  \textbf{Random networks}
  
  \textbf{Pure, abstract random networks:}
    
    
      Consider set of all networks with $N$ labelled nodes
      and $m$ edges.
     
      Standard random network = \\
      one \alert{randomly chosen} network from this set.
     
      To be clear: each network is \alert{equally} probable.
     
      Sometimes equiprobability is a good assumption, but
      it is always an assumption.
     
      Known as \erdosrenyi\ random networks 
      or \alertb{ER graphs}.
    
  
  

\begin{frame}[label=]
  \textbf{Random network generator for $N=3$:}

  \includegraphics[width=\textwidth]{2011-02-26random-network-generator-tp-5}

  
   
    Get your own exciting generator
    \wordwikilink{\coursewebsite/docs/2011-02-26random-network-generator.png}{here}.
   
    As $N \nearrow$, our polyhedral die rapidly becomes a ball...
  



\begin{frame}[label=]
  \textbf{Random networks---basic features:}

    
     
      Number of possible edges:
      $$
      0 \le m \le \binom{N}{2} = \frac{N(N-1)}{2}
      $$
    
      Limit of $m = 0$: empty graph.
          
      Limit of $m = \binom{N}{2}$: complete or fully-connected graph.
    
      Number of possible networks with $N$
      labelled nodes: 
      $$
      2^{\binom{N}{2}} \sim e^{\frac{\ln{2}}{2} N^2}.
      $$
    
      Given $m$ edges, there are 
      $ \binom{\binom{N}{2}}{m} $
      different possible networks.
    
      Crazy factorial explosion for $1 \ll m \ll \binom{N}{2}$.
    
      \alertb{Real world:} links are usually costly
      so real networks are almost always \alert{sparse}.
    
  


\subsection{How\ to\ build}

\begin{frame}[label=]
  \textbf{Random networks}

  \textbf{How to build standard random networks:}
    
     Given $N$ and $m$.
     Two probablistic methods {(we'll see a third later on)}
    
    
     Connect each of the $\binom{N}{2}$ pairs 
      with appropriate probability $p$.
        
         \alert{Useful for theoretical work.}
        
       Take $N$ nodes and add exactly $m$ links
        by selecting edges without replacement.
        
         \alert{Algorithm:} Randomly choose a pair of nodes $i$ and $j$, $i \ne j$,
          and connect if unconnected; repeat until all $m$ edges
          are allocated.
         Best for adding relatively small numbers of links (most cases).
         \alertb{1} and \alertb{2} are effectively equivalent
          for large $N$.
        
    
  


  \textbf{Random networks}

  \textbf{A few more things:}
  
  
    For method \alertb{1}, \# links is
    probablistic:
    $$
    \avg{m} = p \binom{N}{2} {= p\frac{1}{2}N(N-1)}
    $$
  
    So the expected or \alert{average degree} is 
    $$
    \avg{k} = \frac{\alert{2} \avg{m}}{N} 
    $$
    $$
    {
      = \frac{2}{N} p\frac{1}{2}N(N-1)
    }
    {
      = \frac{\cancel{2}}{\cancel{N}} p\frac{1}{\cancel{2}}\cancel{N}(N-1)
      }
    {
      = p(N-1).
    }
    $$
   Which is what it should be...
   
    If we keep $\tavg{k}$ constant then
    $p \propto 1/N \rightarrow 0$ as $N \rightarrow \infty$.
  
    
  
  


\subsection{Some\ visual\ examples}

\begin{frame}[label=]
  \textbf{Random networks: examples}

  \textbf{Next slides:}
    Example realizations of random networks
    
     $N=500$
     Vary $m$, the number of edges from 100 to 1000.
     Average degree $\tavg{k}$ runs from 0.4 to 4.
     Look at full network plus the largest component.
    
  



%% series of random network examples
%% varying m
\input{nw_purerandom_graphviz01}

%% entire networks on one page
\input{nw_purerandom_graphviz01all}

%% largest components on one page
\input{nw_purerandom_graphviz01giants}

%% now keeping m constant
%% entire networks on one page
\input{nw_purerandom_graphviz02all}

%% largest components on one page
\input{nw_purerandom_graphviz02giants}

\section{Structure}

\subsection{Clustering}

  \textbf{Clustering in random networks:}
  
  
     
      For method \alertb{1}, 
      what is the clustering coefficient for a finite network?
    
      Consider triangle/triple clustering coefficient:\cite{newman2003a}
      $$ C_2 = \frac{3 \times \textnormal{\#triangles}}{\textnormal{\#triples}} $$ 
  
    
      
    \includegraphics[width=\textwidth]{2011-02-25clustering-randomnetworks-tp-10.pdf}
    
    
     
      Recall:
      $C_2$ = probability that two friends
      of a node are also friends. 
     
      Or: $C_2$ = probability that a triple is part of a triangle.
     
      For standard random networks, we have simply
      that 
      $$\alertb{C_2 = p}.$$
    
      

  \textbf{Other ways to compute clustering:}

  
   
    Expected number of triples in entire network:
    $$
    \frac{1}{2}N(N-1)(N-2)p^2
    $$
    (Double counting dealt with by $\frac{1}{2}$.)
   
    Expected number of triangles in entire network:
    $$
    \frac{1}{6}N(N-1)(N-2)p^3
    $$
    (Over-counting dealt with by $\frac{1}{6}$.)
  
    $$ 
    C_2 = \frac{3 \times \textnormal{\#triangles}}{\textnormal{\#triples}} 
    = 
    \frac{
      3 \times \frac{1}{6}N(N-1)(N-2)p^3
    }
    {
      \frac{1}{2}N(N-1)(N-2)p^2
    }
    = p.
    $$
  


  \textbf{Other ways to compute clustering:}

  
  
    \alertb{Or:} take any three nodes, call them $a$, $b$, and $c$.
  
    Triple $a$-$b$-$c$ centered at $b$ occurs with probability
    $p^2 \times (1-p) + p^2 \times p = p^2$.
  
    Triangle occurs with probability $p^3$.
  
    Therefore, 
    $$
    C_2 = \frac{p^3}{p^2} = p.
    $$
  


  \textbf{Clustering in random networks:}
  
      
    \includegraphics[width=\textwidth]{2011-02-26branching-networks-local-tp-5}
    %%     \includegraphics[width=\textwidth]{nw_purerandom_graphviz01maincluster_7}
    
    
     
      So for large random networks ($N \rightarrow \infty$),
      clustering drops to zero.
     
      Key structural feature of random networks is that
      they locally look like\\ 
      \alertb{pure branching networks}
     
      No small loops.
    
  

\subsection{Degree\ distributions}

  \textbf{Random networks}
  
  \textbf{Degree distribution:}
    
     
      Recall $P_k$ = probability that a randomly
      selected node has degree $k$.
     
      Consider method \alertb{1} for constructing
      random networks: each possible link is realized
      with probability $p$.
     
      Now consider one node: there are `$N-1$ choose $k$'
      ways the node can be connected to $k$ of 
      the other $N-1$ nodes.
    
      Each connection occurs with probability $p$,
      each non-connection with probability $(1-p)$.
     
      Therefore have a binomial distribution:
      $$
      P(k;p,N) = \binom{N-1}{k} p^k (1-p)^{N-1-k}.
      $$
  
  

\begin{frame}[label=]
  \textbf{Random networks}

  \textbf{Limiting form of $P(k;p,N)$:}
    
    
      Our degree distribution:
      $
      P(k;p,N) = \binom{N-1}{k} p^k (1-p)^{N-1-k}.
      $
    
      What happens as $N \rightarrow \infty$?
    
      We must end up with the normal distribution right?
    
      If $p$ is fixed, then we would end up
      with a Gaussian with average 
      degree $\tavg{k} \simeq pN \rightarrow \infty$.
    
      But we want to keep $\tavg{k}$ fixed...
    
      So examine limit of $P(k;p,N)$
      when \alert{$p \rightarrow 0$} and \alert{$N \rightarrow \infty$}
      with \alertb{$\tavg{k} = p(N-1)$ = constant}.
    
  


\begin{frame}[label=]
  \textbf{Limiting form of $P(k;p,N)$:}

    \small
    
    
      Substitute $p = \frac{\tavg{k}}{N-1}$ into $P(k;p,N)$
      and hold $k$ fixed:
      $$
      P(k;p,N) = \binom{N-1}{k} 
      \left(
      \frac{\tavg{k}}{N-1}
      \right)^k 
      \left(
      1-\frac{\tavg{k}}{N-1}
      \right)^{N-1-k}
      $$
    []
      $$
      = 
      \frac{(N-1)!}
      {k!(N-1-k)!}
      \frac{\tavg{k}^k}{(N-1)^k}
      \left(
      1-\frac{\tavg{k}}{N-1}
      \right)^{N-1-k}
      $$
    []
      $$
      = 
      \frac{\alert{(N-1)(N-2)\cdots(N-k)}}
      {k!}
      \frac{\tavg{k}^k}{(N-1)^k}
      \left(
      1-\frac{\tavg{k}}{N-1}
      \right)^{N-1-k}
      $$
    []
              
      $$
      =
      \frac{\alert{N^k}
        {(1-\frac{1}{N})\cdots(1-\frac{k}{N})}}
      {k! \alert{N^k}}
      \frac{\tavg{k}^k}{(1-\frac{1}{N})^k}
      \left(
      1-\frac{\tavg{k}}{N-1}
      \right)^{N-1-k}
      $$
        
      $$
      =
      \frac{\cancel{\alert{N^k}}
        {(1-{\frac{1}{N}})\cdots(1-\frac{k}{N})}}
      {k! \cancel{\alert{N^k}}}
      \frac{\tavg{k}^k}{(1-\frac{1}{N})^k}
      \left(
      1-\frac{\tavg{k}}{N-1}
      \right)^{N-1-k}
      $$
        
      $$
      \simeq
      \frac{\cancel{\alert{N^k}}
        {(1-\cancel{\frac{1}{N}})\cdots(1-\cancel{\frac{k}{N}})}}
      {k! \cancel{\alert{N^k}}}
      \frac{\tavg{k}^k}{(1-\cancel{\frac{1}{N}})^k}
      \left(
      1-\frac{\tavg{k}}{N-1}
      \right)^{N-1-k}
      $$
          


\begin{frame}[label=]
  \textbf{Limiting form of $P(k;p,N)$:}
  \small

  
  
    We are now here:
    $$
    P(k;p,N)
    \simeq
    \frac{\tavg{k}^k}
    {k!}
    \left(
      1-\frac{\tavg{k}}{N-1}
    \right)^{N-1-k}
    $$
  
    Now use the excellent result:
    $$
    \lim_{n \rightarrow \infty}
    \left(
      1+\frac{x}{n}
    \right)^n
    = e^{x}.
    $$
    {
      (Use l'H\^{o}pital's rule to prove.)
    }
  
    Identifying $n=N-1$ and $x=-\tavg{k}$:
    $$
    P(k;\tavg{k})
    \simeq
    \frac{\tavg{k}^k}
    {k!}
    e^{-\tavg{k}}
    \left(
      1-\frac{\tavg{k}}{N-1}
    \right)^{-k}
    {
      \rightarrow
      \alert{
        \frac{\tavg{k}^k}
        {k!}
        e^{-\tavg{k}}
      }
    }
    $$
  
    This is a 
    \wordwikilink{http://en.wikipedia.org/wiki/Poisson_distribution}
    {Poisson distribution}
    with mean $\tavg{k}$.
  


\begin{frame}[label=]
  \textbf{Poisson basics:}

  %% example figures
  %% note when the Poisson appears in general

      
    $$
    \boxed{
      \alertb{
      P(k;\lambda)
      =
      \frac{\lambda^k}
      {k!}
      e^{-\lambda}}
  }
    $$
    \bigskip
    \includegraphics[width=\textwidth]{325px-Poisson_pmf-svg-tp-10.pdf}\\
    \mbox{}\hfill\includegraphics[width=.1\textwidth]{wikipedia-tp.pdf}
    
    
     
      $\lambda > 0$
     
      $k=0, 1, 2, 3, \ldots$
     
      Classic use: probability
      that an event occurs $k$ 
      times in a given time period,
      given an average rate of occurrence.
     
      e.g.:\\ 
      phone calls/minute,\\ 
      horse-kick deaths.
     
      `Law of small numbers'
    
  


\begin{frame}[label=]
  \textbf{Poisson basics:}

  
   
    Normalization: we must have
    $$ 
    \sum_{k=0}^\infty
    P(k;\tavg{k})
    = 1
    $$
  
    Checking:
    $$ 
    \sum_{k=0}^\infty
    P(k;\tavg{k})
    =
    \sum_{k=0}^\infty
    \frac{\tavg{k}^k}
    {k!}
    e^{-\tavg{k}}
    $$
  []
    $$
    =
    e^{-\tavg{k}}
    \sum_{k=0}^\infty
    \frac{\tavg{k}^k}
    {k!}
    $$
  []
    $$
    =
    e^{-\tavg{k}}
    e^{\tavg{k}}
    {
      =
      1
      \alert{\faCheck}
    }
    $$
  



\begin{frame}[label=]
  \textbf{Poisson basics:}

  \small
  
   Mean degree: we must have
    $$
    \tavg{k}
    =
    \sum_{k=0}^\infty
    k
    P(k;\tavg{k}).
    $$
  
    Checking:
    $$ 
    \sum_{k=0}^\infty
    k P(k;\tavg{k})
    =
    \sum_{k=0}^\infty
    k \frac{\tavg{k}^k}
    {k!}
    e^{-\tavg{k}}
    $$
  []
    $$
    = e^{-\tavg{k}}
    \sum_{k=\alert{1}}^\infty
    \frac{\tavg{k}^k}
    {\alert{(k-1)!}}
    $$
  []
    $$
    = 
    \alert{\tavg{k}}
    e^{-\tavg{k}}
    \sum_{k={1}}^\infty
    \frac{\tavg{k}^{\alert{k-1}}}
    {{(k-1)!}}
    $$
  []
    $$
    = 
    \tavg{k}
    e^{-\tavg{k}}
    \sum_{\alert{i=0}}^\infty
    \frac{\tavg{k}^{i}}
    {\alert{i!}}
    {
      =
      \tavg{k}
    e^{-\tavg{k}}
    e^{\tavg{k}}
    }
    {
      =
      \alertb{\tavg{k}}
      \alert{\faCheck}
    }
    $$
  
    Note: We'll get to a better and crazier way of doing this...
  


\begin{frame}[label=]
  \textbf{Poisson basics:}

    
     The \alertb{variance} of degree distributions
      for random networks turns out to be \alert{very important}.
     Using calculation similar to one for finding $\tavg{k}$
      we find the \alertb{second moment} to be:
      $$
      \tavg{k^2} = \tavg{k}^2 + \tavg{k}.
      $$
      \insertassignmentquestionsoft{05}{5}
     
      Variance is then
      $$
      \sigma^2 = \alert{\tavg{k^2}} - \alertb{\tavg{k}^2} 
      {
        = 
        \alert{\tavg{k}^2 + \tavg{k}} - \alertb{\tavg{k}^2}
      }
      {
        = 
        \tavg{k}.
      }
      $$
    
      So standard deviation $\sigma$ is equal to $\sqrt{\tavg{k}}$.
    
      Note: This is a special property of Poisson distribution 
      and can trip us up...
    


\subsection{Configuration\ model}

\begin{frame}[label=]
  \textbf{General random networks}

  
  
    So... standard random networks have a Poisson degree distribution
  
    Generalize to arbitrary degree distribution $P_k$.
  
    Also known as the \alertb{configuration model}.\cite{newman2003a}
  
    Can generalize construction method from ER random networks.
  
    Assign each node a weight $w$ from some distribution $P_w$
    and form links with probability
    $$ P(\mbox{link between $i$ and $j$}) \propto w_i w_j.$$
  
    But we'll be more interested in 
    
    
      Randomly wiring up (and rewiring) already existing nodes
      with fixed degrees.
    
      Examining mechanisms that lead to networks with
      certain degree distributions.
    
  


\begin{frame}[label=]
  \textbf{Random networks: examples}

  \textbf{Coming up:}
    Example realizations of random networks with
    power law degree distributions:
    
     $N=1000$.
     $P_k \propto k^{-\gamma}$ for $k \ge 1$.
     Set $P_0 = 0$ (no isolated nodes).
     Vary exponent $\gamma$ between 2.10 and 2.91.
     Again, look at full network plus the largest component.
     Apart from degree distribution, wiring is random.
    
  



%% examples with power law degree distributions
%% varying gamma
% \input{nw_purerandom_graphviz01}

%% entire networks on one page
\input{nw_powerlaw_graphviz01all}

%% largest components on one page
\input{nw_powerlaw_graphviz01giants}

\subsection{Random\ friends\ are\ strange}

\begin{frame}[label=]
  \textbf{The edge-degree distribution:}

  
   
    The degree distribution $P_k$ is fundamental for our description
    of many complex networks
   
    Again: $P_k$ is the degree of \alert{randomly chosen node}.
   
    A second very important distribution arises
    from \alertb{choosing randomly on edges} rather than on nodes.
  
    Define \alertb{$Q_k$} to be the 
    probability the node at a \alert{random end}
    of a \alert{randomly chosen edge} has
    degree \alertb{$k$}.
   Now choosing nodes based on their degree (i.e., size):
    $$
    \boxed{
      \alertb{Q_k \propto k P_k}
    }
    $$
  
    Normalized form:
    $$
    \alertb{Q_k} = \frac{kP_k}{\sum_{k'=0}^{\infty} k' P_{k'}}
    
    {
      \alertb{= \frac{kP_k}{\tavg{k}}}.
    }
    $$
  


\begin{frame}[label=]
  \textbf{The edge-degree distribution:}

    
     
      For random networks, $Q_k$ is also the probability
      that a friend (neighbor) of a random node has \alert{$k$ friends}.
     
      Useful variant on $Q_k$:\\
      \bigskip
      \alertb{
        $
        R_k 
        $
      }
      = probability that a friend of a random node
      has \alert{$k$ other friends}.
    
      $$
      R_k
      =
      \frac{(k+1)P_{k+1}}
      {\sum_{k'=0} (k'+1) P_{k'+1}}
      {
        =
        \frac{(k+1)P_{k+1}}
        {\tavg{k}}
      }
      $$
    
      Equivalent to friend having degree $k+1$.
    
      \alertb{Natural question}: what's the expected
      number of other friends that one friend has?
    


\begin{frame}[label=]
  \textbf{The edge-degree distribution:}
  \small
    
     Given $R_k$ is the probability that a friend
      has $k$ other friends, then the average number of
      \alert{friends' other friends} is
      $$
      \avg{k}_{R}
      =
      \sum_{k=0}^{\infty}
      k R_k
      {
        =
        \sum_{k=0}^{\infty}
        k
        \frac{(k+1)P_{k+1}}
        {\tavg{k}}
      }
      $$
      {
        $$
        =
        \frac{1}{\tavg{k}}
        \sum_{\alert{k=1}}^{\infty}
        k(k+1)P_{k+1}
        $$
      }
      {
        $$
        =
        \frac{1}{\tavg{k}}
        \sum_{{k=1}}^{\infty}
        \left((k+1)^2 - (k+1)\right) 
        P_{k+1}
        $$
        (where we have sneakily matched up indices)
      }
      {
        $$
        =
        \frac{1}{\tavg{k}}
        \sum_{\alert{j=0}}^{\infty}
        (j^2 - j)
        P_j
        \mbox{\quad (using j = k+1)}
        $$
      }
      {
        $$
        =
        \frac{1}{\tavg{k}}
        \left(
          \tavg{k^2} - \tavg{k}
        \right)
        $$
      }
    


  \textbf{The edge-degree distribution:}

  
   Note: our result, 
    $
    \avg{k}_{R}
    =
    \frac{1}{\tavg{k}}
    \left(
      \tavg{k^2} - \tavg{k}
    \right)
    $,
    is true for \alert{all} random networks, 
    \alertb{independent of degree distribution}.
  
    For standard random networks, recall
    $$
    \tavg{k^2} = \tavg{k}^2 + \tavg{k}.
    $$
  
    Therefore:
    $$
    \avg{k}_{R}
    =
    \frac{1}{\tavg{k}}
    \left(
      \alert{\tavg{k}^2 + \tavg{k}} - \tavg{k}
    \right)
    {
      = \alertb{\tavg{k}}
    }
    $$
  
    Again, neatness of results is a special property of the Poisson distribution.
  
    So friends on average have $\tavg{k}$ other friends,
    and $\tavg{k}+1$ total friends...
  


  \textbf{The edge-degree distribution:}

  
  
    In fact, $R_k$ is rather special for pure random networks \ldots
  
    Substituting
    $$
    P_k
    =
    \frac{\tavg{k}^k}
    {k!}
    e^{-\tavg{k}}
    $$
    into
    $$
    R_k
    =
    \frac{(k+1)P_{k+1}}
    {\tavg{k}}
    $$
  []
    we have
    $$
    R_k
    =
    \frac{(k+1)}
    {\tavg{k}}
    \frac{\tavg{k}^{(k+1)}}
    {(k+1)!}
    e^{-\tavg{k}}
    {
    =
    \frac{\cancel{(k+1)}}
    {\cancel{\tavg{k}}}
    \frac{\tavg{k}^{(k+\cancel{1})}}
    {\cancel{(k+1)} k!}
    e^{-\tavg{k}}
    }
    $$
  []
    $$
    =
    \frac{\tavg{k}^{k}}
    {k!}
    e^{-\tavg{k}}
    {
    \equiv P_k.}
    $$
  
    \#samesies.
  
 

\begin{frame}[label=]
  \textbf{Two reasons why this matters}
  
  \small
  
  \textbf{Reason \#1:}
    
     
      Average \# friends of friends per node
      is 
      $$
      \tavg{k_2}
      =
      \tavg{k} \times \tavg{k}_{R}
      {
        =
        \tavg{k}
        \frac{1}{\tavg{k}}
        \left(
          \tavg{k^2} - \tavg{k}
        \right)
      }
      {
        = \alertb{\tavg{k^2} - \tavg{k}}.
      }
      $$
    
      Key: Average depends on the \alert{1st and 2nd
        moments} of $P_k$ and \alertb{not just the 1st moment}.
    
      Three peculiarities:
      
      
        We might guess $\tavg{k_2} = \tavg{k}(\tavg{k}-1)$
        but it's actually $\tavg{k(k-1)}$.
      
        If $P_k$ has a \alert{large second moment},\\
        then $\tavg{k_2}$ will be big.\\
        {(e.g., in the case of a power-law distribution)}
      
        Your friends really are different from you...\cite{feld1991a,newman2003h}
      
        See also: class size paradoxes (nod to: Gelman)
      
    
    
  
  

\begin{frame}[label=]
  \textbf{Two reasons why this matters}
  
  \textbf{More on peculiarity \#3:}
    
    
      A node's average \# of friends: \alertb{$\tavg{k}$}
     
      Friend's average \# of friends: $\frac{\tavg{k^2}}{\tavg{k}}$
    
      Comparison:
      $$
      \frac{\tavg{k^2}}{\tavg{k}}
      = \alertb{\tavg{k}} \alert{\frac{\tavg{k^2}}{\tavg{k}^2}}
      {
        = \alertb{\tavg{k}} \alert{\frac{\sigma^2 + \tavg{k}^2}{\tavg{k}^2}}
      }
      {
        = \alertb{\tavg{k}} 
        \left(
          \alert{1 + \frac{\sigma^2}{\tavg{k}^2}}
        \right)
      }
      {
        \ge \alertb{\tavg{k}}
      }
      $$
    
      So only if everyone has the same degree (variance$ = \sigma^2=0$)
      can a node be the same as its friends.
    
      Intuition: for random networks,
      the more connected a node, the more likely it is to be chosen
      as a friend.
    
  
  

\begin{frame}[label=]
  \textbf{Two reasons why this matters}
  
  \textbf{(Big) Reason \#2:}
    
     
      $\tavg{k}_R$ is \alertb{key} to understanding
      how well random networks are connected together.
     
      e.g., we'd like to know what's the 
      size of the largest component within a network.
     
      As $N \rightarrow \infty$, does 
      our network have a \alert{giant component}?
     
      \alert{Defn:}
      Component = connected subnetwork of nodes
      such that $\exists$ path between each
      pair of nodes in the subnetwork, and no
      node outside of the subnetwork is connected to it.
     
      \alert{Defn:}
      Giant component = component that comprises
      a non-zero fraction of a network as $N \rightarrow \infty$.
     
      Note: Component = Cluster
    
    
  
  

\subsection{Largest\ component}

\begin{frame}[label=]
  \textbf{Giant component}

  \includegraphics[height=0.9\textheight,angle=-90]{figphastrans3}


\begin{frame}[label=]
  \textbf{Structure of random networks}
  
  \textbf{Giant component:}
    
     
      A giant component exists if when
      we follow a random edge, we are likely
      to hit a node with \alert{at least 1} other
      outgoing edge.
     Equivalently, expect exponential growth in
      node number as we move out from 
      a random node.
     
      All of this is the same as requiring $\tavg{k}_R > 1$.
     
      \alert{Giant component condition} (or percolation condition):
      $$
      \tavg{k}_R 
      =
      \frac{\tavg{k^2}-\tavg{k}}{\tavg{k}}
      > 
      1
      $$
    
      Again, see that the second moment is an essential
      part of the story.
    
      Equivalent statement:
      $
      \tavg{k^2} > 2\tavg{k}
      $
    
    
  
  

\begin{frame}[label=]
  \textbf{Giant component}

  \textbf{Standard random networks:}
    
     
      Recall $\tavg{k^2} = \tavg{k}^2 + \tavg{k}$.
    
      Determine condition for giant component:
      $$
      \tavg{k}_R = 
      \frac{\tavg{k^2} - \tavg{k}}
      {\tavg{k}}
      {
        =
      \frac{\alert{\tavg{k}^2 + \tavg{k}} - \tavg{k}}
      {\tavg{k}}
      }
      {
        = \tavg{k}
      }
      $$
    
      Therefore when $\tavg{k} > 1$,
      standard random networks have a giant component.
    
      When $\tavg{k} < 1$,
      all components are finite.
    
      Fine example of a continuous
      \wordwikilink{http://en.wikipedia.org/wiki/Phase_transition}
      {phase transition}.
    
      We say $\tavg{k}=1$ marks the critical point of the system.
    
  
  

\begin{frame}[label=]
  \textbf{Giant component}

  \textbf{Random networks with skewed $P_k$:}
    
     
      e.g, if $P_k = c k^{-\gamma}$ with $2 < \gamma < 3$, $k \ge 1$, 
      then
      $$
      \tavg{k^2} 
      = c \sum_{k=1}^{\infty}
      k^2 k^{-\gamma}
      $$
      {
        $$
        \sim
        \int_{x=1}^{\infty} x^{2-\gamma} \dee{x}
        $$
      }
      {
        $$
        \propto
        \left.
        x^{3-\gamma}
        \right|_{x=1}^{\infty}
        {
          \alertb{=\infty}
        }
        {
          \quad \alertb{(\gg \tavg{k})}.
        }
        $$
      }
    
      So giant component \alert{always exists}
      for these kinds of networks.
    
      Cutoff scaling is $k^{-3}$: if $\gamma>3$ then
      we have to look harder at $\tavg{k}_R$.
    
      How about $P_k = \delta_{kk_0}$?
    
  
  

\begin{frame}[label=]
  \textbf{Giant component}
  
  \small
  \textbf{And how big is the largest component?}
    
    
      Define \alertb{$S_1$} as the \alert{size of the largest component}.
    
      Consider an infinite ER random network with average
      degree $\tavg{k}$.
     
      Let's find $S_1$ with a back-of-the-envelope argument.
     
      Define \alert{$\delta$} as the probability that a randomly
      chosen node \alert{does not} belong to the largest component.
     Simple connection: $\delta = 1 - S_1$.
    
      \alertb{Dirty trick:} If a randomly chosen node is not part
      of the largest component, then none of its neighbors are.
    
      So
      $$
      \delta = \sum_{k=0}^\infty P_k \delta^k
      $$
    
      Substitute in Poisson distribution...
    
  


\begin{frame}[label=]
  \textbf{Giant component}
  
  
   
    Carrying on:
    $$ 
    \alert{\delta} = \sum_{k=0}^{\infty} P_k \delta^k 
    {
      = \sum_{k=0}^{\infty} \frac{\tavg{k}^k}{k!} e^{-\tavg{k}} \delta^k 
    }
    $$
    {
      $$
      = e^{-\tavg{k}} \sum_{k=0}^{\infty} \frac{(\tavg{k} \delta)^k}{k!}
      $$
    }
    {
    $$
    =
    e^{-\tavg{k}}
    e^{\tavg{k}\delta}
    }
    {
    =
    \alert{e^{-\tavg{k}(1-\delta)}}.
    $$
    }
  
    Now substitute in $\delta = 1 -S_1$ and rearrange
    to obtain:
    $$
    S_1 = 1 - e^{-\tavg{k}S_1}.
    $$
  



\begin{frame}[label=]
  \textbf{Giant component}
  
  
   
    We can figure out some limits and details for 
    $
    S_1 = 1 - e^{-\tavg{k}S_1}.
    $
   
    First, we can write $\tavg{k}$ in terms of $S_1$:
    $$
    \tavg{k} = \frac{1}{S_1}\ln{\frac{1}{1-S_1}}.
    $$
  
    As $\tavg{k} \rightarrow 0$, $S_1 \rightarrow 0$.
  
    As $\tavg{k} \rightarrow \infty$, $S_1 \rightarrow 1$.
  
    Notice that at $\tavg{k}=1$, the critical point, $S_1=0$.
  
    Only solvable for $S_1>0$ when $\tavg{k}>1$.
  
    Really a transcritical bifurcation.\cite{strogatz1994a}
  


\begin{frame}[label=]
  \textbf{Giant component}

  \includegraphics[height=0.9\textheight,angle=-90]{figphastrans3}


\begin{frame}[label=]
  \textbf{Giant component}

  \textbf{Turns out we were lucky...}
    
     
      Our dirty trick \alert{only works for} ER random networks.
     
      \alert{The problem:} We assumed that neighbors have the same
      probability $\delta$ of belonging to the
      largest component.
     
      But we know our friends are different from us...
    
      Works for ER random networks because $\tavg{k} = \tavg{k}_R$.
    
      We need a separate probability $\delta'$ for
      the chance that an edge \alert{leads to}
      the giant (infinite) component.
    
      We can sort many things out with \alertb{sensible probabilistic arguments}...
    
      More detailed investigations will profit from
      a spot of \alertb{Generatingfunctionology}.\cite{wilf2006a}
    
  


