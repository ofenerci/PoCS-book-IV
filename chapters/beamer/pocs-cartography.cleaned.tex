http://strangemaps.wordpress.com/2009/10/26/419-france-reconstructed-from-apparently-inadequate-data/


How little information do you need to be able to draw a map? This
zen-like question provided the basis for a short article in the May
21st, 1971 issue of Nature, intriguingly entitled Construction of Maps
from ``Odd Bits of Information.''  The article, according to its
author David G. Kendall of Cambridge University's Statistical
Laboratory, starts from a ``rather general principle in historical
geography'', i.e. that maps can indeed be produced from apparently
inadequate data, and goes on to describe a research programme based on
that principle, carried out by Kendall's lab.  The research
concentrated on setting up a suitable (dis)simularity matrix believed
to ``lie naturally'' in a Euclidian space of k dimensions, making use of
a computer programme called MD-SCAL. The article mentions two
experiments, the first one involving the mapping of eight parishes of
the district of Otmoor in Oxfordshire. Amazingly, a fairly accurate
map for the eight parishes was extrapolated solely from data on the
intermarriage rates between them for the period 1600-1850.  The second
experiment involved a map of 88 French departments (excluding the
Corsican and Parisian ones), with the only information available being
``whether or not one of the 3,828 pairs of departments shares a common
boundary.'' The map thus computer-produced is one ``in which each
department is represented by a point, but this system of linked points
is converted to a honeycomb of cells by exploiting a natural duality.''
Mr Kendall finally mentions a future experiment with MD-SCAL: ``The
next step [...] will be to attempt to reconstruct a fifteenth century
manor from the abuttals in a contemporary cartulary.''  These maps show
France as it really is, and France reconstructed from abuttal
data. Please note that the departments are numbered not in the usual
alphabetical order, but by ``an alternative which approximately orders
the departments first by longitude and then by latitude.''

