%% change drunkard to zombie?

%% \changelogo{.18}{Tardis-tp-3.pdf}

%%%%%%%%%%%%%%%%%%%
% gaussians
% proof of central limit theorem?
% renormalization group approach
% random walks
%%%%%%%%%%%%%%%%%%%

%%% 
%%% really fix up this slides
%%% 
%%% next time:
%%%
%%% check on holtsbrook (any data?)
%%%
%%% roughness and hurst exponents
%%% add sub and super diffusion
%%%
%%% more on recurrence
%%% show that probability of return = 1

%% Add plinko video!
%% 
%% /Users/dodds/flow/2007/2007-10-11plinko.mov
%% 
%% add gaussian!
%% Limit of Gaussian from a binomial
%% good problem for problem set
%% 
%% talk about stable distributions
%% 
%% prove central limit theorem
%% problem set?

\changelogo{.18}{2013-11-26pocsmap-distributions-mountains-roads-sketches_postcard0_polaroid.png}

  \includegraphics[width=\textwidth]{2013-11-26pocsmap-distributions-mountains-roads-sketches_postcard0_polaroid.png}

  \includegraphics[width=\textwidth]{2013-11-26pocsmap-distributions-mountains-roads-sketches_postcard1_polaroid.png}

\section{Random\ Walks}

  \textbf{Mechanisms:}

  \textbf{A powerful story in the rise of complexity:}
    
    
      \alertg{structure arises out of randomness.}
    
      \alertg{Exhibit A:} 
      \wordwikilink{http://en.wikipedia.org/wiki/Random_walk}{Random walks.}
    
  

  \textbf{The essential random walk:}
    
    
      One spatial dimension.
    
      Time and space are discrete
    
      Random walker (e.g., a drunk) starts at origin $x=0$.
    
      Step at time $t$ is $\epsilon_t$:
      $$
      \epsilon_t = 
      \left\{
        \begin{array}{ll}
          +1 & \mbox{with probability 1/2} \\
          -1 & \mbox{with probability 1/2} \\
        \end{array}
      \right.
      $$
    
  


  \textbf{A few random random walks:}

  
    \includegraphics[width=\textwidth]{figrandwalk1_noname.pdf}\\
    \includegraphics[width=\textwidth]{figrandwalk2_noname.pdf}\\
    \includegraphics[width=\textwidth]{figrandwalk3_noname.pdf}\\
    \includegraphics[width=\textwidth]{figrandwalk5_noname.pdf}
  


  \textbf{Random walks:}

  \textbf{Displacement after $t$ steps:}
    $$x_t = \sum_{i=1}^{t} \epsilon_i$$  
  

  \medskip

  \textbf{Expected displacement:}
    $$
    \avg{x_t} = \avg{\sum_{i=1}^{t} \epsilon_i}
    {= \sum_{i=1}^{t} \avg{\epsilon_i}}
    {= 0}
    $$
    
    
      At any time step, we `expect' our drunkard to be back at the pub.
    
      Obviously fails for odd number of steps...
    
      But as time goes on, the chance of our drunkard lurching back
      to the pub must diminish, right?
    
  


%%  \textbf{Random walks}

  \textbf{\wordwikilink{http://en.wikipedia.org/wiki/Variance\#Variance\_of\_the\_sum\_of\_uncorrelated\_variables}{Variances sum:}$^*$}
    \alertb{$$ \var(x_t) = \var\left( \sum_{i=1}^{t} \epsilon_i \right) $$}
    $$
    {= \sum_{i=1}^{t} \var\left(  \epsilon_i \right) }
    {= \sum_{i=1}^{t} 1}
    {= t }
    $$
    {\small $^*$ Sum rule = a good reason for using the variance to measure spread; 
      only works for independent distributions.}
  

  \textbf{So typical displacement from the origin scales as:}
    $$\boxed{\sigma = t^{1/2}}$$
    
     
      A non-trivial scaling law arises out of\\
      \qquad \alertb{additive aggregation} or \alertb{accumulation}.
    
  
  


%%%%%%%%%%%%%%%%%%%
  %% gaussians
  %% proof of central limit theorem?
  %% renormalization group approach
  %% random walks
%%%%%%%%%%%%%%%%%%%


  Stock Market randomness:

  \begin{center}

    \includemedia[
    label=vid.stockmarket,
    %% width=1\linewidth,height=0.75\linewidth,
    width=1\linewidth,height=0.5625\linewidth,
    activate=pageopen,
    flashvars={
      modestbranding=1 % no YT logo in control bar
      &autoplay=0 % 
      &autohide=1 % controlbar autohide
      &showinfo=0 % no title and other info before start
      &end=120 % finish here
      &rel=0 % no related videos after end, must be included in URL
      %% as ?rel
    }
    ]
    {}
    {http://www.youtube.com/v/AUSKTk9ENzg?rel=0}

  \end{center}

    Also known as the \wordwikilink{http://en.wikipedia.org/wiki/Bean_machine}{bean machine},
    the
    \wordwikilink{http://www.mathsisfun.com/data/quincunx.html}{quincunx
    (simulation)},
    and the Galton box.


  
  
  \textbf{Great moments in Televised Random Walks:}
    \youtubevideo{05gqx6eSyO0}{}{}

    \wordwikilink{http://en.wikipedia.org/wiki/Plinko}{Plinko!}
    from the Price is Right.
  



  \textbf{Random walk basics:}

  \textbf{Counting random walks:}
    
    
      Each \alertr{specific} random walk of
      length $t$ appears with a chance $1/2^{t}$.
    
      We'll be more interested in how many random walks
      end up at the same place.
     
      Define
      $
      N(i,j,t)
      $
      as \# distinct walks that start at $x=i$
      and end at $x=j$ after $t$ time steps.
     
      Random walk must displace by $+(j-i)$ after $t$ steps.
    
      \insertassignmentquestionsoft{02}{2}
      $$ 
      N(i,j,t) = \binom{t}{(t+j-i)/2} 
      $$
    
  



  \textbf{How does $P(x_{t})$ behave for large $t$?}
    
    
      Take time $t=2n$ to help ourselves.
    
      $x_{2n} \in \{0, \pm 2, \pm 4, \ldots, \pm 2n\}$
    
      $x_{2n}$ is even so set $x_{2n} = 2k$.
    
      Using our expression $N(i,j,t)$ with
      $i=0$, $j=2k$, and $t=2n$, we have
      $$
      \mathbf{Pr}(x_{2n} \equiv 2k) 
      \propto 
      \binom{2n}{n+k}
      $$
    
      For large $n$, the binomial
      deliciously approaches the Normal Distribution of Snoredom:
      $$
      \mathbf{Pr}(x_{t} \equiv x) 
      \simeq
      \frac{1}{\sqrt{2\pi t}}
      e^{-\frac{x^2}{2t}}.
      $$
      \insertassignmentquestionsoft{02}{2}
     
      The whole is different from the parts. \hfill \alertg{\#nutritious}
     
      See also: \wordwikilink{http://en.wikipedia.org/wiki/Stable_distribution}{Stable Distributions}
    
  



  \textbf{\wordwikilink{http://en.wikipedia.org/wiki/Universality\_(dynamical\_systems)}{Universality} is also not left-handed:}

  
    \includegraphics[width=0.75\textwidth]{2013-01-24random-walk-normal-distribution-crop-tp-3.pdf}
    
     
      This is \wordwikilink{http://en.wikipedia.org/wiki/Diffusion}{Diffusion}: 
      the most essential kind of spreading (more later).
     
    View as Random Additive Growth Mechanism.
  
  


  \includegraphics[width=\textwidth]{2013-11-26pocsmap-distributions-mountains-roads-sketches_postcard0_polaroid.png}

  \includegraphics[width=\textwidth]{2013-11-26pocsmap-distributions-mountains-roads-sketches_postcard2_polaroid.png}

\changelogo{.18}{2013-11-26pocsmap-distributions-mountains-roads-sketches_postcard2_polaroid.png}

%%  \textbf{Random walks}

  \textbf{Random walks are even weirder than you might think...}
    
    
      $\xi_{r,t}$ = the probability that by time step $t$,
      a random walk has crossed the origin $r$ times.
    
      Think of a coin flip game with ten thousand tosses.
    
      If you are behind early on, what are the chances you
      will make a comeback?
    
      The most likely number of lead changes is...  
      {\alertb{0.}}
    
      In fact:
      $\xi_{0,t} > \xi_{1,t} > \xi_{2,t} > \cdots $
    
      \alertb{Even crazier:\\ 
        The expected time between tied scores = $\infty$}
    
    {\small See Feller, Intro to Probability Theory, Volume I\cite{feller1968a}} 
  



%% %%   \textbf{Random walks---some examples}
%% 
%%   \includegraphics[width=\textwidth]{figrandwalk4_noname.pdf}\\
%%   \includegraphics[width=\textwidth]{figrandwalk6_noname.pdf}
%% 
%% 

\subsection{The\ First\ Return\ Problem}

  \textbf{Random walks \hfill \alertg{\#crazytownbananapants}}

  \textbf{The problem of first return:}
    
     
      What is the probability that a random walker
      in one dimension returns to the origin
      for the first time after $t$ steps?
     
      Will our drunkard always return to the origin?
     
      What about higher dimensions?
    
  

  \textbf{Reasons for caring:}
    
     
      We will find a power-law size distribution
      with an \alertb{interesting} exponent.
     
      Some physical structures may result from random walks.
     
      We'll start to see how different scalings relate to each other.
    
  



%% %%   \textbf{Random Walks}
%% 
%%   \includegraphics[width=\textwidth]{figrandwalk2_noname.pdf}
%% 
%%   \includegraphics[width=\textwidth]{figrandwalk5_noname.pdf}
%% 
%%   {
%%   Again: expected time between ties = $\infty$...
%% 
%%   Let's find out why...\cite{feller1968a}}
%% 
%% 

%%%%%%%%%%%%%%%%%%%
% gaussians
% proof of central limit theorem?
% renormalization group approach
% random walks
%%%%%%%%%%%%%%%%%%%

  %% \textbf{First Returns}

  \textbf{For random walks in 1-$d$:}
    \includegraphics[width=\textwidth]{figrandomwalk_firstreturn_noname.pdf}
    
    
      A \alertg{return} to origin can only happen when $t = 2n$.
    
      In example above, returns occur at $t=8$, 10, and 14.
    
      Call $P_{\textrm{fr}(2n)}$ the probability of \alertg{first return} at $t=2n$.
     
      Probability calculation $\equiv$ Counting problem \\
      (combinatorics/statistical mechanics).
    
      \alertb{Idea:} Transform first return problem into an easier return problem.
      %% $$P_{\textrm{fr}(2n) = 2 Pr(x_{t} \ge 1, t=1,\ldots,2n-1, \ \mbox{and} \  x_{2n} =0). $$
    
  



  %% \textbf{First Returns}
  \small

  
          
      \includegraphics[width=.95\textwidth]{figrandomwalk_firstreturn2_noname.pdf}%
      
      \includegraphics[width=.95\textwidth]{figrandomwalk_firstreturn3_noname.pdf}
        %% \includegraphics[width=\textwidth]{figrandomwalk_firstreturn4_noname.pdf}
    
    
      Can assume drunkard first lurches to $x=1$.
     
      Observe walk first returning at $t=16$ stays at or above $x=1$ for $1 \le t \le 15$
      (dashed \alertb{red} line).
     
      Now want walks that can return many times to $x=1$.
     
      $P_{\textrm{fr}}(2n) = $ \\
      $2\cdot\frac{1}{2}Pr(x_{t} \ge 1, 1 \le t \le 2n-1, \ \mbox{and} \  x_1 = x_{2n-1} = 1) $
     
      The $\frac{1}{2}$ accounts for $x_{2n}=2$ instead of 0.
     
      The $2$ accounts for drunkards that first lurch to $x=-1$.
    
  


  \textbf{Counting first returns:}

  \textbf{Approach:}
    
    
      Move to counting numbers of walks.
    
      Return to probability at end.
    
      Again,
      $N(i,j,t)$ is the \# of possible walks between $x=i$ and $x=j$ taking $t$ steps.
     
      Consider \alertb{all paths} starting at $x=1$ and ending at $x=1$ after $t=2n-2$ steps.
     
      \alertb{Idea:} If we can compute the number of walks that hit $x=0$ at least once, then we can
      subtract this from the total number to find the ones that maintain $x \ge 1$.
     
      Call walks that drop below $x=1$ \alertb{excluded walks}.
     
      We'll use a method of images to identify these excluded walks.
    
  


%%  \textbf{First Returns:}

  \textbf{Examples of excluded walks:}
    \begin{center}
      \includegraphics[width=.4\textwidth]{figrandomwalk_firstreturn5_noname.pdf}
      \includegraphics[width=.4\textwidth]{figrandomwalk_firstreturn6_noname.pdf}
    \end{center}
  

    \textbf{Key observation for excluded walks:}
      
        
        For any path starting at $x$$=$$1$ that hits 0,
        there is a unique matching path starting at $x$$=$$-1$.
         
        Matching path first mirrors and then tracks after first reaching $x$$=$$0$.
      
        \# of $t$-step paths starting and ending at $x$$=$$1$
        and hitting $x$$=$$0$ at least once\\
        {= \# of $t$-step paths starting at $x$$=$$-$1 and ending at $x$$=$$1$}
        {= $N(-1,1,t)$\\}
      
        So \alertb{$N_{\textrm{first\ return}}(2n) = N(1,1,2n-2) - N(-1,1,2n-2)$}
      
    



%% %%   \textbf{First Returns}
%% 
%%   Probability of first return at time $t=2n$ \\
%%   is \alertb{the same} as\\
%%   the probability of a walk returning at time $t=2n-2$
%%   such that $x_{t} \ge 0$ until then.
%%   
%%   $$P_{\textrm{first\ return}}(2n) = \frac{1}{2} P_{\textrm{return}}(2n-2)$$
%% 
%% 

  \textbf{Probability of first return:}

  \textbf{\insertassignmentquestionsoft{02}{2}:}
    
     
      Find 
      $$
      \boxed{
        N_{\textrm{fr}}(2n) \sim 
        \frac{
          2^{2n-3/2}
        }
        {
          \sqrt{2 \pi} n^{3/2}
        }.
      }
      $$
     
      Normalized number of paths gives probability.
     
      Total number of possible paths = $2^{2n}$.
    
      $$ 
      P_{\textrm{fr}}(2n) = \frac{1}{2^{2n}} N_{\textrm{fr}}(2n)
      $$
      {
        $$ 
        \simeq
        \frac{1}{2^{2n}}
        \frac{
          2^{2n-3/2}
        }
        {
          \sqrt{2 \pi} n^{3/2}
        }
        $$}
      $$
      {
        =  \frac{1}{\sqrt{2 \pi}}
        (2n)^{-3/2}
      }
      {
        \propto t^{-3/2}.
      }
      $$
    
    
  

  \small
%%  \textbf{First Returns}

  
  
  We have 
     $P(t) \propto t^{-3/2},\  \gamma = 3/2.$
   
    Same scaling holds for continuous space/time walks.
   
    $P(t)$ is normalizable.
   
    \alertb{Recurrence:} Random walker always returns to origin 
   
    But mean, variance, and all higher moments are infinite.
    \hfill \alertg{\#totalmadness}
   
    Even though walker must return, expect a long wait...
   
    \alertg{One moral}: 
    Repeated gambling against an infinitely wealthy opponent
    must lead to ruin.
  


\textbf{\wordwikilink{http://en.wikipedia.org/wiki/Random\_walk\#Higher\_dimensions}{Higher dimensions}:}
  
   
    Walker in $d=2$ dimensions must also return
   
    Walker may not return in $d \ge 3$ dimensions
  
  



  \textbf{Random walks}
  
  \textbf{On finite spaces:}
    
     In any finite homogeneous space, a random walker will visit every
      site with equal probability
%%     Random walking $\equiv$ Diffusion
     Call this probability the \alertb{Invariant Density} of
      a dynamical system
     Non-trivial Invariant Densities arise in chaotic systems.
    
  

  \textbf{On networks:}
    
    
      On networks, a random walker visits each node
      with frequency $\propto$ node degree
      \hfill \alertg{\#groovy}
    
      Equal probability still present:\\
      walkers traverse
      \alertb{edges} with equal frequency. \\
      \mbox{} \hfill \alertg{\#totallygroovy}
    
  



% ???


% %   \textbf{Random walks on networks}
%   
%   Some slides explaining diffusion
%
% 
% add section on
% linear algebra


% \subsection{Random\ River\ Networks}

\subsection{Examples}

  \textbf{Scheidegger Networks\cite{scheidegger1967b,dodds2000a}}

  
    \includegraphics[width=\textwidth]{scheidmodel.pdf}
    
     
      Random directed network on triangular lattice.
     
      Toy model of real networks.
     
      `Flow' is southeast or southwest with equal probability.
    
  


\changelogo{.18}{scheidmodel.pdf}

  \textbf{Scheidegger networks}

  
  
   
    Creates basins with random walk boundaries.
   
    \alertg{Observe} that subtracting one random walk from another
    gives random walk with increments:
    $$
    \epsilon_t = 
    \left\{
      \begin{array}{cl}
        +1 & \mbox{with probability $1/4$} \\
        0 & \mbox{with probability $1/2$} \\
        -1 & \mbox{with probability $1/4$} \\
      \end{array}
    \right.
    $$
   
    Random walk with probabilistic pauses.
   
    Basin termination = first return random walk problem.
   
    Basin length $\msl$ distribution: $P(\msl) \propto \msl^{-3/2}$
   
    For real river networks, generalize 
    to $P(\msl) \propto \msl^{-\gamma}$.
  
  
  

% \subsection{Scaling\ Relations}

  \textbf{Connections between exponents:}

  
  
     
    For a basin of length $\msl$, width $\propto \msl^{1/2}$
     
    Basin area $a \propto \msl\cdot \msl^{1/2} = \msl^{3/2}$
     
    Invert: $ \msl \propto a^{\, 2/3} $
     
    $ \dee{\msl} \propto \dee{(a^{2/3})} = 2/3 a^{-1/3} \dee{a} $
    
    \alertb{$
    \mathbf{Pr}(\mbox{basin area} = a) \dee{a}
    $}\\
    $
    =
    \mathbf{Pr}(\mbox{basin length} = \msl) \dee{\msl}
    $\\
    {
      $
      \propto
      \msl^{-3/2} \dee{\msl} 
      $\\}
    {
      $
      \propto
      (a^{2/3})^{-3/2} a^{-1/3} \dee{a} 
      $\\}
    {
      $
      =
      a^{-4/3} \dee{a} 
      $\\}
    {
      \alertb{$
      =
      a^{-\tau} \dee{a}
      $}\\}
  
  


  \textbf{Connections between exponents:}

  
  
   
    Both basin area and length obey power law distributions
   
    Observed for real river networks
   
    Reportedly: $1.3 < \tau < 1.5$ and $1.5 < \gamma < 2$
  
  

  \textbf{Generalize relationship between area and length:}
  
   
    Hack's law\cite{hack1957a}:
    $$\msl \propto a^h.$$
   
    For real, large networks $h \simeq 0.5$
   
    Smaller basins possibly $h>1/2$ (later: allometry).
   
    Models exist with interesting values of $h$.
   
    \alertb{Plan:} Redo calc with $\gamma$, $\tau$, and $h$.
  
  
  

  \textbf{Connections between exponents:}

  
  
      
    Given $$ \msl \propto a^{h}, \ P(a) \propto a^{-\tau},\ \mbox{and} \  P(\msl) \propto \msl^{-\gamma}$$
      
    $ \dee{\msl} \propto \dee{(a^{h})} = h a^{h-1} \dee{a} $    
     
    Find $\tau$ in terms of $\gamma$ and $h$.
     
    $
    \mathbf{Pr}(\mbox{basin area} = a) \dee{a}
    $\\
    $
    =
    \mathbf{Pr}(\mbox{basin length} = \msl) \dee{\msl}
    $\\
    {
    $
    \propto
    \msl^{-\gamma} \dee{\msl} 
    $\\}
    {
    $
    \propto
    (a^{h})^{-\gamma} a^{h-1} \dee{a} 
    $\\}
    {
    $
    =
    a^{-(1+h\, (\gamma-1))} \dee{a} 
    $\\}
    
      $$\boxed{\tau = 1 + h(\gamma-1)} $$
    
      Excellent example of the \alert{Scaling Relations}
      found between exponents describing power laws
      for many systems.
  
  



  \textbf{Connections between exponents:}

  \textbf{With more detailed description of network structure,
      $\tau = 1 + h(\gamma-1)$ simplifies to:\cite{dodds1999a}}
    $$\boxed{\tau = 2 - h}$$
    and
    $$\boxed{\gamma = 1/h}$$
  

  
    
     
      Only one exponent is independent (take $h$).
     
      Simplifies system description.
     
      Expect Scaling Relations where power laws are found.
    
      Need only characterize 
      \wordwikilink{http://en.wikipedia.org/wiki/Universality\_(dynamical\_systems)}{Universality}
      class with independent exponents.
  
  


%% ??? 
% make connection

  \includegraphics[width=\textwidth]{2013-11-26pocsmap-distributions-mountains-roads-sketches_postcard1_polaroid.png}

\changelogo{.18}{2013-11-26pocsmap-distributions-mountains-roads-sketches_postcard1_polaroid.png}

  \textbf{Other First Returns or First Passage Times:}

  \textbf{Failure:}
    
     A very simple model of failure/death:\cite{weitz2001a}
     $x_t$ = entity's `health' at time $t$
     Start with $x_0 > 0$.
     Entity fails when $x$ hits 0.
    
  

  \bigskip

  \textbf{Streams}
    
    
      Dispersion of suspended sediments in streams.
     
      Long times for clearing.
    
  



%% ???  Inverse Gaussian

% \end{comment}

  \includegraphics[width=\textwidth]{2013-11-26pocsmap-distributions-mountains-roads-sketches_postcard3_polaroid.png}

  \textbf{More than randomness}
  
  
    
    
      Can generalize to Fractional Random Walks\cite{montroll1982a,montroll1983a,montroll1984a}
     
      Levy flights, Fractional Brownian Motion
    
      See Montroll and Shlesinger for example:\cite{montroll1984a}\\
      ``On $1/f$ noise and other distributions with long tails.''\\
      Proc. Natl. Acad. Sci., 1982.
     
      In 1-d, standard deviation $\sigma$ scales as
      $$
      \sigma \sim t^{\, \alpha}
      $$
      \\
      {
        $\alpha = 1/2$ --- \alertb{diffusive}
      }
      \\
      {
        $\alpha > 1/2$ --- \alertb{superdiffusive}
      }
      \\
      {
        $\alpha < 1/2$ --- \alertb{subdiffusive}
      }
     Extensive memory of path now matters...
    
  

\end{frame}  

% how to measure exponents
% roughness

% ???
% %    \textbf{Random walks in nature}
%
% 
\neuralreboot{cBkWhkAZ9ds}{}{}{Desert rain frog/Squeaky toy:}

\section{Variable\ transformation}

\subsection{Basics}

%% ??? add a figure explaining transformation

  \textbf{Variable Transformation}

  \textbf{Understand power laws as arising from}
    
    
      Elementary distributions (e.g., exponentials).
    
      Variables connected by power relationships.
    
  

  
  
   
    Random variable $X$ with known distribution $P_x$
   
    Second random variable $Y$ with $y=f(x)$.
  
      
    
     
      $ P_Y(y) \dee{y} = = \sum_{x | f(x) = y} P_X(x) \dee{x} $ \\
      $ = \sum_{y | f(x) = y}
      P_X( f^{-1}(y) )
      \frac{\dee{y}}
      {
        \left|
          f'(f^{-1}(y))
        \right|
      }
      $
     Often easier to do by hand...
    
    
%%    Figure...
    




  \textbf{General Example}
    
    
      Assume relationship between $x$ and $y$ is 1-1.
     Power-law relationship between variables: \\
      $y = c x^{-\alpha}$, $\alpha > 0$
     Look at \alertb{$y$ large} and \alertb{$x$ small}
     $$ \dee{y} = \dee{\left(cx^{-\alpha }\right)}  $$
      {$$  = c(-\alpha)x^{-\alpha-1} \dee{x}  $$}
      {
        $$ \mbox{\alertb{invert:}} \ \dee{x} = \frac{-1}{c\alpha}
        x^{\alpha+1} \dee{y}
        $$
      }
      {
        $$ \dee{x} = \frac{-1}{c\alpha}
        \left(\frac{y}{c}\right)^{-(\alpha+1)/\alpha} \dee{y}
        $$
      }
      {
        $$ \dee{x} = \frac{-c^{1/\alpha}}{\alpha}
        y^{-1-1/\alpha} \dee{y}
        $$
      }
    
  



  \textbf{Now make transformation:}  
      $$
      P_y(y) \dee{y}
      = P_x(x) \dee{x}
      $$
      {$$
      P_y(y) \dee{y}
      = P_x
      \overbrace{
      \left(
        \alertb{
          \left(
            \frac{y}{c}
          \right)^{-1/\alpha}}
      \right)
        }^{(x)}
      \overbrace{
      \frac{c^{1/\alpha}}{\alpha}
      \tc{red}{y^{-1 - 1/\alpha}} \dee{y}}^{\dee{x}}
      $$}

      \bigskip

      
        
        If $P_x(x) \rightarrow$ non-zero constant
        as $x \rightarrow 0$ then
        $$
        P_x(y) 
        \propto 
        \tc{red}{y^{-1 -1/\alpha}} 
        \mbox{\ as \ } 
        y \rightarrow \infty. 
        $$
      
        If $P_x(x) \rightarrow x^{\beta}$ as $x \rightarrow 0$ then
        $$ 
        P_y(y) 
        \propto 
        \tc{red}{y^{-1 -1/\alpha - \beta/\alpha}} 
        \mbox{\ as \ } 
        y \rightarrow \infty.
        $$
      
  


%% %%   \textbf{General Example}
%% 
%%   $$
%%   P_y(y) \dee{y}
%%   = P_x
%%   \left(
%%     \alertb{
%%       \left(
%%         \frac{y}{c}
%%       \right)^{-1/\alpha}}
%%   \right)
%%   \frac{c^{1/\alpha}}{\alpha}
%%   \tc{red}{y^{-1 - 1/\alpha}} \dee{y}
%%   $$
%%   
%%   
%%   
%%     If $P_x(x) \rightarrow x^{\beta}$ as $x \rightarrow 0$ then
%% 
%%     $$ P_y(y) \propto \tc{red}{y^{-1 -1/\alpha - \beta/\alpha}} 
%%     \mbox{\ as \ } y \rightarrow \infty $$
%%   
%% 
%% 
  \textbf{Example}

  \textbf{Exponential distribution}
    Given \alertb{$P_x(x) = \frac{1}{\lambda} e^{-x/\lambda}$} 
    and \alertb{$y = c x^{-\alpha}$}, then
    $$ P(y) \propto y^{\tc{red}{-1-1/\alpha}} + O\left(y^{-1-2/\alpha}\right)$$
    
    
      Exponentials arise from randomness (easy)...
    
      More later when we cover robustness.
    
  



% %   \textbf{Real example}
% 
% %  ???
% 
% 
\insertvideo{XZ1kRxgKft4}{225}{260}{}

\subsection{Holtsmark's\ Distribution}

\changelogo{.18}{pocslogo100.pdf}

  \textbf{Gravity}
  
      
    
    
     
      Select a random point in the universe $\vec{x}$
     
      Measure the force of gravity $F(\vec{x})$
     
      Observe that $P_F(F) \sim F^{-5/2}$.
    
    
    
    \includegraphics[width=\textwidth]{Tardis-tp-3.pdf}
  

\changelogo{.18}{Tardis-tp-3.pdf}

%%  

  \textbf{Matter is concentrated in stars:\cite{sornette2003a}}
    
     
      $F$ is distributed unevenly
     
      Probability of being a distance $r$ from a
      single star at $\vec{x} = \vec{0}$:
      $$ P_r(r) \dee{r} \propto r^{2} \dee{r} $$
     
      Assume stars are distributed randomly in space (oops?)
     
      Assume only one star has significant effect at $\vec{x}$.
     
      Law of gravity: $$ F \propto r^{-2} $$
     
      \alertb{invert}: $$ r \propto F^{-1/2} $$
    
      Also invert:\\
      $ \dee{F} \propto \dee{(r^{-2})} \propto r^{-3} \dee{r} $
      $\rightarrow $
      $ \dee{r} \propto r^3 \dee{F} \propto F^{-3/2} \dee{F}.$
    
  


%% %%   \textbf{Transformation:}
%% 
%%   
%%     {
%%       $$ \dee{F} \propto \dee{(r^{-2})} $$
%%     }
%%     {
%%       $$ \propto r^{-3} \dee{r} $$
%%     }
%%     {
%%       \alertb{invert:} 
%%       $$ \dee{r} \propto r^3 \dee{F} $$ 
%%     }
%%     { 
%%       $$ \propto F^{-3/2} \dee{F} $$
%%     }
%%   
%%   
%% 
  \textbf{Transformation:}

  
    Using
    $\boxed{r \propto F^{-1/2}}$\ ,
    $\boxed{\dee{r} \propto F^{-3/2} \dee{F}}$\ ,
    and
    $\boxed{P_r(r) \propto r^{2}}$
    
     
      $$ P_F(F)\dee{F} = P_r(r) \dee{r} $$
     
      $$  \propto P_r(\mbox{const} \times F^{-1/2}) F^{-3/2} \dee{F} $$
     
      $$ \propto \left(F^{-1/2}\right)^{2}  F^{-3/2} \dee{F} $$
     
      $$ = F^{-1 -3/2} \dee{F} $$
     
      $$ = \tc{red}{F^{-5/2}} \dee{F}. $$
    
  


  \textbf{Gravity:}

  \textbf{$$ P_F(F) = \tc{red}{F^{-5/2}} \dee{F} $$}
  
     
    $$ \gamma = 5/2 $$
   
    Mean is finite.
   
    Variance = $\infty$.
   
    A \alertb{wild} distribution.
   
    \alertb{Upshot:} Random sampling of space usually safe\\
    but can end badly...
  
  


\insertvideo{Adk1ujjmguo}{}{}{Doctorin' the Tardis}

\begin{frame}[plain]
  \textbf{$\Box$ Todo: Build Dalek army.}
  
  \includegraphics[height=0.9\textheight]{dalek-blueprints-1024x768-wallpaper-911351.jpg}


\changelogo{.18}{2011-02-07no-mouse-click-tp-10}

\subsection{PLIPLO}

  \textbf{Extreme Caution!}

  
    
     
      \alertb{PLIPLO} = \alertg{Power law in, power law out}
     
      Explain a power law as resulting 
      from another unexplained power law.
    
      Yet another \wordwikilink{http://en.wikipedia.org/wiki/Homunculus_argument}{homunculus argument}...
    
      \alertb{Don't do this!!!}  (slap, slap)
    
      We need mechanisms!
    
  
  


