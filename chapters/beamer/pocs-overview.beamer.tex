
%% \changelecturelogo{.18}{2012-07-16compstorylab-stamp-circle-tp-1.pdf}
\changelecturelogo{.18}{2012-12-12computational-story-lab-logo-dark-tp-3.pdf}

%% \section{Computational\ Story\ Lab}
\begin{frame}
%%  \frametitle{Computational Story Lab:}

  \includegraphics[width=\textwidth]{2012-12-27collaborators-twitter.pdf}\\
  Funding: NSF, NASA, MITRE.
  
%%  \includegraphics[width=\textwidth]{collaborators-onehappybird-no-bliss1.jpg}
%%  \includegraphics[width=\textwidth]{collaborators-onehappybird-no-bliss1-tp-1.pdf}
%%  \includegraphics[width=\textwidth]{collaborators-onehappybird-no-bliss1-tp-3.pdf}
  %%  \includegraphics[width=\textwidth]{collaborators-onehappybird.jpg}
  %%   \includegraphics[width=\textwidth]{/Users/danforth/Local/collaborators-twitter.pdf}

\end{frame}

%% \begin{frame}
%%   \frametitle{To be clear, I work with this guy:}
%% 
%%   \includegraphics[width=\textwidth]{2012-12-18danforth-jumping-over-fire.jpg}
%% 
%% \end{frame}

\changelecturelogo{0.18}{lightbulb-idea-calculus-circle-tp-1.pdf}

\begin{frame}

  \setlength{\fboxsep}{0pt}
  \setlength{\fboxrule}{0.5pt}

  \begin{center}
    \href{\papersdir/1999/dodds1999a.pdf}{\fbox{\includegraphics[height=0.120\textheight]{dodds1999a_fig2_400px.jpg}}}
    \href{\papersdir/2000/dodds2000a.pdf}{\fbox{\includegraphics[height=0.120\textheight]{dodds2000a_fig1_400px.jpg}}}
    \href{\papersdir/2001/dodds2001a.pdf}{\fbox{\includegraphics[height=0.120\textheight]{dodds2001a_fig8_400px.jpg}}}
    \href{\papersdir/2001/dodds2001b.pdf}{\fbox{\includegraphics[height=0.120\textheight]{dodds2001b_fig1_400px.jpg}}}
    \href{\papersdir/2001/dodds2001c.pdf}{\fbox{\includegraphics[height=0.120\textheight]{dodds2001c_fig1_400px.jpg}}}
    \href{\papersdir/2001/dodds2001d.pdf}{\fbox{\includegraphics[height=0.120\textheight]{dodds2001d_figextra1_400px.jpg}}}
    \href{\papersdir/2002/dodds2002a.pdf}{\fbox{\includegraphics[height=0.120\textheight]{dodds2002a_fig3_400px.jpg}}}
    \href{\papersdir/2002/watts2002b.pdf}{\fbox{\includegraphics[height=0.120\textheight]{watts2002b_fig1_400px.jpg}}}
    \href{\papersdir/2003/dodds2003a.pdf}{\fbox{\includegraphics[height=0.120\textheight]{dodds2003a_fig2_400px.jpg}}}
    \href{\papersdir/2003/dodds2003b.pdf}{\fbox{\includegraphics[height=0.120\textheight]{dodds2003b_fig1_400px.jpg}}}
    \href{\papersdir/2003/dodds2003c.pdf}{\fbox{\includegraphics[height=0.120\textheight]{dodds2003c_fig2_400px.jpg}}}
    \href{\papersdir/2004/dodds2004a.pdf}{\fbox{\includegraphics[height=0.120\textheight]{dodds2004a_fig2_400px.jpg}}}
    \href{\papersdir/2005/dodds2005a.pdf}{\fbox{\includegraphics[height=0.120\textheight]{dodds2005a_fig1c_400px.jpg}}}
    \href{\papersdir/2005/watts2005a.pdf}{\fbox{\includegraphics[height=0.120\textheight]{watts2005a_fig2_400px.jpg}}}
    \href{\papersdir/2006/salganik2006a.pdf}{\fbox{\includegraphics[height=0.120\textheight]{salganik2006a_figextra1_400px.jpg}}}
    \href{\papersdir/2007/hanaki2007a.pdf}{\fbox{\includegraphics[height=0.120\textheight]{hanaki2007a_fig1_400px.jpg}}}
    \href{\papersdir/2007/watts2007a.pdf}{\fbox{\includegraphics[height=0.120\textheight]{watts2007a_fig1_400px.jpg}}}
    \href{\papersdir/2008/hartmann2008a.pdf}{\fbox{\includegraphics[height=0.120\textheight]{hartmann2008a_figextra1_400px.jpg}}}
    \href{\papersdir/2009/dodds2009a.pdf}{\fbox{\includegraphics[height=0.120\textheight]{dodds2009a_fig1_400px.jpg}}}
    \href{\papersdir/2009/dodds2009b.pdf}{\fbox{\includegraphics[height=0.120\textheight]{dodds2009b_fig6_400px.jpg}}}
    \href{\papersdir/2009/payne2009a.pdf}{\fbox{\includegraphics[height=0.120\textheight]{payne2009a_fig5_400px.jpg}}}
    \href{\papersdir/2010/dodds2010a.pdf}{\fbox{\includegraphics[height=0.120\textheight]{dodds2010a_fig1_400px.jpg}}}
    \href{\papersdir/2011/dodds2011b.pdf}{\fbox{\includegraphics[height=0.120\textheight]{dodds2011b_fig1_400px.jpg}}}
    \href{\papersdir/2011/payne2011a.pdf}{\fbox{\includegraphics[height=0.120\textheight]{payne2011a_fig1_400px.jpg}}}
    \href{\papersdir/2011/dodds2011d.pdf}{\fbox{\includegraphics[height=0.120\textheight]{dodds2011d_fig1_400px.jpg}}}
    \href{\papersdir/2011/dodds2011a.pdf}{\fbox{\includegraphics[height=0.120\textheight]{dodds2011a_fig2_400px.jpg}}}
    \href{\papersdir/2011/kloumann2011a.pdf}{\fbox{\includegraphics[height=0.120\textheight]{kloumann2011a_fig1_400px.jpg}}}
    \href{\papersdir/2012/bliss2012a.pdf}{\fbox{\includegraphics[height=0.120\textheight]{bliss2012a_figA4_400px.jpg}}}
    \href{\papersdir/2012/price2012a.pdf}{\fbox{\includegraphics[height=0.120\textheight]{price2012a_figextra1_400px.jpg}}}
    \href{\papersdir/2012/dodds2012a.pdf}{\fbox{\includegraphics[height=0.120\textheight]{dodds2012a_fig3_400px.jpg}}}
    \href{\papersdir/2013/mitchell2013a.pdf}{\fbox{\includegraphics[height=0.120\textheight]{mitchell2013a_fig1_400px.jpg}}}
    \href{\papersdir/2013/harris2013a.pdf}{\fbox{\includegraphics[height=0.120\textheight]{harris2013a_fig7_400px.jpg}}}
    \href{\papersdir/2013/frank2013a.pdf}{\fbox{\includegraphics[height=0.120\textheight]{frank2013a_fig3_400px.jpg}}}
    \href{\papersdir/2011/dodds2011d.pdf}{\fbox{\includegraphics[height=0.120\textheight]{dodds2011d_fig1_400px.jpg}}}
    \href{\papersdir/2013/bliss2013a.pdf}{\fbox{\includegraphics[height=0.120\textheight]{bliss2013a_fig1_400px.jpg}}}
    \href{\papersdir/2013/gottesman2013a.pdf}{\fbox{\includegraphics[height=0.120\textheight]{gottesman2013a_fig1_400px.jpg}}}
  \end{center}

\end{frame}

%% heroes

\changelecturelogo{.18}{sealie-lambie-productions.jpg}


\section{Orientation}

\subsection{Course\ Information}

\begin{frame}
  \frametitle{Basics:}

  \begin{block}{}
  \begin{itemize}
  \item 
    \alertb{Instructor:} \myname
  \item 
    \alertb{Lecture room and meeting times:}\\
    \lectureroom, \meetingtimes
  \item 
    \alertb{Office:} \officelocation
  \item 
    \alertb{email:} \myemail
  \item 
    \alertb{Course Website:} \newline
    {\small
      \wordwikilinklong{\coursewebsite}{\coursewebsite}}
  \item
    \alertb{Course Twitter handle:} \coursehandle
  \item
    \alertb{Course hashtag:} \coursehashtag
  \end{itemize}
  \end{block}

\end{frame}

\begin{frame}
%%  \frametitle{Admin:}

  \begin{block}<1->{Potential paper products:}
    \begin{itemize}
    \item<1-> 
      The \syllabuslink{\coursewebsite/docs/\courseprefix}
      and a 
      \posterlink{\coursewebsite/docs/\courseprefix}.
    \end{itemize}
  \end{block}

  \begin{block}<2->{Office hours:}
    \begin{itemize}
    \item<2-> 
      \officehours, \\ \officelocation
    \end{itemize}
  \end{block}

  \begin{block}<3->{Graduate Certificate:}
    \begin{itemize}
    \item
      Principles of Complex Systems is one of two core requirements
      for UVM's five course
      \wordwikilink{http://www.uvm.edu/~cems/complexsystems/?Page=gradApplications/certificate.php}{Certificate of Graduate Study in Complex Systems}.
    \item
     Other required course:  Prof.\ Maggie Eppstein's ``Modelling Complex Systems'' (CSYS/CS 302).
    \item
      The Sequel to PoCS: ``Complex Networks'' (CSYS/MATH 303).
    \end{itemize}
  \end{block}

\end{frame}

\begin{frame}

  \frametitle{Exciting details regarding these slides:}

  \begin{block}<1->{}
    \begin{itemize}
    \item<+->
      Three versions (all in pdf): 
      \begin{enumerate}
      \item 
        Presentation,
      \item
        Flat Presentation,
      \item
        Handout (3x2 slides per page).
      \end{enumerate}
    \item<+->
      Presentation versions are \alertb{hyperly navigable}:\newline
      \insertbackfindforwardnavigationsymbol
      $\equiv$
      back + search + forward.
    \item<+->
      Web links look 
      \wordwikilink{http://www.google.com}{like this}
      and are eminently clickable.
    \item<+->
      References in slides link to full citation at end.\cite{anderson1972a}
    \item<+->
      Citations contain links to pdfs for papers (if available).
    \item<+->
      Some books will be linked to on amazon.
    \item<+->
      Brought to you by a frightening melange of 
      \wordwikilink{http://en.wikipedia.org/wiki/LaTeX}{\LaTeX}, 
      \wordwikilink{http://en.wikipedia.org/wiki/Beamer\_(LaTeX)}{Beamer}, 
      \wordwikilink{http://www.perl.org/}{perl}, 
      \wordwikilink{http://www.ctan.org/tex-archive/macros/latex/contrib/perltex/}{PerlTeX},
      \wordwikilink{http://en.wikipedia.org/wiki/Command-line_interface}{fevered command-line madness},
      and 
      \wordwikilink{http://www.youtube.com/watch?v=uprjmoSMJ-o}{an almost fanatical devotion}
      to the indomitable \wordwikilink{http://en.wikipedia.org/wiki/Emacs}{emacs}.\newline
      \uncover<+->{\alertr{\#superpowers}}
    \end{itemize}
  \end{block}

\end{frame}

\begin{frame}

  \begin{block}{More super exciting details:}
    \begin{itemize}
    \item
      This is Season \courseseason\ of \coursename.
    \item
      Lectures will be called Episodes.
    \item
      All lectures are 
      \wordwikilink{http://en.wikipedia.org/wiki/Bottle_episode}{bottle} 
      \wordwikilink{http://tvtropes.org/pmwiki/pmwiki.php/Main/BottleEpisode}{episodes}.
    \item
      \wordwikilink{http://tvtropes.org}{Other tropes} will be involved.
    \end{itemize}
  \end{block}

  \begin{center}
    \includemedia[
    label=vid.episodeexample,
    width=.8\linewidth,height=0.45\linewidth,
    activate=pageopen,
    flashvars={
      modestbranding=0 % no YT logo in control bar
      &autoplay=1
      &fs=1
      &autohide=1 % controlbar autohide
      &showinfo=0 % no title and other info before start
      &rel=0
      &start=30
      &end=90
      % &rel=0 % no related videos after end
    }
    ]
    {}
    {http://www.youtube.com/v/6lzrm3SQynQ?rel=0}

    \begin{itemize}
    \item
      Last season's Episodes are \wordwikilink{http://www.uvm.edu/~pdodds/teaching/courses/2013-01UVM-300/content/lectures.html}{here}.
    \end{itemize}
  \end{center}
  
\end{frame}



\begin{frame}
  \frametitle{Grading breakdown:}

  \begin{block}{}
    \begin{itemize}
    \item 
      \alertb{Projects/talks (36\%)}---Students will 
      work on semester-long projects.  Students will develop a
      proposal in the first few weeks of the course which will be discussed
      with the instructor for approval.  

      \smallskip

      Details: 12\% for the first talk, 12\% for the final talk,
      and 12\% for the written project.
    \end{itemize}
  \end{block}
  \begin{block}{}
    \begin{itemize}
    \item 
      \alertb{Assignments (60\%)}---All assignments will be 
      of equal weight and there will be nine or ten of them.
    \end{itemize}
  \end{block}
  \begin{block}{}
    \begin{itemize}
    \item 
      \alertb{General attendance/Class participation (4\%)}
    \end{itemize}
  \end{block}

\end{frame}

\begin{frame}
  \frametitle{How grading works:}

  \begin{block}{Questions are worth 3 points according to the following scale:}
    \begin{itemize}
    \item 
      3 = correct or very nearly so.
    \item 
      2 = acceptable but needs some revisions.
    \item 
      1 = needs major revisions.
    \item 
      0 = way off.
    \end{itemize}

  \end{block}


\end{frame}


\begin{frame}
  \frametitle{Important things:}

  \begin{block}{}
    \importantdates
  \end{block}

  \begin{block}{}
    \alertb{Do} check your zoo account for updates regarding the course.
  \end{block}

  \begin{block}{}
    \alertb{Academic assistance:} Anyone who requires assistance in any way 
    (as per the ACCESS program
    or due to athletic endeavors), please see or contact me as soon as possible.
  \end{block}

\end{frame}

\changelecturelogo{0.18}{MacGillivray_William_John_Dory_cut.png}

\subsection{Topics}

\begin{frame}

  \begin{block}<+->{The nature of PoCS:}
    \begin{itemize}
    \item
      Transitional from standard coursework to research-focused work.
      \uncover<+->{\alertr{\#alittlescary}}
    \end{itemize}
  \end{block}

  \begin{block}<+->{Major themes:}
    \begin{itemize}
    \item<+-> 
      The Complexity Manifesto;
    \item<+-> 
      Complex Systems $\equiv$ Modern, Normal Science;
    \item<+->
      Roles and limits of Data, Theory, and Experiment;
    \item<+->
      Emergence;
    \item<+->
      Universality and Accidents of History;
    \item<+-> 
      Structure and Stories: Micro-to-macro Mechanisms;
    \item<+->
      Elements: 
      Scaling, Surprise, Networks, 
      Robustness, Failure,
      and
      Spreading.
    \item<+-> 
      The Theory of Anything: Why Complexify?
    \end{itemize}
  \end{block}


%%    Measuring complexity

\end{frame}


\begin{frame}
  \frametitle{Topics:}
  
  \begin{block}{Scaling phenomena:}
    \begin{itemize}
    \item 
      Power law size distributions and non-Gaussian statistics and 
    \item 
      Zipf's law
    \item 
      Key mechanisms for generating power law size distributions
    \item 
      Allometry
    \item 
      Scaling of social phenomena: crime, creativity, and consumption.
    \item 
      Scaling in biology (elephants and platypuses).
    \item 
      Renormalization techniques
    \end{itemize}
  \end{block}

%%  \begin{block}{Measures of complexity}
%%  \end{block}

  %% \item Growth mechanisms (part of above story)
  %%   \begin{itemize}
  %%   \item Size distributions
  %%   \item Agglomeration processes
  %%   \item Accretion processes
  %% % read juanico2006ua.pdf
  %% % look for general agglomeration stuff
  %%   \end{itemize}

\end{frame}

\begin{frame}
  \frametitle{Topics:}

  \begin{block}{Complex networks:}
    \begin{itemize}
    \item 
      Structure and Dynamics;
    \item 
      Statistical Mechanics;
    \item 
      Phase transitions;
    \item 
      Random Networks;
    \item 
      Scale-free Networks;
    \item 
      Small-world Networks.
    \end{itemize}
  \end{block}

  \begin{block}{Multiscale complex systems:}
    \begin{itemize}
    \item 
      Hierarchies and Scaling;
    \item 
      Modularity;
%%    \item 
%%      Form and context in design.
    \end{itemize}
  \end{block}

\end{frame}

\begin{frame}
  \frametitle{Topics:}

  \begin{block}{Integrity of complex systems:}
    \begin{itemize}
    \item 
      Generic failure mechanisms
    \item 
      Network robustness
    \item 
      Highly Optimized Tolerance (HOT): Robustness and fragility
    \item 
      Predictablity
%%    \item 
%%    Normal accidents and high reliability theory
    \end{itemize}
  \end{block}

  \begin{block}{Information and Language:}
    \begin{itemize}
    \item 
      Search in networked systems \\ (e.g., the web, social systems)
    \item 
      Search on scale-free networks
    \item 
      Knowledge trees, metadata and tagging
    \item 
      Evolution and structure of natural languages
    \end{itemize}
  \end{block}

\end{frame}

\begin{frame}
  \frametitle{Topics:}


  \begin{block}{Sociotechnical Systems:}
    \begin{itemize}
    \item 
      Biological and social spreading models;
    \item 
      Schelling's model of segregation;\cite{schelling1971a}
    \item 
      Granovetter's model of imitation;\cite{granovetter1978a}
    \item 
      Collective behavior and Synchrony;
    \item 
      Global cooperation from bad actors;
    \item 
      Global conflicts from good actors;
    \item 
      Stories (Homo Narrativus);
    \item 
      The Sociotechnocene.
    \end{itemize}
  \end{block}

\end{frame}

\begin{frame}
  \frametitle{Topics:}

  \begin{block}{Large-scale social patterns:}
    \begin{itemize}
    \item 
      Movement of individuals;
    \item 
      Cities;
    \item 
      Happiness;
    \item 
      Twitter.
    \end{itemize}
  \end{block}

  \begin{block}{Collective decision making:}
    \begin{itemize}
    \item 
      Wisdom and madness of crowds;
%%      Theories of social choice;
    \item 
      Systems of voting;
    \item 
      The role of randomness and chance;
%%    \item 
%%      Juries;
    \item 
      Success inequality: superstardom;
    \end{itemize}
  \end{block}

    %% \item Distribution and redistribution networks
    %%   \begin{itemize}
    %%   \item Single source supply networks
    %%   \item Distributed source supply networks
    %%   \end{itemize}

\end{frame}

%% start of spreading video ...
\insertvideo{tcRudblV-eM}{}{}{The Secret of Success will be revealed:}

\begin{frame}

  \begin{block}{Season's Narrative Arc (or Places We Will Go):}
    \begin{itemize}
    \item<+->
      Overview of Complexity with bonus Manifesto.
    \item<+->
      Size distributions of system elements:
      \begin{itemize}
      \item<+->
        Power-law size distributions.
      \item<+->
        Description and Mechanisms of Becoming.
      \end{itemize}
    \item<+-> 
      Robustness of Complex Systems.
    \item<+-> 
      Complex networks---how system elements are connected:
      \begin{itemize}
      \item<+-> 
        Structure, Growth Mechanisms, Processes on Networks.
      \item<+-> 
        Social Contagion, Voting, Fame and Fate, Stories.
      \end{itemize}
    \item<+->
      Allometric scaling in complex systems.
    \item<+->
      Happiness.
    \item<+->
      Complexification: The Theory of Anything.
    \end{itemize}
  \end{block}

\end{frame}

\begin{frame}[plain]
  \frametitle{Schedule in detail:}

  \tiny
  \begin{block}{}
    \rowcolors[]{1}{blue!20}{blue!10}
    \lectureschedule
  \end{block}

\end{frame}


\begin{frame}<1 | handout:0 | trans:1>

  \frametitle{Richard Feynmann on the  Social Sciences:}
  
  \includemedia[
  width=1\linewidth,height=0.5625\linewidth,
  activate=pageopen,
  flashvars={
    rel=0
    &modestbranding=1 % no YT logo in control bar
    &autoplay=0
    &autohide=1
    &showinfo=0
    % controlbar autohide
    % no title and other info before start
    % allow related videos after end because Feynmann is awesome
  }
  ]{}{http://www.youtube.com/v/IaO69CF5mbY}

\end{frame}

\begin{frame}<1 | handout:0 | trans:1>
  
\frametitle{Sheldon Cooper on the Social Sciences:}

%% ``I agree, the social sciences are largely hokum''

\includemedia[
width=1\linewidth,height=0.5625\linewidth,
activate=pageopen,
flashvars={
  rel=0
  &modestbranding=1 % no YT logo in control bar
  &autoplay=0
  &autohide=1
  &showinfo=0
  % controlbar autohide
  % no title and other info before start
  % no related videos after end
}
]{}{http://www.youtube.com/v/XNiSRx7Mpk0?rel=0}

\end{frame}



\subsection{Projects}
\begin{frame}
  \frametitle{Projects}

  \begin{block}{}
  \begin{itemize}
  \item<+->
    Semester-long projects, teams of 2 or 3.
  \item<+-> 
    Develop proposal in first few weeks.
  \item<+-> 
    May range from novel research to investigation of an established area of complex systems.
  \item<+-> 
    Two talks + written piece.
  \item<+->
    Usage of 
    \wordwikilink{http://www.uvm.edu/~vacc/}{the VACC}
    is encouraged (ability to code well = super powers).
  \item<+->
    Massive data sets available, including Twitter.
  \item<+->
    Academic output (journal papers) resulting from Principles
    of Complex Systems and Complex Networks can be found
    \wordwikilink{\coursewebsite/output/}{here}.  Add more!
  \item<+->
    We'll go through a list of possible projects soon.
  \end{itemize}
  \end{block}

\end{frame}

\begin{frame}
  \frametitle{Projects}

  \begin{block}{The \alertb{narrative hierarchy}---explaining things on all scales:}
    \begin{itemize}
    \item 
      1 to 3 word encapsulation = a soundbite = a buzzframe,
    \item 
      1 sentence, title,
    \item 
      \wordwikilink{http://en.wikipedia.org/wiki/Log\_line}{Log line},
    \item 
      few sentences,
    \item 
      a paragraph, abstract,
    \item 
      short paper, essay,
    \item 
      long paper,
    \item 
      chapter,
    \item 
      book,
    \item 
      \ldots
    \end{itemize}
  \end{block}

\end{frame}


\subsection{Centers,\ Books,\ Resources}

\begin{frame}
  \frametitle{Popular Science Books:}

  Historical artifact:
  \bigskip
  \amazonbook{waldrop1993a}
  \bigskip
  
  \begin{overprint}
    \onslide<1| handout: 0 | trans: 0>
    \onslide<2| handout: 1 | trans: 1>
    \begin{columns}
      \column{0.75\textwidth}
      Shout-out:
      \wordwikilink{http://www.parkvilleneurosurgery.com/who-we-are}{Dr.\ Andrew P.\ Morokoff}, \newline
      MBBS PhD FRACS 
      \wordwikilink{http://discworld.wikia.com/wiki/Doctor\_of\_Thaumatology}{D.Thau (Bug)}
      \column{0.25\textwidth}
      \includegraphics[width=\textwidth]{andrew-morokoff.jpg}
    \end{columns}
  \end{overprint}

\end{frame}

\begin{frame}
  \frametitle{Popular Science Books:}

  \amazonbook{johnson2009a}
  \medskip
  \amazonbook{mitchell2009a}
  \medskip
  \amazonbook{gleick2011a}

\end{frame}

\begin{frame}
  \frametitle{On complex sociotechnical systems:}

  \amazonbook{zipf1949a}
  \medskip
  \amazonbook{schelling1978a}
  \medskip
  \amazonbook{ball2004a}

\end{frame}

\begin{frame}
  \frametitle{A few textbooky books:}

  \amazonbook{miller2007a}
  \medskip
  \amazonbook{sornette2006a}
  \medskip
  \amazonbook{boccara2010a}

\end{frame}

%% \begin{frame}
%%   \frametitle{Complex networks}
%% \end{frame}

\begin{frame}
  \frametitle{Relevant online courses:}

  \begin{itemize}
  \item 
    Melanie Mitchell (Santa Fe Institute): \\
    \wordwikilink{http://www.santafe.edu/education/schools/sfi-mooc/}{Introduction to Complexity}
  \item   
    Lada Adamic (Michigan): \\
    \wordwikilink{https://www.coursera.org/course/sna}{Social Network Analysis}
  \end{itemize}

\end{frame}

\begin{frame}
  \frametitle{Centers:}

  \begin{block}{}
  \begin{itemize}
  \item 
    Santa Fe Institute (SFI)
  \item 
    New England Complex Systems Institute (NECSI)
  \item 
    Michigan's Center for the Study of Complex Systems 
    (\wordwikilink{http://www.cscs.umich.edu/}{CSCS}) 
  \item 
    Northwestern Institute on Complex Systems 
    (\wordwikilink{http://www.northwestern.edu/nico/}{NICO})
  \item 
    Also: Indiana, Davis, Brandeis, University of Illinois, Duke, Warsaw, Melbourne, ..., 
  \item
    \includegraphics[height=0.1\textheight]{roboctopus.png}
    \wordwikilink{http://www.uvm.edu/~cems/complexsystems/}{Vermont Complex Systems Center}
  \end{itemize}
  \end{block}

\end{frame}


\begin{frame}
  \frametitle{Other inputs:}

  \begin{block}{}
    \begin{itemize}
    \item
      Complexity Digest:
      
      {\small
        \wordwikilink{http://www.comdig.org}{http://www.comdig.org}\\
        \wordwikilink{https://twitter.com/@cxdig}{https://twitter.com/@cxdig}
      }
    \end{itemize}
    \begin{columns}
      \column{0.05\textwidth}
      \column{0.35\textwidth}
      \includegraphics[width=\textwidth]{nautilus-cover.png}
      \column{0.55\textwidth}
      \begin{itemize}
      \item 
        Nautilus Magazine: 
        \wordwikilink{http://nautil.us/}{http://nautil.us/}
      \end{itemize}
    \end{columns}
  \end{block}

\end{frame}



%%%%%%%%%%%%%%%%%%%%%%%%%%%%%%%%%%%%%
%% basic definitions
%%%%%%%%%%%%%%%%%%%%%%%%%%%%%%%%%%%%%

\changelecturelogo{.18}{icons-lightbulb-tp.pdf}

\section{Fundamentals}

\subsection{Complexity}

\begin{frame}
  \frametitle{Definitions}

  %% dictionary definition
  \alertb{Complex:} (Latin = with + fold/weave (com + plex))
  \hfill
  \includegraphics[width=.07\textwidth]{wikipedia-tp.pdf}

  \medskip

  \begin{block}{Adjective:}
    \begin{enumerate}
    \item Made up of multiple parts; intricate or detailed.
    \item Not simple or straightforward.
    \end{enumerate}
  \end{block}

\end{frame}


\begin{frame}
  \frametitle{Definitions}

  \begin{block}{Complicated versus Complex:}
    \begin{itemize}
    \item <1-> Complicated: Mechanical watches, airplanes, ...
    \item <2-> Engineered systems can be made to be \alertr{highly robust
        but not adaptable}.
    \item <3-> But engineered systems can become complex (power grid, planes).
    \item <4-> They can also \alertr{fail spectacularly}.
    \item <5-> Explicit distinction: \alertb{Complex Adaptive Systems}.
    \end{itemize}
  \end{block}

%% add this!!!!
%% http://www.nytimes.com/2013/01/17/business/faa-orders-grounding-of-us-operated-boeing-787s.html

\end{frame}

\begin{frame}
  \frametitle{Definitions}

  \begin{block}<+->{
      The Wikipedia on \alertb{Complex Systems:}
      \hfill \includegraphics[width=.07\textwidth]{wikipedia-tp.pdf} 
    }   
    ``Complexity science is not a single theory: 
    it encompasses more than one theoretical framework and is highly
    interdisciplinary, seeking the answers to some fundamental questions
    about living, adaptable, changeable systems.''
  \end{block}

  \begin{block}<+->{
      \alertb{Nino Boccara} in \textit{Modeling Complex Systems}:}\cite{boccara2004a}
    ``... there is no universally accepted definition
    of a complex system ... most researchers would describe
    a system of connected agents that exhibits
    an emergent global behavior not imposed by a central
    controller, but resulting from the interactions between
    the agents.''
  \end{block}


\end{frame}


\begin{frame}
  \frametitle{Definitions}

  \begin{block}<+->{\alertb{Philip Ball} in \textit{Critical Mass}:}\cite{ball2004a}
    ``...complexity theory seeks to understand how order and 
    stability arise from the interactions of many components
    according to a few simple rules.''
  \end{block}


\end{frame}

%% \begin{frame}
%%   \frametitle{Buzzword Definitions}
%% 
%%   \begin{block}{\alertb{Nonlinear} (OED)}
%%     1. a. Math. and Physics. Not linear; ...
%%     involving or possessing the property that the magnitude of an
%%     effect or output is not linearly or proportionally related to that
%%     of the cause or input.
%%     \medskip
%%     First cited use 1844.
%%   \end{block}
%% 
%% \end{frame}
%% 
%% \begin{frame}
%%   \frametitle{Buzzword Definitions}
%% 
%%   \begin{block}{\alertb{Nonlinear} (OED)}
%%     b. \textit{colloq.} \textbf{to go non-linear:} 
%%     to lose one's head; to rave, esp. about a particular obsession.
%%     \medskip
%%     First cited use 1985.
%%   \end{block}
%% 
%% \end{frame}


\begin{frame}
  \frametitle{Definitions}

  %%  \hfill
  %%  \includegraphics[width=.07\textwidth]{wikipedia-tp.pdf}

  \begin{block}{A meaningful definition of a \alertb{Complex System}:}
    \begin{itemize}
    \item<1->
      Distributed system of many interrelated (possibly networked) parts
      with no centralized control
      exhibiting 
      emergent behavior---`More is Different'\cite{anderson1972a}
    \end{itemize}    
  \end{block}

  \begin{block}<2->{A few optional features:}
    \begin{itemize}
    \item<2->
      Explicit nonlinear relationships
    \item<2->
      Presence of feedback loops
    \item<2->
      Being open or driven, opaque boundaries
    \item<2->
      Presence of memory
    \item<2->
      Modular (nested)/multiscale structure
    \end{itemize}
  \end{block}

\end{frame}


\begin{frame}
  \frametitle{Examples of Complex Systems:}

  \begin{block}{}
    \begin{columns}[t] 
      \begin{column}{0.45\textwidth} 
        \begin{itemize}
        \item human societies 
        \item financial systems
        \item cells     
        \item ant colonies 
        \item weather systems 
        \item ecosystems     
        \end{itemize}
      \end{column} 
      \begin{column}{0.45\textwidth} 
        \begin{itemize}
        \item animal societies     
        \item disease ecologies    
        \item brains               
        \item social insects       
        \item geophysical systems  
        \item the world wide web   
        \end{itemize}
      \end{column} 
    \end{columns} 
  \end{block}

  \begin{itemize}
  \item<+->
    i.e., everything that's interesting...
  \end{itemize}


\end{frame}

\begin{frame}
  \frametitle{Relevant fields:}

  \begin{block}{}
    \begin{columns}
      \column{0.3\textwidth}
      \begin{itemize}
      \item 
        Physics
      \item 
        Economics
      \item 
        Sociology
      \item 
        Psychology
      \item 
        Information Sciences
      \end{itemize}
      \column{0.05\textwidth}
      \column{0.3\textwidth}
      \begin{itemize}
      \item 
        Cognitive Sciences
      \item 
        Biology
      \item 
        Ecology
      \item 
        Geociences
      \item 
        Geography
      \end{itemize}
      \column{0.05\textwidth}
      \column{0.3\textwidth}
      \begin{itemize}
      \item 
        Medical Sciences
      \item 
        Systems Engineering
      \item 
        Computer Science
      \item 
      \ldots
      \end{itemize}
    \end{columns}
  \end{block}

  \begin{itemize}
  \item<+->
    i.e., everything that's interesting...
  \end{itemize}

\end{frame}

\begin{frame}
  \frametitle{Reductionism:}

  \begin{columns}
    \column{0.25\textwidth}
    \includegraphics[width=\textwidth]{200px-Democritus2.jpg}\\
    \column{0.75\textwidth}
    \begin{block}{
    \wordwikilink{http://en.wikipedia.org/wiki/Democritus}{Democritus}\\
    (ca. 460 BC -- ca. 370 BC)
  }
    \begin{itemize}
    \item 
      Atomic hypothesis
    \item 
      Atom $\sim$ a (not) -- temnein (to cut)
    \item 
      Plato allegedly wanted his books burned.
    \end{itemize}
    \end{block}
  \end{columns}

  \medskip

  \begin{columns}
    \column{0.25\textwidth}
    \includegraphics[width=\textwidth]{240px-Dalton_John_desk.jpg}\\
    \column{0.75\textwidth}
    \begin{block}{
        \wordwikilink{http://en.wikipedia.org/wiki/John\_Dalton}{John Dalton}\\
        1766--1844
      }
      \begin{itemize}
    \item 
      Chemist, Scientist
    \item 
      Developed atomic theory
    \item 
      First estimates of atomic weights
    \end{itemize}
    \end{block}
  \end{columns}

\end{frame}

\begin{frame}
  \frametitle{Reductionism:}

  \begin{columns}
    \column{0.3\textwidth}
    \includegraphics[width=\textwidth]{225px-Boltzmann2.jpg}\\
    \column{0.7\textwidth}
    \begin{block}{
        \wordwikilink{http://en.wikipedia.org/wiki/Ludwig\_Boltzmann}{Ludwig
          Boltzmann}, 1844--1906.\\
        Atomic Theory.
      }
      \begin{itemize}
      \item
        \small
        ``Boltzmann's kinetic theory of gases seemed to presuppose the
        reality of atoms and molecules, but almost all German philosophers and
        many scientists like Ernst Mach and the physical chemist Wilhelm
        Ostwald disbelieved their existence.''
      \end{itemize}
    \end{block}
  \end{columns}

  \begin{block}{}
    \begin{itemize}
    \item 
      \small
      ``In 1904 at a physics conference in St. Louis most
      physicists seemed to reject atoms and he was not even invited
      to the physics section. Rather, he was stuck in a section
      called "applied mathematics," he violently attacked
      philosophy, especially on allegedly Darwinian grounds but
      actually in terms of Lamarck's theory of the inheritance of
      acquired characteristics that people inherited bad philosophy
      from the past and that it was hard for scientists to overcome
      such inheritance.''
    \end{itemize}
  \end{block}

\end{frame}

\begin{frame}
  \frametitle{Reductionism:}

  \begin{columns}
    \column{0.25\textwidth}
    \includegraphics[width=\textwidth]{220px-Einstein_1921_portrait2.jpg}\\
    \column{0.75\textwidth}
    \begin{block}{
    \wordwikilink{http://en.wikipedia.org/wiki/Albert\_Einstein}{Albert Einstein}
    1879--1955}
    \begin{itemize}
    \item 
      \wordwikilink{http://en.wikipedia.org/wiki/Annus_Mirabilis_papers}{Annus Mirabilis paper:} ``the Motion of Small Particles Suspended in a Stationary Liquid, as Required by the Molecular Kinetic Theory of Heat''\cite{einstein1905a,einstein1956a}
    \item 
      Showed \wordwikilink{http://en.wikipedia.org/wiki/Brownian_motion}{Brownian motion} 
      followed from an atomic model giving rise to diffusion.
    \end{itemize}
  \end{block}
  \end{columns}

  \medskip

  \begin{columns}
    \column{0.25\textwidth}
    \includegraphics[width=\textwidth]{180px-Jean_Baptiste_Perrin.jpg}\\
    \column{0.75\textwidth}
    \begin{block}{
        \wordwikilink{http://en.wikipedia.org/wiki/Jean\_Perrin}{Jean Perrin}
        1870--1942}
      \begin{itemize}
      \item 
        1908: Experimentally verified Einstein's work and Atomic Theory.
      \end{itemize}
    \end{block}
  \end{columns}

\end{frame}

\begin{frame}
  \small

  \begin{block}{Feynmann:}
    If, in some cataclysm, all of scientific knowledge were to be
    destroyed, and only one sentence passed on to the next generation of
    creatures, what statement would contain the most information in the
    fewest words? I believe it is the atomic hypothesis that all things
    are made of atoms---little particles that move around in perpetual
    motion, attracting each other when they are a little distance apart,
    but repelling upon being squeezed into one another. In that one
    sentence, you will see, there is an enormous amount of information
    about the world, if just a little imagination and thinking are
    applied.
  \end{block}

  \tiny
  Snared from 
  \hwlink{http://www.brainpickings.org/index.php/2012/09/11/richard-feynman-lectures-on-physics/}{brainpickings.org}

\end{frame}


\begin{frame}
%%  \frametitle{}
  \small

  \begin{block}{Complexity Manifesto:}
    \begin{enumerate}
    \item<+->
      Systems are ubiquitous and systems matter.
    \item<+->
      Consequently, much of science is about understanding
      how pieces dynamically fit together.
    \item<+->
      1700 to 2000 = Golden Age of Reductionism.
      \begin{itemize}
      \item 
        Atoms!, sub-atomic particles, DNA, genes, people, ...
      \end{itemize}
    \item<+->
      Understanding and creating systems (including new `atoms')
      is the greater part of science and engineering.
    \item<+->
      Universality: systems with quantitatively different micro details
      exhibit qualitatively similar macro behavior.
    \item<+->
      Computing advances make the Science of Complexity possible:
      \begin{enumerate}
      \item<+->
        We can measure and record enormous amounts of data,
        research areas continue to transition from data scarce to data rich.
      \item<+->
        We can simulate, model, and create complex systems
        in extraordinary detail.  
      \end{enumerate}
    \end{enumerate}
  \end{block}

\end{frame}




\begin{frame}

  \wordwikilink{http://www.economist.com/node/15557443}{Data, Data, Everywhere---the Economist, Feb 25, 2010}

  \begin{columns}
    \column{0.45\textwidth}
    \includegraphics[width=\textwidth]{201009SRC696-economist.png}
    \begin{itemize}
    \item 
      Exponential growth: $\sim$ 60\% per year.
    \end{itemize}
    \column{0.55\textwidth}
    \begin{block}{Big Data Science:}
    \begin{itemize}
    \item 
      2013: year traffic on Internet estimate to reach 2/3 Zettabytes \\
      (1ZB = $10^3$EB = $10^6$PB = $10^9$TB)
    \item 
      Large Hadron Collider: 40 TB/second.\\
    \item 
      2016---Large Synoptic Survey Telescope:\\
      140 TB every 5 days.
    \item 
      Facebook: $\sim$ 250 billion photos (mid 2013)
    \item 
      Twitter: $\sim$ 500 billion tweets (mid 2013)
    \end{itemize}
    \end{block}      
  \end{columns}

\end{frame}

\begin{frame}[plain]
  \frametitle{No really, that's a lot of data}

  \includegraphics[width=1.2\textwidth]{201009SRC722-economist.png}
\end{frame}


\begin{frame}
  \frametitle{Big Data---Culturomics:}

  \small{``Quantitative analysis of culture using millions of
    digitized books'' by Michel et al., Science, 2011\cite{michel2011a}}

  \includegraphics[width=0.45\textwidth]{michel2011a_fig3a.pdf} 
  \includegraphics[width=0.45\textwidth]{michel2011a_fig3e.pdf} \\
  \includegraphics[width=0.45\textwidth]{michel2011a_fig3f.pdf}
  \includegraphics[width=0.35\textwidth]{michel2011a_fig4f.pdf}

  {\small
    \wordwikilink{http://www.culturomics.org/}{http://www.culturomics.org/}\\
    \wordwikilink{http://ngrams.googlelabs.com/}{Google Books ngram viewer}
  }

\end{frame}


\begin{frame}
  \frametitle{Basic Science $\simeq$ Describe + Explain:}

  \begin{columns}
    \column{0.4\textwidth}
    \includegraphics[width=\textwidth]{lordkelvin-aip.jpg}
    \column{0.6\textwidth}
    \begin{block}{Lord Kelvin (possibly):}
      \begin{itemize}
      \item<+->
        \alertg{``To measure is to know.''}
      \item<+-> 
        \alertg{``If you cannot measure it, you cannot improve it.''}
      \end{itemize}
    \end{block}
    \begin{block}<+->{Bonus:}
      \begin{itemize}
      \item<+->
        \alertg{``X-rays will prove to be a hoax.''}
      \item<+->
        \alertg{``There is nothing new to be discovered in physics now, 
          All that remains is more and more precise measurement.''}
      \end{itemize}
    \end{block}

  \end{columns}

\end{frame}


\begin{frame}
  \frametitle{The Newness of being a Scientist:}

  \includegraphics[width=\textwidth]{2013-01-14ngrams-scientist.jpg}

\end{frame}


%% \begin{frame}
%%   \frametitle{Outreach}
%% 
%%   \begin{center}
%%     \includegraphics[height=0.8\textheight]{complexity-society-frontpage.pdf}
%%   \end{center}
%% 
%% \end{frame}
%% 
%% \begin{frame}
%%   \frametitle{Outreach}
%% 
%%   ``The society objectives are to promote the \alertb{theory of complexity} in
%%   education, government, the health service and business as well as the
%%   beneficial application of complexity in a wide variety of social,
%%   economic, scientific and technological contexts such as sources of
%%   competitive advantage, business clusters and knowledge management.''
%% 
%%   \medskip
%% 
%%   \visible<2->{
%%     ``Complexity includes ideas such as complex adaptive systems,
%%     self-organisation, co-evolution, agent based computer models, chaos,
%%     networks, emergence, and fractals.''
%%   }
%% 
%% \end{frame}

\subsection{Emergence}

\begin{frame}
  \frametitle{Emergence:}

  \begin{block}{The Wikipedia on \alertb{Emergence}:
      \hfill
      \includegraphics[width=.07\textwidth]{wikipedia-tp.pdf}
    }
    ``In philosophy, systems theory and the sciences, emergence refers to
    the way complex systems and patterns arise out of a multiplicity of
    relatively simple interactions. 
    \visible<2->{... \alertb{emergence is central
        to the physics of complex systems and yet very controversial}.''}
  \end{block}

\begin{block}<3->{}
    The philosopher G. H. Lewes first
    used the word explicity in 1875.
\end{block}


\end{frame}

\begin{frame}<1 | handout:0 | trans:1>
  \frametitle{Fireflies $\Rightarrow$ Synchronized Flashes:}

  \includemedia[
  label=vid.fireflies,
  width=1\linewidth,height=0.75\linewidth,
  activate=pageopen,
  flashvars={
    modestbranding=1 % no YT logo in control bar
    &autoplay=0
    &autohide=1 % controlbar autohide
    &showinfo=0 % no title and other info before start
    %% &rel=0 % no related videos after end, must be included in URL
    %% as ?rel
  }
  ]
  {\includegraphics{videos/2010/fireflies-strogatz-frame.jpg}}
  {http://www.youtube.com/v/s972vNE5gLA?rel=0}

  \small

  Film: Sir David Attenborough, BBC.

  Voiceover: Steve Strogatz on 
  \wordwikilink{http://www.radiolab.org/2007/aug/14/}{Radiolab's Emergence, S1E3}.

\end{frame}

\begin{frame}
%%  \frametitle{Emergence:}

  \begin{block}{Emergence:}
    Tornadoes, financial collapses, human emotion aren't
    found in water molecules, dollar bills, or carbon atoms.
    %%      There's no tornado in a water molecule, no financial
    %%      collapse in a single dollar bill,
    %%      no human emotion in a carbon atom;
    %%      these are `emergent' properties.
  \end{block}

  \begin{block}<2->{Examples:}
    \begin{itemize}
    \item<2-> 
      Fundamental particles $\Rightarrow$ Life, the Universe, and Everything
    \item<2-> 
      Genes $\Rightarrow$ Organisms
    \item<2->
      Neurons etc. $\Rightarrow$ Brain $\Rightarrow$ Thoughts
    \item<2-> 
      People $\Rightarrow$ Religion, Collective behaviour
    \item<2->
      People $\Rightarrow$ The Web
    \item<2->
      People $\Rightarrow$ Language, and rules of language
    \item<2->
      ? $\Rightarrow$ time; 
      ? $\Rightarrow$ gravity;
      ? $\Rightarrow$ reality.
    \end{itemize}
  \end{block}

  \begin{block}<3->{}
    \alertb{``The whole is more than the sum of its parts''}
    --Aristotle
  \end{block}

\end{frame}


\begin{frame}
  \frametitle{Emergence:}

  \begin{block}{\wordwikilink{http://en.wikipedia.org/wiki/Thomas\_Schelling}{Thomas Schelling}
      \smallskip
      (Economist/Nobelist):}
    \medskip
    \begin{columns}
      \column{0.3\textwidth}
      \includegraphics[width=\textwidth]{eggs1.jpg}\\
      \includegraphics[width=\textwidth]{eggs2.jpg}\\
      \includegraphics[width=\textwidth]{eggs3.jpg}\\
      \wordwikilink{http://www.youtube.com/watch?v=JjfihtGefxk}{[youtube]}
      \column{0.7\textwidth}
      \begin{center}
      \includegraphics[width=0.7\textwidth]{thomas_schelling_nobelprice_eco_2005.png}
      \begin{itemize}
      \item 
        ``Micromotives and Macrobehavior''\cite{schelling1978a}
        \begin{itemize}
        \item
          Segregation\cite{schelling1971a,schelling2006a}
        \item
          Wearing hockey helmets\cite{schelling1973a}
        \item
          Seating choices
        \end{itemize}
      \end{itemize}
      \end{center}
    \end{columns}
  \end{block}

\end{frame}



\begin{frame}
  \frametitle{Emergence:}

  \begin{block}{
      \wordwikilink{http://en.wikipedia.org/wiki/Friedrich_Hayek}{Friedrich Hayek} 
      \smallskip
      (Economist/Philospher/Nobelist):
    }
    \begin{itemize}
    \item<1-> Markets, legal systems, political systems are emergent and not designed.
    \item<2->   
      `Taxis' = made order (by God, Sovereign, Government, ...)
    \item<3->
      `Cosmos' = grown order
    \item<4->
      Archetypal limits of \alertb{hierarchical} and \alertr{decentralized} structures.
    \item<5->
      \alertb{Hierarchies} arise once problems are solved.\cite{dodds2003c}
    \item<6->
      \alertr{Decentralized structures} help solve problems.
    \item<7->
      Dewey Decimal System versus tagging.
    \end{itemize}
  \end{block}


\end{frame}

\begin{frame}
  \frametitle{Emergence:}

  \begin{block}{
      \wordwikilink{http://en.wikipedia.org/wiki/James\_Samuel\_Coleman}{James Coleman} 
      \smallskip
      in \textit{Foundations of Social Theory}:}
    \includegraphics[width=0.9\textwidth]{coleman.pdf}
  \end{block}

  \bigskip

  \begin{block}{}
  \begin{itemize}
  \item 
    Understand macrophenomena arises from microbehavior
    which in turn depends on macrophenomena.\cite{coleman1994a}
  \item<2->
    More on Coleman 
    \wordwikilink{http://en.wikipedia.org/wiki/James_Samuel_Coleman}{here}.
  \end{itemize}
  \end{block}

\end{frame}

\begin{frame}
  \frametitle{Emergence:}

  \begin{block}<+->{Higher complexity:}
    \begin{itemize}
    \item 
      Many system scales (or levels) \\ 
      that
      interact with each other.
    \item<+->
      Potentially much harder to explain/understand.
    \end{itemize}
  \end{block}

  \begin{block}<+->{Even mathematics:\cite{foote2007a}}
    \begin{columns}
      \column{0.02\textwidth}
      \column{0.2\textwidth}
      \includegraphics[width=0.97\textwidth]{kurt_godel.jpg}
      \column{0.6\textwidth}
      \wordwikilink{http://bit.ly/VdbsWU}{G\"{o}del's Theorem}:\\
      we can't prove every theorem that's true \ldots
%%      \column{0.2\textwidth}
%%      \includegraphics[width=0.97\textwidth]{hofst.jpg}
    \end{columns}
  \end{block}

  \begin{block}<+->{}
    \begin{itemize}
    \item 
      Suggests a \alertb{strong form of emergence}:
      \uncover<+->{
        Some phenomena cannot be analytically deduced
        from elementary aspects of a system.
      }
    \end{itemize}
  \end{block}

\end{frame}


\begin{frame}
  \frametitle{Emergence:}

  \begin{block}{}
    Roughly speaking, there are \alertg{two types} of \alertb{emergence}:  
  \end{block}

  \begin{block}<2->{\alertb{I. Weak emergence:}}
    System-level phenomena is
    different from that of its constituent parts
    yet can be connected theoretically.
  \end{block}

  \begin{block}<3->{\alertb{II. Strong emergence:}}
    System-level 
    phenomena fundamentally cannot
    be deduced from how parts interact.
  \end{block}

%%   \smallskip
%%   \visible<4->{
%%     Strong emergence could be called \alertr{magic}...\\
%%     See Bedau (1997)\cite{bedau1997a}
%%     }

\end{frame}

\begin{frame}
  \frametitle{Emergence:}

  \begin{block}{}
    \begin{itemize}
      %% \item<+->
      %%   Complex Systems enthusiasts often decry \alertb{reductionist} approaches \ldots
      %% \item<+->
      %%   But reductionism seems to be misunderstood.
    \item<+->
      \alertb{Reductionist} techniques can explain weak emergence.
    \item<+->
      \alertb{Magic} explains strong emergence.\cite{bedau1997a}
    \item<+->
      But: maybe \alertb{magic} should be interpreted
      as an \alertb{inscrutable yet real mechanism} that cannot
      ever be \alertr{simply described}.    
    \item<+->
      Gulp.
    \end{itemize}
  \end{block}

\end{frame}

\begin{frame}

  \includemedia[
  label=audio.radiolab-limits,
  activate=pageopen,
  addresource=sound/2010/radiolab041610c.mp3,
  flashvars={
    source=sound/2010/radiolab041610c.mp3
    &autoPlay=0
    &showinfo=1
  },
  transparent
  ]{\color{grey}\framebox[\linewidth][c]{Limits of Science | Radiolab}}{APlayer.swf}
  
  \begin{block}{}
    \begin{columns}
      \column{0.02\textwidth}
      \column{0.18\textwidth}
      \includegraphics[width=\textwidth]{radiolab-limits-150430.jpg}
      \column{0.78\textwidth}
      Listen to Steve Strogatz, Hod Lipson, and Michael Schmidt (Cornell) in the 
      \wordwikilink{http://www.radiolab.org/2010/apr/05/limits-of-science/}{last piece}
      (11:16) on Radiolab's show 
      \wordwikilink{http://www.radiolab.org/2010/apr/05/}{`Limits'} (April 5, 2010).
      %% (51:40)
      \column{0.02\textwidth}
    %% \url{http://blogs.wnyc.org/radiolab/2010/04/05/limits/}
    \end{columns}
    \bigskip
    \includegraphics[width=\textwidth]{radiolab-limits-science-tp-1.pdf}

    Bonus: 
    \wordwikilink{http://www.youtube.com/watch?v=6XSncCrQzwk}{Mike Schmidt's talk on Eureqa}
    at \newline
    \wordwikilink{http://www.uvm.edu/~tedxuvm/?Page=archive/2011/default.php}{UVM's 2011 TEDx event ``Big Data, Big Stories.''}

\end{block}

\end{frame}

\begin{frame}
  \frametitle{The emergence of taste:}

  \begin{block}{}
    \begin{itemize}
    \item 
      Molecules $\Rightarrow$ Ingredients $\Rightarrow$ Taste 
    \item 
      See Michael Pollan's
      \wordwikilink{http://www.nytimes.com/2007/01/28/magazine/28nutritionism.t.html}{article on nutritionism} in the New York Times, January 28, 2007.

      \medskip

      \includegraphics[width=0.7\textwidth]{2007-01-28unhappymeals.jpg}\\
      {\tiny \url{nytimes.com}}
    \end{itemize}
  \end{block}

\end{frame}


\begin{frame}
  \frametitle{Reductionism}

  \begin{block}{Reductionism and food:}
    \begin{itemize}
%    \item 
%      {\small \alertb{\textit{Unhappy Meals}}, Michael Pollan, NY Times, January 2007}
    \item<2-> 
      \alertb{Pollan:}
      ``even the simplest food is a
      hopelessly complex thing to study, a virtual wilderness of chemical
      compounds, many of which exist in complex and dynamic relation to one
      another...''
    \item<3->
      ``So ... break the
      thing down into its component parts and study those one by one, even
      if that means ignoring complex interactions and contexts, as well as
      the fact that the whole may be more than, or just different from, the
      sum of its parts. This is what we mean by reductionist science.''
    \end{itemize}
  \end{block}

\end{frame}

\begin{frame}
  \frametitle{Reductionism}

  \begin{block}{}
  \begin{itemize}
  \item<1-> ``people don't eat nutrients, they eat foods, and foods can behave
    very differently than the nutrients they contain.''
  \item<2-> Studies suggest diets high in fruits and vegetables help prevent cancer.
  \item<3-> So...  find the nutrients responsible and eat more of them
  \item<4-> But ``in the case of \alertb{beta
    carotene ingested as a supplement}, scientists have discovered that it
    actually \alertr{increases the risk of certain cancers}. Oops.''
  \end{itemize}
  \end{block}

\end{frame}


\begin{frame}
  \frametitle{Reductionism}

  \begin{columns}
    \column{0.7\textwidth}
    \begin{block}{\alertg{Thyme's known antioxidants:}}
      4-Terpineol, alanine, anethole, apigenin, ascorbic acid, beta
      carotene, caffeic acid, camphene, carvacrol, chlorogenic acid,
      chrysoeriol, eriodictyol, eugenol, ferulic acid, gallic acid,
      gamma-terpinene isochlorogenic acid, isoeugenol, isothymonin,
      kaempferol, labiatic acid, lauric acid, linalyl acetate, luteolin,
      methionine, myrcene, myristic acid, naringenin, oleanolic acid,
      p-coumoric acid, p-hydroxy-benzoic acid, palmitic acid, rosmarinic
      acid, selenium, tannin, thymol, tryptophan, ursolic acid, vanillic
      acid.
    \end{block}
    \column{0.3\textwidth}
    \includegraphics[width=\textwidth]{thyme-lrg.jpg}\\
    {\mbox{} \hfill \tiny [cnn.com]}
  \end{columns}

\end{frame}

\begin{frame}
  \frametitle{Reductionism}

  \begin{block}{}
  ``It would be great to know how this all works, but \alertb{in the meantime} we
  can enjoy thyme in the knowledge that it probably doesn't do any harm
  (since people have been eating it forever) and that it may actually do
  some good (since people have been eating it forever) and that even if
  it does nothing, we like the way it tastes.''

  \bigskip

  \visible<2->{
    \alertb{Gulf between theory and practice (see baseball and bumblebees).}
  }
  \end{block}

\end{frame}

\subsection{Self-Organization}

\begin{frame}
  \frametitle{Definitions}

  \begin{block}{Self-Organization
      \hfill
      \includegraphics[width=.07\textwidth]{wikipedia-tp.pdf}
    }
    ``\wordwikilink{http://en.wikipedia.org/wiki/Self-organization}{Self-organization} is a process in which the internal organization 
    of a system, normally an open system, increases in complexity without 
    being guided or managed by an outside source.''
    (also: Self-assembly)
  \end{block}

  \medskip

  \begin{block}<2->{}
    \begin{itemize}
    \item<2-> 
      Self-organization refers to a broad array of decentralized processes 
      that lead to emergent phenomena.
    \end{itemize}
  \end{block}

\end{frame}


 \begin{frame}
   \frametitle{Examples of self-organization:}

   \begin{block}{}
     \begin{itemize}
     \item
       Molecules/Atoms liking each other $\rightarrow$ Gas-liquid-solids
     \item
       Spin alignment $\rightarrow$ Magnetization
     \item 
       Imitation $\rightarrow$ Herding, flocking, stock market
     \end{itemize}

     \medskip

     \visible<2->{
       Fundamental question: how likely is `complexification'?
     }
   \end{block}

 \end{frame}

%% \begin{frame}
%%   \frametitle{Buzzword Definitions}
%% 
%%   \alertb{Emergence but no Self-Organization?}
%% 
%%   \bigskip
%% 
%%   \visible<2->{
%%     H$_2$0 molecules $\Rightarrow$ Water
%%   }
%% 
%%   \bigskip
%%   
%%   \visible<3->{
%%     Random walks $\Rightarrow$ Normal distributions
%%   }
%%   
%% \end{frame}
%% 
%% 
%% \begin{frame}
%%   \frametitle{Buzzword Definitions}
%% 
%%   \alertb{Self-organization but no Emergence?}
%% 
%%   \medskip
%% 
%%   \visible<2->{
%%     Water above and near the freezing point.
%%   }
%% 
%%   \medskip
%% 
%%   \visible<3->{
%%     Emergence may be limited to a low scale of a system.
%%   }
%% 
%% \end{frame}

%% \begin{frame}
%%  \frametitle{Economics}
%%Eric Beinhocker (\textit{The Origin of Wealth}):
%%
%%{\small
%%
%%\begin{tabular}{lll}
%% & Complexity Economics  &  Traditional Economics \\
%%Dynamic & 
%%Open, dynamic, non-linear systems, far from equilibrium & 
%%Closed, static, linear systems in equilibrium \\
%%Agents  &
%%Modelled individually; use inductive rules of thumb to make decisions; have incomplete information; are subject to errors and biases; learn to adapt over time  &
%%Modelled collectively; use complex deductive calculations to make decisions; have complete information; make no errors and have no biases; have no need for learning or adaptation (are already perfect) \\
%%Networks  &  
%%Explicitly model bi-lateral interactions between individual agents; networks of relationships change over time &
%%Assume agents only interact indirectly through market mechanisms (e.g. auctions) \\
%%Emergence   &    
%%No distinction between micro/macro economics; 
%%macro patterns are emergent result of micro level behaviours and interactions.  &  
%%Micro-and macroeconomics remain separate disciplines \\
%%Evolution   &    
%%The evolutionary process of differentiation, selection and amplification provides the system with novelty and is responsible for its growth in order and complexity  &
%%No mechanism for endogenously creating novelty, or growth in order and complexity \\
%%\end{tabular}
%%
%%}
%% 
%% \end{frame}

%% \begin{frame}
%%   \frametitle{Economics}
%% 
%%   Eric Beinhocker (\textit{The Origin of Wealth}):\cite{beinhocker2006a}
%% 
%%   \begin{block}{\alertb{Dynamic:}}
%%     \begin{itemize}
%%     \item 
%%       \tc{red}{Complexity Economics:}
%%       Open, dynamic, non-linear systems, far from equilibrium
%%     \item 
%%       \tc{red}{Traditional Economics:}
%%       Closed, static, linear systems in equilibrium
%%     \end{itemize}
%%   \end{block}
%% 
%% \end{frame}
%% 
%% \begin{frame}
%%   \frametitle{Economics}
%% 
%%   \begin{block}{\alertb{Agents:}}
%%     \begin{itemize}
%%     \item 
%%       \tc{red}{Complexity Economics:}
%%       
%%       Modelled individually; use inductive rules of thumb to make
%%       decisions; have incomplete information; are subject to errors
%%       and biases; learn to adapt over time
%% 
%%     \item \tc{red}{Traditional Economics:} 
%%       Modelled collectively; use
%%       complex deductive calculations to make decisions; have complete
%%       information; make no errors and have no biases; have no need for
%%       learning or adaptation (are already perfect)
%%     \end{itemize}
%%     
%%   \end{block}
%% 
%% \end{frame}
%% 
%% \begin{frame}
%%   \frametitle{Economics}
%% 
%%   \begin{block}{\alertb{Networks:}}
%%     \begin{itemize}
%%     \item 
%%       \tc{red}{Complexity Economics:}
%%       Explicitly model bi-lateral interactions between individual agents; networks of relationships change over time
%%     \item 
%%       \tc{red}{Traditional Economics:}
%%       Assume agents only interact indirectly through market mechanisms (e.g. auctions)
%%     \end{itemize}
%%   \end{block}
%%   
%% 
%% \end{frame}
%% 
%% \begin{frame}
%%   \frametitle{Economics}
%% 
%%   \begin{block}{\alertb{Emergence:}}
%%     \begin{itemize}
%%     \item
%%       \tc{red}{Complexity Economics:}
%%       No distinction between micro/macro economics; 
%%       macro patterns are emergent result of micro level behaviours and interactions
%%     \item
%%       \tc{red}{Traditional Economics:}
%%       Micro-and macroeconomics remain separate disciplines
%%     \end{itemize}
%%   \end{block}
%% 
%% \end{frame}
%% 
%% \begin{frame}
%%   \frametitle{Economics}
%% 
%%   \begin{block}{\alertb{Evolution:}}
%%     \begin{itemize}
%%     \item 
%%       \tc{red}{Complexity Economics:}
%%       
%%       The evolutionary process of differentiation, selection and
%%       amplification provides the system with novelty and is
%%       responsible for its growth in order and complexity
%% 
%%     \item 
%% 
%%       \tc{red}{Traditional Economics:}
%%   
%%       No mechanism for endogenously creating novelty, or growth in
%%       order and complexity
%% 
%%     \end{itemize}
%%     
%%   \end{block}
%%   
%% \end{frame}

\subsection{Our\ Framing}

\begin{frame}
  \frametitle{Upshot}

  \begin{block}{}
  \begin{itemize}
  \item<+->
    The central concepts \alertg{Complexity} and
    \alertg{Emergence} are \alertb{not precisely defined}.
  \item<+->
    There is \alertb{no general theory of Complex Systems}.
  \item<+->
    But the problems exist...\\
    \qquad \qquad Complex (Adaptive) Systems abound...
  \item<+->
    Framing: Science's focus is moving to Complex Systems 
    \alertg{because it finally can}.
  \item<+->
    \alertb{We use whatever tools we need.}
  \item<+->
    Reality is theoretically weak.
  \item<+->
    Science $\simeq$ Describe + Explain.
  \end{itemize}
  \end{block}

\end{frame}

\begin{frame}
  \frametitle{Rather silly but great example of real science:}

  \wordwikilink{http://www.sciencemag.org/content/early/2010/11/10/science.1195421}{``How Cats Lap: {W}ater Uptake by \textit{{F}elis catus}''}\\
  Reis et al., \textit{Science}, 2010.

  \medskip

  \includegraphics[width=\textwidth]{12cats_graphic-popup-v2.jpg}

  Amusing interview \wordwikilink{http://video.nytimes.com/video/2010/11/11/science/1248069317702/how-cats-lap.html}{here}

\end{frame}


\subsection{Modeling}

\begin{frame}
  \frametitle{Models}

  \begin{block}{\alertb{Nino Boccara} in \textit{Modeling Complex Systems}:}
    ``Finding the emergent global behavior of a large
    system of interacting agents using methods is usually
    hopeless, and researchers therefore must rely on
    computer-based models.''
  \end{block}

  \begin{block}{Focus is on dynamical systems models:}
    \begin{itemize}
    \item<+-> 
      differential and difference equation models
    \item<+-> 
      dynamical systems theory
    \item<+-> 
      cellular automata
    \item<+-> 
      networks
    \item<+-> 
      power-law distributions
    \end{itemize}
  \end{block}

\end{frame}

\begin{frame}
  \frametitle{}

  \begin{block}{Tools and techniques:}
    \begin{itemize}
    \item<2-> 
      Differential equations, difference equations, linear algebra.
    \item<3-> 
      Statistical techniques for comparisons and descriptions.
    \item<4-> 
      Methods from statistical mechanics and computer science.
    \item<5-> 
      Computer modeling, everything from
      \begin{itemize}
      \item 
        Toy models (e.g., using \wordwikilink{http://ccl.northwestern.edu/netlogo/}{Netlogo})
      \item 
        to kitchen sink models.
      \end{itemize}
    \end{itemize}
  \end{block}

  \begin{block}<6->{Key advance (to repeat):}
    \begin{itemize}
    \item<6-> Representation of \alertb{complex interaction patterns} as \alertb{dynamic networks}.
    \item<7-> The driver: \alertr{Massive amounts of Data}
    \item<8-> More later...
    \end{itemize}
  \end{block}

\end{frame}

\begin{frame}
  \frametitle{Models}

  \begin{block}{\alertb{Philip Ball} in \textit{Critical Mass}:}\cite{ball2004a}
    ``... very often what passes today for `complexity science'
    is really something much older, dressed up in fashionable apparel.
    The main themes in complexity theory have been studied
    for more than a hundred years by physicists who evolved
    a tool kit of concepts and techniques to which complexity
    studies have barely added a handful of new items.''
  \end{block}

  \begin{block}<2->{Old School:}
  \begin{itemize}
  \item<2-> 
    Statistical Mechanics is \alertr{``a science of collective behavior.''}
  \item<3->
    \alertb{Simple rules} give rise to \alertb{collective phenomena.}
  \end{itemize}
  \end{block}

\end{frame}

\subsection{Statistical\ Mechanics}


\begin{frame}
%%  \frametitle{The stuff of statistical mechanics:}

  \begin{block}{\wordwikilink{http://en.wikipedia.org/wiki/Ising_model}{The Ising Model} of a ferromagnet:}
    \begin{columns}
      \column{0.02\textwidth}
      \column{0.27\textwidth}
      \includegraphics[width=\textwidth]{2013-01-21ising-model-sketch-crop-tp-1.pdf}\\
      \includegraphics[width=\textwidth]{2013-01-22ising-model-sketch-crop-tp-1.pdf}
      \column{0.69\textwidth}
      \begin{itemize}
      \item<+-> 
        Each atom is assumed to have a local spin 
        that can be \tc{red}{up} or \tc{red}{down}: $ S_i = \pm 1$.
      \item<+-> 
        Spins are assumed to be arranged on a lattice.
      \item<+-> 
        In isolation, spins like to align with each other.
      \item<+-> 
        Increasing temperature breaks these alignments.
      \item<+-> 
        The \wordwikilink{http://en.wikipedia.org/wiki/Drosophila}{drosophila} of statistical mechanics.
      \item<+-> 
        Criticality: Power-law distributions at critical points.
      \end{itemize}
      \column{0.02\textwidth}
    \end{columns}
  \end{block}
  \begin{block}<+->{Example 2-d Ising model simulation:}
    \wordwikilink{http://dtjohnson.net/projects/ising}{http://dtjohnson.net/projects/ising}
  \end{block}

\end{frame}


\begin{frame}
  \frametitle{Phase diagrams}

  \includegraphics[width=0.9\textwidth]{Phase-diag.jpg}

  \medskip

  Qualitatively distinct macro states.

\end{frame}

\begin{frame}
  \frametitle{Phase diagrams}

  Oscillons, bacteria, traffic, snowflakes, ...

  \medskip
  
  \includegraphics[width=0.45\textwidth]{osc1.pdf}
  \includegraphics[width=0.45\textwidth]{osc2.pdf}

  \medskip

  Umbanhowar et al., \textit{Nature}, 1996\cite{umbanhowar1996a}

\end{frame}

\begin{frame}
  \frametitle{Phase diagrams}

  \begin{center}
    \includegraphics[height=0.8\textheight]{oscillon_phasediagram.pdf}
  \end{center}

\end{frame}

\begin{frame}
  \frametitle{Phase diagrams}

  \begin{center}
    \includegraphics[height=0.7\textheight]{bacteria_phase_diagram.jpg}
  \end{center}

  {\tiny $W_0$ = initial wetness, $S_0$ = initial nutrient supply\\
    \url{http://math.arizona.edu/~lega/HydroBact.html}}

\end{frame}

\begin{frame}
  \frametitle{Ising model}

  \begin{block}{Analytic issues:}
    \begin{itemize}
    \item<1-> 1-d: simple (Ising \& Lenz, 1925)
    \item<2-> 2-d: hard (Onsager, 1944)
    \item<3-> 3-d: extremely hard...
    \item<4-> 4-d and up: simple.
    \end{itemize}
  \end{block}

\end{frame}

\begin{frame}
  \frametitle{Statistics}

  \begin{block}{Historical surprise:}
  \begin{itemize}
  \item<1-> Origins of Statistical Mechanics are in the studies of people...
    (Maxwell and co.)
  \item<2-> Now physicists are using their techniques to study everything else
    including people...
  \item<3-> See Philip Ball's ``Critical Mass''\cite{ball2004a}
  \end{itemize}
  \end{block}

\end{frame}


