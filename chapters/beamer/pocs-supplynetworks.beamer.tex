%%\section{Introduction}

\begin{frame}
  \frametitle{Optimal supply networks}

  \begin{block}<1->{What's the best way to distribute stuff?}
    \begin{itemize}
    \item<2-> Stuff = medical services, energy, people, 
    \item<3-> \alert{Some} fundamental network problems:
      \begin{enumerate}
      \item<4-> Distribute stuff from a \alert{single source} to
        \alert{many sinks}
      \item<5-> Distribute stuff from \alert{many sources} to
        many sinks
      \item<6-> \alert{Redistribute} stuff between nodes that
        are both sources and sinks
      \end{enumerate}
    \item<7-> Supply and Collection are equivalent problems
    \end{itemize}
  \end{block}
  
\end{frame}


\begin{frame}
  \frametitle{Single source optimal supply}

  \begin{block}{Basic question for distribution/supply networks:}
    \begin{itemize}
    \item<1->
      How does flow behave given cost:
      $$
      C 
      = 
      \sum_{j} I_j^{\, \gamma} Z_j
      $$
      where \\
      \alertb{$
      I_j 
      $
      = current on link $j$}\\
      and\\
      \alertb{$Z_j$ = link $j$'s impedance}?
    \item<2->
      Example: $\gamma=2$ for electrical networks.
    \end{itemize}
  \end{block}
  
\end{frame}


\begin{frame}
  \frametitle{Single source optimal supply}

  \includegraphics[width=\textwidth]{bohn2007a_fig2}
  
  \begin{itemize}
  \item[(a)]
    $\gamma > 1$: \alert{Braided} (bulk) flow 
  \item[(b)]
    $\gamma < 1$:
    Local minimum: \alert{Branching} flow
  \item[(c)] 
    $\gamma < 1$:
    Global minimum: \alert{Branching} flow
    
  \end{itemize}

  \medskip
  
  {\small From Bohn and Magnasco\cite{bohn2007a}}

  {\small See also Banavar et al.\cite{banavar2000a}}

\end{frame}

\begin{frame}
  \frametitle{Single source optimal supply}

  Optimal paths related to transport (Monge) problems:

  \includegraphics[width=0.48\textwidth]{xia2003a_fig1.pdf}
  \includegraphics[width=0.48\textwidth]{xia2003a_fig6.pdf}

  Xia (2003)\cite{xia2003a}

\end{frame}

\begin{frame}
  \frametitle{Growing networks:}

  \begin{center}
    \includegraphics[width=0.75\textwidth]{xia2007a_fig1.pdf}
  \end{center}

%%  \includegraphics[width=0.49\textwidth]{xia2007a_fig2}

  Xia (2007)\cite{xia2007a}

\end{frame}

\begin{frame}
  \frametitle{Growing networks:}

  \begin{center}
    \includegraphics[width=0.75\textwidth]{xia2007a_fig3.pdf}
  \end{center}

%%  \includegraphics[width=0.49\textwidth]{xia2007a_fig4}

  Xia (2007)\cite{xia2007a}

\end{frame}

\begin{frame}
  \frametitle{Single source optimal supply}

  \begin{block}{An immensely controversial issue...}
    \begin{itemize}
    \item<1->
      The form of river networks and blood networks:
      optimal or not?\cite{west1997a,banavar1999a,dodds2001a,dodds2010a}
    \end{itemize}
  \end{block}

  \begin{block}<2->{Two observations:}
    \begin{itemize}
    \item<2->
      Self-similar networks appear everywhere in nature
      for single source supply/single sink collection.
    \item<3->
      Real networks \alertb{differ} in \alert{details of scaling}
      but reasonably \alertb{agree} in \alert{scaling relations}.
    \end{itemize}
  \end{block}

\end{frame}


\begin{frame}
  \frametitle{River network models}

  \begin{block}<1->{Optimality:}
    \begin{itemize}
    \item<1->Optimal channel networks\cite{rodriguez-iturbe1997a}
    \item<1->Thermodynamic analogy\cite{scheidegger1991a}
    \end{itemize}
  \end{block}
  \uncover<2->{versus...}
  \begin{block}<2->{Randomness:}
    \begin{itemize}
    \item<2->Scheidegger's directed random networks
    \item<2->Undirected random networks
    \end{itemize}
  \end{block}
  
\end{frame}

\section{Optimal\ branching}

\subsection{Murray's\ law}

\begin{frame}
  \frametitle{Optimization---Murray's law}

  \begin{columns}
    \column{0.45\textwidth}
    \includegraphics[width=\textwidth]{murrays-law-tp-10}
    \column{0.55\textwidth}
    \begin{itemize}
    \item<1-> Murray's law (1926) connects branch radii at forks:\cite{murray1926a,murray1926b,murray1927a,labarbera1990a,thompson1961a}
      $$ \boxed{ \alert{r_0^{3} = r_1^{3} + r_2^{3}}} $$
      where $r_0$ = radius of main branch,
      and $r_1$ and $r_2$ are radii of sub-branches.
    \end{itemize}
  \end{columns}
    \begin{itemize}
    \item<2-> Holds up well for outer branchings of blood networks.
    \item<3-> Also found to hold for trees\cite{murray1927a,mcculloh2003a,mcculloh2004a}.
    \item<4-> See D'Arcy Thompson's ``On Growth and Form'' for background inspiration\cite{thompson1952a,thompson1961a}.
    \end{itemize}

\end{frame}


%% \begin{frame}
%%   \frametitle{Optimization approaches}
%% 
%%   \begin{block}{Cardiovascular networks:}
%%     \begin{itemize}
%%     \item<1-> Murray's law (1926) connects branch radii at forks:\cite{murray1926a,murray1926b,murray1927a,labarbera1990a,thompson1961a}
%%       $$ \boxed{ \alert{r_0^{3} = r_1^{3} + r_2^{3}}} $$
%%       where $r_0$ = radius of main branch\\
%%       and $r_1$ and $r_2$ are radii of sub-branches.
%%     \item<1-> See D'Arcy Thompson's ``On Growth and Form'' for background inspiration\cite{thompson1952a,thompson1961a}.
%%     \item<2-> Calculation assumes 
%%       \wordwikilink{http://en.wikipedia.org/wiki/Hagen-Poiseuille_equation}{Poiseuille flow}.
%%     \item<3-> Holds up well for outer branchings of blood networks.
%%     \item<4-> Also found to hold for trees\cite{murray1927a,mcculloh2003a,mcculloh2004a}.
%%     \item<5-> Use hydraulic equivalent of Ohm's law:
%%       $$
%%       \Delta p = \Phi Z  \Leftrightarrow V = IR
%%       $$
%%       where $\Delta p$ = pressure difference, $\Phi$ = flux.
%%     \end{itemize}
%%   \end{block}
%% \end{frame}

\begin{frame}
%%  \frametitle{Optimization---Murray's law}

    \begin{itemize}
    \item<1-> Use hydraulic equivalent of Ohm's law:
      $$
      \Delta p = \Phi Z  \Leftrightarrow V = IR
      $$
      where $\Delta p$ = pressure difference, $\Phi$ = flux.
      \begin{columns}
        \column{0.4\textwidth}
        \includegraphics[width=\textwidth]{poiseuille-flow-tp-10}
        \column{0.6\textwidth}
        \begin{itemize}
        \item<2->
          Fluid mechanics: 
          \wordwikilink{http://en.wikipedia.org/wiki/Hagen-Poiseuille_equation}{Poiseuille impedance}
          for smooth 
          \wordwikilink{http://en.wikipedia.org/wiki/Hagen-Poiseuille_equation}{Poiseuille flow}
          in a tube
          of radius $r$ and length $\ell$:
          $$ Z = \frac{8\eta \ell}{\pi r^4} $$
        \end{itemize}
      \end{columns}
    \item<2->
      $\eta$ = 
      \wordwikilink{http://en.wikipedia.org/wiki/Dynamic_viscosity}{dynamic viscosity}
      (units: $ML^{-1}T^{-1}$).
    \item<3-> 
      Power required to overcome impedance: 
      $$ P_{\textnormal{drag}} = \Phi \Delta p  = \Phi^2 Z. $$
    \item<4-> 
      Also have rate of energy expenditure in maintaining blood
      given metabolic constant $c$:
      $$ P_{\textnormal{metabolic}} = c r^2 \ell  $$
    \end{itemize}

\end{frame}

\begin{frame}
  \frametitle{Optimization---Murray's law}

  \begin{block}<1->{Aside on $P_{\textnormal{drag}}$}
  \begin{itemize}
  \item<2-> 
    Work done = $F \cdot d$ = energy transferred by force $F$
  \item<3-> 
    Power = $P$ = rate work is done = $F \cdot v$
  \item<4-> $\Delta p$ = Force per unit area
  \item<5-> $\Phi$ = Volume per unit time \\ = cross-sectional area $\cdot$ velocity
  \item<6-> So $\Phi \Delta p$ = Force $\cdot$ velocity
  \end{itemize}
  \end{block}

\end{frame}


\begin{frame}
  \frametitle{Optimization---Murray's law}

  \begin{block}{Murray's law:}
    \begin{itemize}
    \item<1-> Total power (cost):
      $$ 
      P = P_{\textnormal{drag}} + P_{\textnormal{metabolic}}
      \uncover<2->{=
      \Phi^2 \frac{8\eta \alert{\ell}}{\pi \alert{r^4}}
      + c \alert{r^2 \ell}}
      $$
    \item<3-> Observe power increases linearly with $\ell$
    \item<4-> But $r$'s effect is nonlinear: 
      \begin{itemize}
      \item<5->  
        increasing $r$
        makes flow easier \alert{but increases metabolic cost} (as $r^2$)
      \item<6->
        decreasing $r$
        decrease metabolic cost \alert{but impedance goes up} (as $r^{-4}$)
      \end{itemize}
    \end{itemize}
  \end{block}

\end{frame}


\begin{frame}
  \frametitle{Optimization---Murray's law}

  \begin{block}{Murray's law:}
    \begin{itemize}
    \item<1-> Minimize $P$ with respect to $r$:
      $$
      \partialdiff{P}{r}
      = 
      \partialdiff{}{r} 
      \left( 
        \Phi^2 \frac{8\eta {\ell}}{\pi {r^4}}
      + c {r^2 \ell}
      \right)
    $$
      \uncover<2->{
        $$
        = 
        -4 \Phi^2 \frac{8\eta {\ell}}{\pi {r^5}}
        + c {2r \ell}
        \uncover<3->{\alert{=0}}
      $$
      \item<4-> Rearrange/cancel/slap:
        $$
        \alert{\Phi^2} = \frac{c \pi r^6}{16 \eta} \uncover<5->{= k^2 \alert{r^6}}
        $$
        \uncover<5->{where $k$ = constant.}
        
      }
    \end{itemize}
  \end{block}

\end{frame}


\begin{frame}
  \frametitle{Optimization---Murray's law}

  \begin{block}{Murray's law:}
    \begin{itemize}
    \item<1-> So we now have:
      $$
      \Phi = k r^3 
      $$
    \item<2-> 
      Flow rates at each branching have to add up
      (else our organism is in serious trouble...):
      $$
      \Phi_0 = \Phi_1 + \Phi_2
      $$
      where again 0 refers to the  main branch and 1 and 2 refers
      to the offspring branches
    \item<3->
      All of this means we have a groovy cube-law:
      $$ 
      \boxed{\alert{r_0^3 = r_1^3 + r_2^3}}
      $$
      
    \end{itemize}
  \end{block}

\end{frame}

\subsection{Murray\ meets\ Tokunaga}


\begin{frame}
  \frametitle{Optimization}

  \begin{block}{Murray meets Tokunaga:}
    \begin{itemize}
    \item<1-> 
      $\Phi_\om$ = volume rate of flow into an order
      $\om$ vessel segment
    \item<2-> 
      Tokunaga picture:
      $$ 
      \Phi_\om
      = 
      2 \Phi_{\om - 1}
      +
      \sum_{k=1}^{\om-1}
      T_k
      \Phi_{\om-k}
      $$
    \item<3->
      Using $\phi_\om = k r_{\om}^{3}$
      $$
      r_\om^3
      = 
      2 r_{\om - 1}^{3}
      +
      \sum_{k=1}^{\om-1}
      T_k
      r_{\om-k}^{3}
      $$
    \item<4->
      Find Horton ratio for vessel radius $R_r = r_{\om}/r_{\om-1}$...
    \end{itemize}
  \end{block}

\end{frame}


\begin{frame}
  \frametitle{Optimization}

  \begin{block}{Murray meets Tokunaga:}
    \begin{itemize}
    \item<1-> 
      Find $R_r^{\, 3}$ satisfies same equation as $R_n$ and $R_v$\\
      ($v$ is for volume):
      $$
      \boxed{ \alert{R_r^3 = R_n = R_v} }
      $$
    \item<2->
      Is there more we could do here to constrain the Horton
      ratios and Tokunaga constants?
    \end{itemize}
  \end{block}

\end{frame}


\begin{frame}
  \frametitle{Optimization}

  \begin{block}{Murray meets Tokunaga:}
    \begin{itemize}
    \item<1-> 
      Isometry: $V_\om \propto \ell_\om^{\, 3}$
    \item<2->
      Gives 
      $$\boxed{\alert{R_\ell^3 = R_v = R_n}}$$
    \item<3-> 
      We need one more constraint...
    \item<4->
      West et al (1997)\cite{west1997a} achieve similar
      results following Horton's laws.
    \item<5->
      So does Turcotte et al. (1998)\cite{turcotte1998a}
      using Tokunaga (sort of).
    \end{itemize}
  \end{block}

\end{frame}

%% \changelecturelogo{.2}{virtualvessels4.pdf}

\section{Single\ Source}

\subsection{Geometric\ argument}

%% \subsection{History}
%% 
%% \begin{frame}
%%   \frametitle{History in brief}
%%   
%%   \begin{block}<1->{Seeking optimal universality:}
%%     \begin{itemize}
%%     \item<1-> Two major real-world branching networks
%%       \begin{enumerate}
%%       \item \alert{Blood networks}
%%       \item \alertb{River networks}
%%       \end{enumerate}
%%     \item<2-> Blood networks argued
%%       to lead to 
%%       $$B \propto M^{\alpha}$$
%%       where
%%       \begin{itemize}
%%       \item 
%%         $B$ = basal metabolic rate and $M$ = body mass
%%       \item 
%%       $\alpha=2/3$ or $3/4$ or something else...\cite{kleiber1961a,west1997a,banavar1999a,dodds2001d}
%%       \end{itemize}
%%     \item<3-> River basins may or may not scale allometrically.
%%     \item<4->
%%       Recall: Hack's law\cite{hack1957a}
%%       $$\ell \propto a^h$$
%%       If $h>1/2$ then basins elongate.
%%     \end{itemize}
%%     
%%   \end{block}
%%   
%% \end{frame}
%% 
%% \subsection{Minimal volume calculation}
%% 
%% \begin{frame}
%%   \frametitle{Geometric argument}
%% 
%%   \begin{itemize}
%%   \item<1-> 
%%     Consider \alert{one source supplying many sinks} in a $d$ dimensional volume
%%   \item<2->
%%     Material draw by sinks is invariant.
%%   \item<3-> 
%%     See network as a bundle of virtual vessels:
%%     \begin{overprint}
%%       \onslide<1-2>        
%%       \onslide<3->        
%%       \begin{center}
%%         \includegraphics[angle=-90,width=0.8\textwidth]{virtualvessels4.pdf}
%%       \end{center}
%%       \end{overprint}
%%   \item<4-> 
%%     \alert{The simplest question}: how does number  of sustainable
%%     sinks $N_{\textnormal{sinks}}$
%%     scale with volume $V$ for the most efficient network design?
%%   \item<5-> 
%%     Or: what is highest $\alpha$ for $N_{\textnormal{sinks}} \propto V^{\alpha}$?
%%   \item<6-> 
%%     Covered in PoCS CSYS 300: we will recap and refine here.
%%   \end{itemize}
%% 
%% %  \item<3-> 
%% %    Assume some cap on flow speed of material, $v_{\textnormal{max}}$
%% 
%% \end{frame}
%% 
%% \begin{frame}
%%   \frametitle{Geometric argument}
%% 
%%   \begin{itemize}
%%   \item<1-> Consider families of systems that grow allometrically.
%%   \item<2-> Family = a basic shape $\Omega$ indexed by volume $V$.
%%   \begin{center}
%%     \includegraphics[angle=-90,width=0.8\textwidth]{shapescaling}    
%%   \end{center}
%%   \bigskip
%%   \item<3-> Orient shape to have dimensions $L_1 \times L_2 \times  ... \times L_d$
%%   \item<4-> In 2-d,
%%     $L_1 \propto A^{\gamma_1}$ and $L_2 \propto A^{\gamma_2}$
%%     where $A$ = area.
%%   \item<5-> In general, have $d$ lengths which scale
%%     as $L_i \propto V^{\gamma_i}$.
%%   \item<6-> For above example, width grows faster than
%%     height: $\gamma_1 > \gamma_2$.
%%   \end{itemize}
%% 
%% \end{frame}
%% 
%% \begin{frame}
%%   \frametitle{Geometric argument}
%% 
%%   \begin{block}<1->{Some generality:}
%%     \begin{itemize}
%%     \item<1-> Consider $d$ dimensional spatial regions living in 
%%       $D$ dimensional ambient spaces.  \uncover<2->{Notation: \alert{$\volume{V}$}.}
%%     \item<3-> River networks: \alertb{$d=2$ and $D=3$}
%%     \item<4-> Cardiovascular networks: \alertb{$d=3$ and $D=3$}
%%     \item<5->
%%       \alert{Star-convexity of $\volume{V}$:} A spatial
%%       region is star-convex if from at least one point, all other
%%       points in the region can be reached by travelling along straight lines
%%       while remaining within the region.
%%     \item<6->
%%       Assume source can be located at a point which has direct line of
%%       sight to all sources.
%%     \item<7->
%%       We can generalize to a much broader class of shapes...
%%     \end{itemize}
%%     
%%   \end{block}
%% 
%% \end{frame}
%% 
%% 
%% \begin{frame}
%%   \frametitle{Geometric argument}
%% 
%%   \begin{itemize}
%%   \item<1-> Reminder of best and worst configurations
%%     \begin{center}
%%       \includegraphics[angle=-90,width=0.8\textwidth]{efficientnetworks5.pdf}
%%     \end{center}
%%     \bigskip
%%   \item<2-> \alert{Basic idea:}
%%     Minimum volume of material in system $V_{\textnormal{net}} \propto$ sum of distance
%%     from the source to the sinks.
%%   \item<3-> See what this means for sink density $\rho$ if sinks do not
%%     change their feeding habits with overall size.
%%   \end{itemize}
%% 
%% \end{frame}

\begin{frame}
%%  \frametitle{Geometric argument}

  \begin{block}{
      ``Optimal Form of Branching Supply and Collection Networks''\cite{dodds2010a}
    }
    P.\ S.\ Dodds, Phys. Rev. Lett., \textbf{104}, 048702, 2010.
    \begin{itemize}
    \item<1-> 
      Consider \alert{one source} supplying \alert{many sinks} in a 
      volume $V$ \alertb{$d$-dim.} region
      in a \alertb{$D$-dim.} ambient space.
    \item<1->
      Assume \alertb{sinks are invariant}.
    \item<1->
      Assume \alert{$\rho = \rho(V)$}, i.e., $\rho$ may vary with volume $V$.
    \item<2-> 
      See network as a bundle of virtual vessels:
      \begin{center}
        \begin{overprint}
          \onslide<1 | handout:0| trans:0>
          \onslide<2-| handout:1| trans:1>
          \includegraphics[angle=-90,width=0.8\textwidth]{virtualvessels4.pdf}
        \end{overprint}
      \end{center}
    \item<3-> 
      \alert{Q:} how does the number of sustainable
      sinks $N_{\textnormal{sinks}}$
      scale with volume $V$ for the most efficient network design?
    \item<4-> 
      \alert{Or:} what is the highest $\alpha$ for $N_{\textnormal{sinks}} \propto V^{\alpha}$?
    \end{itemize}
  \end{block}

\end{frame}

\begin{frame}
  \frametitle{Geometric argument}

  \begin{itemize}
  \item<1-> Allometrically growing regions:
%  \item<2-> Family = a basic shape $\Omega$ indexed by volume $V$.
  \begin{center}
    \includegraphics[width=0.8\textwidth]{shapescaling-unrotated}    
  \end{center}
  \bigskip
%  \item<3-> Orient shape to have dimensions $L_1 \times L_2 \times  ... \times L_d$
%  \item<4-> In 2-d,
%    $L_1 \propto A^{\gamma_1}$ and $L_2 \propto A^{\gamma_2}$
%    where $A$ = area.
  \item<1-> Have $d$ length scales which scale
    as 
    {
      $$
      \alertb{L_i} \propto \alertb{V}^{\alertb{\gamma_i}}
      \mbox{\ where $\gamma_1 + \gamma_2 + \ldots + \gamma_d = 1$.}
      $$
    }
  \item<1-> 
    For \alert{isometric} growth, $\gamma_i = 1/d$.
  \item<1->
    For \alert{allometric} growth, 
    we must have at least two of the $\{\gamma_i\}$ being different
%  \item<6-> For above example, width grows faster than
%    height: $\gamma_1 > \gamma_2$.
  \end{itemize}

\end{frame}


\begin{frame}
  \frametitle{Geometric argument}

  \begin{itemize}
  \item<1-> Best and worst configurations (Banavar et al.)
    \begin{center}
      \includegraphics[angle=-90,width=0.8\textwidth]{efficientnetworks5.pdf}
    \end{center}
    \bigskip
  \item<2-> \alert{Rather obviously:}\\
    $\min V_{\textnormal{net}} \propto \sum$
    distances
    from source to sinks.

%  \item<3-> See what this means for sink density $\rho$ if sinks do not
%    change their feeding habits with overall size.
  \end{itemize}

\end{frame}

\begin{frame}
  \frametitle{Minimal network volume:}

  Real supply networks are close to optimal:

  \includegraphics[width=\textwidth]{gastner2006a_fig1.pdf}

  \bigskip

  {\small (2006)
    Gastner and Newman\cite{gastner2006a}:
    ``Shape and efficiency in spatial distribution networks'' }

\end{frame}

\begin{frame}
  \frametitle{Minimal network volume:}

  \begin{block}{We add one more element:}
    \includegraphics[width=\textwidth]{shapes-virtualv-4c.pdf}
    \begin{itemize}
    \item Vessel cross-sectional area
      may vary with distance from the source.
    \item
      Flow rate increases as cross-sectional area decreases.
    \item e.g., a collection network may
      have vessels tapering as they approach
      the central sink.
    \item
      Find that vessel volume $v$ must scale
      with vessel length $\ell$ to affect overall
      system scalings.
    \end{itemize}
  \end{block}
\end{frame}

\begin{frame}
  \frametitle{Minimal network volume:}

  \begin{block}{Effecting scaling:}
    \includegraphics[width=\textwidth]{shapes-virtualv-4c.pdf}
    \begin{itemize}
    \item
      Consider vessel radius $r \propto (\ell+1)^{-\epsilon}$,
      tapering from $r=r_{\max}$ where $\epsilon \ge 0$.
    \item
      Gives
      $
      v \propto \ell^{1-2\epsilon}
      $ if $\epsilon < 1/2$
    \item
      Gives
      $
      v \propto 1 - \ell^{-(2\epsilon-1)} \rightarrow 1$ for large $\ell$
      if $\epsilon > 1/2$
    \item
      Previously, we looked at $\epsilon=0$ only.
    \end{itemize}
  \end{block}
\end{frame}

\begin{frame}
  \frametitle{Minimal network volume:}

  For $0 \le \epsilon < 1/2$, approximate network volume by integral over region:
  $$ 
  \alertb{\min V_{\textnormal{net}}}  \propto 
  \int_{\volume{V}} \alertb{\rho} \, ||\vec{x}||^{1-2\epsilon} \, \dee{\vec{x}} 
  $$
  %%   \visible<2->{
  %%     $$
  %%     \rightarrow 
  %%     \rho V^{1+\gamma_{\max}}
  %%     \int_{\volume{c}} (c_1^{2} u_1^2 + \ldots + c_k^{2} u_k^2 )^{(1-2\epsilon)/2}
  %%     \dee{\vec{u}}
  %%     $$
  %%   }
  \visible<2->{\insertassignmentquestion{03}{3}{1}}
  \visible<2->{
    $$
    \propto
    \alert{ \rho V^{1+\gamma_{\max}(1-2\epsilon)} } 
    \
    \mbox{where}
    \
    \gamma_{\max} = \max_{i} \gamma_i.
    $$
  }
  \visible<3->{
    For $\epsilon > 1/2$, find simply that 
    $$
    \alertb{\min V_{\textnormal{net}}}  
    \propto 
    \rho V
    $$
  }
  \begin{itemize}
  \item<4->
    So if supply lines can taper fast enough and without
    limit, minimum network volume can be made negligible.
%  \item<5->
%    \alert{The problem:} must eventually reach a limiting speed
%    or size (e.g., blood velocity and cells).
  \end{itemize}
\end{frame}

\begin{frame}
  \frametitle{Geometric argument}

  \begin{block}{For $0 \le \epsilon < 1/2$:}
    \begin{itemize}
    \item<1-> 
      $
      \boxed{\alert{
          \min V_{\textnormal{net}} 
          \propto
          \rho V^{1+\gamma_{\max}(1-2\epsilon)} 
        }}
      $
    \item<2-> 
      If scaling is \alertb{isometric}, we have $\gamma_{\max} = 1/d$:
      $$
      \min V_{\textnormal{net/iso}} 
      \propto
      \rho V^{1+(1-2\epsilon)/d}
      $$
    \item<3-> 
      If scaling is \alertb{allometric}, we have
      $\gamma_{\max} = \gamma_{\textnormal{allo}} > 1/d$:
      and 
      $$
      \min V_{\textnormal{net/allo}} 
      \propto
      \rho V^{1+(1-2\epsilon)\gamma_{\textnormal{allo}}}
      $$
    \item<4-> 
      Isometrically growing volumes 
      \alert{require less network volume} 
      than allometrically growing volumes:
      $$
      \frac{\min V_{\textnormal{net/iso}}}{\min V_{\textnormal{net/allo}}} \rightarrow 0 
      \mbox{\ as $V \rightarrow \infty$}
      $$
    \end{itemize}    
    
  \end{block}
\end{frame}

\begin{frame}
  \frametitle{Geometric argument}

  \begin{block}{For $\epsilon > 1/2$:}
    \begin{itemize}
    \item<1-> 
      $
      \boxed{\alert{
          \min V_{\textnormal{net}} 
          \propto
          \rho V
        }}
      $
    \item<2-> 
      Network volume scaling is now independent 
      of overall shape scaling.
    \end{itemize}
  \end{block}

  \medskip

  \begin{block}<3->{Limits to scaling}
    \begin{itemize}
    \item 
      Can argue that $\epsilon$ must effectively be 0
      for real networks over large enough scales.
    \item 
      Limit to how fast material can move,
      and how small material packages can be.
    \item 
      e.g., blood velocity and blood cell size.
    \end{itemize}
  \end{block}
\end{frame}


\subsection{Real\ networks}
%% \subsection{Blood\ networks}

\begin{frame}
  \frametitle{Blood networks}

  \begin{itemize}
  \item<1-> Velocity at capillaries and 
    aorta approximately constant across body size\cite{weinberg2006a}: 
    $\epsilon = 0$.
  \item<2-> \alert{Material costly} $\Rightarrow$ expect lower optimal bound of 
    $V_{\textnormal{net}} \propto \rho V^{(d+1)/d}$ to be followed closely.
  \item<3->
    For cardiovascular networks, \alert{$d=D=3$}.
  \item<4->
    Blood volume scales linearly with body
    volume\cite{stahl1967a}, $V_{\textnormal{net}} \propto V$.
  \item<5->
    Sink density must $\therefore$ decrease as volume increases:
    $$
    \alertb{\rho \propto V^{-1/d}}.
    $$
  \item<6->
    Density of suppliable sinks \alert{decreases} with organism size.
  \end{itemize}      

\end{frame}


\begin{frame}
  \frametitle{Blood networks}

  \begin{itemize}
  \item<1-> Then $P$, the rate of overall energy 
    use in $\Omega$, can at most scale with volume as
    $$
    P \propto \rho V 
    \visible<2->{
      \propto \rho \, M
    }
    \visible<3->{
      \propto M^{\, (d-1)/d}
    }
    $$
  \item<4-> 
    For $d=3$ dimensional organisms, we have 
    $$\alertb{\boxed{ P \propto M^{\, 2/3}}}$$
  \item<5-> 
    Including other constraints may raise scaling exponent
    to a higher, less efficient value.
  \item<6->
    \alertb{Exciting bonus:} 
    Scaling obtained by the supply network story and the surface-area law
    \alert{only match} for isometrically growing shapes.\\
    \insertassignmentquestion{03}{3}{3}
  \end{itemize}    

\end{frame}

\begin{frame}
  \frametitle{Recap:}

  \begin{block}{}
  \begin{itemize}
  \item<+-> 
    The exponent $\alpha = 2/3$ works for all birds and
    mammals up to 10--30 kg
  \item<+-> 
    For mammals $>$ 10--30 kg, maybe we have a new scaling regime
  \item<+-> 
    Economos: limb length break in scaling around 20 kg
  \item<+-> 
    White and Seymour, 2005: unhappy with large herbivore measurements.
    Find $\alpha \simeq 0.686 \pm 0.014$
  \end{itemize}
  \end{block}

\end{frame}

\begin{frame}
  \frametitle{General unhappiness:}

  \begin{block}{Everyone is confused:}
    \begin{itemize}
    \item
      White et al., Ecology, 2007:
      ``Allometric exponents do not support a universal metabolic allometry''\cite{white2007a}
    \item
      Savage et al., PLoS Computational Biology:
      ``Sizing Up Allometric Scaling Theory.''\cite{savage2008a}
    \item 
      Banavar et al., PNAS, 2010:
      ``A general basis for quarter-power scaling in animals.''\cite{banavar2010a}
  \end{itemize}
  \end{block}

\end{frame}


%% \begin{frame}
%%   \frametitle{Prefactor:}
%% 
%%   \begin{block}<1->{Stefan-Boltzmann law:}
%%     \begin{itemize}
%%     \item<1->
%%       $$\diff{E}{t} = \sigma S T^4$$
%%       where $S$ is surface and $T$ is temperature.
%%     \item<2-> 
%%       Very rough estimate of prefactor based on scaling
%%       of normal mammalian body temperature and surface
%%       area $S$:
%%       $$B \simeq 10^5M^{2/3} \mbox{erg/sec}.$$
%%     \item<3->
%%       Measured for $M \leq 10$ kg:
%%       $$B=2.57\times 10^5M^{2/3} \mbox{erg/sec}.$$
%%     \end{itemize}
%%   \end{block}
%% 
%% \end{frame}

%% \subsection{River\ networks}

\begin{frame}
  \frametitle{River networks}

  \begin{itemize}
  \item<1-> View river networks as collection networks.
  \item<1-> Many sources and one sink.
  \item<2-> $\epsilon$?
  \item<3-> Assume $\rho$ is constant over time and $\epsilon=0$:
    $$V_{\textnormal{net}} \propto \rho V^{(d+1)/d} = \mbox{constant} \times V^{\, 3/2} $$
  \item<4-> Network volume grows faster than
    basin `volume' (really area).
  \item<5-> \alert{It's all okay:}\\ 
    Landscapes are $d$=2 surfaces living in $D$=3 dimensions.
  \item<6->
    Streams can grow not just in width but in depth...
  \item<7->
    If $\epsilon > 0$, $V_{\textnormal{net}}$ will grow more slowly
    but 3/2 appears to be confirmed from real data.
  \end{itemize}

\end{frame}

%% \begin{frame}
%%   \frametitle{Hack's law}
%% 
%%   \begin{itemize}
%%   \item<1-> Volume of water in river network can be calculated 
%%     by adding up basin areas
%%   \item<1-> Flows sum in such a way that 
%%     $$ V_{\textnormal{net}} = \sum_{\mbox{\scriptsize all pixels}} a_{\mbox{\scriptsize pixel $i$}} $$
%%   \item<1-> Hack's law again:
%%     $$
%%     \ell \sim a^{\, h}
%%     $$
%%   \item<1-> 
%%     Can argue     
%%     $$ V_{\textnormal{net}} \propto V_{\textnormal{basin}}^{1+h} = a_{\textnormal{basin}}^{1+h}$$
%%     where 
%%     $h$ is Hack's exponent.
%%   \item<1-> 
%%     $\therefore$ minimal volume calculations gives 
%%     $$
%%     \boxed{
%%       h=1/2
%%     }
%%     $$
%%   \end{itemize}
%% 
%% \end{frame}
%% 
%% \begin{frame}
%%   \frametitle{Real data:}
%% 
%%   \begin{columns}
%%     \column{0.4\textwidth}
%%     \begin{itemize}
%%     \item<1-> Banavar et al.'s approach\cite{banavar1999a} is okay 
%%       because $\rho$ \alertb{really is constant}.
%%     \item<3-> \alert{The irony:} shows optimal basins are isometric
%%     \item<4-> Optimal Hack's law: $\msl \sim a^{h}$ with
%%       $h=1/2$ 
%%     \item<5-> \visible<5->{(Zzzzz)}
%%     \end{itemize}
%%     \column{0.6\textwidth}
%%     \begin{overprint}
%%       \onslide<2-| handout:1| trans:1>
%%       \includegraphics[width=\textwidth]{banavar1999fig2.png}\\
%%       {\small From Banavar et al. (1999)\cite{banavar1999a}}
%%     \end{overprint}
%%   \end{columns}
%% \end{frame}
%% 
%% \begin{frame}
%%   \frametitle{Even better---prefactors match up:}
%% 
%%   \begin{center}
%%     \includegraphics[width=0.8\textwidth]{figwatervolume02_noname.pdf}
%%   \end{center}
%% 
%% \end{frame}


\changelecturelogo{.18}{gastner2006c_fig6b-tp-1}

\section{Distributed\ Sources}

%% \subsection{Facility location}

\begin{frame}
  \frametitle{Many sources, many sinks}
  
  \begin{block}<1->{How do we distribute sources?}
    \begin{itemize}
    \item<2-> Focus on 2-d (results generalize to higher dimensions).
    \item<3-> Sources = hospitals, post offices, pubs, ...
    \item<4-> \alert{Key problem:} How do we cope with uneven population densities?
    \item<5-> Obvious: if density is uniform then sources are best distributed
      \alert{uniformly}.
    \item<6-> Which lattice is optimal? \uncover<7->{The \alert{hexagonal lattice}}
    \item<8-> \alert{Q2:} Given population density is uneven, what do we do?
    \item<9-> We'll follow work by Stephan (1977, 1984)\cite{stephan1977a,stephan1984a},
      Gastner and Newman (2006)\cite{gastner2006c}, 
      Um \etal (2009)\cite{um2009a} and work cited by them.
    \end{itemize}
  \end{block}

\end{frame}


\begin{frame}
  \frametitle{Optimal source allocation}

  \begin{block}<1->{Solidifying the basic problem}
    \begin{itemize}
    \item<1-> Given a region with some population distribution $\rho$, most likely uneven.
    \item<2-> Given resources to build and maintain $N$ facilities.
    \item<3-> \alert{Q:} How do we locate these $N$ facilities so as to
      \alert{minimize the average distance} between an \alertb{individual's residence} and 
      the \alertb{nearest facility}?
    \end{itemize}
  \end{block}
\end{frame}

\begin{frame}
  \frametitle{Optimal source allocation}

  {\centering
    \includegraphics[width=0.9\textwidth]{gastner2006c_fig1}
  }

  {\small 
    From Gastner and Newman (2006)\cite{gastner2006c}\\
    \begin{itemize}
    \item<1-> Approximately optimal location of 5000 facilities.
    \item<1-> Based on 2000 Census data.
    \item<1-> Simulated annealing + Voronoi tessellation.
    \end{itemize}
  }

\end{frame}

\begin{frame}
  \frametitle{Optimal source allocation}

  \begin{center}
    \includegraphics[width=0.7\textwidth]{gastner2006c_fig2}
  \end{center}
  {\small
    From Gastner and Newman (2006)\cite{gastner2006c}
  }
  \begin{itemize}
  \item<1-> Optimal facility density $\rhofac$ vs.\ population density $\rhopop$.
  \item<2-> Fit is $\rhofac \propto \rhopop^{0.66}$ with $r^2 = 0.94$.
  \item<3-> Looking good for a 2/3 power...
  \end{itemize}
  
\end{frame}


\subsection{Size-density\ law}

\begin{frame}
  \frametitle{Optimal source allocation}

  \begin{block}<1->{Size-density law:}
    \begin{itemize}
    \item<1->
    $$
    \boxed{\alert{\rhofac \propto \rhopop^{2/3}}}
    $$
    \item<2->
      Why?
    \item<3->
      Again: Different story to branching networks where
      there was either one source or one sink.
    \item<4->
      Now sources \& sinks are distributed 
      throughout region...
    \end{itemize}
  \end{block}

\end{frame}

\begin{frame}
  \frametitle{Optimal source allocation}

    \begin{itemize}
    \item<+-> 
      We first examine Stephan's treatment (1977)\cite{stephan1977a,stephan1984a}
    \item<+->
      \alertb{``Territorial Division: The Least-Time Constraint
        Behind the Formation of Subnational Boundaries''} (Science, 1977)
    \item<+->
      Zipf-like approach: invokes \alert{principle of minimal effort}.
    \item<+->
      Also known as the Homer principle.
    \end{itemize}

\end{frame}

\begin{frame}
  \frametitle{Optimal source allocation}

  \begin{itemize}
  \item<1-> 
    Consider a region of area $A$ and population $P$ with
    a single functional center that everyone needs to access
    every day.
  \item<2->
    Build up a general cost function based on time expended
    to \alert{access and maintain center}.
  \item<3->
    Write \alert{average travel distance} to center as $\bar{d}$ and 
    assume \alert{average speed of travel} is $\bar{v}$.
  \item<4->
    Assume \alertb{isometry}: average travel distance $\bar{d}$ will be on the length
    scale of the region which is $\sim$ \alertb{$A^{1/2}$}
  \item<5->
    Average time expended per person in accessing facility
    is therefore
    \alertb{
      $$
      \bar{d}/\bar{v} = c A^{1/2} / \bar{v}
      $$
    }
    where $c$ is an unimportant shape factor.
  \end{itemize}

\end{frame}

\begin{frame}
  \frametitle{Optimal source allocation}

  \begin{itemize}
  \item<1-> Next assume facility requires regular maintenance (person-hours per day)
  \item<2-> Call this quantity $\tau$
  \item<3-> If burden of mainenance is shared then average cost per person
    is \alert{$\tau/P$} where $P$ = population.
  \item<4-> Replace $P$ by $\rhopop A$ where $\rhopop$ is density.
  \item<5-> Total average time cost per person:
    $$
    T = \bar{d}/\bar{v} + \tau/(\rhopop A) 
    \uncover<6->{= c \alert{A^{1/2}}/\bar{v} + \tau/(\rhopop \alert{A}).}
    $$
  \item<7-> Now Minimize with respect to $A$...
  \end{itemize}

\end{frame}

\begin{frame}
  \frametitle{Optimal source allocation}

  \begin{itemize}
  \item<1-> Differentiating...
    $$
    \partialdiff{T}{A} = 
    \partialdiff{}{A} \left( c A^{1/2}/\bar{v} + \tau/(\rhopop A) \right)
    $$
    $$
    \uncover<2->{
      =
      \frac{c}{2\bar{v} A^{1/2}}
      -\frac{\tau}{\rhopop A^2}
    }
    \uncover<3->{
      \alert{ = 0 }
    }
    $$
  \item<4-> Rearrange:
    $$
    A = 
    \left(
      \frac{2 \bar{v} \tau}
      {c \rhopop}
    \right)^{2/3}
    \uncover<5->{
      \propto \rhopop^{-2/3}
    }
    $$
  \item<6-> \# facilities per unit area $\rhofac$:
    $$ 
    \rhofac
    \propto
    \alert{A^{-1}  \propto \rhopop^{2/3}}
    $$
  \item<7-> Groovy...
    
  \end{itemize}

\end{frame}

\begin{frame}
  \frametitle{Optimal source allocation}

  \begin{block}{An issue:}
    \begin{itemize}
    \item<1-> Maintenance ($\tau$) is assumed to be 
      \alert{independent} of population
      and area ($P$ and $A$)
    \end{itemize}
  \end{block}
  
\end{frame}

\begin{frame}
  \frametitle{Optimal source allocation}

  \begin{itemize}
  \item 
    Stephan's online book\\
    \alert{``The Division of Territory in Society''}
    is
    \wordwikilink{http://www.edstephan.org/Book/contents.html}{here}.
  \item 
    (It used to be
    \wordwikilink{http://www.ac.wwu.edu/~stephan/Book/contents.html}{here}.)
  \item 
    The 
    \wordwikilink{http://www.edstephan.org/Book/chap0/0.html}{Readme} 
    is well worth reading (1995).
  \end{itemize}

  %% winner of the first Zipf award!
  %% George Kingsley Zipf 
  %% Memorial Award
  %% 1984 Population Association of America
\end{frame}

\subsection{Cartograms}

\begin{frame}
  \frametitle{Cartograms}

  Standard world map:
  \includegraphics[width=\textwidth]{newman_world1024x512.png}

\end{frame}

\begin{frame}
  \frametitle{Cartograms}

  Cartogram of countries `rescaled' by population:
  \includegraphics[width=\textwidth]{newman_population1024x512.png}\\
  \includegraphics[width=0.25\textwidth]{newman_world1024x512.png}
\end{frame}

\begin{frame}
  \frametitle{Cartograms}

  \begin{block}<1->{Diffusion-based cartograms:}
    \begin{itemize}
    \item<2-> Idea of cartograms is to \alert{distort areas} to 
      more accurately represent
      some local density $\rhopop$ (e.g. population).
    \item<3-> Many methods put forward---typically involve
      some kind of physical analogy to \alert{spreading or repulsion}.
    \item<4-> Algorithm due to Gastner and Newman (2004)\cite{gastner2004a}
      is based on \alertb{standard diffusion}:
      $$ 
      \nabla^2 \rhopop - \partialdiff{\rhopop}{t} = 0. 
      $$
    \item<5-> Allow density to diffuse and trace the 
      movement of individual elements and boundaries.
    \item<6-> Diffusion is constrained by boundary condition
      of surrounding area having density $\bar{\rhopop}$.
    \end{itemize}
  \end{block}

\end{frame}

\begin{frame}
  \frametitle{Cartograms}

  Child mortality:
  \includegraphics[width=\textwidth]{newman_childmort1024x512.png}

\end{frame}

\begin{frame}
  \frametitle{Cartograms}

  Energy consumption:
  \includegraphics[width=\textwidth]{newman_energyconsump1024x512.png}
\end{frame}

\begin{frame}
  \frametitle{Cartograms}

  Gross domestic product:
  \includegraphics[width=\textwidth]{newman_gdp1024x512.png}
\end{frame}

\begin{frame}
  \frametitle{Cartograms}

  Greenhouse gas emissions:
  \includegraphics[width=\textwidth]{newman_greenhouse1024x512.png}
\end{frame}

\begin{frame}
  \frametitle{Cartograms}

  Spending on healthcare:
  \includegraphics[width=\textwidth]{newman_healthcare1024x512.png}
\end{frame}

\begin{frame}
  \frametitle{Cartograms}
  
  People living with HIV:
  \includegraphics[width=\textwidth]{newman_hiv1024x512.png}
\end{frame}


\begin{frame}
  \frametitle{Cartograms}

  \begin{itemize}
  \item<1-> The preceding sampling of Gastner \& Newman's cartograms
    lives \wordwikilink{http://www-personal.umich.edu/~mejn/cartograms/}{here}.
  \item<1->
    A larger collection can be found
    at \wordwikilink{http://www.worldmapper.org/}{worldmapper.org}.

    \bigskip

    \includegraphics[width=0.5\textwidth]{worldmapper.png}
  \end{itemize}

\end{frame}

\begin{frame}
  \frametitle{Size-density law}

  \includegraphics[width=\textwidth]{gastner2006c_fig3}

  \begin{itemize}
  \item <1-> \alert{Left:} population density-equalized cartogram.
  \item <2-> \alert{Right:} (population density)$^{2/3}$-equalized cartogram.
  \item <3-> Facility density is uniform for $\rhopop^{2/3}$ cartogram.
  \end{itemize}
  {\small
    From Gastner and Newman (2006)\cite{gastner2006c}
  }
\end{frame}

%% \begin{frame}
%%   \frametitle{}
%% 
%%   \includegraphics[width=\textwidth]{gastner2006c_fig4}
%% 
%%   From Gastner and Newman (2006)\cite{gastner2006c}
%% \end{frame}

\begin{frame}
  \frametitle{Size-density law}

  \includegraphics[width=\textwidth]{gastner2006c_fig5}

  {\small
    From Gastner and Newman (2006)\cite{gastner2006c}
  }
  \begin{itemize}
  \item Cartogram's Voronoi cells are somewhat hexagonal.
  \end{itemize}
  
\end{frame}

\subsection{A\ reasonable\ derivation}

\begin{frame}
  \frametitle{Size-density law}

  \begin{block}<1->{Deriving the optimal source distribution:}
    \begin{itemize}
    \item<2-> \alert{Basic idea:} Minimize the average distance
      from a random individual to the nearest facility.\cite{gastner2006c}
    \item<3-> Assume given a fixed population density $\rhopop$ defined
      on a spatial region $\Om$.
    \item<4-> Formally, we want to find the locations of 
      \alert{$n$ sources} $\{\vec{x}_1,\ldots,\vec{x}_n\}$
      that minimizes the \alert{cost function}
      $$
      F(\{\vec{x}_1,\ldots,\vec{x}_n\})
      =
      \int_{\Om}
      \alert{\rhopop(\vec{x})}
      \,
      \alertb{\min_{i}
      || \vec{x} - \vec{x}_i ||}
      \dee{\vec{x}}.
      $$
    \item<5-> Also known as the p-median problem.
    \item<6-> Not easy...  \uncover<6->{in fact this one is an NP-hard problem.\cite{gastner2006c}}
    \item<7-> Approximate solution originally due to
      Gusein-Zade\cite{gusein-zade1982a}.
    \end{itemize}
  \end{block}

\end{frame}

\begin{frame}
  \frametitle{Size-density law}

  \begin{block}{Approximations:}
    \begin{itemize}
    \item<1-> For a given set of source placements $\{\vec{x}_1,\ldots,\vec{x}_n\}$,
      the region $\Om$ is divided up into 
      \wordwikilink{http://en.wikipedia.org/wiki/Voronoi_diagram}{Voronoi cells},
      one per source.
    \item<2->
      Define \alert{$A(\vec{x})$} as the \alert{area} of the 
      Voronoi cell containing $\vec{x}$.
    \item<3-> As per Stephan's calculation, estimate
      typical distance from $\vec{x}$ to the nearest source (say $i$)
      as 
      $$
      \alertb{c_i A(\vec{x})^{1/2}}
      $$
      where $c_i$ is a shape factor for the $i$th Voronoi cell.
    \item<4-> 
      Approximate $c_i$ as a constant $c$.
    \end{itemize}
  \end{block}

\end{frame}

\begin{frame}
  \frametitle{Size-density law}

  \begin{block}{Carrying on:}
    \begin{itemize}
    \item<1-> The cost function is now
      $$
      F
      =
      c \int_{\Om}
      \alertb{\rhopop(\vec{x})}
      \alertb{ A(\vec{x})^{1/2}}
      \dee{\vec{x}}.
      $$
    \item<2-> We also have that the \alert{constraint} that
      Voronoi cells divide up the overall area
      of $\Om$:
      $
      \sum_{i=1}^{n} A(\vec{x}_i) = A_\Om.
      $
    \item<3-> Sneakily turn this into an integral constraint:
      $$
      \int_\Om
      \frac{\dee{\vec{x}}}
      {A(\vec{x})}
      = n.
      $$
    \item<4->
      Within each cell, $A(\vec{x})$ is constant.
    \item<5->
      So... integral over each of the $n$ cells equals 1.
    \end{itemize}
  \end{block}

\end{frame}

\begin{frame}
  \frametitle{Size-density law}

  \begin{block}{Now a Lagrange multiplier story:}
    \begin{itemize}
    \item<1-> By varying $\{\vec{x}_1,...,\vec{x}_n\}$, minimize
      $$
      G(A) = 
      c \int_{\Om}
      \alertb{\rhopop(\vec{x})}
      \alertb{ A(\vec{x})^{1/2}}
      \dee{\vec{x}}
      -
      \lambda
      \left(n -
        \int_\Om
        \left[A(\vec{x})\right]^{-1}
        \dee{\vec{x}}
      \right)
      $$
    \item<2->
      I Can Haz
      \wordwikilink{http://en.wikipedia.org/wiki/Calculus\_of\_variations}{Calculus of Variations}?
    \item<3->
      Compute
      $\delta G / \delta A$,
      the \wordwikilink{http://en.wikipedia.org/wiki/Functional_derivative}{functional derivative}
      of the functional $G(A)$.
    \item<4-> This gives
      $$
      \int_{\Om}
      \left[
        \frac{c}{2} \alertb{\rhopop(\vec{x})}
        \alertb{ A(\vec{x})^{-1/2}}
        -
        \lambda
        \left[A(\vec{x})\right]^{-2}
      \right]
      \dee{\vec{x}} = 0.
      $$
    \item<5-> Setting the integrand to be zilch, we have:
      $$
      \rhopop(\vec{x})
      =
      2\lambda
      c^{-1}
      A(\vec{x})^{-3/2}.
      $$
    \end{itemize}
  \end{block}

\end{frame}


\begin{frame}
  \frametitle{Size-density law}

  \begin{block}{Now a Lagrange multiplier story:}
    \begin{itemize}
    \item<1-> Rearranging, we have
      $$
      A(\vec{x}) = (2{\lambda} c^{-1})^{2/3} \rhopop^{-2/3}.
      $$
    \item<2->
      Finally, we indentify $1/A(\vec{x})$ as $\rhofac(\vec{x})$,
      an approximation of the local source density.
    \item<3-> Substituting $\rhofac=1/A$, we have
      $$
      \alert{\rhofac(\vec{x})
      = \left( 
        \frac{c}{2{\lambda}}
        \rhopop
    \right)^{2/3}}.
      $$
    \item<4-> Normalizing (or solving for $\lambda$):
      $$
      \alert{\rhofac(\vec{x})}
      =  n 
      \frac{[\rhopop(\vec{x})]^{2/3}}
      {\int_{\Om} [\rhopop(\vec{x})]^{2/3} \dee{\vec{x}}}
      \alert{\propto [\rhopop(\vec{x})]^{2/3}}.
      $$
    \end{itemize}
  \end{block}

\end{frame}

\subsection{Global\ redistribution}

\begin{frame}
  \frametitle{Global redistribution networks}

  \begin{block}<1->{One more thing:}
    \begin{itemize}
    \item<1-> How do we supply these facilities?
    \item<2-> How do we best redistribute mail?  People?
    \item<3-> How do we get beer to the pubs?
    \item<4-> Gaster and Newman model: cost is 
      a function of basic maintenance and travel time:
      $$
      C_{\textnormal{maint}} + \gamma C_{\textnormal{travel}}.
      $$
    \item<5-> Travel time is more complicated:
      Take `distance' between nodes to be a composite
      of shortest path distance $\ell_{ij}$ and 
      number of legs to journey:
      $$
      (1-\delta) \ell_{ij} + \delta (\# \mbox{hops}).
      $$
    \item<6-> When $\delta=1$, only number of hops matters.
      
      
    \end{itemize}
  \end{block}

\end{frame}

\begin{frame}
  \frametitle{Global redistribution networks}

  \includegraphics[width=\textwidth]{gastner2006c_fig6}

  From Gastner and Newman (2006)\cite{gastner2006c}
\end{frame}

\subsection{Public\ versus\ Private}

\begin{frame}
  \frametitle{Public versus private facilities}

  \begin{block}<1->{Beyond minimizing distances:}
    \begin{itemize}
    \item<2->
      ``Scaling laws between population and facility densities'' by
      Um et al., Proc. Natl. Acad. Sci., 2009.\cite{um2009a}
    \item<3->
      Um et al.\ find empirically and argue theoretically that the connection
      between facility and population density
      $$
      \rhofac \propto \rhopop^{\alpha}
      $$
      \alertb{does not universally hold} with $\alpha=2/3$.
    \item<4->
      \alert{Two idealized limiting classes}:
      \begin{enumerate}
      \item<4->
        For-profit, commercial facilities: \alertb{$\alpha = 1$};
      \item<5->
        Pro-social, public facilities: \alertb{$\alpha = 2/3$}.
      \end{enumerate}
    \item<5->
      Um et al.\ investigate facility locations in the United States
      and South Korea.
    \end{itemize}
  \end{block}
  
\end{frame}


\begin{frame}
  \frametitle{Public versus private facilities: evidence}

  \includegraphics[width=0.49\textwidth]{um2009a_fig1A.pdf}
  \includegraphics[width=0.49\textwidth]{um2009a_fig1B.pdf}

  \begin{itemize}
  \item<1->
    \alert{Left plot:} ambulatory hospitals in the U.S.
  \item<1-> 
    \alert{Right plot:} public schools in the U.S.
  \item<2->
    Note: break in scaling for public schools.
    Transition from $\alpha \simeq 2/3$ to 
    $\alpha = 1$ around $\rhopop \simeq 100$.
  \end{itemize}

\end{frame}

\begin{frame}
  \frametitle{Public versus private facilities: evidence}

  \begin{columns}
    \column{0.7\textwidth}
    \includegraphics[width=\textwidth]{um2009a_tab1A.pdf}\\
    \includegraphics[width=\textwidth]{um2009a_tab1B.pdf}
    \column{0.3\textwidth}
      {\small
        Rough \alertb{transition} between public and private at $\alpha \simeq 0.8$.

        \medskip

        Note: * indicates analysis is at state/province level; otherwise county level.}
  \end{columns}

\end{frame}

\begin{frame}
  \frametitle{Public versus private facilities: evidence}

  \includegraphics[width=\textwidth]{um2009a_fig2.pdf}

  \alert{A, C:} ambulatory hospitals in the U.S.;
  \alert{B, D:} public schools in the U.S.;
  \alert{A, B:} data; 
  \alert{C, D:} Voronoi diagram from model simulation.

\end{frame}

\begin{frame}
  \frametitle{Public versus private facilities: the story}

  \begin{block}<1->{So what's going on?}
    \begin{itemize}
    \item<1->
      Social institutions seek to \alertb{minimize distance of travel}.
    \item<2->
      Commercial institutions seek to \alertb{maximize the
      number of visitors}.
    \item<3->
      \alertb{Defns:} For the $i$th facility and its Voronoi cell $V_i$, define
      \begin{itemize}
      \item
        $n_i$ = population of the $i$th cell;
      \item
        $\tavg{r_i}$ = the average travel distance
        to the $i$th facility.
      \item
        $s_i$ = area of $i$th cell.
      \end{itemize}
    \item<4->
      Objective function to maximize for a facility (highly constructed):
      $$ 
      \alertb{v_i = n_i \tavg{r_i}^\beta}
      \
      \mbox{with}
      \
      0 \le \beta \le 1.
      $$
    \item<5->
      Limits:
      \begin{itemize}
      \item $\beta = 0$: purely commercial.
      \item $\beta = 1$: purely social.
      \end{itemize}
    \end{itemize}
  \end{block}

\end{frame}

\begin{frame}
  \frametitle{Public versus private facilities: the story}

  \begin{itemize}
  \item<1-> 
    Proceeding as per the Gastner-Newman-Gusein-Zade calculation,
    Um et al.\ obtain:
    $$
    \alert{\rhofac(\vec{x})}
    =  n
    \frac{[\rhopop(\vec{x})]^{2/(\beta+2)}}
    {\int_{\Om} [\rhopop(\vec{x})]^{2/(\beta+2)} \dee{\vec{x}}}
    \alert{\propto [\rhopop(\vec{x})]^{2/(\beta+2)}}.
    $$
  \item<2-> 
    For $\beta=0$, $\alpha=1$: commercial scaling is linear.
  \item<3-> 
    For $\beta=1$, $\alpha=2/3$: social scaling is sublinear.
  \item<3-> 
    You can try this too: \insertassignmentquestion{04}{4}{3}.
  \end{itemize}

\end{frame}

