\section{Cascade size}

%% idea

\subsection{Distributed seeds}

\begin{frame}
  \frametitle{}
  
\end{frame}

\subsection{Theory}

\begin{frame}
  \frametitle{}
  
\end{frame}


\subsection{Comparison to simulations}

\begin{frame}
  \frametitle{Comparison to data}

  \begin{columns}
    \column{0.5\textwidth}
    \includegraphics[width=\textwidth]{gleeson2007a_fig1.pdf}
    \column{0.5\textwidth}
  \end{columns}
\end{frame}

\begin{frame}
  \frametitle{Comparison to data}

  \begin{columns}
    \column{0.5\textwidth}
    \includegraphics[width=\textwidth]{gleeson2007a_fig2.pdf}
    \column{0.5\textwidth}
  \end{columns}
\end{frame}

\section{Groups}

\begin{frame}
  \frametitle{Extensions}

  \begin{itemize}
  \item<1-> Assumption of sparse interactions is good
  \item<2-> Degree distribution is (generally) key to a network's function
  \item<3-> Still, random networks don't represent all networks
  \item<4-> Major element missing: \alert{group structure}
  \end{itemize}

\end{frame}

\begin{frame}
  \frametitle{Group structure---Ramified random networks}
 
   \centering
   \includegraphics[width=0.48\textwidth]{ramifiednetwork}
 
   $p$ = intergroup connection probability\\
   $q$ = intragroup connection probability.
 
\end{frame}

\begin{frame}
  \frametitle{Bipartite networks}
 
   \centering
   \includegraphics[height=0.75\textheight]{bipartite}
 
   % boards of directors
   % movies
   % transportation
 
 
 
 
\end{frame}

\begin{frame}
  \frametitle{Context distance}
 
   \centering
   \includegraphics[width=1\textwidth]{bipartite2}
  
\end{frame}

\begin{frame}
  \frametitle{Generalized affiliation model}
 
   \centering
   \includegraphics[width=1\textwidth]{generalcontext2}
 
   (Blau \& Schwartz, Simmel, Breiger)
 
 
 % \includegraphics[width=0.6\textwidth]{figgroupcascade_good}
 
 %% cascade windows for group based networks

\end{frame}

\begin{frame}
  \frametitle{Generalized affiliation model networks with triadic closure}

  \begin{itemize}
  \item<1-> Connect nodes with probability $\propto \exp^{-\alpha d}$\\
  where\\
  $\alpha$ = homophily parameter\\
  and \\
  $d$ = distance between nodes (height of lowest common ancestor)
  \item<2->
  $\tau_1$ = intergroup probability of friend-of-friend connection
  \item<3->
  $\tau_2$ = intragroup probability of friend-of-friend connection
  \end{itemize}
 
\end{frame}

\begin{frame}
  \frametitle{Cascade windows for group-based networks}

  \includegraphics[width=1\textwidth]{figgroupcascade3}\\

\end{frame}

%%  
%% %%   
%%
%% \begin{frame}
%%   \frametitle{Starting cascades with seed groups---group-based networks}
%% 
%%   \centering
%%   \includegraphics[width=0.4\textwidth]{fignetwork_ramify_cw21_3b_noname}\\
%%   \includegraphics[width=0.4\textwidth]{fignetwork_multibip_cw03_3_noname}
%% 
%%   
%% 
%%   
%% \end{frame}
%% %%
%% \begin{frame}
%%   \frametitle{Starting cascades with seed groups---group-based networks}
%%   
%%   \begin{tabular}{cc}
%%     \includegraphics[width=0.4\textwidth]{fignetwork_ramify_cw21_5b_noname} &
%%     \includegraphics[width=0.4\textwidth]{fignetwork_ramify_cw22_4b_noname} \\
%%     \includegraphics[width=0.4\textwidth]{fignetwork_multibip_cw03_5_noname} & 
%%     \includegraphics[width=0.4\textwidth]{fignetwork_multibip_cw03_4_noname}
%%   \end{tabular}
%% 

\begin{frame}
  \frametitle{Assortativity in group-based networks}

  \centering

  \includegraphics[width=0.6\textwidth]{fignw_thr_ramify_startprob211_mod2_1_noname}

  \begin{itemize}
  \item<2-> Very surprising: the most connected nodes aren't always the most influential
  \item<3-> \alert{Assortativity} is the reason
  \end{itemize}
  
\end{frame}

\begin{frame}
  \frametitle{Social contagion}

  \begin{block}{Summary}
    \begin{itemize}

    \item<1-> \alert{`Influential vulnerables'} are key to spread.
    \item<2-> Early adopters are mostly vulnerables.
    \item<3-> Vulnerable nodes important but not necessary.
    \item<4-> Groups may greatly facilitate spread.
    \item<5-> Seems that cascade condition is a global one.
    \item<6-> Most extreme/unexpected cascades occur in highly connected networks 
    \item<7-> `Influentials' are posterior constructs.\\
    \item<8-> Many potential influentials exist.
    \end{itemize}
  \end{block}
  
\end{frame}

%% \begin{frame}
%%   \frametitle{Summary}
%% 
%%   \begin{itemize}
%% %  \item<1-> Cascade initiators not greatly above average.
%% %  \item<2-> Average initiators easier to find and influence.
%% %  \item<4-> Early adopters may be above or below average.
%%   \end{itemize}
%% 
%% \end{frame}

\begin{frame}
  \frametitle{Social contagion}

  \begin{block}{Implications}
  \begin{itemize}
  \item<1->
    Focus on \tc{blue}{the influential vulnerables}.
  \item<2->
    Create entities that can be transmitted successfully
    through many individuals rather than broadcast from one `influential.'
  \item<3->
    Only \tc{blue}{simple ideas} can spread by word-of-mouth.\\
    \qquad (Idea of opinion leaders spreads well...)
  \item<4->
    Want enough individuals who will adopt and display.
  \item<5->
    Displaying can be \tc{blue}{passive} = free (yo-yo's, fashion),\\
    or \tc{blue}{active} = harder to achieve (political messages).
  \item<6->
    Entities can be novel or designed to combine with others,
    e.g. block another one.
  \end{itemize}
  \end{block}

\end{frame}


%% \begin{frame}
%%    \frametitle{Social Sciences: Threshold models}
                                %
                                %  At time $t+1$, fraction rioting
                                %  = fraction with $\gamma \le \phi_t$.
                                %
                                %  \[ \phi_{t+1} = \int_{0}^{\phi_t} f(\gamma) \dee{\gamma}
                                %  = \left. F(\gamma) \right|_{0}^{\phi_t} = F(\phi_t) \]
                                %
                                %  $\Rightarrow$ Iterative maps of the interval.
                                %

                                %
                                %  \end{frame}
%%
%% \begin{frame}
%%    \frametitle{Social Sciences: Threshold models}
                                %
                                %  Distribution of individual thresholds, $f(\gamma)$ $\Rightarrow$ interval maps
                                %  \includegraphics{[],figthreshold2_noname.ps,width=0.45\textwidth}
                                %  \includegraphics{[],figthresholdF2b_noname.ps,width=0.45\textwidth}\\
                                %  $\Rightarrow$ Single stable state model
                                %


                                % \end{frame}

%% \begin{frame}
%%   \frametitle{Social Sciences---Threshold models}
%% 
%%   \includegraphics[width=0.45\textwidth]{figthreshold3_noname}
%%   \includegraphics[width=0.45\textwidth]{figthresholdF3b_noname}
%% 
%%   Critical mass model
%% 
%% \end{frame}
%% 




                                %
  %%
  %% \begin{frame}
  %%    \frametitle{Cascade windows for group-based networks}
                                %
                                %  \centering
                                %  \includegraphics[width=0.65\textwidth]{fignetwork_ud_meanfield03_pdd_noname}
                                %
                                %


                                %  \end{frame}

%% \begin{frame}
%%   \frametitle{Starting cascades with seed groups---random networks}
%% 
%%   \centering
%%   \includegraphics[width=0.6\textwidth]{fignw_thr_ramify_startprob211_mod2_1_noname}
%% 
%% 
%% 
%% 
%% 
%% \end{frame}





                                % definition of model
                                % figure



%%   \caption{
%%     Examples of networks where (A) cascades are possible
%%     even when no vulnerable cluster of nonzero fractional
%%     size exists, and (C) activation spreads at an
%%     exponential rate also in the absence of a vulnerable
%%     cluster.  In both networks, all nodes have the
%%     same threshold $\phi=1/3$.  
%%     In plot A, the initiator $i_0$ and $i_0$'s 
%%     two neighbors are the only vulnerable nodes
%%     in the network.  The ladder then allows
%%     activation to propagate from left to right, and the cascade
%%     grows at a linear rate.
%%     The network shown in plot C is a renormalized
%%     version of a trivalent Bethe Lattice, which is shown in plot B.
%%     Each node of the Bethe lattice is replaced by two nodes and additional edges
%%     as shown.  Activation in the resulting network will spread providing
%%     both nodes in a single group are initially activated together,
%%     and the rate of spread will be exponential.


