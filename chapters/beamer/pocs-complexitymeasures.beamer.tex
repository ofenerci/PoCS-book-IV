Add sections on

Computational Irreducibility

Add this to modelling section.

Prediction.

Add section on information?

\section{Measures\ of\ Complexity}

\begin{frame}
  \frametitle{Measures of Complexity}

  \alertb{How do we measure the complexity of a system?}

  \medskip

  (1) \alert{Entropy}: number of microstates that could underlie
  a particular macrostate.

  \medskip

  \begin{itemize}
  \item<1-> Used in information theory and statistical mechanics/thermodynamics.
    %% shannon and von neumann story
    %% call it entropy because no one understands that...
  \item<2-> Measures how uncertain we are about the details of a system.
  \item<3-> Problem: Randomness maximizes entropy, perfect order minimizes.
  \item<4->Our idea of `maximal complexity' is somewhere in between...
  \end{itemize}

\end{frame}

\begin{frame}
  \frametitle{Hmmm}

  \alertb{(Aside)}

  \bigskip

  What about entropy and self-organization?

  \bigskip

  \visible<2->{
    Isn't entropy supposed to always increase?
  }

\end{frame}

\begin{frame}
  \frametitle{Hmmm}

  Two ways for order to appear in a system
  without offending the second law of
  thermodynamics:

  \bigskip

  \visible<2->{
    \alertb{(1)} Entropy of the system decreases at
    the expense of entropy increasing in the
    environment.
  }

  \bigskip

  \visible<3->{
    \alertb{(2)} The system becomes more ordered macroscopically
    while becoming more disordered microscopically.
  }

\end{frame}

\begin{frame}
  \frametitle{Measures of Complexity}

  \begin{block}{(2) Various kinds of information complexity: }
    \begin{itemize}
    \item<1-> Roughly, what is the size of a program required to reproduce a 
      string of numbers?
    \item<2-> Again maximized by random strings.
    \item<3-> Very hard to measure.
    \end{itemize}
    
  \end{block}

\end{frame}

\begin{frame}
  \frametitle{Measures of Complexity}

  (3) Variation on (2): what is the size of a program
  required to reproduce members of an ensemble of
  a string of numbers?

  \bigskip

  \visible<2->{
    \alertb{Now: Random strings have very low complexity.}
  }

\end{frame}

\begin{frame}
  \frametitle{Measures of Complexity}

  \alertb{Large problem:} given any one example, how do we know
  what ensemble it belongs to?

  \bigskip

  \visible<2->{ 
    \alertb{One limited solution:} divide the string up into subsequences
    to create an ensemble.

    \bigskip

    \visible<3->{
      See \textit{Complexity} by Badii \& Politi\cite{badii1997a}
    }
  }
  %% ????

\end{frame}

\begin{frame}
  \frametitle{Measures of Complexity}

  \begin{block}{So maybe no one true measure of complexity exists.}

    \medskip

    \visible{
      \alert{Cosma Shalizi:}

      \medskip

      ``Every few months seems to produce another paper proposing yet
      another measure of complexity, generally a quantity which can't be
      computed for anything you'd actually care to know about, if at
      all. These quantities are almost never related to any other variable,
      so they form no part of any theory telling us when or how things get
      complex, and are usually just quantification for quantification's own
      sweet sake.''
    }
  \end{block}

\end{frame}
