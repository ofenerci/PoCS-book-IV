\section{History}

\begin{frame}
  \frametitle{Some problems for people thinking about people?:}
  
  \begin{block}<1->{How are social networks structured?}
    \begin{itemize}
    \item<1-> How do we define connections?
    \item<1-> How do we measure connections?
    \item<1-> (remote sensing, self-reporting)
    \end{itemize}
  \end{block}

  \begin{block}<2->{What about the dynamics of social networks?}
    \begin{itemize}
    \item<2-> How do social networks evolve? 
    \item<2-> How do social movements begin? 
    \item<2-> How does collective problem solving work? 
    \item<2-> How is information transmitted through social networks?
    \end{itemize}
  \end{block}

\end{frame}

\begin{frame}
  \frametitle{Social Search}

  \begin{block}<1->{A small slice of the pie:}
    \begin{itemize}
    \item<1-> 
      \alert{Q.} Can people pass messages between distant individuals 
      using only their existing social connections?
    \item<2->
      \alert{A.} Apparently yes...
    \end{itemize}
  \end{block}

  \begin{block}<3->{Handles:}
    \begin{itemize}
    \item<4-> 
      The Small World Phenomenon
    \item<5->
      or ``Six Degrees of Separation.''
    \end{itemize}
  \end{block}

\end{frame}

\begin{frame}
  \frametitle{The problem}

  \begin{block}{Stanley Milgram et al., late 1960's:}
    \begin{itemize}
    \item Target person worked in Boston as a stockbroker.
    \item 296 senders from Boston and Omaha.
    \item 20\% of senders reached target.
    \item average chain length $\simeq$ 6.5.
    \end{itemize}
  \end{block}

\end{frame}

\begin{frame}
   \frametitle{The problem}

   \begin{columns}
     \column{0.65\textwidth}
     \begin{block}{Lengths of successful chains:}
       \includegraphics[width=\textwidth]{figbasicmilg_noname}
     \end{block}
     \column{0.35\textwidth}
     From Travers and Milgram (1969) in Sociometry:\cite{travers1969a}\\
     \alertb{``An Experimental Study of the Small World Problem.''}
   \end{columns}

\end{frame}

\begin{frame}
  \frametitle{The problem}

  \alertb{Two features}
  characterize a social `Small World':

  \begin{enumerate}
  \item<2-> Short paths exist
  \item[]<2-> and 
  \item<3-> People are good at finding them.
  \end{enumerate}

\end{frame}

\section{An online experiment}


\begin{frame}
  \frametitle{Social Search}


  \begin{block}{Milgram's small world experiment with e-mail\cite{dodds2003b}}
    \begin{center}
      \includegraphics[width=0.8\textwidth]{window2}
    \end{center}
  \end{block}

\end{frame}

\begin{frame}
  \frametitle{Social search---the Columbia experiment}

  \begin{itemize}
  \item<1-> 
    60,000+ participants in 166 countries
  \item<2-> 
    18 targets in 13 countries including
    \begin{itemize}
    \item<3-> 
      a professor at an Ivy League university,\\
    \item<4-> 
      an archival inspector in Estonia,\\
    \item<5-> 
      a technology consultant in India,\\
    \item<6-> 
      a policeman in Australia,\\
    \item[]<7-> 
      and 
    \item<7-> 
      a veterinarian in the Norwegian army.
    \end{itemize}
  \item<8->
    24,000+ chains
  \end{itemize}

\end{frame}

\begin{frame}
  \frametitle{Social search---the Columbia experiment}

  \begin{itemize}
  \item<1->
    Milgram's participation rate was roughly 75\%
  \item<2->
    Email version: Approximately 37\% participation rate.
  \item<3->
    Probability of a chain of length 10 getting through:
    $$.37^{10} \simeq 5 \times 10^{-5}$$
  \item<4->
  $\Rightarrow$ 384 completed chains (1.6\% of all chains).
  \end{itemize}
  
\end{frame}

\begin{frame}
  \frametitle{Social search---the Columbia experiment}

  % results

  \begin{itemize}
  \item<1->  
    Motivation/Incentives/Perception matter.
  \item<2->  
    If target \textit{seems} reachable\\
    $\Rightarrow$ participation more likely.
  \item<3->  
    Small changes in attrition rates\\
    $\Rightarrow$ large changes in completion rates
  \item<4->  
    e.g., $\searrow$ 15\% in attrition rate \\
    $\Rightarrow$ $\nearrow$ 800\% in completion rate
  \end{itemize}

\end{frame}



\begin{frame}
  \frametitle{Social search---the Columbia experiment}

  \begin{block}<1->{Successful chains disproportionately used}
    \begin{itemize}
    \item<2-> 
      weak ties (Granovetter)
    \item<3-> 
      professional ties (34\% vs.\ 13\%)
    \item<4-> 
      ties originating at work/college
    \item<5-> 
      target's work (65\% vs.\ 40\%)
    \end{itemize}
  \end{block}
  
  \begin{block}<6->{\ldots and disproportionately avoided}
    \begin{itemize}
    \item<7-> 
      hubs (8\% vs. 1\%) (+ no evidence of funnels)
    \item<8->
      family/friendship ties (60\% vs. 83\%)
    \end{itemize}
  \end{block}

  \begin{block}<9->{Geography $\rightarrow$ Work}
  \end{block}  

\end{frame}


\begin{frame}
  \frametitle{Social search---the Columbia experiment}


  Senders of successful messages showed\\
  \tc{blue}{little absolute dependency} on
  \begin{itemize}
  \item<1->
    age, gender
  \item<2->
    country of residence
  \item<3-> 
    income
  \item<4-> 
    religion
  \item<5-> 
    relationship to recipient
  \end{itemize}

  \bigskip

  \uncover<6->{
    Range of completion rates for subpopulations: \\
    \mbox{} \hfill 30\% to 40\%
  }

\end{frame}


\begin{frame}
  \frametitle{Social search---the Columbia experiment}

% Age 30-39   39.3%
% Australia     40.0%
% Graduate level education 41.9%
% Gender Male 39.6%
% occupation > 20 counts: mass media 47.0%
% position 
% high school student 31.0%
% college student 32.2%
% retired 32.9%
% nreligion christian 36.2%, buddhism 33.5%, islam 32.3%


% Age 17 or under 32.8%
% Canada 34.7%
% Elementary school 28.3%
% Gender female 37.1%
% occupation > 20 counts: consumer services 29.2%
% position 
% `other' 40%
% specialist/engineer 39.8%
% university student 39.8%
% religion none 40.5%

% 69 countries
% Canada, Italy, France, U.S.
% Australia, Germany, Norway, Finland

% uber sender
% mass media
% > 100k
% graduate
% male

Nevertheless, some weak discrepencies do exist...

\begin{block}<1->{An above average connector:}
  Norwegian, secular male, aged 30-39, earning over \$100K, 
  with graduate level education working in mass media or science,
  who uses relatively weak ties to people
  they met in college or at work.
\end{block}

\begin{block}<2->{A below average connector:}
  Italian, Islamic or Christian female earning less than \$2K,
  with elementary school education and retired,
  who uses strong ties to family members.
\end{block}

\end{frame}

\begin{frame}
  \frametitle{Social search---the Columbia experiment}

  \begin{block}{Mildly bad for continuing chain:}
    choosing recipients because 
    \alert{``they have lots of friends''}
    or because they will 
    \alert{``likely continue the chain.''}
  \end{block}

  \begin{block}<2->{Why:}
    \begin{itemize}
    \item<2-> 
      Specificity important
    \item<3-> 
      Successful links used relevant information.\\
      (e.g. connecting to someone who shares same profession as target.)
    \end{itemize}
  \end{block}

\end{frame}


% \begin{frame}
%  \frametitle{Social search---the Columbia experiment}
%
%  \includegraphics[height=0.86\textheight]{figsw_2_r_invert_all3_mod_noname}
%
%\end{frame}

\begin{frame}
  \frametitle{Social search---the Columbia experiment}
  \begin{block}{Basic results:}
    
    \begin{itemize}
    \item<1->
      $\avg{L} = 4.05$ for all completed chains
    \item<2->
      $L_\ast$ = Estimated `true' median chain length (zero attrition)
    \item<3->
      Intra-country chains: $L_\ast = 5$ 
    \item<4->
      Inter-country chains:
      $L_\ast = 7$ 
    \item<5->
      All chains:
      $L_\ast = 7$ 
    \item<6->
      Milgram:
      $L_\ast \simeq$ 9
    \end{itemize}
  \end{block}

\end{frame}

\section{Previous theoretical work}

\begin{frame}
  \frametitle{Previous work---short paths}
  
  \begin{itemize}
  \item<1->
    Connected \alertb{random networks}
    have short average path lengths:
    $$\tavg{d_{AB}} \sim \log(N)$$
  \item[]<1->
    $N$ = population size,
  \item[]<1->
    $d_{AB}$ = distance between nodes $A$ and $B$.
  \item<2->
  \alert{But: social networks aren't random...}
  \end{itemize}


\end{frame}


\begin{frame}
  \frametitle{Previous work---short paths}

  \begin{columns}
    \column{0.6\textwidth}
    \includegraphics[width=\textwidth]{clustering}
    \column{0.4\textwidth}
    Need \alert{``clustering''} (your friends are likely to know each other):
  \end{columns}

\end{frame}


\begin{frame}
  \frametitle{Non-randomness gives clustering}

  \begin{center}
    \includegraphics[height=0.65\textheight]{lattice3}
  \end{center}

  $d_{AB}=10$ $\rightarrow$ too many long paths.

\end{frame}

\begin{frame}
  \frametitle{Randomness + regularity}

  \begin{center}
    \includegraphics[height=0.65\textheight]{latticeshortcut3}
  \end{center}

  \alert{Now have $d_{AB}=3$}
  \hfill $\tavg{d}$ decreases overall
\end{frame}

\begin{frame}
  \frametitle{Small-world networks}

  Introduced by\\
  Watts and Strogatz (Nature, 1998)\cite{watts1998a}\\
  ``Collective dynamics of `small-world' networks.''

  \begin{block}<2->{Small-world networks were found everywhere:}
    \begin{itemize}
    \item<2-> neural network of C. elegans,
    \item<3-> semantic networks of languages,
    \item<4-> actor collaboration graph,
    \item<5-> food webs,
    \item<6-> social networks of comic book characters,...
    \end{itemize}
  \end{block}

  \begin{block}<7->{Very weak requirements:}
    \begin{itemize}
    \item<7-> \alert{local regularity}
      \uncover<8->{+ random \alertb{short cuts}}
    \end{itemize}
  \end{block}

  
\end{frame}

\begin{frame}
  \frametitle{Toy model}

    \includegraphics[width=\textwidth]{watts1998a_fig1.pdf}

\end{frame}

\begin{frame}
  \frametitle{The structural small-world property}

    \includegraphics[width=\textwidth]{watts1998a_fig2.pdf}

\end{frame}



\begin{frame}
  \frametitle{Previous work---finding short paths}


  But are these short cuts findable?

  \bigskip

  \uncover<2->{\alert{No.}}

  \bigskip

  \uncover<3->{
  Nodes \alertb{cannot} find each other quickly\\ 
  with \alertb{any local search method}.
  }

\end{frame}


\begin{frame}
  \frametitle{Previous work---finding short paths}

  \begin{itemize}
  \item<1-> What can a local search method reasonably use?
  \item<2->  How to find things without a map?
  \item<3-> \alertb{Need some measure of distance between friends
      and the target.}
  \end{itemize}
  
  \bigskip

  \begin{block}<4->{Some possible knowledge:}
    \begin{itemize}
    \item<1-> Target's identity
    \item<1-> Friends' popularity 
    \item<1-> Friends' identities 
    \item<1-> Where message has been 
    \end{itemize}
  \end{block}

\end{frame}

\begin{frame}
  \frametitle{Previous work---finding short paths}

  Jon Kleinberg (Nature, 2000)\cite{kleinberg2000a}\\
   ``Navigation in a small world.''

   \bigskip
   
   \begin{block}<2->{Allowed to vary:}
     \begin{enumerate}
     \item<2-> local search algorithm
     \item[]<3-> and
     \item<3-> network structure.
     \end{enumerate}
   \end{block}

\end{frame}

\begin{frame}
  \frametitle{Previous work---finding short paths}

  \begin{block}<1->{Kleinberg's Network:}
    \begin{enumerate}
    \item<2->
      Start with
      regular d-dimensional cubic lattice.
    \item<3-> 
      Add local links so 
      nodes know all nodes within a distance $q$.
    \item<4->
      Add $m$ short cuts per node.
    \item<5->  
      Connect $i$ to $j$ with probability 
      $$ p_{ij} \propto {d_{ij}}^{-\alpha}. $$
    \end{enumerate}
  \end{block}

  \begin{itemize}
  \item<6-> 
    \alert{$\alpha=0$}: random connections.
  \item<6->  
    \alert{$\alpha$ large}: reinforce local connections.
  \item<6-> 
    \alert{$\alpha=d$}: same number of connections at all scales.
  \end{itemize}


\end{frame}

\begin{frame}
  \frametitle{Previous work---finding short paths}

  \begin{block}{Theoretical optimal search:}
    \begin{itemize}
    \item<1-> 
      ``Greedy'' algorithm.
    \item<2-> 
      Same number of connections at all scales: $\alpha=d$.
    \end{itemize}

    \bigskip
    \visible<3->{
      Search time grows slowly with system size (like $\log^2N$).
      }

 %  For $\alpha \ne d$, polynomial factor $N^\beta$ appears.

    \bigskip
    \visible<4->{
      \alert{But: social networks aren't lattices plus links.}
    }
    
  \end{block}
  

\end{frame}


\begin{frame}
  \frametitle{Previous work---finding short paths}

  \begin{itemize}
  \item<1-> 
    If networks have \alertb{hubs} can 
    also search well: Adamic et al. (2001)\cite{adamic2001a}
    $$ P(k_i) \propto k_i^{-\gamma}$$
    where $k$ = degree of node $i$ (number of friends).
  \item<2->
    Basic idea: get to hubs first\\
    (airline networks).
  \item<3->   
    \alert{But: hubs in social networks are limited.}
\end{itemize}
  
\end{frame}

\section{An improved model}

\begin{frame}
  \frametitle{The problem}

  If there are no hubs and no underlying lattice,
  how can search be efficient?

  \includegraphics[width=0.45\textwidth]{barenetwork}%
  \raisebox{8ex}{\begin{tabular}{l}
      \\
      Which friend of \alertb{a} is closest \\
      to the target \alertb{b}?\\
      \\
      What does `closest' mean?\\
      \\
      What is
      `social distance'?  \\
      \end{tabular}}

\end{frame}


\begin{frame}
  \frametitle{The model}

  One approach: incorporate \alertb{identity}.\\
  \small{(See ``Identity and Search in Social Networks.'' Science, 2002,  Watts, Dodds, and Newman\cite{watts2002b})}

  \bigskip

  \begin{block}<2->{\alertb{Identity is formed from attributes such as:}}
    \begin{itemize}
    \item<2-> 
      Geographic location
    \item<2-> 
      Type of employment
    \item<2-> 
      Religious beliefs
    \item<2-> 
      Recreational activities.
    \end{itemize}
  \end{block}

  \bigskip

  \uncover<3->{
    \alertb{Groups} are formed by people with at least one similar attribute.
  }

  \bigskip

  \uncover<4->{
    Attributes $\Leftrightarrow$ 
    Contexts $\Leftrightarrow$ 
    Interactions $\Leftrightarrow$ 
    Networks.
  }

\end{frame}

\begin{frame}
  \frametitle{Social distance---Bipartite affiliation networks}

  \centering
  \includegraphics[height=0.75\textheight]{bipartite}

% boards of directors
% movies
% transportation

\end{frame}



\begin{frame}
  \frametitle{Social distance---Context distance}

  \centering
  \includegraphics[width=\textwidth]{bipartite2}

\end{frame}

\begin{frame}
  \frametitle{The model}

  Distance between two individuals $x_{ij}$ 
  is the height of lowest common ancestor.

  \begin{center}
    \includegraphics[width=0.8\textwidth]{fig01_hierarchy_againA}
  \end{center}

  \alertb{$x_{ij}=3$, $x_{ik}=1$, $x_{iv}=4$.}

\end{frame}

\begin{frame}
  \frametitle{The model}

  \begin{itemize}
  \item<1-> 
    Individuals are more
    likely to know each other the closer they are
    within a hierarchy.
  \item<2-> 
    Construct $z$ connections for each node
    using
    \alertb{$$p_{ij} =c\exp\{-\alpha x_{ij}\}.$$}
  \item<3-> 
    \alert{$\alpha=0$}: random connections.
  \item<4-> 
    \alert{$\alpha$ large}: local connections.
  \end{itemize}

\end{frame}


\begin{frame}
  \frametitle{Social distance---Generalized context space}

  \centering
  \includegraphics[width=1\textwidth]{generalcontext2}

  (Blau \& Schwartz, Simmel, Breiger)
\end{frame}

% \begin{frame}
%   \frametitle{The model}
% 
%   Six propositions about social networks:\\
%   (Blau \& Schwartz, Simmel, Breiger)
% 
%   \alert{P1:} Individuals have identities and belong to
%   various groups that reflect these identities.
% 
%   \alert{P2:} Individuals break down
%   the world into a hierarchy of categories.
% 
% \end{frame}
% 
% % \begin{frame}
% %   \frametitle{The model}
% % 
% %   A Geographic example: The United States.
% % 
% %   \alertb{Level 1:} The country.
% % 
% %   \alertb{Level 3:} Regions: South, North East, Midwest, West coast, South West, Alaska.
% % 
% %   \alertb{Level 4:} States within regions\\ (New York, Connecticut, Massachusetts,\ldots).
% % 
% %   \alertb{Level 5:} Cities/areas within States\\ (New York city, Boston, the Berkshires).
% % 
% %   \alertb{Level 6:} Suburbs/towns/smaller cities\\ (Brooklyn, Cambridge).
% %   
% %   \alertb{Level 7:} Neighborhoods\\ (the Village, Harvard Square).
% % \end{frame}
% 

 
% \begin{frame}
%   \frametitle{The model}
% 
%   \alert{P4:}  Each attribute
%   of identity $\equiv$ hierarchy.
% \end{frame}


\begin{frame}
  \frametitle{The model}

  \begin{center}
    \includegraphics[width=\textwidth]{fig01_hierarchy_againD}
  \end{center}

  \begin{center}

    $\vec{v}_i = [ 1 \  1 \ 1 ]^T$, $\vec{v}_j = [ 8 \ 4 \ 1]^T$ \hfill
    Social distance:\\
    \alertb{$x^1_{ij} = 4$, \ $x^2_{ij} = 3$, \ $x^3_{ij} = 1$.}
    \hfill
    $ \boxed{y_{ij} = \min_h x^h_{ij}.} $

  \end{center}

\end{frame}

% \begin{frame}
%   \frametitle{The model}
% 
%   \alert{P5:}   ``Social distance'' is the minimum distance
%   between two nodes in all hierarchies.
% 
%   $$ \boxed{y_{ij} = \min_h x^h_{ij}.} $$
% 
% \vfill
% 
%   Previous slide:
%   \begin{center}
%     
%     \alertb{$x^1_{ij} = 4$, \ $x^2_{ij} = 3$, \ $x^3_{ij} = 1$.}
% 
%     $\Rightarrow  y_{ij} = 1$.
% 
%   \end{center}
% 
% \end{frame}


\begin{frame}
  \frametitle{The model}

  Triangle inequality doesn't hold:

  \begin{center}
    \includegraphics[width=1\textwidth]{fig01_hierarchy_againE}
  \end{center}

  \begin{center}
    \alert{$y_{ik} = 4 > y_{ij} + y_{jk} = 1 + 1 = 2.$}
  \end{center}
 \end{frame}


\begin{frame}
  \frametitle{The model}

  \begin{itemize}
  \item<1-> 
    Individuals know the identity
    vectors of
    \begin{enumerate}
    \item<2-> 
      themselves,
    \item<3->  
      their friends,
    \item[]<4->  
      and
    \item<4->  
      the target.
    \end{enumerate}
  \item<5->
    Individuals can estimate the social distance
    between their friends and the target.
  \item<6->
    Use a greedy algorithm + allow searches to fail randomly.
  \end{itemize}
  
\end{frame}


\begin{frame}
   \frametitle{The model-results---searchable networks}
 
   $\alpha=0$ versus $\alpha=2$ for $N \simeq 10^5$:
   \centering
   \includegraphics[height=0.4\textheight]{figHalphavar02ultp_talk2_noname}%
 \raisebox{12ex}{
   \begin{tabular}{l}
   \alertb{$q \ge r$} \\
   \alert{$q<r$} \\
   $r= 0.05$
 \end{tabular}}

$q$ = probability an arbitrary message
chain reaches a target.

\begin{itemize}
\item<1-> A few dimensions help.\\
\item<1-> Searchability decreases as population increases.\\
\item<1-> Precise form of hierarchy largely doesn't matter.
\end{itemize}

\end{frame}

 
\begin{frame}
  \frametitle{The model-results}

  Milgram's Nebraska-Boston data:

  \begin{columns}
    \column{0.6\textwidth}
    \includegraphics[width=\textwidth]{figmilgram_talk_noname}%
    \column{0.4\textwidth}
    \begin{block}{Model parameters:}
      \begin{itemize}
      \item<1->
        $N=10^8$, 
      \item<1->
        $z=300$, $g=100$,
      \item<1->
        $b=10$,  
      \item<1->
        $\alpha=1$, $H=2$; 
      \item[]<1->
      \item<1->
        $\tavg{L_{\textnormal{model}}} \simeq 6.7$
      \item<1->
        ${L_{\textnormal{data}}} \simeq 6.5$
      \end{itemize}
    \end{block}
  \end{columns}

\end{frame}

\begin{frame}
  \frametitle{Social search---Data}

  \begin{block}{Adamic and Adar (2003)}
    \begin{itemize}
    \item<1->
      For HP Labs, found probability of connection
      as function of organization distance
      well fit by exponential distribution.
    \item<2->
      Probability of connection as function of
      real distance $\propto 1/r$.
    \end{itemize}
  \end{block}

\end{frame}

\begin{frame}
  \frametitle{Social Search---Real world uses}

  \begin{itemize}
  \item 
  Tags create identities for objects
  \item 
  Website tagging:
  \url{http://www.del.icio.us}
  \item 
  (e.g., Wikipedia)
  \item 
  Photo tagging:
  \url{http://www.flickr.com}
  \item 
  Dynamic creation of metadata
  plus links between information objects.
  \item 
  Folksonomy: collaborative creation of metadata
  \end{itemize}
  
\end{frame}

\begin{frame}
  \frametitle{Social Search---Real world uses}

  \begin{block}{Recommender systems:}
    \begin{itemize}
    \item<1->
      Amazon uses people's actions to build
      effective connections between books.  
    \item<2->
      Conflict between `expert judgments' and\\
      tagging of the hoi polloi.
    \end{itemize}
  \end{block}

%  Q: Does tagging lead to a flat structure or 
%  can we identify categories?  (Community detection.)

  % some information scientists decry tagging
  % as poorly directed

\end{frame}

\begin{frame}

  \frametitle{Conclusions}

  \begin{itemize}
  \item<1->
    Bare networks are typically unsearchable.
  \item<2-> 
    Paths are findable if nodes understand how network is formed.
  \item<3-> 
    Importance of identity (interaction contexts).
  \item<4-> 
    Improved social network models.
  \item<5-> 
    Construction of peer-to-peer networks.
  \item<6-> 
    Construction of searchable information databases.
  \end{itemize}

\end{frame}



