In the beginning, there were no words, but there was the start of a story.


This is a first edition.

The framing of complex systems has often been confusing
but it shouldn't be.

One of our great successes in studying complexity
is fluid mechanics.
The Navier-Stokes equations were formulated in 18??.
and 
From a microscopic description of fluids, we are able
to derive the partial differential equations
\footnote{Our predilection for paying tribute to the
discoverers 
with eponymous laws.
Unfortunately, this is often undercut by Stigler's law~\cite{stigler1980a}}
}

I'll return to fluids later in the book.




Complexify.


A key algorithm is not to see everything as algorithms.

Charlotte is playing with bean seeds.  Dormant algorithms,
waiting to operate in physical space, working with the
laws of physics.

Crossbreeding of plants and animals = searching for
better algorithms.  A coarse, rough method for combining
growing/building/creating algorithms.



\section{Preface}

I want to understand everything.

But my time is limited.

Intellectually, I suppose I am by nature a kind of explorer.
I'm not a charlatan, or a snake oil salesman.  You will find them in the
uncharted regions of science.

I like the frontier.

There are the ancient questions many have asked and many will ask: 
Why is anything here?  Why is there a here, why is there a now?
Why is the universe here?  Why am I here?  What am I?  What is the meaning of life?

But I think these questions are not all eternal.  

We are reaching



My motivation in creating this narrative, 


An application more than a book, an app in the current parlance.


\section{Introduction}


\subsection{The Golden Age of Reductionism}


\subsection{The Complexity Manifesto}


Words wash around the scaffolds of meaning, of truth,
of what's possible, what happened, what makes sense.
Find the shape, sometimes erode, change,
sometimes just find the shape.




Physics?
Higher level, complex phenomena is more algorithmic in nature.

Michael Brenner's story on water droplet formation
and cell tranfer.

There are no if-then statements in simple physics.
There is not if-then in $e=mc^2$, no regular
expression pattern matching in the Navier-Stokes equations.

We created computers using physics allowing us to, like biology,
abstract and lift away from raw physics.


\usepackage{lipsum}
\usepackage[framemethod=default]{mdframed}

\newmdenv[linecolor=red,backgroundcolor=yellow]{myframe}

\begin{document}

\begin{myframe}
\lipsum[1]
\end{myframe}




Code that builds its own code.

Code that is modular and evolvable.

Social pheone


Appify


Why Complexify?


Physicists have
long sort the Grand Unified Theory of Everything.

History of this.

Einstein.

And it would be grand to have, but it would
tell us nothing about anything interesting.

evolution, about intelligence,
the life

The whole is greater than the sum of the parts.

More is different.



water droplet -> hurricane
dollar bill -> financial collapse 
...
whole is more than the sum of the parts,
and moreover often fundamentally different from each part.
we love homunculus stories.

\section{The Big Transitions}

\subsection{Big Life}

\subsection{Big Algorithms}

\section{Fluid Mechanics}

\section{Universality and Accidents of History, Inevitability versus Randomness}

\section{Statistical Mechanics}


Information

\section{}

The rise of algorithms, life.

Algorithms.

Manners are about algorithms.

\section{The meaning of life}

We create meaning.


\section{Social spreading}

Anecdotes about failures to see future talent.

Why do we have such poor intuition?


\section{The Unreasonable Ineffectiveness of Mathematicians in the Natural Sciences}

It's about mechanisms.  

\cite{wigner1960a}

\section{Why collective cooperation is such a mystery}

\todo{Lede}

In 19XX, the game theory first ...

\todo{History}

economists 
have Prisoner's Dilemma to 


Theory and methodological approaches.

If you have a favorite theory, one that
you treasure, that has beautiful internal
consistency, the last thing you should ever
do is test it against reality.

The list of ideas about nature that have
proved to be wrong is vast and spans
the ludicrous to the commonsensical.

So where do theories come from?
In a data free environment?
From what other people have said in the past?
Because somehow we gloriously intuit
the truths of the universe while sitting
at a desk?  


The correct models of the real
world that do survive are adapted
and refined.

Some ideas about the world move in
and out of the right and wrong sections.
Lamarck has seen a resurgence of rightness
first at the level of bacteria~\cite{bacteria}
and more recently for people.

Darwin

Science is never purely moving along,
sorting out the best descriptions.
There are choices made about what to
study next that depend on past successes
as well as the ambient culture.


And we have religion.  Even allowing
that one major religion may be entirely
correct, then all the others are at best
on the wrong track and at worst wrong.



Nature includes people.



Many mathematical pathways are not human narratives.

\hwlink{http://www.slate.com/blogs/how_not_to_be_wrong/2014/06/03/number_sentences_stephen_colbert_thinks_they_re_silly_they_re_not.html}{Equations
  as number sentences.}
Not well done.  Math struggles with stories.

Numbers numb, letters let.

Equations eek.

Equations eke.





Me: ``Evolution is not just the game of survival, it's about flourishing,
and all points in between.''

Me: ``The game of evolution starts with bare survival but
through inherent creativity, also leads all the way up to flourishing.''
