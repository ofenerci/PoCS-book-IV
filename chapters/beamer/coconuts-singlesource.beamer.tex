\section{Single Source}

%% \subsection{History}
%% 
%% \begin{frame}
%%   \frametitle{History in brief}
%%   
%%   \begin{block}<1->{Seeking optimal universality:}
%%     \begin{itemize}
%%     \item<1-> Two major real-world branching networks
%%       \begin{enumerate}
%%       \item \alert{Blood networks}
%%       \item \alertb{River networks}
%%       \end{enumerate}
%%     \item<2-> Blood networks argued
%%       to lead to 
%%       $$B \propto M^{\alpha}$$
%%       where
%%       \begin{itemize}
%%       \item 
%%         $B$ = basal metabolic rate and $M$ = body mass
%%       \item 
%%       $\alpha=2/3$ or $3/4$ or something else...\cite{kleiber1961a,west1997a,banavar1999a,dodds2001d}
%%       \end{itemize}
%%     \item<3-> River basins may or may not scale allometrically.
%%     \item<4->
%%       Recall: Hack's law\cite{hack1957a}
%%       $$\ell \propto a^h$$
%%       If $h>1/2$ then basins elongate.
%%     \end{itemize}
%%     
%%   \end{block}
%%   
%% \end{frame}
%% 
%% \subsection{Minimal volume calculation}
%% 
%% \begin{frame}
%%   \frametitle{Geometric argument}
%% 
%%   \begin{itemize}
%%   \item<1-> 
%%     Consider \alert{one source supplying many sinks} in a $d$ dimensional volume
%%   \item<2->
%%     Material draw by sinks is invariant.
%%   \item<3-> 
%%     See network as a bundle of virtual vessels:
%%     \begin{overprint}
%%       \onslide<1-2>        
%%       \onslide<3->        
%%       \begin{center}
%%         \includegraphics[angle=-90,width=0.8\textwidth]{virtualvessels4.pdf}
%%       \end{center}
%%       \end{overprint}
%%   \item<4-> 
%%     \alert{The simplest question}: how does number  of sustainable
%%     sinks $N_{\textnormal{sinks}}$
%%     scale with volume $V$ for the most efficient network design?
%%   \item<5-> 
%%     Or: what is highest $\alpha$ for $N_{\textnormal{sinks}} \propto V^{\alpha}$?
%%   \item<6-> 
%%     Covered in PoCS CSYS 300: we will recap and refine here.
%%   \end{itemize}
%% 
%% %  \item<3-> 
%% %    Assume some cap on flow speed of material, $v_{\textnormal{max}}$
%% 
%% \end{frame}
%% 
%% \begin{frame}
%%   \frametitle{Geometric argument}
%% 
%%   \begin{itemize}
%%   \item<1-> Consider families of systems that grow allometrically.
%%   \item<2-> Family = a basic shape $\Omega$ indexed by volume $V$.
%%   \begin{center}
%%     \includegraphics[angle=-90,width=0.8\textwidth]{shapescaling}    
%%   \end{center}
%%   \bigskip
%%   \item<3-> Orient shape to have dimensions $L_1 \times L_2 \times  ... \times L_d$
%%   \item<4-> In 2-d,
%%     $L_1 \propto A^{\gamma_1}$ and $L_2 \propto A^{\gamma_2}$
%%     where $A$ = area.
%%   \item<5-> In general, have $d$ lengths which scale
%%     as $L_i \propto V^{\gamma_i}$.
%%   \item<6-> For above example, width grows faster than
%%     height: $\gamma_1 > \gamma_2$.
%%   \end{itemize}
%% 
%% \end{frame}
%% 
%% \begin{frame}
%%   \frametitle{Geometric argument}
%% 
%%   \begin{block}<1->{Some generality:}
%%     \begin{itemize}
%%     \item<1-> Consider $d$ dimensional spatial regions living in 
%%       $D$ dimensional ambient spaces.  \uncover<2->{Notation: \alert{$\volume{V}$}.}
%%     \item<3-> River networks: \alertb{$d=2$ and $D=3$}
%%     \item<4-> Cardiovascular networks: \alertb{$d=3$ and $D=3$}
%%     \item<5->
%%       \alert{Star-convexity of $\volume{V}$:} A spatial
%%       region is star-convex if from at least one point, all other
%%       points in the region can be reached by travelling along straight lines
%%       while remaining within the region.
%%     \item<6->
%%       Assume source can be located at a point which has direct line of
%%       sight to all sources.
%%     \item<7->
%%       We can generalize to a much broader class of shapes...
%%     \end{itemize}
%%     
%%   \end{block}
%% 
%% \end{frame}
%% 
%% 
%% \begin{frame}
%%   \frametitle{Geometric argument}
%% 
%%   \begin{itemize}
%%   \item<1-> Reminder of best and worst configurations
%%     \begin{center}
%%       \includegraphics[angle=-90,width=0.8\textwidth]{efficientnetworks5.pdf}
%%     \end{center}
%%     \bigskip
%%   \item<2-> \alert{Basic idea:}
%%     Minimum volume of material in system $V_{\textnormal{net}} \propto$ sum of distance
%%     from the source to the sinks.
%%   \item<3-> See what this means for sink density $\rho$ if sinks do not
%%     change their feeding habits with overall size.
%%   \end{itemize}
%% 
%% \end{frame}


\begin{frame}
  \frametitle{Geometric argument}

  \begin{itemize}
  \item<1-> 
    Consider \alert{one source} supplying \alert{many sinks} in a \alertb{$d$-dim.} volume
    in a \alertb{$D$-dim.} ambient space.
  \item<1->
    Assume \alertb{sinks are invariant}.
  \item<1->
    Assume \alert{$\rho = \rho(V)$}.
  \item<2-> 
    See network as a bundle of virtual vessels:
    \begin{center}
      \begin{overprint}
        \onslide<1 | handout:0| trans:0>
        \onslide<2-| handout:1| trans:1>
        \includegraphics[angle=-90,width=0.8\textwidth]{virtualvessels4.pdf}
      \end{overprint}
    \end{center}
  \item<3-> 
    \alert{Q:} how does the number of sustainable
    sinks $N_{\textnormal{sinks}}$
    scale with volume $V$ for the most efficient network design?
  \item<4-> 
    \alert{Or:} what is the highest $\alpha$ for $N_{\textnormal{sinks}} \propto V^{\alpha}$?
  \end{itemize}

\end{frame}

\begin{frame}
  \frametitle{Geometric argument}

  \begin{itemize}
  \item<1-> Allometrically growing regions:
%  \item<2-> Family = a basic shape $\Omega$ indexed by volume $V$.
  \begin{center}
    \includegraphics[angle=-90,width=0.8\textwidth]{shapescaling}    
  \end{center}
  \bigskip
%  \item<3-> Orient shape to have dimensions $L_1 \times L_2 \times  ... \times L_d$
%  \item<4-> In 2-d,
%    $L_1 \propto A^{\gamma_1}$ and $L_2 \propto A^{\gamma_2}$
%    where $A$ = area.
  \item<1-> Have $d$ length scales which scale
    as 
    {
      $$
      \alertb{L_i} \propto \alertb{V}^{\alertb{\gamma_i}}
      \mbox{\ where $\gamma_1 + \gamma_2 + \ldots + \gamma_d = 1$.}
      $$
    }
  \item<1-> 
    For \alert{isometric} growth, $\gamma_i = 1/d$.
  \item<1->
    For \alert{allometric} growth, 
    we must have at least two of the $\{\gamma_i\}$ being different
%  \item<6-> For above example, width grows faster than
%    height: $\gamma_1 > \gamma_2$.
  \end{itemize}

\end{frame}


\begin{frame}
  \frametitle{Geometric argument}

  \begin{itemize}
  \item<1-> Best and worst configurations (Banavar et al.)
    \begin{center}
      \includegraphics[angle=-90,width=0.8\textwidth]{efficientnetworks5.pdf}
    \end{center}
    \bigskip
  \item<2-> \alert{Rather obviously:}\\
    $\min V_{\textnormal{net}} \propto \sum$
    distances
    from source to sinks.

%  \item<3-> See what this means for sink density $\rho$ if sinks do not
%    change their feeding habits with overall size.
  \end{itemize}

\end{frame}

\begin{frame}
  \frametitle{Minimal network volume:}

  Real supply networks are close to optimal:

  \includegraphics[width=\textwidth]{gastner2006a_fig1.pdf}

  \bigskip

  {\small (2006)
    Gastner and Newman\cite{gastner2006a}:
    ``Shape and efficiency in spatial distribution networks'' }

\end{frame}

\begin{frame}
  \frametitle{Minimal network volume:}

  \begin{block}{Add one more element:}
    \begin{itemize}
    \item Vessel cross-sectional area
      may vary with distance from the source.
    \item
      Flow rate increases as cross-sectional area decreases.
    \item e.g., a collection network may
      have vessels tapering as they approach
      the central sink.
    \item
      Find that vessel volume $v$ must scale
      with vessel length $\ell$ to affect overall
      system scalings.
    \item
      Consider vessel radius $r \propto (\ell+1)^{-\epsilon}$,
      tapering from $r=r_{\max}$ where $\epsilon \ge 0$.
    \item
      Gives
      $
      v \propto \ell^{1-2\epsilon}
      $ if $\epsilon < 1/2$
    \item
      Gives
      $
      v \propto 1 - \ell^{-(2\epsilon-1)} \rightarrow 1$ for large $\ell$
      if $\epsilon > 1/2$
    \item
      Previously, we looked at $\epsilon=0$ only.
    \end{itemize}
  \end{block}
\end{frame}

\begin{frame}
  \frametitle{Minimal network volume:}

  For $0 \le \epsilon < 1/2$, approximate network volume by integral over region:
  $$ 
  \alertb{\min V_{\textnormal{net}}}  \propto 
  \int_{\volume{V}} \alertb{\rho} \, ||\vec{x}||^{1-2\epsilon} \, \dee{\vec{x}} 
  $$
  %%   \visible<2->{
  %%     $$
  %%     \rightarrow 
  %%     \rho V^{1+\gamma_{\max}}
  %%     \int_{\volume{c}} (c_1^{2} u_1^2 + \ldots + c_k^{2} u_k^2 )^{(1-2\epsilon)/2}
  %%     \dee{\vec{u}}
  %%     $$
  %%   }
  \visible<2->{\insertassignmentquestionsoft{02}{2}}
  \visible<2->{
    $$
    \propto
    \alert{ \rho V^{1+\gamma_{\max}(1-2\epsilon)} } 
    $$
  }
  \visible<3->{
    For $\epsilon > 1/2$, find simply that 
    $$
    \alertb{\min V_{\textnormal{net}}}  
    \propto 
    \rho V
    $$
  }
  \begin{itemize}
  \item<4->
    So if supply lines can taper fast enough and without
    limit, minimum network volume can be made negligible.
%  \item<5->
%    \alert{The problem:} must eventually reach a limiting speed
%    or size (e.g., blood velocity and cells).
  \end{itemize}
\end{frame}

\begin{frame}
  \frametitle{Geometric argument}

  \begin{block}{For $0 \le \epsilon < 1/2$:}
    \begin{itemize}
    \item<1-> 
      $
      \boxed{\alert{
          \min V_{\textnormal{net}} 
          \propto
          \rho V^{1+\gamma_{\max}(1-2\epsilon)} 
        }}
      $
    \item<2-> 
      If scaling is \alertb{isometric}, we have $\gamma_{\max} = 1/d$:
      $$
      \min V_{\textnormal{net/iso}} 
      \propto
      \rho V^{1+(1-2\epsilon)/d}
      $$
    \item<3-> 
      If scaling is \alertb{allometric}, we have
      $\gamma_{\max} = \gamma_{\textnormal{allo}} > 1/d$:
      and 
      $$
      \min V_{\textnormal{net/allo}} 
      \propto
      \rho V^{1+(1-2\epsilon)\gamma_{\textnormal{allo}}}
      $$
    \item<4-> 
      Isometrically growing volumes 
      \alert{require less network volume} 
      than allometrically growing volumes:
      $$
      \frac{\min V_{\textnormal{net/iso}}}{\min V_{\textnormal{net/allo}}} \rightarrow 0 
      \mbox{\ as $V \rightarrow \infty$}
      $$
    \end{itemize}    
  \end{block}
\end{frame}

\begin{frame}
  \frametitle{Geometric argument}

  \begin{block}{For $0 \le \epsilon < 1/2$:}
    \begin{itemize}
    \item<1-> 
      $
      \boxed{\alert{
          \min V_{\textnormal{net}} 
          \propto
          \rho V
        }}
      $
    \item<2-> 
      Network volume scaling is now independent 
      of overall shape scaling.
    \end{itemize}
  \end{block}

  \medskip

  \begin{block}<3->{Limits to scaling}
    \begin{itemize}
    \item 
      Can argue that $\epsilon$ must effectively be 0
      for real networks over large enough scales.
    \item 
      Limit to how fast material can move,
      and how small material packages can be.
    \item 
      e.g., blood velocity and blood cell size.
    \end{itemize}
  \end{block}
\end{frame}


\subsection{Blood networks}

\begin{frame}
  \frametitle{Blood networks}

  \begin{itemize}
  \item<1-> Velocity at capillaries and 
    aorta approximately constant across body size\cite{weinstein2006a}: 
    $\epsilon = 0$.
  \item<2-> \alert{Material costly} $\Rightarrow$ expect lower optimal bound of 
    $V_{\textnormal{net}} \propto \rho V^{(d+1)/d}$ to be followed closely.
  \item<3->
    For cardiovascular networks, \alert{$d=D=3$}.
  \item<4->
    Blood volume scales linearly with blood 
    volume\cite{stahl1967a}, $V_{\textnormal{net}} \propto V$.
  \item<5->
    Sink density must $\therefore$ decrease as volume increases:
    $$
    \alertb{\rho \propto V^{-1/d}}.
    $$
  \item<6->
    Density of suppliable sinks \alert{decreases} with organism size.
  \end{itemize}      

\end{frame}


\begin{frame}
  \frametitle{Blood networks}

  \begin{itemize}
  \item<1-> Then $P$, the rate of overall energy 
    use in $\Omega$, can at most scale with volume as
    $$
    P \propto \rho V 
    \visible<2->{
      \propto \rho \, M
    }
    \visible<3->{
      \propto M^{\, (d-1)/d}
    }
    $$
  \item<4-> 
    For $d=3$ dimensional organisms, we have 
    $$\alertb{\boxed{ P \propto M^{\, 2/3}}}$$
  \item<5-> 
    Including other constraints may raise scaling exponent
    to a higher, less efficient value.
  \end{itemize}    

\end{frame}

\begin{frame}
  \frametitle{Recap:}

  \begin{itemize}
  \item<1-> 
    The exponent $\alpha = 2/3$ works for all birds and
    mammals up to 10--30 kg
  \item<2-> For mammals $>$ 10--30 kg, maybe we have a new scaling regime
  \item<3-> Economos: limb length break in scaling around 20 kg
  \item<4-> White and Seymour, 2005: unhappy with large herbivore measurements.
Find $\alpha \simeq 0.686 \pm 0.014$
  \end{itemize}

\end{frame}

%% \begin{frame}
%%   \frametitle{Prefactor:}
%% 
%%   \begin{block}<1->{Stefan-Boltzmann law:}
%%     \begin{itemize}
%%     \item<1->
%%       $$\diff{E}{t} = \sigma S T^4$$
%%       where $S$ is surface and $T$ is temperature.
%%     \item<2-> 
%%       Very rough estimate of prefactor based on scaling
%%       of normal mammalian body temperature and surface
%%       area $S$:
%%       $$B \simeq 10^5M^{2/3} \mbox{erg/sec}.$$
%%     \item<3->
%%       Measured for $M \leq 10$ kg:
%%       $$B=2.57\times 10^5M^{2/3} \mbox{erg/sec}.$$
%%     \end{itemize}
%%   \end{block}
%% 
%% \end{frame}

\subsection{River networks}

\begin{frame}
  \frametitle{River networks}

  \begin{itemize}
  \item<1-> View river networks as collection networks.
  \item<1-> Many sources and one sink.
  \item<2-> $\epsilon$?
  \item<3-> Assume $\rho$ is constant over time and $\epsilon=0$:
    $$V_{\textnormal{net}} \propto \rho V^{(d+1)/d} = \mbox{constant} \times V^{\, 3/2} $$
  \item<4-> Network volume grows faster than
    basin `volume' (really area).
  \item<5-> \alert{It's all okay:}\\ 
    Landscapes are $d$=2 surfaces living in $D$=3 dimension.
  \item<6->
    Streams can grow not just in width but in depth...
  \item<7->
    If $\epsilon > 0$, $V_{\textnormal{net}}$ will grow more slowly
    but 3/2 appears to be confirmed from real data.
  \end{itemize}

\end{frame}

%% \begin{frame}
%%   \frametitle{Hack's law}
%% 
%%   \begin{itemize}
%%   \item<1-> Volume of water in river network can be calculated 
%%     by adding up basin areas
%%   \item<1-> Flows sum in such a way that 
%%     $$ V_{\textnormal{net}} = \sum_{\mbox{\scriptsize all pixels}} a_{\mbox{\scriptsize pixel $i$}} $$
%%   \item<1-> Hack's law again:
%%     $$
%%     \ell \sim a^{\, h}
%%     $$
%%   \item<1-> 
%%     Can argue     
%%     $$ V_{\textnormal{net}} \propto V_{\textnormal{basin}}^{1+h} = a_{\textnormal{basin}}^{1+h}$$
%%     where 
%%     $h$ is Hack's exponent.
%%   \item<1-> 
%%     $\therefore$ minimal volume calculations gives 
%%     $$
%%     \boxed{
%%       h=1/2
%%     }
%%     $$
%%   \end{itemize}
%% 
%% \end{frame}
%% 
%% \begin{frame}
%%   \frametitle{Real data:}
%% 
%%   \begin{columns}
%%     \column{0.4\textwidth}
%%     \begin{itemize}
%%     \item<1-> Banavar et al.'s approach\cite{banavar1999a} is okay 
%%       because $\rho$ \alertb{really is constant}.
%%     \item<3-> \alert{The irony:} shows optimal basins are isometric
%%     \item<4-> Optimal Hack's law: $\msl \sim a^{h}$ with
%%       $h=1/2$ 
%%     \item<5-> \visible<5->{(Zzzzz)}
%%     \end{itemize}
%%     \column{0.6\textwidth}
%%     \begin{overprint}
%%       \onslide<2-| handout:1| trans:1>
%%       \includegraphics[width=\textwidth]{banavar1999fig2.png}\\
%%       {\small From Banavar et al. (1999)\cite{banavar1999a}}
%%     \end{overprint}
%%   \end{columns}
%% \end{frame}
%% 
%% \begin{frame}
%%   \frametitle{Even better---prefactors match up:}
%% 
%%   \begin{center}
%%     \includegraphics[width=0.8\textwidth]{figwatervolume02_noname.pdf}
%%   \end{center}
%% 
%% \end{frame}

