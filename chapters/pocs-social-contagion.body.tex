


cludegraphics[width=\textwidth]{boy-names-in-us_00.png}
  \includegraphics[width=\textwidth]{boy-names-in-us_01.png}
  \includegraphics[width=\textwidth]{boy-names-in-us_02.png}
  \includegraphics[width=\textwidth]{boy-names-in-us_03.png}
  \includegraphics[width=\textwidth]{boy-names-in-us_04.png}
  \includegraphics[width=\textwidth]{boy-names-in-us_05.png}
  \includegraphics[width=\textwidth]{boy-names-in-us_06.png}
  \includegraphics[width=\textwidth]{boy-names-in-us_07.png}
  \includegraphics[width=\textwidth]{boy-names-in-us_08.png}
  \includegraphics[width=\textwidth]{boy-names-in-us_09.png}
  \includegraphics[width=\textwidth]{boy-names-in-us_10.png}
  \includegraphics[width=\textwidth]{boy-names-in-us_11.png}
  \includegraphics[width=\textwidth]{boy-names-in-us_12.png}
  \includegraphics[width=\textwidth]{boy-names-in-us_13.png}
  \includegraphics[width=\textwidth]{boy-names-in-us_14.png}
  \includegraphics[width=\textwidth]{boy-names-in-us_15.png}
  \includegraphics[width=\textwidth]{boy-names-in-us_16.png}
  \includegraphics[width=\textwidth]{boy-names-in-us_17.png}
  \includegraphics[width=\textwidth]{boy-names-in-us_18.png}
  \includegraphics[width=\textwidth]{boy-names-in-us_19.png}
  \includegraphics[width=\textwidth]{boy-names-in-us_20.png}
  \includegraphics[width=\textwidth]{boy-names-in-us_21.png}
  \includegraphics[width=\textwidth]{boy-names-in-us_22.png}
  \includegraphics[width=\textwidth]{boy-names-in-us_23.png}
  \includegraphics[width=\textwidth]{boy-names-in-us_24.png}
  \includegraphics[width=\textwidth]{boy-names-in-us_25.png}
  \includegraphics[width=\textwidth]{boy-names-in-us_26.png}
  \includegraphics[width=\textwidth]{boy-names-in-us_27.png}
  \includegraphics[width=\textwidth]{boy-names-in-us_28.png}
  \includegraphics[width=\textwidth]{boy-names-in-us_29.png}
  \includegraphics[width=\textwidth]{boy-names-in-us_30.png}
  \includegraphics[width=\textwidth]{boy-names-in-us_31.png}
  \includegraphics[width=\textwidth]{boy-names-in-us_32.png}
  \includegraphics[width=\textwidth]{boy-names-in-us_33.png}
  \includegraphics[width=\textwidth]{boy-names-in-us_34.png}
  \includegraphics[width=\textwidth]{boy-names-in-us_35.png}
  \includegraphics[width=\textwidth]{boy-names-in-us_36.png}
  \includegraphics[width=\textwidth]{boy-names-in-us_37.png}
  \includegraphics[width=\textwidth]{boy-names-in-us_38.png}
  \includegraphics[width=\textwidth]{boy-names-in-us_39.png}
  \includegraphics[width=\textwidth]{boy-names-in-us_40.png}
  \includegraphics[width=\textwidth]{boy-names-in-us_41.png}
  \includegraphics[width=\textwidth]{boy-names-in-us_42.png}
  \includegraphics[width=\textwidth]{boy-names-in-us_43.png}
  \includegraphics[width=\textwidth]{boy-names-in-us_44.png}
  \includegraphics[width=\textwidth]{boy-names-in-us_45.png}
  \includegraphics[width=\textwidth]{boy-names-in-us_46.png}
  \includegraphics[width=\textwidth]{boy-names-in-us_47.png}
  \includegraphics[width=\textwidth]{boy-names-in-us_48.png}
  \includegraphics[width=\textwidth]{boy-names-in-us_49.png}
  \includegraphics[width=\textwidth]{boy-names-in-us_50.png}
  \includegraphics[width=\textwidth]{boy-names-in-us_51.png}
  \includegraphics[width=\textwidth]{boy-names-in-us_52.png}

  \wordwikilink{http://www.theatlantic.com/technology/archive/2013/10/americas-most-popular-boys-names-since-1960-in-1-spectacular-gif/280852/}{From
    the Atlantic}

ames-in-us_00.png}
  \includegraphics[width=\textwidth]{girl-names-in-us_01.png}
  \includegraphics[width=\textwidth]{girl-names-in-us_02.png}
  \includegraphics[width=\textwidth]{girl-names-in-us_03.png}
  \includegraphics[width=\textwidth]{girl-names-in-us_04.png}
  \includegraphics[width=\textwidth]{girl-names-in-us_05.png}
  \includegraphics[width=\textwidth]{girl-names-in-us_06.png}
  \includegraphics[width=\textwidth]{girl-names-in-us_07.png}
  \includegraphics[width=\textwidth]{girl-names-in-us_08.png}
  \includegraphics[width=\textwidth]{girl-names-in-us_09.png}
  \includegraphics[width=\textwidth]{girl-names-in-us_10.png}
  \includegraphics[width=\textwidth]{girl-names-in-us_11.png}
  \includegraphics[width=\textwidth]{girl-names-in-us_12.png}
  \includegraphics[width=\textwidth]{girl-names-in-us_13.png}
  \includegraphics[width=\textwidth]{girl-names-in-us_14.png}
  \includegraphics[width=\textwidth]{girl-names-in-us_15.png}
  \includegraphics[width=\textwidth]{girl-names-in-us_16.png}
  \includegraphics[width=\textwidth]{girl-names-in-us_17.png}
  \includegraphics[width=\textwidth]{girl-names-in-us_18.png}
  \includegraphics[width=\textwidth]{girl-names-in-us_19.png}
  \includegraphics[width=\textwidth]{girl-names-in-us_20.png}
  \includegraphics[width=\textwidth]{girl-names-in-us_21.png}
  \includegraphics[width=\textwidth]{girl-names-in-us_22.png}
  \includegraphics[width=\textwidth]{girl-names-in-us_23.png}
  \includegraphics[width=\textwidth]{girl-names-in-us_24.png}
  \includegraphics[width=\textwidth]{girl-names-in-us_25.png}
  \includegraphics[width=\textwidth]{girl-names-in-us_26.png}
  \includegraphics[width=\textwidth]{girl-names-in-us_27.png}
  \includegraphics[width=\textwidth]{girl-names-in-us_28.png}
  \includegraphics[width=\textwidth]{girl-names-in-us_29.png}
  \includegraphics[width=\textwidth]{girl-names-in-us_30.png}
  \includegraphics[width=\textwidth]{girl-names-in-us_31.png}
  \includegraphics[width=\textwidth]{girl-names-in-us_32.png}
  \includegraphics[width=\textwidth]{girl-names-in-us_33.png}
  \includegraphics[width=\textwidth]{girl-names-in-us_34.png}
  \includegraphics[width=\textwidth]{girl-names-in-us_35.png}
  \includegraphics[width=\textwidth]{girl-names-in-us_36.png}
  \includegraphics[width=\textwidth]{girl-names-in-us_37.png}
  \includegraphics[width=\textwidth]{girl-names-in-us_38.png}
  \includegraphics[width=\textwidth]{girl-names-in-us_39.png}
  \includegraphics[width=\textwidth]{girl-names-in-us_40.png}
  \includegraphics[width=\textwidth]{girl-names-in-us_41.png}
  \includegraphics[width=\textwidth]{girl-names-in-us_42.png}
  \includegraphics[width=\textwidth]{girl-names-in-us_43.png}
  \includegraphics[width=\textwidth]{girl-names-in-us_44.png}
  \includegraphics[width=\textwidth]{girl-names-in-us_45.png}
  \includegraphics[width=\textwidth]{girl-names-in-us_46.png}
  \includegraphics[width=\textwidth]{girl-names-in-us_47.png}
  \includegraphics[width=\textwidth]{girl-names-in-us_48.png}
  \includegraphics[width=\textwidth]{girl-names-in-us_49.png}
  \includegraphics[width=\textwidth]{girl-names-in-us_50.png}
  \includegraphics[width=\textwidth]{girl-names-in-us_51.png}
  \includegraphics[width=\textwidth]{girl-names-in-us_52.png}

  \wordwikilink{http://www.theatlantic.com/technology/archive/2013/10/a-wondrous-gif-shows-the-most-popular-baby-names-for-girls-since-1960/280709/}{From
    the Atlantic}

k{http://www.buzzfeed.com}{buzzfeed.com}:

  \begin{center}
    \includegraphics[width=1.0\textwidth]{2012-10-16buzzfeed-categories.pdf}
  \end{center}

  {
    {\huge
      + News ...
    }
  }

ter}
    \includegraphics[width=0.8\textwidth]{2012-10-16buzzfeed-fail-cropped-tp-2.pdf}

    %%    \includegraphics[width=1.0\textwidth]{2012-10-16buzzfeed-categories.pdf}
  \end{center}



  \includegraphics[width=\textwidth]{enhanced-buzz-7600-1349980225-3.jpg}
{center}
    \includegraphics[height=0.8\textheight]{tattoo-cross}
  \end{center}

ter}
    \includegraphics[height=0.8\textheight]{2000px-Rubiks_cube-tp-1.pdf}
  \end{center}
  
\includegraphics[width=.05\textwidth]{wikipedia-tp-3.pdf}

 imitation
%% I'll Have What She's Having: Mapping Social Behavior 

%% add romero, kleinberg

\section{Social\ Contagion\ Models}


\subsection{Background}

tagion works so well:}

  \begin{center}
    \includegraphics[height=0.8\textheight]{sheeple.png}\\
    \wordwikilink{http://xkcd.com/610/}{http://xkcd.com/610/}
  \end{center}

}

  \includegraphics[width=0.9\textwidth]{044thepumafad_640}

}

  \textbf{Examples abound}
                  
          fashion
          striking
         \wordwikilink{http://content.nejm.org/cgi/content/short/358/21/2249}{smoking}\cite{christakis2008a}
          residential segregation\cite{schelling1971a}
          ipods                                   
         \wordwikilink{http://content.nejm.org/cgi/content/full/357/4/370}{obesity}\cite{christakis2007a}
        
                    
         Harry Potter         
         voting               
         gossip               
         Rubik's cube \includegraphics[height=12pt]{rubiks_cube.jpg}
         religious beliefs    
         \alert{leaving lectures}
        
            

  \textbf{SIR and SIR\alert{S} contagion possible}
    
     Classes of behavior versus specific behavior{: \alert{dieting}}
    
  

g to start spreading
\insertvideo{FEaCfIp9qR4}{}{}{Market much?}


%% faberge shampoo ad
\insertvideo{TgDxWNV4wWY}{}{}{Mixed messages: Please copy, but also, don't copy \ldots}


%% 
\insertlocalvideo{2014-04-16scrubsS1E21contagious-smile-perry-cox.mp4}{}{}{}

gham heart study:}

  \textbf{Evolving network stories (Christakis and Fowler):}
  
   
    The spread of 
    \wordwikilink{http://www.nejm.org/doi/full/10.1056/NEJMsa0706154}{quitting smoking}\cite{christakis2008a}
   
    The spread of 
    \wordwikilink{http://www.nejm.org/doi/full/10.1056/NEJMsa066082}{spreading}\cite{christakis2007a}
   
    Also: \wordwikilink{http://www.bmj.com/content/337/bmj.a2338.full}{happiness}\cite{fowler2008a}, 
    loneliness, ...
  
    The book: 
    \wordwikilink{http://www.amazon.com/Connected-Surprising-Power-Social-Networks/dp/0316036145}{Connected: The Surprising Power of Our Social Networks and How They Shape Our Lives}
  
  

  \textbf{Controversy:}
    
     
    \wordwikilink{http://www.nytimes.com/2009/09/13/magazine/13contagion-t.html?pagewanted=all}{Are your friends making you fat?} (Clive Thomspon, NY Times, September 10, 2009).
    
      \wordwikilink{http://www.slate.com/id/2250102/}{Everything is contagious}---Doubts about the social plague stir in the human superorganism (Dave Johns, Slate, April 8, 2010).
    
  

sert PNAS paper on emotional contagion
%% http://www.pnas.org/content/early/2014/05/29/1320040111

tagion}

  \textbf{Two focuses for us}
    
     Widespread media influence
     Word-of-mouth influence
    
  

  \textbf{We need to understand influence}
    
     Who influences whom?  {Very hard to measure...}
     What kinds of influence response functions are there?
     Are some individuals super influencers?\\
      {Highly popularized by Gladwell\cite{gladwell2000a} as `connectors'}
     The infectious idea of opinion leaders (Katz and Lazarsfeld)\cite{katz1955a}
    
  
ce}

  \begin{center}
    \includegraphics[width=0.8\textwidth]{influence_model_hypodermic}    
  \end{center}


ce\cite{katz1955a}}

  \begin{center}
    \includegraphics[width=0.8\textwidth]{influence_model_twostep}    
  \end{center}

fluence}

  \begin{center}
    \includegraphics[width=0.8\textwidth]{influence_model_general}    
  \end{center}


}

  \textbf{Why do things spread?}
    
     Because of properties of special individuals?
     Or system level properties?
     Is the match that lights the fire important?
     Yes.  But only because we are narrative-making machines...
     We like to think things happened for reasons...
     Reasons for success are usually ascribed to intrinsic
      properties (e.g., Mona Lisa, next slide)
     System/group properties harder to understand
     Always good to examine what is said before and after the fact...
    
  

%%   \textbf{The big deal here}
%%     %%       %%         {Global properties} \\
%%         
%%       %%       %%         Local \\
%%       %%     %%   
%%     \alert{}: Global versus local
%%      System-level propertives versus individual properties
%%      \alert{Difficult to understand}  \alert{Easy to understand}

{center}
    \includegraphics[height=0.5\textheight]{monalisa.pdf}%
    \includegraphics[height=0.5\textheight]{NewYorkerMonaMonica.pdf}
    \includegraphics[height=0.5\textheight]{monalisasimpson.jpg}
    \includegraphics[height=0.5\textheight]{s_monalisa.jpg}
  \end{center}

  
  
  ``Becoming Mona Lisa: The Making of a Global Icon''---David Sassoon
   
  Not the world's greatest painting from the start...
  
  Escalation through theft, vandalism, {\alert{parody}, ...}
  
  
 this!
%% http://www.nytimes.com/2012/10/21/sports/how-armstrongs-wall-fell-one-rider-at-a-time.html?hpw

  \wordwikilink{http://www.nytimes.com/2012/10/18/sports/cycling/inquiry-into-kayle-leogrande-led-to-lance-armstrongs-eventual-fall.html?hp}{`Tattooed Guy' Was Pivotal in Armstrong Case [nytimes]}

  \begin{center}
  \includegraphics[width=\textwidth]{CYCLING-articleLarge.jpg}
  \end{center}

  
  
    ``... Leogrande's doping \alertb{sparked} a series of events ...''
%%   
%%    ``Without Leogrande, who knows, the Armstrong investigation maybe never would have happened,'' 
%%    Tygart said.
  


%%  \begin{center}
%%    \includegraphics[width=0.7\textwidth]{cheat-to-win-bracelet-the-onion.jpg}
%%    \wordwikilink{http://store.theonion.com/p-5045-cheat-to-win-bracelet.aspx}{[the onion]}
%%  \end{center}

 Europe}

  \includegraphics[height=0.7\textheight]{berlinwall.jpg}

  Timur Kuran:\cite{kuran1991a,kuran1997a} 
  ``Now Out of Never: The Element of Surprise in the East European Revolution of 1989''

ter}
    \includegraphics[width=\textwidth]{hp1_posters.jpg}
%    \includegraphics[height=0.5\textheight]{Dumbledore_and_Elder_Wand.jpg}
  \end{center}

  \includegraphics[width=.05\textwidth]{wikipedia.jpg}

gelecturelogo{.18}{sendak1963a-cover.jpg}

{block}{
      From a 2013 \wordwikilink{http://www.believermag.com}{Believer Magazine}
      \wordwikilink{http://www.believermag.com/issues/201211/?read=interview\_sendak}{interview
        with Maurice Sendak}:}
    BLVR: Did the success of Where the Wild Things Are ever feel like an albatross?\\
    \medskip
    {
      MS: It's a nice book. 
    }
    {
      It's perfectly nice. 
    }
    {
      I can't complain about
      it. 
    }
    {
      I remember Herman Melville said, ``When I die no one is going to
      mention Moby-Dick. 
    }
    {
      They're all going to talk about my first book,
      about f$\ast$$\ast$$\ast$ing maidens in Tahiti.'' 
    }
    {
      He was right. 
    }
    {
      No mention of
      Moby-Dick then. 
    }
    {
      Everyone wanted another Tahitian book, a beach
      book. 
    }
    {
      But then he kept writing deeper and deeper and then came
      Moby-Dick and people hated it. 
    }
    {
      The only ones who liked it were Mr. and
      Mrs. Nathaniel Hawthorne. 
    }
    {
      Moby-Dick didn't get famous until 1930.
    }
  

  
  
    Sendak named his dog Herman.
  
    The essential Colbert interview:
    \wordwikilink{http://www.colbertnation.com/the-colbert-report-videos/406796/january-24-2012/grim-colberty-tales-with-maurice-sendak-pt--1}{Pt.\ 1} 
    and 
    \wordwikilink{http://www.colbertnation.com/the-colbert-report-videos/406902/january-25-2012/grim-colberty-tales-with-maurice-sendak-pt--2}{Pt.\ 2}.
  

gelecturelogo{.18}{brady.png}

{block}{
  \wordwikilink{http://www.nytimes.com/interactive/2013/04/25/sports/football/picking-the-best-in-the-nfl-draft.html}{Drafting
    success in the NFL:}
}
  \includegraphics[width=\textwidth]{2013-04-25nytimes-top-players.png}\\
  \medskip
  \includegraphics[width=\textwidth]{2013-04-25nytimes-top-players-example.png}
  

gelecturelogo{.18}{figfullspread_circ5-rot.pdf}

tagion}

  \textbf{Messing with social connections}
    
     
      Ads based on message content \\
      {(e.g., Google and email)}    
     
      \wordwikilink{http://about.bzzagent.com/}{BzzAgent}
     
      One of Facebook's early advertising attempts: 
      \wordwikilink{http://en.wikipedia.org/wiki/Facebook\_Beacon}{Beacon}
     
      All of Facebook's advertising attempts.
    
  

gs for you}

  
    A very good book: \alertb{`Influence'}\cite{cialdini2000a} by 
    \wordwikilink{http://en.wikipedia.org/wiki/Robert\_Cialdini}{Robert Cialdini}
  

  \textbf{Six modes of influence:}
    
     
      \alert{Reciprocation}: \textit{The Old Give and Take... and Take};\\
      \alertg{e.g., Free samples, Hare Krishnas.}
     
      \alert{Commitment and Consistency}: \textit{Hobgoblins of the Mind};
      \alertg{e.g., Hazing.}
     
      \alert{Social Proof}: \textit{Truths Are Us};\\
      \alertg{e.g.,} \wordwikilink{http://en.wikipedia.org/wiki/Jonestown}{Jonestown},\\
      \wordwikilink{http://en.wikipedia.org/wiki/Murder\_of\_Kitty_Genovese}{Kitty Genovese} (contested).
      %% jim jones, 1978, 910 dead
      %% overstated: queens, 38 witness, over half an hour long attack
      %% no one could explain their inaction
     
      \alert{Liking}: \textit{The Friendly Thief};
      \alertg{e.g., Separation into groups is enough to cause problems.}
     
      \alert{Authority}: \textit{Directed Deference};\\
      \alertg{e.g.,} 
      \wordwikilink{http://www.npr.org/templates/story/story.php?storyId=124838091}{Milgram's obedience to authority experiment.}
     
      \alert{Scarcity}: \textit{The Rule of the Few};
      \alertg{e.g., Prohibition.}
    
  

}

  
    
     
      Cialdini's modes are heuristics that help up us get through life.
     
      Useful but can be leveraged...
    
  

  \textbf{Other acts of influence:}
  
   
    Conspicuous Consumption (Veblen, 1912)
   
    Conspicuous Destruction (Potlatch)
  
  
  

}

  \textbf{Some important models:}
    
     
      Tipping models---Schelling (1971)\cite{schelling1971a,schelling1973a,schelling1978a}\\
      
       
        Simulation on checker boards
       
        Idea of thresholds
       
        Explore the \wordwikilink{http://ccl.northwestern.edu/netlogo/}{Netlogo} 
        \wordwikilink{http://ccl.northwestern.edu/netlogo/models/run.cgi?Segregation.734.460}{online implementation}\cite{wilensky1998a}
      
     
      Threshold models---Granovetter (1978)\cite{granovetter1978a}
     
      Herding models---Bikhchandani, Hirschleifer, Welch (1992)\cite{bikhchandani1992a,bikhchandani1998a}
      
      
        Social learning theory, Informational cascades,...
      
    
  

 models}

 \textbf{Thresholds}
   
    Basic idea: individuals adopt a behavior when a
     \alert{certain fraction of others} have adopted
    `Others' may be everyone in a population,
     an individual's close friends, any reference group.
    Response can be probabilistic or deterministic.
    Individual thresholds can vary
    Assumption: order of others' adoption does not matter...
     {\alert{(unrealistic)}.}
    Assumption: level of influence per person is uniform \\
     {\alert{(unrealistic)}.}
   
 

}

  \textbf{Some possible origins of thresholds:}
    
     Inherent, evolution-devised inclination 
      to coordinate, to conform, to imitate.\cite{bentley2011a}
     \alert{Lack of information}:
        impute the worth of a good or behavior
        based on degree of adoption (social proof)
     Economics: \alert{Network effects} or \alert{network externalities}
      
       Externalities = Effects on others not directly involved in a transaction
       Examples: telephones, fax machine, Facebook, operating systems
       An individual's utility increases with
        the adoption level among peers and the population in general
      
    
  

%% qwerty
%% 

%% skateboarder madness
\insertvideo{LL2GQrvbb1k}{}{}{Shareworthy interlude}

\subsection{Granovetter's\ model}

se functions}

  
    \includegraphics[width=0.45\textwidth]{figthreshold_eg1_noname}
    \includegraphics[width=0.45\textwidth]{figthreshold_eg2_noname}\\

    
     Example threshold influence response functions:\\
      \alert{deterministic} and \alert{stochastic}
     $\phi$ = fraction of contacts `on' \alert{(e.g., rioting)}
     Two states: S and I.
    
  

 based on perceived behavior of others:}
    \includegraphics[width=1\textwidth]{figthreshold_noname}
    
     Two states: S and I.
     $\phi$ = fraction of contacts `on' (e.g., rioting)
     Discrete time update (strong assumption!)
     This is a \alert{Critical mass model}
  
  

other example of critical mass model:}
     \begin{center}
     \includegraphics[width=.45\textwidth]{figthreshold3_distribution_noname}
     \includegraphics[width=.45\textwidth]{figthresholdF3b_noname}
     \end{center}
   

gle stable state model:}
    \begin{center}
      \includegraphics[width=.45\textwidth]{figthreshold2_noname}
      \includegraphics[width=.45\textwidth]{figthresholdF2b_noname}
    \end{center}
  

ovetter1986a,granovetter1988a,dodds2013a,harris2013a}}
  \begin{center}
    \includegraphics[width=0.4\textwidth]{figtentmap2_noname} 
    \includegraphics[width=0.4\textwidth]{figtentmap_noname} 
  \end{center}

  
   Period doubling arises as map amplitude $r$ is increased.
   Synchronous update assumption is crucial
  
  

  

s for collective action theory:}
    
     Collective uniformity $\not\Rightarrow$ individual uniformity
     Small individual changes $\Rightarrow$ large global changes
    
  

s
%% 2+. morris?

%%%%%%%%%%%%%% 2 mins
%% 2+. lopez? 

%%%%%%%%%%%%%% 2 mins
%% 2+. kempe, kleinberg

%% !!!!!!!!!!!!!!!!!
%%%%%%%%%%%%%% 2 mins
%% bifurcation in standard
%% disease models
%% heterogeneity of groups



\subsection{Network\ version}

 a network}

  \small
  \textbf{Many years after Granovetter and Soong's work:}
    
     
      ``A simple model of global cascades on random networks''\\      
      D. J. Watts.  Proc. Natl. Acad. Sci., 2002\cite{watts2002a}
      
       
        Mean field model $\rightarrow$ network model
       
        Individuals now have a limited view of the world
      
    
  

  \textbf{We'll also explore:}
    
     
      ``Seed size strongly affects cascades on random networks''\cite{gleeson2007a}\\
      Gleeson and Cahalane, Phys. Rev. E, 2007.
    
      ``Direct, phyiscally motivated derivation of the contagion condition for spreading processes on generalized random networks''\cite{dodds2011b}
      Dodds, Harris, and Payne, Phys. Rev. E, 2011
     
      ``Influentials, Networks, and Public Opinion Formation''\cite{watts2007a}\\
      Watts and Dodds, J. Cons. Res., 2007.
     
      ``Threshold models of Social Influence''\cite{watts2009a}\\
      Watts and Dodds, The Oxford Handbook of Analytical Sociology, 2009.
    
  


etwork}

  
  
   
    Interactions between individuals 
    now represented by a network
   
    Network is \alert{sparse}
   
    Individual $i$ has $k_i$ contacts
   
    Influence on each link is \alert{reciprocal} and of \alert{unit weight}
   
    Each individual $i$ has a fixed threshold $\phi_i$
   
    Individuals repeatedly poll contacts on network
   
    Synchronous, discrete time updating 
   
    Individual $i$ becomes active when\\
    fraction of active contacts \alertb{$\frac{a_i}{k_i} \ge \phi_i $}
   
    Individuals remain active when switched (no recovery = SI model)
  
  

{block}{Word-of-mouth contagion:}
%%  




 a network}

  
    \begin{center}
      \includegraphics[angle=-90,width=1\textwidth]{contagioncondition3a}%
      \includegraphics[angle=-90,width=1\textwidth]{contagioncondition3b}%
      \includegraphics[angle=-90,width=1\textwidth]{contagioncondition3c}%
    \end{center}

    
     All nodes have threshold $\phi=0.2$.
    
  


%% previous work
%% definition of vulnerables
%% global condition

g}

  \textbf{First study random networks:}
    
     Start with $N$ nodes with a degree distribution $P_k$
     Nodes are randomly connected (carefully so)
     Aim: Figure out when activation will propagate
     Determine a \alert{cascade condition}
    
  

  \textbf{The Cascade Condition:}
    
    
      If one individual is initially activated,
      what is the probability that
      an activation will spread over a network?
    
      What features of a network determine whether
      a cascade will occur or not?
    
  

etwork structure:}

      
    \includegraphics[width=\textwidth]{2011-04-04random-network-contagion-sketch_3a-tp-5.pdf}
    
    
     
      $\Omega_{\textrm{crit}}$ = $\Omega_{\textrm{vuln}}$ = critical mass = global vulnerable component
     
      $\Omega_{\textrm{trig}}$ = triggering component
     
      $\Omega_{\textrm{final}}$ = potential extent of spread
     
      $\Omega$ = entire network
    
    \bigskip
  $$
  \Omega_{\textrm{crit}} 
  \subset
  \Omega_{\textrm{trig}};
  \
  \Omega_{\textrm{crit}} 
  \subset
  \Omega_{\textrm{final}};
  \
  \mbox{and}
  \
  \Omega_{\textrm{trig}},
  \Omega_{\textrm{final}} 
  \subset
  \Omega.
  $$
g}

  \textbf{Follow active links}
    
     An active link is a link connected to an activated node.
     If an infected link leads to \alert{at least 1 more infected link},
      then \alert{activation spreads}.
     We need to understand which nodes can be activated when
      only one of their neigbors becomes active.
    
  

erables:}
    
     We call individuals who can be activated by
    just one contact being active \alert{vulnerables}
     The vulnerability condition for node $i$:
      $$1/k_i \ge \phi_i$$
    
      Which means \# contacts  $k_{i} \le \lfloor 1/\phi_i \rfloor$
    
      For global cascades on random networks, must have a
      \alertb{\textit{global cluster of vulnerables}}\cite{watts2002a}
    
      \alert{Cluster of vulnerables} = \alert{critical mass}
    
      Network story: 1 node $\rightarrow$ critical mass $\rightarrow$ everyone.
    
  

   ation of cascade window

%%

%%
 dition}

    \textbf{Back to following a link:}
      
       A randomly chosen link, traversed in a random direction,
         leads to a degree $k$ node with probability \alertb{$\propto k P_k$}.
       Follows from there being $k$ ways
        to connect to a node with degree $k$.
       Normalization:
        $$\sum_{k=0}^\infty {kP_k} = \avg{k}$$
       So
        $$ 
        P(\mbox{linked node has degree $k$})
        =
        \frac{kP_k}{\tavg{k}}
        $$
      
    

  }

    \textbf{Next: Vulnerability of linked node}
      
       
        Linked node is \alert{vulnerable}
        with probability 
        $$\beta_k = \int_{\phi_\ast'=0}^{1/k} f(\phi_\ast') \dee{\phi_\ast'}$$
       If linked node is \alertb{vulnerable}, it produces 
        \alert{$k-1$ new}
        outgoing active links
       If linked node is \alertb{not vulnerable}, it produces \alert{no} active links.
      
    

  }

    \textbf{Putting things together:}
      
      
        Expected number of active edges produced by an active edge:
      []
        $$
        R =
        \sum_{k=1}^{\infty} 
        \underbrace{
          \alert{(k-1)} 
          \cdot
          \beta_k 
          \cdot
          \frac{k P_k}{\tavg{k}} 
        }_{\textrm{success}}
        \quad
        + 
        \quad
        {
          \underbrace{
            \alert{0}
            \cdot
            (1 - \beta_k) 
            \cdot
            \frac{k P_k}{\tavg{k}}
          }_{\textrm{failure}}}
        $$
      []
        $$
        =
        \sum_{k=1}^{\infty} 
        (k-1) 
        \cdot
        \beta_k 
        \cdot
        \frac{k P_k}{\tavg{k}}
        $$
      
    

  }

    
      So... for random networks with fixed degree distributions,
      cacades take off when:
      $$
      \sum_{k=1}^{\infty}
      (k-1) 
      \cdot
      \beta_k 
      \cdot
      \frac{k P_k}{\tavg{k}} > 1.
      $$
      
       $\beta_k =$ probability a degree $k$ node is vulnerable.
       $P_k =$ probability a node has degree $k$.
      
    

}

    \textbf{Two special cases:}
      
       
        (1) Simple disease-like spreading succeeds: $\beta_k = \beta$
      []
        $$
        \beta 
        \cdot 
        \sum_{k=1}^{\infty} 
        (k-1) 
        \cdot 
        \frac{k P_k}{\tavg{k}} > 1.
        $$
       
        (2) Giant component exists: $\beta = 1$
      []
        $$
        1 \cdot
        \sum_{k=1}^{\infty} 
        (k-1) 
        \cdot 
        \frac{k P_k}{\tavg{k}}
        > 1.
        $$
      
    

dow 1
%% figure of basic cascade window outline

 random networks}
  
          
      \includegraphics[width=\textwidth,angle=-90]{figclusters_networks}\\
      
      
%       \alert{Top curve} = final fraction infected if successful.
%       \alert{Bottom curve} = fractional size of vulnerable cluster.
       Cascades occur only if size of max vulnerable cluster $>0$.\\
       
        System may be \alertb{`robust-yet-fragile'}.\nocite{carlson1999a,carlson2000a,sornette2003a}
       
        \alert{`Ignorance'} facilitates spreading.
      
      

 %   
%    

dom networks}

  \begin{center}
  \includegraphics[angle=-90,width=0.9\textwidth]{figcascadewindow_cut2}
  \end{center}

  
   \alert{`Cascade window'} widens as threshold $\phi$ decreases.
   Lower thresholds enable spreading.
  

dom networks}

  \begin{center}
  \includegraphics[width=\textwidth]{figclusters_networks_gimp2.png}
  \end{center}

etworks}
  \centering

  
    \includegraphics[height=.85\textheight]{figcascwind5_noname}\\
  

{block}{For our simple model of a uniform threshold:}
    
     
      \alertb{Low $\tavg{k}$:} No cascades in poorly connected networks.\\
      No global clusters of any kind.
     
      \alertb{High $\tavg{k}$:} Giant component exists but not enough vulnerables.
     
      \alertb{Intermediate $\tavg{k}$:} Global cluster of vulnerables exists.\\
      Cascades are possible in \alert{``Cascade window.''}
    
  

ess
\insertvideo{kMyIFrEZtnw}{15}{135}{Shareworthy interlude}
%% \insertvideo{kMyIFrEZtnw}{360}{480}{Shareworthy interlude}

\subsection{Final\ size}

tagion on random networks}

  
  
  
    \alert{Next:} Find expected fractional size of spread.
  
    Not obvious even for uniform threshold problem.
  
    Difficulty is in figuring out if and when
    nodes that need $\ge 2$ hits switch on.
  
    Problem \alertb{beautifully solved} for infinite seed case by Gleeson and Cahalane:\\
    ``Seed size strongly affects cascades on random networks,'' 
    Phys.\ Rev.\ E, 2007.\cite{gleeson2007a}
  
    Developed further by Gleeson
    in ``Cascades on correlated and modular random networks,'' 
    Phys.\ Rev.\ E, 2008.\cite{gleeson2008a}
  
  

{itemize}
     
      Randomly turn on a fraction $\phi_0$ of nodes at time $t=0$
     
      Capitalize on local branching network structure of random
      networks (again)
     
      Now think about what must happen for
      a specific node $i$ to become active at time $t$:
    [$\bullet$]
      \alertb{$t=0$:} $i$ is one of the seeds (prob = $\phi_0$)
    [$\bullet$]
      \alertb{$t=1$:} $i$ was not a seed but enough of $i$'s friends switched
      on at time $t=0$ so that $i$'s threshold is now exceeded.
    [$\bullet$] 
      \alertb{$t=2$:} enough of $i$'s friends and friends-of-friends switched
      on at time $t=0$ so that $i$'s threshold is now exceeded.
    [$\bullet$] 
      \alertb{$t=n$:} enough nodes within $n$ hops of $i$ 
      switched on at $t=0$ and their effects have propagated to reach $i$.
    
  

gle=-90,width=\textwidth]{figfullspread.pdf}
  \includegraphics[angle=-90,width=\textwidth]{figfullspread2.pdf}
  \includegraphics[angle=-90,width=\textwidth]{figfullspread3.pdf}
  \includegraphics[angle=-90,width=\textwidth]{figfullspread4.pdf}
  \includegraphics[angle=-90,width=\textwidth]{figfullspread5.pdf}

gle=-90,width=\textwidth]{figfullspread_circ1.pdf}
  \includegraphics[angle=-90,width=\textwidth]{figfullspread_circ2.pdf}
  \includegraphics[angle=-90,width=\textwidth]{figfullspread_circ3.pdf}
  \includegraphics[angle=-90,width=\textwidth]{figfullspread_circ4.pdf}
  \includegraphics[angle=-90,width=\textwidth]{figfullspread_circ5.pdf}

{itemize}
     
      Calculations are possible
      if nodes do not become inactive (strong restriction).
    
      Not just for threshold model---works
      for a wide range of contagion processes.
    
      We can analytically determine the entire time evolution,
      not just the final size.
    
      We can in fact determine \\
      $\Prob$(node of degree $k$ switching on at time $t$).
    
      Asynchronous updating can be handled too.
    
  

tness:}
    
     
      \alertb{Taking off from a single seed story} is about \alert{expansion} away from
      a node.
     
      \alertb{Extent of spreading story} is about \alert{contraction} at
      a node.
    
    \includegraphics[width=\textwidth]{contraction-expansion-tp-10}
  

{itemize}
  
    \alert{Notation:}
    $ \phi_{k,t} = 
    \Prob(\mbox{a degree $k$ node is active at time $t$}) $.
  
    \alert{Notation:}
    $\infprob_{k j} = \Prob$ (a degree $k$ node becomes active
    if $j$ neighbors are active).
   
    Our starting point: $ \phi_{k,0} = \phi_0$.
  
    $ 
    \alertb{\binom{k}{j}
      \phi_0^{\, j}
      (1-\phi_0)^{k-j} }
    $ 
    =
    $\Prob$ ($j$ of a degree $k$ node's neighbors were seeded at time $t=0$).
   
    Probability a degree $k$ node was a seed at $t=0$ is $\alert{\phi_0}$ (as above).
   
    Probability a degree $k$ node was not a seed at $t=0$ is $\alert{(1-\phi_0)}$.
   
    Combining everything, we have:
    $$ 
    \phi_{k,1}
    = 
    \alert{\phi_{0}}
    + 
    \alert{(1-\phi_{0})}
    \sum_{j=0}^{k}
    \alertb{\binom{k}{j}
    \phi_0^{\, j}
    (1-\phi_0)^{k-j} }
  \infprob_{k j}.
    $$
  
  

{itemize}
  
    For general $t$, we need to know
    the probability an edge coming into a degree $k$ node
    at time $t$ is active.
  
    \alert{Notation:} call this probability $\theta_t$.
  
    We already know $\theta_0 = \phi_0$.
  
    Story analogous to $t=1$ case.  For node $i$:
    $$
    \phi_{i,t+1}
    = 
    \alert{\phi_{0}}
    + 
    \alert{(1-\phi_{0})}
    \sum_{j=0}^{k_i}
    \alertb{\binom{k_i}{j}
    \theta_t^{\, j}
    (1-\theta_t)^{k_i-j}}
    \infprob_{k_i j}.
    $$
  
    Average over all nodes to obtain expression for $\phi_{t+1}$:
    $$
    \phi_{t+1}
    = 
    \alert{\phi_{0}}
    + 
    \alert{(1-\phi_{0})}
    \sum_{k=0}^{\infty} P_k 
    \sum_{j=0}^{k}
    \alertb{\binom{k}{j}
    \theta_t^{\, j}
    (1-\theta_t)^{k-j}}
    \infprob_{kj}.
    $$
  
    So we need to compute $\theta_t$...  {\alertb{massive excitement...}}
  
  
  
nect $\theta_0$ to $\theta_1$:}
    
    
      $
      \theta_{1}
      =
      \phi_0 +
      $
      $$
      (1-\phi_0)
      \sum_{k=1}^{\infty}
      \alert{\frac{k P_k}{\tavg{k}}}
      \alertb{\sum_{j=0}^{k-1}}
      \binom{k-1}{j}
      \theta_{0}^{\ j}
      (1-\theta_{0})^{k-1-j}
      \infprob_{kj}
      $$
    
      $ \alert{\frac{kP_k}{\tavg{k}}} = R_k$ = $\Prob$ (edge connects to a degree $k$ node).
    
      \alertb{$\sum_{j=0}^{k-1}$} piece gives $\Prob$(degree node $k$ activates)
      of its neighbors $k-1$ incoming neighbors are active.
    
      $\phi_0$ and $(1-\phi_0)$ terms account for state of node at time $t=0$.
    
      See this all generalizes to give $\theta_{t+1}$ in terms of $\theta_{t}$...
    
  
d then nodes}
    
    
      $
      \theta_{t+1}
      =
      \underbrace{\phi_0}_{\alertb{\mbox{exogenous}}} 
      $
      $$
      +
      (1-\phi_0)
      \underbrace{
      \sum_{k=1}^{\infty}
      \frac{k P_k}{\tavg{k}}
      \sum_{j=0}^{k-1}
      \binom{k-1}{j}
      \theta_{t}^{\ j}
      (1-\theta_{t})^{k-1-j}
      \infprob_{kj}
      }_{\alertb{\mbox{social effects}}}
      $$
      with $\theta_0 = \phi_0$.
    
      $ 
      \phi_{t+1}
      = 
      $
      $$
      \underbrace{\phi_0}_{\alertb{\mbox{exogenous}}} 
      + 
      (1-\phi_{0})
      \underbrace{
      \sum_{k=0}^{\infty}
      P_k
      \sum_{j=0}^{k}
      \binom{k}{j}
      \theta_t^{\, j}
      (1-\theta_t)^{k-j} 
      \infprob_{kj}
      }_{\alertb{\mbox{social effects}}}.
      $$
    
  

derbrace{\phi_0}_{\alertb{\mbox{exogenous}}} 
      $
      $$
      +
      (1-\phi_0)
      \underbrace{
      \sum_{k=1}^{\infty}
      \frac{k P_k}{\tavg{k}}
      \sum_{j=0}^{k-1}
      \binom{k-1}{j}
      \theta_{t}^{\ j}
      (1-\theta_{t})^{k-1-j}
      \infprob_{kj}
      }_{\alertb{\mbox{social effects}}}
      $$
      $$
      \alert{= G(\theta_t;\phi_0)}
      $$
  

{itemize}
    
      Retrieve cascade condition for 
      spreading from a single seed in limit $\phi_0 \rightarrow 0$.
     
      Depends on map $\theta_{t+1} = G(\theta_{t};\phi_0)$.
     
      First: if self-starters are present, some activation is assured:
      $$
      G(0;\phi_0) = 
      \sum_{k=1}^{\infty} 
      \frac{kP_k}{\tavg{k}}
      \bullet
      \infprob_{k0} > 0.
      $$
      meaning $\infprob_{k0}>0$ for at least one value of $k \ge 1$.
     
      If $\theta=0$ is a fixed point of $G$ (i.e., $G(0;\phi_0) = 0$)
      then spreading occurs if
      $$
      G'(0;\phi_0) = 
      \sum_{k=0}^{\infty} 
      \frac{k P_k}{\tavg{k}}
      \bullet
      (k-1) 
      \bullet
      \infprob_{k1} > 1.
      $$
      \insertassignmentquestionsoft{09}{9}
    
  

 words:}    
    
     
      If $G(0;\phi_0) > 0$, spreading must occur because
      some nodes turn on for free.
     
      If $G$ has an \alert{unstable fixed point} at $\theta = 0$,
      then cascades are also always possible.
    
  

  \textbf{Non-vanishing seed case:}
    
     
      Cascade condition is more complicated for
      $\phi_0 > 0$.
     
      If $G$ has a \alert{stable fixed point} at $\theta = 0$,
      and an \alert{unstable fixed point} for some $0 < \theta_\ast < 1$,
      then for $\theta_0  > \theta_\ast$, spreading takes off.
    
      Tricky point: $G$ depends on $\phi_0$, so as we change
      $\phi_0$, we also change $G$.
    
  

t story:}

  
          
      \includegraphics[angle=90,width=\textwidth]{figGfunction01.pdf}
      
      \includegraphics[angle=90,width=\textwidth]{figGfunction02.pdf}
      
      \includegraphics[angle=90,width=\textwidth]{figGfunction03.pdf}
    
    
     
      Given $\theta_0 (= \phi_0)$, $\theta_\infty$ will be 
      the nearest stable fixed point, either above or below.
     
      n.b., adjacent fixed points must have opposite stability types.
     
      \alert{Important:}
      Actual form of $G$ depends on $\phi_0$.  
    
      So choice of $\phi_0$ dictates both $G$ and starting
      point---can't start anywhere for a given $G$.
    
  

{Spreading success}



%% %%%%%%%%%%%%%%
%%   %% initiators %
%% %%%%%%%%%%%%%%
%% 
%%   %% cascade window
%%   %% comparison between different types of initiators
%%   %% 
%% %%
%% itiators, $\phi=0.18$}
%%   Activate random individuals:\\
%%   \centering
%%   \includegraphics[width=0.62\textwidth]{figtest_nw_threshold_cwi04c_noname}\\
%%   Cascade initiators for $k_{\textrm{init}}=1$, 2, 3, 4, 6, and 9.
%%   %% 
%%   %% 
%% 
%%   %% cascade window
%%   %% comparison between different types of initiators
%% 
%%   %%  
%% tering
%%   \includegraphics[width=0.65\textwidth]{figtest_nw_threshold_cwi04i_noname}\\
%%   Averagely connected nodes versus nodes in top 10 percent.
%%   
%%   
%%   
%%   
%%   %% average degree of everyone versus initiators
%%   
%%   \centering
%% cludegraphics[width=0.6\textwidth]{figtest_network_threshold2_19x1e_noname}\\
%%   \alertb{dashed line:} mean degree of all individuals with $k>0$.\\
%%   \alertb{solid line:} mean degree of cascade initiators.
%%   
%%   
%%  of initiator degrees
%% 
%% %%
%% %% itiators---degree distributions}
%% %%   \[
%% %%   \begin{array}{l}
%% %%     P_{k,\textrm{init}}  = [1 - (1-S)^k] \cdot P_k\\
%% %%     \\
%% %%     \qquad = P({\textrm{cascade}} | k) \cdot P_k \\
%% %%     \\
%% %%     \\
%% %%     S = \mbox{size of} \\
%% %%     \qquad \mbox{vulnerable cluster} \\
%% %%   \end{array}
%% %%   \ \ \raisebox{-5cm}{
%% %%     \includegraphics[width=0.5\textwidth]{figtest_nw_thr2_06fh7_noname}
%% %%   }
%% %%   \]
%% %% 
%% %% 
%% %% 
%% %% 
%% %%   alysis}
%% %%   Sampling individuals $\propto$ number sampled $n$.\\
%% %%   Triggering relatively very costly $\rightarrow$ trigger one individual only.\\
%% %%   Choose most connected individual from $n$ samples.
%% %%   \includegraphics[width=0.45\textwidth]{figtest_nw_thr2_06fcost3_noname}
%% %%   \includegraphics[width=0.45\textwidth]{figtest_nw_thr2_06fcost4_noname}\\
%% %%   $\bullet$ $n \nearrow$ as cost $\searrow$.\\
%% %%   $\bullet$ Chosen individual's degree increases slowly with $n$.
%% %%   
%% %% 
%% %% 
%% 
%% 
%% %% %% mean degree of early adopters
%% %% %% + way disease spreads
%% %% 
%% %%   }
%% 
%%   \raisebox{-4cm}{    
%%     \includegraphics[width=0.62\textwidth]{figtest_nw_thr2_06eh2_6_noname}
%%   }
%%   \begin{tabular}{l}
%%     \alert{---} vulnerables  \\
%%     \alertb{---} non-vulnerables \\
%%     --- total
%%   \end{tabular}
%% 
%% 
%% %% 
%% %%
%% 
%%  to disease spreading models}
%%   
%%   \raisebox{-4cm}{    
%%     \includegraphics[width=0.45\textwidth]{figtest_nw_SIR_01e2_noname}
%%   }
%%   \begin{tabular}{l}
%%     SIR model on random graph \\
%%     \\
%%     early adopters \textit{always}\\
%%     above average \\
%%   \end{tabular}
%% 
%%   Probability of connecting
%%   to a $k$ degree node $\propto k P_k$.
%%   
%% {block}{}
    \begin{tabular}{ccccc}
      $t = 0 $ & {$t = 1$} & $ {t = 2$} & {$t =3 $} \\
      \includegraphics[width=0.2\textwidth]{figtest_nw_thr2_06e_1_mod_noname}&
      {\includegraphics[width=0.2\textwidth]{figtest_nw_thr2_06e_2_mod_noname}}&
      {\includegraphics[width=0.2\textwidth]{figtest_nw_thr2_06e_3_mod_noname}}&
      {\includegraphics[width=0.2\textwidth]{figtest_nw_thr2_06e_4_mod_noname}}\\
      {$t = 4$} & {$t = 6$} & {$ t = 8$} & {$t = 10 $} \\
      {\includegraphics[width=0.2\textwidth]{figtest_nw_thr2_06e_5_mod_noname}}&
      {\includegraphics[width=0.2\textwidth]{figtest_nw_thr2_06e_6_mod_noname}}&
      {\includegraphics[width=0.2\textwidth]{figtest_nw_thr2_06e_7_mod_noname}}&
      {\includegraphics[width=0.2\textwidth]{figtest_nw_thr2_06e_8_mod_noname}}\\
      {$t = 12} $ & {$t = 14$} & {$ t = 16$} & {$t = 18 $} \\
      {\includegraphics[width=0.2\textwidth]{figtest_nw_thr2_06e_9_mod_noname}}&
      {\includegraphics[width=0.2\textwidth]{figtest_nw_thr2_06e_10_mod_noname}}&
      {\includegraphics[width=0.2\textwidth]{figtest_nw_thr2_06e_11_mod_noname}}&
      {\includegraphics[width=0.2\textwidth]{figtest_nw_thr2_06e_12_mod_noname}}\\
      %% \includegraphics{figtest_nw_thr2_06e_13_mod_noname.ps,width=0.2\textwidth}&
      %% \includegraphics{figtest_nw_thr2_06e_14_mod_noname.ps,width=0.2\textwidth}&
      %% \includegraphics{figtest_nw_thr2_06e_15_mod_noname.ps,width=0.2\textwidth}
      %% & 
    \end{tabular}
    $$P_{k,t} \mbox{\ versus\ } k$$
  

tering
      \includegraphics[angle=-90,width=0.9\textwidth]{figaccinfl_paper_multiplier10_avg_x}

      
       Fairly uniform levels of individual influence.
       Multiplier effect is mostly below 1.
      
    

    

%    \includegraphics[width=0.48\textwidth]{fignw_threshold_ramify_multiplier20_21comb3mod_noname}
%% 
%% 

tering
      \includegraphics[angle=-90,width=0.9\textwidth]{fignw_threshold_gamma_multiplier13_avg_x}

      
       Skewed influence distribution example.
      
    
    

%    \includegraphics[width=0.48\textwidth]{fignw_threshold_ramify_multiplier20_21comb3mod_noname}
%% 
%% 

connected, room for a general model
%%    rumours don't spread like SIR models
%%    diseases aren't like threshold models


%%%%%%%%%%%%%% extensions
%%%%%%%%%%%%%% threshold model
%%%%%%%%%%%%%% 1. group structure

etworks can act as triggers}

  
    \includegraphics[width=0.8\textwidth]{betheladder5}
    
     $\phi=1/3$ for all nodes
    
  
  
  
{Groups}

{columns}
   
   \includegraphics[height=0.8\textheight]{despair_teamwork.jpg}\\
   {\tiny \url{despair.com}}
   
   \alertb{
     ``A few harmless flakes working together can unleash
     an avalanche of destruction.''
     }
 
s}

  
    
     
      Assumption of sparse interactions is good
     
      Degree distribution is (generally) key to a network's function
     
      Still, random networks don't represent all networks
     
      Major element missing: \alert{group structure}
    
  

etworks}

  
    \centering
    \includegraphics[width=0.48\textwidth]{ramifiednetwork}
    
    $p$ = intergroup connection probability\\
    $q$ = intragroup connection probability.
  
 
tering
   \includegraphics[height=0.75\textheight]{bipartite}
 
   % boards of directors
   % movies
   % transportation
 
 
 
 
ce}
 
   \centering
   \includegraphics[width=1\textwidth]{bipartite2}
  
 model}
 
   \centering
   \includegraphics[width=1\textwidth]{generalcontext2}
 
   (Blau \& Schwartz, Simmel, Breiger)
 
 
 % \includegraphics[width=0.6\textwidth]{figgroupcascade_good}
 
 %% cascade windows for group based networks

 model networks with triadic closure}

  
    
     Connect nodes with probability $\propto \exp^{-\alpha d}$\\
      where\\
      $\alpha$ = homophily parameter\\
      and \\
      $d$ = distance between nodes (height of lowest common ancestor)
    
      $\tau_1$ = intergroup probability of friend-of-friend connection
    
      $\tau_2$ = intragroup probability of friend-of-friend connection
    
  
 
etworks}

  \includegraphics[width=1\textwidth]{figgroupcascade3}\\

etworks}
%% 
%%   \centering
%%   \includegraphics[width=0.4\textwidth]{fignetwork_ramify_cw21_3b_noname}\\
%%   \includegraphics[width=0.4\textwidth]{fignetwork_multibip_cw03_3_noname}
%% 
%%   
%% 
%%   
%% etworks}
%%   
%%   \begin{tabular}{cc}
%%     \includegraphics[width=0.4\textwidth]{fignetwork_ramify_cw21_5b_noname} &
%%     \includegraphics[width=0.4\textwidth]{fignetwork_ramify_cw22_4b_noname} \\
%%     \includegraphics[width=0.4\textwidth]{fignetwork_multibip_cw03_5_noname} & 
%%     \includegraphics[width=0.4\textwidth]{fignetwork_multibip_cw03_4_noname}
%%   \end{tabular}
%% 

etworks:}

    
      \centering
      \includegraphics[angle=-90,width=0.8\textwidth]{fignw_threshold_ramify_multiplier20_21comb3mod_x}
      
       Multiplier almost always below 1.
      
    

etworks}


  
    \centering
    \includegraphics[angle=-90,width=\textwidth]{fignw_thr_ramify_startprob211_mod2_1_x}
    
     The most connected nodes aren't always the most `influential.'
     \alert{Degree assortativity} is the reason.
    
  
  
}

  \textbf{Summary}
    

     \alert{`Influential vulnerables'} are key to spread.
     Early adopters are mostly vulnerables.
     Vulnerable nodes important but not necessary.
     Groups may greatly facilitate spread.
     Seems that cascade condition is a global one.
     Most extreme/unexpected cascades occur in highly connected networks 
     `Influentials' are posterior constructs.\\
     Many potential influentials exist.
    
  
  
itiators not greatly above average.
%% %   Average initiators easier to find and influence.
%% %   Early adopters may be above or below average.
%%   
%% 
%% }

  \textbf{Implications}
  
  
    Focus on \alertb{the influential vulnerables}.
  
    Create entities that can be transmitted successfully
    through many individuals rather than broadcast from one `influential.'
  
    Only \alertb{simple ideas} can spread by word-of-mouth.\\
    \qquad (Idea of opinion leaders spreads well...)
  
    Want enough individuals who will adopt and display.
  
    Displaying can be \alertb{passive} = free (yo-yo's, fashion),\\
    or \alertb{active} = harder to achieve (political messages).
  
    Entities can be novel or designed to combine with others,
    e.g. block another one.
  
  

sertvideo{EwOA8AfeHM4}{}{}{Spreading and unspreading: Empires}


%% ces: Threshold models}
                                %
                                %  At time $t+1$, fraction rioting
                                %  = fraction with $\phi_\ast \le \phi_t$.
                                %
                                %  \[ \phi_{t+1} = \int_{0}^{\phi_t} f(\phi_\ast) \dee{\phi_\ast}
                                %  = \left. F(\phi_\ast) \right|_{0}^{\phi_t} = F(\phi_t) \]
                                %
                                %  $\Rightarrow$ Iterative maps of the interval.
                                %

                                %
                                %   of individual thresholds, $f(\phi_\ast)$ $\Rightarrow$ interval maps
                                %  \includegraphics{[],figthreshold2_noname.ps,width=0.45\textwidth}
                                %  \includegraphics{[],figthresholdF2b_noname.ps,width=0.45\textwidth}\\
                                %  $\Rightarrow$ Single stable state model
                                %


                                % cludegraphics[width=0.45\textwidth]{figthreshold3_noname}
%%   \includegraphics[width=0.45\textwidth]{figthresholdF3b_noname}
%% 
%%   Critical mass model
%% 
%% etworks}
                                %
                                %  \centering
                                %  \includegraphics[width=0.65\textwidth]{fignetwork_ud_meanfield03_pdd_noname}
                                %
                                %


                                %  dom networks}
%% 
%%   \centering
%%   \includegraphics[width=0.6\textwidth]{fignw_thr_ramify_startprob211_mod2_1_noname}
%% 
%% 
%% 
%% 
%% 
%% ition of model
                                % figure



%%   \caption{
%%     Examples of networks where (A) cascades are possible
%%     even when no vulnerable cluster of nonzero fractional
%%     size exists, and (C) activation spreads at an
%%     exponential rate also in the absence of a vulnerable
%%     cluster.  In both networks, all nodes have the
%%     same threshold $\phi=1/3$.  
%%     In plot A, the initiator $i_0$ and $i_0$'s 
%%     two neighbors are the only vulnerable nodes
%%     in the network.  The ladder then allows
%%     activation to propagate from left to right, and the cascade
%%     grows at a linear rate.
%%     The network shown in plot C is a renormalized
%%     version of a trivalent Bethe Lattice, which is shown in plot B.
%%     Each node of the Bethe lattice is replaced by two nodes and additional edges
%%     as shown.  Activation in the resulting network will spread providing
%%     both nodes in a single group are initially activated together,
%%     and the rate of spread will be exponential.


