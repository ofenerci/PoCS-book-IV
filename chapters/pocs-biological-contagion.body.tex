%% Community epidemiology!!!

%% add page/cites to papers on SARS
%% R_0
%% show how models worked

%% show how rhodes and anderson's sampling
%% is just very, very bad
%% an assignment question perhaps?

%% cites for ferguson et al, germann ..
%% longini
%% Longini, I. M. (1988) Math. Biosci. 90, 367–383.

%%% Todo 
%%% pull inserts out
%%% make new plots


%% to add here
%% herding models from economics
%% 


%%%%%%%%%%%%%%%%%%%%%%%%%

%% Today's menu
%% Simple disease spreading models
%% Epidemic size distributions
%% 
%% Simple social contagion models
%% Threshold models
%% 
%% Generalized model of contagion
%% How many types of contagion are there?
%% Can we categorize real-world contagions?
%% 
%% An online experiment: Contagious tastes


%%%%%%%%%%%%%%%%%%%%%%%%%%%%%%%%%%%%%%%%%%%%%%%%

%% + number sections
%% + colour

%% really a thought piece
%% with some mathematical frills

%% change sweeping through a population

%% experiments: small world phenomenon, influence
%% theory: models of cooperation, search in social networks, contagion
%% collective problem solving
%% data: e-mail logs from columbia coupled with demographic data

%%%%%%%%%
%% forest fire models of disease spreading (5 dimensions)
%% wave equation models
%% 
%%%%%%%%%

%% tattoos

%%%%%%%%%%%%%% 3 mins
%% 1. intro/outline
%% some big deal pages
%% 1a. Outline

%%%%%%%%%%%%%% 3 mins
%% 1+. relationship between search and contagion

%%%%%%%%%%%%%
%% past work

%%%%%%%%%%%%%% 5 mins
%% 2a. disease models---independent interaction models

%%%%%%%%%%%%%% 5 mins
%% 2b. social sciences---threshold models
%%    1-d iterative maps -> critical mass

%%%%%%%%%%%%%% 2 mins
%% 2+. morris?

%%%%%%%%%%%%%% 2 mins
%% 2+. lopez? 

%%%%%%%%%%%%%% 2 mins
%% 2+. kempe, kleinberg


%%%%%%%%%%%%%% 5 mins
%% 3a. our model---definition

%%%%%%%%%%%%%% 5 mins
%% 3b. our model---results

%%%%%%%%%%%%%% 5 mins
%% 3c. our model---what it means
%% ideas for changing contagions

%%%%%%%%%%%%%% 2 mins
%% 4. conclusions
%%    future work

%% overlay
%% use \textcolor{white}

%% humour...
%% blue and red stability lines
%% i imagine those of you hoping for regime change within
%% the US in the coming year will find this colour
%% scheme appealing
%
%% ipods and muggers
%
%% our cooperation model is not cooperating, in fact
%% it appears to have defected just this morning
%% 
%% if you find equations disturbing, then
%% imagine some TV-like scrambling happening here

%% 30 minute talk

%%%%%%%%%%%%%% 3 mins
%% 1. intro/outline
%% some big deal pages

%% Definition
%% Motivation

\section{Introduction}

tagion}

  \textbf{A confusion of contagions:}
    
     
      Was Harry Potter some kind of virus?
     
      What about the Da Vinci Code?
     
      Did Sudoku spread like a disease?
     
      Language?  The alphabet?\cite{gleick2011a}
     
      Religion?
     
      Democracy...?
    
  

  % often bring natural concepts into
  % the description of
  % social and personal ones

}

  \textbf{Naturomorphisms}
    
    
      ``The feeling was contagious.''
    
      ``The news spread like wildfire.''
          
      \alertb{``Freedom is the most contagious virus known to man.''}\\
      ---Hubert H. Humphrey, Johnson's vice president
    
      ``Nothing is so contagious as enthusiasm.''\\
      ---Samuel Taylor Coleridge
    
  

}

  \textbf{Optimism according to \wordwikilink{http://en.wikipedia.org/wiki/Ambrose_Bierce}{Ambrose Bierce:}}
    The doctrine that everything is beautiful, including what is ugly,
    everything good, especially the bad, and everything right that is
    wrong. ...  {\alert{It is hereditary, but fortunately not contagious.}}
  
  
}

  \textbf{Eric Hoffer, 1902--1983}
    {
      There is a grandeur in the uniformity of the mass.
      }
    {
    \alertb{When a fashion, a dance, a song, a slogan or a joke }
      }
    {
      sweeps like \alert{wildfire} from one
      end of the continent to the other, 
    }
    {
      and \alertb{a hundred million people roar with laughter}, 
    }
    {
      sway their bodies in unison, 
      }
    {
    \alertb{hum one song} or \alert{break forth in anger and denunciation}, 
      }
    {
    there is the overpowering feeling 
    that in this country we have come nearer the brotherhood
    of man than ever before.
      }
  

  
   \wordwikilink{http://en.wikipedia.org/wiki/Eric_Hoffer}{Hoffer} was an interesting fellow...
  

{block}{}
    Hoffer's acclaimed work:
    \smallskip
    ``\alert{The True Believer}:\\
    Thoughts On The Nature Of Mass Movements'' (1951)\cite{hoffer1951a}
  

  \textbf{Quotes-aplenty:}
    
      ``We can be absolutely certain only about things we do not understand.''
     \alertb{``Mass movements can rise and spread without belief in a God, but never without belief in a devil.''}
     ``Where freedom is real, equality is the passion of the masses. 
      {Where equality is real, freedom is the passion of a small minority.}''
    
  
  
{columns}
   
   \includegraphics[width=\textwidth]{despair_conformity.jpg}\\
   {\tiny \url{despair.com}}
   
   \alertb{
     ``When people are free to do as they please, they usually imitate each other.''
     }

     \medskip

     ---Eric Hoffer\\
     ``The Passionate State of Mind''\cite{hoffer1954a}
 
s}
   
   \includegraphics[width=\textwidth]{despair_idiocy.jpg}\\
   {\tiny \url{despair.com}}
   
   \alertb{
     ``Never Underestimate the Power of Stupid People in Large Groups.''
     }
 
-disease spreading:}

  \textbf{Interesting infections:}
  
  
    %% walmart
    Spreading of certain buildings in the US:
    \youtubevideo{EGzHBtoVvpc}{}{}
  
    \wordwikilink{http://www.cnnbcvideo.com/?nid=VWB8OWHr.GqH2kYkPxOMwTQ1NDIxODA-}
    {2008 Viral get-out-the-vote video.}
  
  
sertvideo{9ihSeSToXOw}{}{}{Marbleization of the US:}

%% zombieland
%% \insertvideo{HQe2f8L-Rkc}{0}{55}{Rule \#2}

%% outbreak
\insertvideo{IXYUn5BOgHA}{}{}{Community---S2E6: Epidemiology}

g contagious outbreak?}

  \includegraphics[width=\linewidth]{zombies-ngrams.png}
}
  
  \textbf{Definitions}
    
    
      (1) The spreading of a quality or quantity
      between individuals in a population.
    
      (2) A disease itself:\\ the plague, a blight, the dreaded lurgi, ...
    
      from Latin: \alert{\textit{con}} = `together with'  + \alert{\textit{tangere}} `to touch.'
     
      Contagion has unpleasant overtones... 
     
      Just \alert{Spreading} might be a more neutral word
     
      But contagion is kind of exciting...
    
  

s}
  
  \textbf{Two main classes of contagion}
    
     \alert{Infectious diseases}{:\\
        tuberculosis, HIV, ebola, SARS, influenza, zombification, ...}
      \medskip
     \alert{Social contagion}{:\\ fashion, word
        usage, rumors, uprisings, religion, stories about zombies, ...}
    
  

s
%% 2a. disease models---independent interaction models

\section{Simple\ disease\ spreading\ models}

\subsection{Background}

%% 
%
%%   \textbf{I. Simple disease spreading models}
%
%%  \alertb{\ding{228} .}
%%  
%


%%%%%%%%%%%%%% extension of simple model



%% {block}{Mass action compartment models of infectious diseases}
%%     \bigskip
%%     %%       %%         \includegraphics[width=\textwidth]{SIRtransitions_std}
%%       %%       %%         SIR model of infectious disease dynamics:
%% 
%%         \medskip
%% 
%%         Three states:\\
%%         \alertb{S = Susceptible}\\
%%         \alertb{I = Infective/Infectious}\\
%%         \alertb{R = Recovered}\\
%%         (or Removed/Refractory)\\
%%         \medskip
%%         $S(t) + I(t) + R(t) = 1$
%%       %%     %%   
%% 
%% tagion\cite{murray2002a}}
%%    \bigskip
%%    %%      %%        {\alert{SIR model}}
%%
%%        \medskip
%%
%%        {Three states:}\\
%%
%%        \medskip
%%
%%        {\alertb{S = Susceptible}}\\
%%        {\alertb{I = Infective/Infectious}}\\
%%        {\alertb{R = Recovered}}\\
%%        {(or Removed/Refractory)}
%%
%%        \medskip
%%
%%        {$S(t) + I(t) + R(t) = 1$}
%%      %%      %%        {
%%          Discrete time automata example:
%%          \includegraphics[width=\textwidth]{SIRtransitions_std}
%%        }
%%      %%    %%  
%%
%%dard \alert{SIR model}\cite{murray2002a}}
    
     = basic model of disease contagion
     Three states:
      
       \alertb{S = Susceptible}
       \alertb{I = Infective/Infectious}
       \alertb{R = Recovered}
        { or Removed or Refractory}
      
       $S(t) + I(t) + R(t) = 1$
       Presumes random interactions (mass-action principle)
       Interactions are independent (no memory)
       Discrete and continuous time versions
    
  

{columns}
              \includegraphics[width=\textwidth]{SIRtransitions_std}
                    {Transition Probabilities:}

        \bigskip
        {\alertb{$\beta$} for being infected given contact with infected}\\
        {\alertb{$r$} for recovery}\\
        {\alertb{$\rho$} for loss of immunity}\\
            

al models attributed to}
    
     1920's: Reed and Frost
     1920's/1930's: Kermack and McKendrick\cite{kermack1927a,kermack1932a,kermack1933a}
     Coupled differential equations with a mass-action principle
    
  

dent Interaction models}

  \textbf{Differential equations for continuous model}
    $$
    \frac{\dee{}}{\dee{t}} S = - \beta  \alert{I S} + \rho R 
    $$
    $$
    \frac{\dee{}}{\dee{t}} I=  \beta \alert{I S} - r I 
    $$
    $$
    \frac{\dee{}}{\dee{t}} R=  r I - \rho R 
    $$
    $\beta$, $r$, and $\rho$ are now \alert{rates}.
  
  
{block}{\wordwikilink{http://en.wikipedia.org/wiki/Basic\_reproduction\_number}{Reproduction Number $R_0$}}
  
   
    $R_0$ = expected number of infected individuals resulting
    from a single initial infective
   
    Epidemic threshold: If $R_0 > 1$, `epidemic' occurs.
   
    Exponential take off: $R_0^{n}$ where $n$ is the number
    of generations.
   
    Fantastically awful notation convention: $R_0$ and the $R$ in $SIR$.
  
  

{block}{Discrete version:}
    
     Set up: One Infective in a randomly mixing population of Susceptibles
     At time $t=0$, single infective random bumps into a Susceptible
     Probability of transmission = $\beta$
     At time $t=1$, single Infective remains infected with probability $1-r$
     At time $t=k$, single Infective remains infected with probability $(1-r)^k$
    
      

{block}{Discrete version:}
    
     Expected number infected by original infective:
      $$
      R_0 = \beta + (1-r)\beta + (1-r)^2\beta + (1-r)^3\beta + \ldots
      $$
      {
        $$
        = \beta\left( 1 + (1-r) + (1-r)^2 + (1-r)^3 + \ldots \right)
        $$
      }
      $$
      {= \beta \frac{1}{1 - (1-r)}}
      {\alert{= \beta/r}}
      $$
    
    {
      For $S(0) \simeq 1 $ initial susceptibles \\
      ($1-S(0) = R(0)$ = fraction initially immune):
      $$ R_0 = S(0) \beta/r $$
    }
  

dent Interaction models}

  \textbf{For the continuous version}
    
     Second equation:
      $$
      \diff{}{t} I = \beta S I - r I
      $$
      {
        $$
        \diff{}{t} I = (\beta S -r) I
        $$}
    
      Number of infectives grows initially if
      $$
      \beta S(0) - r > 0 
      {
      \  \alertb{\Rightarrow} \
        \beta S(0) > r
      }
      {
      \  \alertb{\Rightarrow} \
      \alert{\beta S(0)/r > 1}
      }
      $$
      where 
      $S(0) \simeq 1$.
     Same story as for discrete model.
    
  

d
\changelogotovideo{HQe2f8L-Rkc}{0}{55}

dependent Interaction models}

  \centering 

  Example of epidemic threshold:

  % add figure here
  \includegraphics[width=0.6\textwidth]{figR0_noname}

  
   {Continuous phase transition.}
   {Fine idea from a simple model.}
  

gelecturelogo{.18}{zombie-pumpkin.png}

dependent Interaction models}

  \textbf{Many variants of the SIR model:} 
    
     \alert{SIS}: susceptible-infective-susceptible 
     \alert{SIRS}: susceptible-infective-recovered-susceptible
     compartment models (age or gender partitions)
     more categories such as `exposed' (\alert{SEIRS})
     recruitment (migration, birth)
    
  

d to save the world:}

  \includegraphics[height=0.80\textheight]{Contagion-movies-wallpaper.jpg}


k{http://www.amazon.com/Z-Man-Games-ZMG-71100-Pandemic/dp/B00A2HD40E/}{Save
        the world yourself:}}
    \begin{center}
      \includegraphics[height=0.70\textheight]{pandemic-box-cover-tp-1.pdf}
    \end{center}
    
     
      Also contagious?: Cooperative games ...
    
    
  

d world with 
  \wordwikilink{http://vax.herokuapp.com}{Vax:}
  \bigskip
  \includegraphics[width=\textwidth]{vax.jpg}
{Prediction}

g models}

%%  \alertb{\ding{202}} 

  \textbf{For novel diseases:}
    
     Can we predict the size of an epidemic?
     How important is the reproduction number $R_0$?
    
  

  \textbf{$R_0$ approximately same for all of the following:}
    
       
      1918-19 ``Spanish Flu'' $\sim$ 500,000 deaths in US
       
      1957-58 ``Asian Flu'' $\sim$ 70,000 deaths in US 
       
      1968-69 ``Hong Kong Flu'' $\sim$ 34,000 deaths in US 
       
      2003 ``SARS Epidemic'' $\sim$ 800 deaths world-wide
    
  

{block}{Size distributions are important elsewhere:}
    
     
      earthquakes (Gutenberg-Richter law)
     
      city sizes, forest fires, war fatalities
     
      wealth distributions
     
      `popularity' (books, music, websites, ideas)
     
      \alert{Epidemics?}
    
  

  {Power laws distributions are common but not obligatory...}

  \textbf{Really, what about epidemics?}

    
      Simply hasn't attracted much attention.
      Data not as clean as for other phenomena.
    
    
  

d as Petri dish
%% map of Iceland

g Ill in \wordwikilink{http://en.wikipedia.org/wiki/Iceland}{Iceland}}
 
   % time series
   % distribution
 
   Caseload recorded monthly for range of diseases in Iceland, 1888-1990

   \bigskip
 
   \includegraphics[width=1\textwidth]{figiceland_all_measles03_noname}

   
    Treat outbreaks separated in time as `novel' diseases.
   
   
 
   % roughly constant population
 
   % rubella, influenza, measles, pertussis
 

 
 Iceland}


  Epidemic size distributions $N(S)$ for \\ Measles, Rubella, and Whooping Cough.

  \bigskip

  \includegraphics[width=1\textwidth]{figiceland_data_noname}\\

  Spike near $S=0$, relatively flat otherwise.

% time series
% distribution
  
serts out
%%% make new plots

cludegraphics[width=.48\textwidth]{fig2A} 
  \includegraphics[width=.45\textwidth]{fig2B} 
  
  {
    \alert{Insert plots:}\\ Complementary cumulative frequency distributions:
    $$\mbox{N}(\Psi' > \Psi) \propto \Psi^{-\gamma+1}$$
  }

  {Limited scaling with a possible break.}

{block}{Measured values of $\gamma$:}
    
     {measles: \alert{1.40} (low $\Psi$) and \alert{1.13} (high $\Psi$)}
     {pertussis: \alert{1.39} (low $\Psi$) and \alert{1.16} (high $\Psi$)}
    
  

  
   Expect $2 \le \gamma < 3$ (finite mean, infinite variance)
   When $\gamma < 1$, can't normalize 
   Distribution is quite \alert{flat}.
  

t size.
%%  
%%    Probability distribution function's tail for large $x$:
%%    $$ \mbox{Pr}(X = x) \propto x^{-\theta-1} $$
%%    
%%    Complementary cumulative distribution function:
%%    $$    \mbox{Pr}(X > x)  = \int_{x}^{\infty} \mbox{Pr}(X=x') \, \dee{x'} \ \ \propto \ \ x^{-\theta} $$
%%  
%%  %  \begin{align*}
%%  %    \mbox{Pr}(X > x) &  = \int_{x}^{\infty} \mbox{Pr}(X=x') \, \dee{x'}\\
%%  %    & \propto x^{-\theta} 
%%  %  \end{align*}
%%  
%%    Expect $1 \le \theta < 2$ (finite mean, infinite variance)
%%  
%% d pertussis}
%% 
%%   \raisebox{.4\textheight}{Measles:} 
%%   \includegraphics[width=.48\textwidth]{fig2A} 
%% 
%%   Complementary cdf: $\mbox{Pr}(X > x) \propto x^{-\theta}$
%%  
%%   $\theta = 0.40$ and $0.13$,
%%   both outside of range $1 \le \theta < 2$.
%% 
%%   \alertb{$\Rightarrow$ Distribution seems `flat.'}
%% 
%% cludegraphics[width=1.0\textwidth]{figSARS_data3_noname}

  
   Epidemic slows...  \\
    {then an infective moves to a new context.}
  Epidemic discovers new `pools' of susceptibles: \alert{Resurgence}.
   \alert{Importance of rare, stochastic events.}
  

terlude
%% the dean
\insertvideo{2CeGQTyj2cg}{0}{43}{Community---S2E6: Epidemiology}


%% %%
%%
%% t models}
%
%%   {\small
%%   \[
%%   \begin{array}{l}
%%     \frac{\dee{S}}{\dee{t}} = - \beta  I S \\
%%     \\
%%     \frac{\dee{I}}{\dee{t}} =  \beta I S - r I \\
%%     \\
%%     \frac{\dee{R}}{\dee{t}} =  r I \\
%%     \\
%%     \mbox{\alertr{Reproduction number}:} \\
%%     \\
%%     \mbox{\alertr{$R_0 = S(0) \beta/r$}} \\
%%   \end{array}
%%   % state transition diagram
%%   \raisebox{-4cm}{\includegraphics[width=0.3\textwidth}]{SIRtransitions_std}
%%   \raisebox{-2cm}{\includegraphics[width=0.3\textwidth}]{figR0_noname}
%%   \]
%% }
%% 
%%   \centering
%%   \alertb{Epidemic threshold: If $R_0 > 1$, `epidemic' occurs.}\\
%%   \ding{228} Fine idea from a simple model.
%% 


\subsection{More\ models}

ge}

  \textbf{So...  can a simple model produce}
    
     \alert{broad epidemic distributions} \\
      and
     \alert{resurgence} ?
    
  
  
  
ipartite networks

s}

            \includegraphics[width=1\textwidth]{fig4A}      
              Simple models typically produce
      \alert{bimodal} or \alert{unimodal} size distributions.
      
  
   This \alert{includes} network models:\\ random, small-world, scale-free, ...
   Exceptions:
    
     Forest fire models
     Sophisticated metapopulation models
    
  
 
gelecturelogowithlink{.18}{http://xkcd.com/793/}{xkcd-793-physicists.png}

%% 

ing through the population}

  \textbf {Forest fire models:\cite{rhodes1996a} }
    
     Rhodes \& Anderson, 1996
     The physicist's approach:\\
      \alert{``if it works for magnets, it'll work for people...''} 
    
   

   \textbf{A bit of a stretch:}
     
     
       Epidemics $\equiv$ forest fires\\ spreading on
       3-d and 5-d lattices.
      
       Claim Iceland and Faroe Islands exhibit power law
       distributions for outbreaks.
      
       Original forest fire model not completely understood.
     
   

cludegraphics[width=\textwidth]{rhodes1996afig1.pdf}

  From Rhodes and Anderson, 1996.
  
gelecturelogo{.18}{icons-lightbulb-tp.pdf}

 models:}

   % Longini has a hard `g'

  
    
      
      Multiscale models suggested earlier by others but not formalized
      (Bailey\cite{bailey1975a}, Cliff and Haggett\cite{cliff1981a}, Ferguson et al.)
     
      Community based mixing (two scales)---Longini.\cite{longini1988a}
     
      Eubank et al.'s 
      EpiSims/\wordwikilink{http://en.wikipedia.org/wiki/Transims}{TRANSIMS}---city simulations.\cite{eubank2004a}
     
      Spreading through countries---Airlines: Germann et al., Colizza et al.\cite{colizza2007a}
    
          
      
      \includegraphics[width=\textwidth]{gleam.jpg}
      
      
       
        \wordwikilink{http://www.gleamviz.org}{GLEAM}: 
        Global pandemic simulations by Vespignani et al.
      
    
  

sertvideo{2CeGQTyj2cg}{44}{95}{Community---S2E6: Epidemiology}

s}

  
    
     
      Vital work but perhaps hard to generalize from...
     
      $\Rightarrow$ Create a simple model involving multiscale travel
     Very big question: \alertr{What is $N$?}
     Should we model SARS in Hong Kong as spreading 
      in a neighborhood, in Hong Kong, Asia, or the world?
     For simple models, we need to know the final size beforehand...
    
  
 
{Toy\ metapopulation\ models}

 g simple models}

  \textbf{Contexts and Identities---Bipartite networks}
    \medskip
    \centering
    \includegraphics[height=0.5\textheight]{bipartite}
  

  
   boards of directors
   movies
   transportation modes (subway)
  
 
{block}{Idea for social networks: incorporate \alertb{identity}}
  
 
   \textbf{Identity is formed from attributes such as:}
   
    Geographic location
    Type of employment
    Age
    Recreational activities
   
   
 
   \textbf{Groups are crucial...}
     
      formed by people with at least one similar attribute
      
       Attributes $\Leftrightarrow$ 
       Contexts $\Leftrightarrow$ 
       Interactions $\Leftrightarrow$ 
       Networks.\cite{watts2002b}
     
   
 
teractions/network from identities}
 
   \includegraphics[width=1\textwidth]{bipartite2}

   \medskip
   Distance makes sense in identity/context space.
 
text space}
 
   \centering
   \includegraphics[width=1\textwidth]{generalcontext2}

   \bigskip

   (Blau \& Schwartz\cite{blau1984a}, Simmel\cite{simmel1902a}, Breiger\cite{breiger1974a})
 model
%%%%%%%%%%%%%%%%%%%%%%%%%%%%%%%%





t-based model:}

  \textbf{``Multiscale, resurgent epidemics in a hierarchical metapopulation model''}
    D.~J.~Watts, R.~Muhamad, D.~C.~Medina, and P~.S~.Dodds\\
    Proc. Natl. Acad. Sci.\\
    pp.~11157--11162, \textbf{102}, 2005.
  
  

  \textbf{Geography: allow people to move between contexts}
    
    
      Locally: standard SIR model with random mixing    
    
      discrete time simulation
    
      $\beta$ = infection probability 
     
      $\gamma$ = recovery probability
     
      \alertb{  $P$ = probability of travel}
     
       \alertb{  \alert{Movement distance:} $\mbox{Pr}(d) \propto \exp(-d/\xi)$ }
     
       $\xi$ = typical travel distance
    
  

{block}{Schematic:}
    \begin{center}
      \includegraphics[width=1\textwidth]{geogspread-fig1}    
    \end{center}
  

{Model\ output}

{block}{}
    
    
      Define $P_0$ = Expected number of infected individuals
      \alert{leaving} initially infected context.
    
      Need \alertb{$P_0 > 1$} for disease to spread (independent of $R_0$).
    
      Limit epidemic size by
      \alertb{restricting frequency of travel and/or range}
    
  
  
g $\xi$:}
    \centering
    \includegraphics[width=.7\textwidth]{fig3cropB}     
  

  
   
    Transition in expected final size based on typical
    movement distance {\alert{(sensible)}}
  
  
g $P_0$:}
    \centering
    \includegraphics[width=.7\textwidth]{fig3cropA} 
  

  
   
    Transition in expected final size based on typical
    number of infectives leaving first group {\alert{(also sensible)}}
  
    Travel advisories: $\xi$ has larger effect than $P_0$.
  

cludegraphics[width=.48\textwidth]{fig4B}
  \includegraphics[width=.48\textwidth]{fig4C}

  
   Flat distributions are possible for certain $\xi$ and $P$.
   Different $R_0$'s may produce similar distributions
   Same epidemic sizes may arise from different $R_0$'s
  

dard model:}
   \bigskip
   \bigskip
   \includegraphics[width=1\textwidth]{fig4D}

% \begin{tabular}{>{\PBS\raggedright\hspace{0pt}}m{0.2\textwidth}m{0.8\textwidth}}
%   \includegraphics[width=.7\textwidth]{fig4D}\\
% \end{tabular}

dard model with transport:}
    \bigskip

    \includegraphics[width=1\textwidth]{fig4E} 

    \includegraphics[width=1\textwidth]{fig4G} 

ce}

  +

  \alertb{broad epidemic size distributions}
  }

{Conclusions}

 clusions}
   
   
     
      
       For this model, epidemic size is highly unpredictable
      
       Model is more complicated than SIR but still simple
      
       We haven't even included normal social responses 
       such as travel bans and self-quarantine.
      
       The reproduction number $R_0$ is not terribly useful.
      
       $R_0$, however measured, is not informative
       about 
       
        
         how likely the observed epidemic size was,
        
         and how likely future epidemics will be.
       
      
       Problem: $R_0$ summarises \alertb{one} epidemic after the fact
       and enfolds movement, the price of bananas, everything.
     
   

 s}

   
     
      Disease spread highly sensitive to population structure
      \alertb{Rare events may matter enormously}\\
       {(e.g., an infected individual taking
         an international flight)}
      \alertb{More support for controlling population movement}\\
       {(e.g., travel advisories, quarantine)}
     
   
 
 s}

 \textbf{What to do:}
   
    Need to separate movement from disease
    $R_0$ needs a friend or two.
    Need $R_0>1$ and $P_0>1$ and $\xi$ sufficiently large\\
     for disease to have a chance of spreading
   
 

 \textbf{More wondering:}
   
    Exactly how important are rare events in disease spreading?
    Again, what is $N$?
   
 

{block}{Valiant attempts to use SIR and co. elsewhere:}
    
     Adoption of ideas/beliefs (Goffman \& Newell, 1964)\cite{goffman1964a}
     Spread of rumors (Daley \& Kendall, 1965)\cite{daley1965a}
     Diffusion of innovations (Bass, 1969)\cite{bass1969a}
     Spread of fanatical behavior (Castillo-Ch\'{a}vez \& Song, 2003)
     Spread of Feynmann diagrams (Bettencourt et al., 2006)
    
  

  % (religion, beliefs, rumors)

{Predicting\ social\ catastrophe}

g social catastrophe isn't easy...}

  \textbf{``Greenspan Concedes Error on Regulation''}
    
    
      \ldots humbled Mr.\ Greenspan admitted that he had put too much
      faith in the self-correcting power of free markets \ldots
%      and had failed
%      to anticipate the self-destructive power of wanton mortgage
%      lending.
    
      ``Those of us who have looked to the self-interest of lending
      institutions to protect shareholders' equity, myself included,
      are in a state of shocked disbelief''
     
      Rep.\ Henry
      A.\ Waxman: ``Do you feel
      that your ideology pushed you to make decisions that you wish
      you had not made?''
     
      Mr. Greenspan conceded: ``Yes, I've found a flaw. I don't know
      how significant or permanent it is. But I've been very
      distressed by that fact.''
    
  

  \wordwikilink{http://www.nytimes.com/2008/10/24/business/economy/24panel.html}{New York Times, October 23, 2008}

omics}  

  \textbf{Alan Greenspan (September 18, 2007):}
    \medskip
          
      {``I've been dealing with these big mathematical models of
        forecasting the economy ...}

      \medskip
      {\alertb{If I could figure out a way to determine whether or
          not people are more fearful or changing to more euphoric,}} 

      \medskip
      {I don't need any of this other stuff.}

      \medskip
      {\alertb{
          I could forecast the economy better than
          any way I know.''}
      }
      
      \includegraphics[width=\textwidth]{Greenspan.jpg}\\
      \tiny{\url{http://wikipedia.org}}
      

omics}  

  \textbf{Greenspan continues:}
    {``The trouble is that we can't figure that out. I've been in the
      forecasting business for 50 years.}
    {
      \alertb{I'm no better than I ever was,}}
    {
      \alertb{and nobody else is.}
    }
    {
      Forecasting 50 years ago was as good or as bad as
      it is today.}
    {
      \alertb{And the reason is that human nature hasn't changed.}
    }
     {
       We can't improve ourselves.''
     }
  

  {
    \textbf{Jon Stewart:}
      \medskip
              
        ``You just bummed the @*!\# out of me.''
        
        \includegraphics[width=\textwidth]{jon_stewart_header_01.jpg}\\
        \tiny{\url{wildbluffmedia.com}}
          
  }

%% full URL:
%% http://www.wildbluffmedia.com/2007/10/04/jon-stewart-launching-new-comedy-series-with-demitri-martin/

  \medskip

  {
    
     From \wordwikilink{http://www.thedailyshow.com}{the Daily Show} (September 18, 2007)
     The full inteview is \wordwikilink{http://thedailyshow.cc.com/videos/cenrt5/alan-greenspan}{here}.
    
  }

omics}  

  \textbf{James K.\ Galbraith:}
    
    
      [NYT] But there are at least 15,000 professional economists in this
      country, and you're saying only two or three of them foresaw the
      mortgage crisis?
      {\alertb{[JKG] Ten or 12 would be closer than two or three.}}
    
      [NYT] What does that say about the field of economics, which claims to be a
      science? 
      {
        \alertb{[JKG] It's an enormous blot on the reputation of the
          profession.}
      }
      {
        \alertb{There are thousands of economists. Most of them teach.}
      }
      {
        \alertb{And most of them teach a theoretical framework that has been shown to be
          fundamentally useless.}
      }
    
  

  From the \wordwikilink{http://www.nytimes.com/2008/11/02/magazine/02wwln-Q4-t.html}{New York Times, 11/02/2008}

