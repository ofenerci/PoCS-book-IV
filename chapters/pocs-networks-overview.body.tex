\section{Basic\ definitions}

{block}{}
    \includegraphics[width=\textwidth]{network_dictionary_cut.pdf}
  

ess:}
    \begin{center}
      \includegraphics[width=0.9\textwidth]{network_thesaurus_cut.pdf}
    \end{center}
  

s}
%% 
%%   \alert{Network:} (net + work, 1500's)
%%   \hfill
%%   \includegraphics[width=.07\textwidth]{wikipedia.jpg}
%% 
%%   \textbf{\alert{Noun:}}
%%     
%%      Any interconnected group or system
%%      Multiple computers and other devices connected together to share information
%%     
%%   
%% 
%%   \textbf{\alert{Verb:}}
%%     
%%      To interact socially for the purpose of getting connections or personal advancement
%%      To connect two or more computers or other computerized devices
%%     
%%   
%% 
%% {block}{
      From Keith Briggs's excellent
      \wordwikilink{http://keithbriggs.info/network.html}{etymological investigation:}}
          
      
      
       
        Opus reticulatum:
       
        A Latin origin?
      
      
      \includegraphics[width=\textwidth]{opus_reticulatum.jpg}\\
      {\tiny [http://serialconsign.com/2007/11/we-put-net-network]}
      

{block}{First known use: Geneva Bible, 1560}
    `And thou shalt make unto it a grate like networke of brass (Exodus xxvii 4).'
  

  \textbf{From the OED via Briggs:}
    
     
      1658--: reticulate structures in animals
     
      1839--: rivers and canals
     
      1869--: railways
     
      1883--: distribution network of electrical cables
     
      1914--: wireless broadcasting networks
    
  

{block}{Net and Work are venerable old words:}
    
    
      \alert{`Net'} first used to mean spider web 
      {\small (King {\AE}lfr\'{e}d, 888)}.
    
      \alert{`Work'} appear to have long meant purposeful action.
    
  

      
    \includegraphics[width=\textwidth]{briggs2005a_fig1}
    
    \includegraphics[width=\textwidth]{briggs2005a_fig2}
  
  
    
    
      `Network' = something built
      based on the idea of natural, flexible lattice or web.
     
      c.f., ironwork, stonework, fretwork.
    
  

{block}{}
  
   
    Many \alert{complex systems}\\ 
    can be viewed as \alert{complex networks}\\
    of physical or abstract interactions.
   
    Opens door to mathematical and numerical analysis.
    
    Dominant approach of last decade of 
    a \alertb{theoretical-physics/stat-mechish} flavor.
    
    Mindboggling amount of work published 
    on complex networks since 1998...
    
    ... largely due to your typical theoretical physicist:
          
      
      \smallskip
              
        \includegraphics[width=\textwidth]{piranha3-tp.pdf}
        
        
         \textit{Piranha physicus}
         Hunt in packs.
         Feast on new and interesting ideas \\
          {\small (see chaos, cellular automata, ...)}
        
            
    
  
{block}{``Collective dynamics of `small-world' networks''\cite{watts1998a}}
    
    [] 
      Duncan Watts and Steve Strogatz\\
      Nature, 1998
    [] 
      \wordwikilink{http://scholar.google.com/citations?view\_op=view\_citation\&hl=en\&user=LhOAiXMAAAAJ\&citation\_for\_view=LhOAiXMAAAAJ:u5HHmVD\_uO8C}
      {Times cited: {\alert{$\sim 23,732$}} }
      {\tiny(as of September 23, 2014)}
        %% http://scholar.google.com/citations?view_op=view_citation&hl=en&user=LhOAiXMAAAAJ&citation_for_view=LhOAiXMAAAAJ:u5HHmVD_uO8C
        %% 
      
    

  \textbf{``Emergence of scaling in random networks''\cite{barabasi1999a}}
    
    [] 
      L\'{a}szl\'{o} Barab\'{a}si and R\'{e}ka Albert\\
      Science, 1999
    [] 
      \wordwikilink{http://scholar.google.com/citations?view\_op=view\_citation\&hl=en\&user=vsj2slIAAAAJ\&citation\_for\_view=vsj2slIAAAAJ:u5HHmVD\_uO8C}
      {Times cited: {\alert{$\sim 20,734$}}}
      {\tiny(as of September 23, 2014)}
      %% http://scholar.google.com/citations?view_op=view_citation&hl=en&user=vsj2slIAAAAJ&citation_for_view=vsj2slIAAAAJ:u5HHmVD_uO8C
    
  
{block}{Review articles:}
    
    
      S. Boccaletti et al.,\\
      Physics Reports, 2006,\\
      \alertb{``Complex networks: structure and dynamics''}\cite{boccaletti2006a}\\
      \wordwikilink{http://scholar.google.com/citations?view\_op=view\_citation\&hl=en\&user=BEC76f4AAAAJ\&citation\_for\_view=BEC76f4AAAAJ:u5HHmVD\_uO8C}{Times
        cited: \alert{$\sim$ 4,925}} 
      {\tiny(as of September 23, 2014)}
      %% http://scholar.google.com/citations?view_op=view_citation&hl=en&user=BEC76f4AAAAJ&citation_for_view=BEC76f4AAAAJ:u5HHmVD_uO8C
     
      M. Newman,\\
      SIAM Review, 2003,\\
      \alertb{``The structure and function of complex networks''}\cite{newman2003a}\\
      \wordwikilink{http://scholar.google.com/scholar?cites=12945060519911641528\&as\_sdt=5,46\&sciodt=0,46\&hl=en}
      {Times cited: \alert{$\sim$ 11,550}} 
      {\tiny(as of September 23, 2014)}
      %% http://scholar.google.com/scholar?cites=12945060519911641528&as_sdt=5,46&sciodt=0,46&hl=en
     
      R.\ Albert and A.-L.\ Barab\'{a}si\\
      Reviews of Modern Physics, 2002,\\
      \alertb{``Statistical mechanics of complex networks''}\cite{albert2002a}\\
      \wordwikilink{http://scholar.google.com/citations?view\_op=view\_citation\&hl=en\&user=d27Ji6kAAAAJ\&citation\_for\_view=d27Ji6kAAAAJ:WF5omc3nYNoC}
      {Times cited: \alert{$\sim$ 14,298}} 
      {\tiny(as of September 23, 2014)}
      %% http://scholar.google.com/citations?view_op=view_citation&hl=en&user=d27Ji6kAAAAJ&citation_for_view=d27Ji6kAAAAJ:WF5omc3nYNoC
    
  

k to other courses around the world on the main site
%%
%% Lada Adamic
%%
%% David Easley and Jon Kleinberg (Economics and Computer Science, Cornell)
%% 
%%  Mark Newman (Physics, Michigan)\\
%%   \textbf{Textbooks:}
%%     
%%        \small
%%         Mark Newman (Physics, Michigan)\\
%%         \alertb{``Networks: An Introduction''}
%%         \wikilink{http://www.amazon.com/Networks-Introduction-Mark-Newman/dp/0199206651}
%%        \small
%%         
%%         \alertb{``Networks, Crowds, and Markets: Reasoning About a Highly Connected World''}
%%         \wikilink{http://www.cs.cornell.edu/home/kleinber/networks-book/}
%%     
%%   
%% 
%% {block}{Textbooks:}
    
       \small
        Mark Newman (Physics, Michigan)\\
        \alertb{``Networks: An Introduction''}
        \wikilink{http://www.amazon.com/Networks-Introduction-Mark-Newman/dp/0199206651}
       \small
        David Easley and Jon Kleinberg (Economics and Computer Science, Cornell)\\
        \alertb{``Networks, Crowds, and Markets: Reasoning About a Highly Connected World''}
        \wikilink{http://www.cs.cornell.edu/home/kleinber/networks-book/}
    
  

gpoint.jpg}
  {The Tipping Point: How Little Things can make a Big Difference}
  {Malcolm Gladwell\cite{gladwell2000a}}

  \bigskip

  \showbook{nexus.jpg}
  {Nexus: Small Worlds and the Groundbreaking Science of Networks}
  {Mark Buchanan}

ked.jpg}
  {Linked: How Everything Is Connected to Everything Else and What It Means}
  {Albert-Laszlo Barab\'{a}si}

  \bigskip

  \showbook{sixdegrees.jpg}
  {Six Degrees: The Science of a Connected Age}
  {Duncan Watts\cite{watts2003a}}

dbook of Graphs and Networks}
%% {editors: Stefan Bornholdt and H. G. Schuster\cite{bornholdt2003a}}
%% 
%% \bigskip
%% 
%% \showbook{evolutionofnetworks.jpg}
%% {Evolution of Networks}
%% {S. N. Dorogovtsev and J. F. F. Mendes\cite{dorogovtsev2003a}}
%% 
%% alysis.jpg}
%% {Social Network Analysis}
%% {Stanley Wasserman and Kathleen Faust\cite{wasserman1994a}}
%% 
%% \bigskip
%% 
%% \showbook{inthebeatofaheart.jpg}
%% {In the Beat of a Heart: Life, Energy, and the Unity of Nature}
%% {John Whitfield}
%% 
%% {itemize}
     
      \alertb{Complex Social Networks}---F. Vega-Redondo\cite{vega-redondo2007a}
     
      \alertb{Fractal River Basins: Chance and Self-Organization}---I. Rodr\'{\i}guez-Iturbe and A. Rinaldo\cite{rodriguez-iturbe1997a}
     
      \alertb{Random Graph Dynamics}---R. Durette
     
      \alertb{Scale-Free Networks}---Guido Caldarelli
     
      \alertb{Evolution and Structure of the Internet: A Statistical Physics Approach}---Romu Pastor-Satorras and Alessandro Vespignani
     
      \alertb{Complex Graphs and Networks}---Fan Chung
     
      \alertb{Social Network Analysis}---Stanley Wasserman and Kathleen Faust
     
      \alertb{Handbook of Graphs and Networks}---Eds: Stefan Bornholdt and H. G. Schuster\cite{bornholdt2003a}
     
      \alertb{Evolution of Networks}---S. N. Dorogovtsev and J. F. F. Mendes\cite{dorogovtsev2003a}
    
  

s
%%%%%%%%%%%%%%%%%%%%%%%%%

s}

  
  
  
    But surely \alert{networks aren't new}...
  
    Graph theory is well established...
  
    Study of social networks started in the 1930's...
  
    So why all this `new' research on networks?
  
    \alert{Answer:} \alertb{Oodles of Easily Accessible Data.}
  
    We can now inform (alas) our theories \\
    with a much more measurable reality.$^\ast$
   
    A worthy goal: establish \alertb{mechanistic explanations}.\\
    \medskip
    {
    {\small 
      $\mbox{}^\ast$\textit{If this is upsetting, maybe string theory is for you...}}
  }
  
  



{block}{}
    
    
      \alertb{Web-scale} data sets can be overly \alert{exciting}.
    
  
  
  \textbf{Witness:}
    
    
      The End of Theory: The Data Deluge Makes the Scientific Theory Obsolete (Anderson, Wired)
      \wikilink{http://www.wired.com/science/discoveries/magazine/16-07/pb\_theory\#}
    
      ``The Unreasonable Effectiveness of Data,''\\ Halevy et al.\cite{halevy2009a}.
    
      c.f. Wigner's ``The Unreasonable Effectiveness of Mathematics in the Natural Sciences''\cite{wigner1960a}
    
  

  \textbf{But:}
  
   
    For scientists, description is only part of the battle.
   
    We still need to \alertb{understand}.
  
  

itions
%%%%%%%%%%%%%%%%%%%%%%%%

itions}

  \textbf{\alert{Nodes} = A collection of entities 
      which have properties that
      are somehow related to each other}
    
      
      e.g., people, forks in rivers, proteins, webpages, organisms,...
    
  

  \textbf{\alert{Links} = Connections between nodes}
    
    
      \alert{Links} may be directed or undirected.
    
      \alert{Links} may be binary or weighted.
    
  

  {
    Other spiffing words: vertices and edges.
  }

s}
%% 
%%   \textbf{\alert{Links} = Connections between nodes}
%%     
%%     
%%       \alert{links}
%%       
%%        
%%       may be real and fixed (rivers),
%%        
%%       real and dynamic (airline routes), 
%%        
%%       abstract with physical impact (hyperlinks),
%%        
%%       or purely abstract (semantic connections between concepts).
%%       
%%     
%%       \alert{Links} may be directed or undirected.
%%     
%%       \alert{Links} may be binary or weighted.
%%     
%%   
%% 
%% s}

  \textbf{\alert{Node degree} = Number of links per node}
    
     Notation: Node $i$'s degree = $k_i$.
     $k_i$ = 0,1,2,\ldots.
     Notation: the average degree of a network = $\avg{k}$ \\
      {(and sometimes $z$)}
    
      Connection between number of edges $m$ and average degree:
      $$
      \tavg{k} = \frac{2m}{N}.
      $$
    
      \alertb{Defn:} ${\cal N}_i$ = the set of $i$'s $k_i$ neighbors
    
  

s}

  \textbf{Adjacency matrix:}
    
    
      We represent a directed network by a 
      matrix $A$ with link weight $a_{ij}$ for nodes $i$ and $j$
      in entry $(i,j)$.
    
      e.g.,
      $$
      A = \left[
        \begin{array}{ccccc}
          0 & 1 & 1 & 1 & 0\\
          0 & 0 & 1 & 0 & 1\\
          1 & 0 & 0 & 0 & 0 \\
          0 & 1 & 0 & 0 & 1 \\
          0 & 1 & 0 & 1 & 0 \\
        \end{array}
      \right]
      $$
    
      (n.b., for numerical work, we 
      always use sparse matrices.)
    
  

{Examples\ of\ Complex\ Networks}

{block}{So what passes for a complex network?}
    
     Complex networks are \alert{large} (in node number)
     Complex networks are \alert{sparse} (low edge to node ratio)
     Complex networks are usually \alert{dynamic} and \alert{evolving}
     Complex networks can be social, economic, natural, informational, abstract, ...
    
Physical networks
        
         River networks
         Neural networks
         Trees and leaves
         Blood networks
        

      
        
         The Internet
         Road networks
         Power grids
        

      

  \medskip

      
    \includegraphics[height=.28\textheight]{opte1105841711-LGL-2D-4000x4000.png} 
      %%      {\centering \tiny (\url{opte.com})}
    
    \includegraphics[height=.28\textheight]{Rivierescr.jpg}
    
    \includegraphics[height=.28\textheight]{BoucleSach_imacr.jpg} 
  
  
    
     \alert{Distribution} (branching) versus \alert{redistribution} (cyclical)
    
  

s}
    
    \textbf{Interaction networks}
      
        The Blogosphere
        Biochemical networks
        Gene-protein networks
        Food webs: who eats whom
        The World Wide Web (?)
        Airline networks
        Call networks (AT\&T)
        The Media
      
    
    
    \includegraphics[width=\textwidth]{datamining-core-2006-06-27.png}\\
    {\tiny \wordwikilink{http://datamining.typepad.com}{datamining.typepad.com}}
  
{columns}
      
      
       
        Hidalgo et al.'s ``The Product Space Conditions the Development of Nations''\cite{hidalgo2007a}
       
        How do products depend on each other, and how does this network evolve?
       
        How do countries depend on
        each other for water, energy, people (immigration), investments?
            
      
      \includegraphics[width=\textwidth]{spacelabelslegends.pdf}
      

s}
    
    \textbf{Interaction networks: social networks}
      
       Snogging
       Friendships
       Acquaintances
       Boards and directors
       Organizations %% formal and informal ties
       
        \wordwikilink{http://www.facebook.com}{facebook}
        \wordwikilink{http://www.twitter.com}{twitter},
      
    
    
    \includegraphics[width=\textwidth]{bearman_sex_network.jpg}\\
    {\tiny (Bearman \etal, 2004)} 
  
  
    
    
      `Remotely sensed' by:
      email activity, 
      instant messaging, 
      phone logs {\alert{(*cough*)}}.
    
  

_sex_network.jpg}\\

al networks}
    
     
      Consumer purchases \\
      {(Wal-Mart, Target, Amazon, ...)}
%% : $\approx 1 \ \mbox{petabyte} \ = 10^{15} \ \mbox{bytes}$)}
     
      Thesauri: Networks of words generated by meanings
     
      Knowledge/Databases/Ideas
     
      Metadata---Tagging:
      \wordwikilink{http://bit.ly}{bit.ly}
      \wordwikilink{http://www.flickr.com}{flickr}
    
  
      
    \includegraphics[width=0.8\textwidth]{delicious.pdf}
  
{center}
    \includegraphics[height=0.75\textheight]{bollen2009a_fig5.png}\\
    Bollen et al.\cite{bollen2009a};
    a higher resolution figure is
    \wordwikilink{http://www.plosone.org/article/slideshow.action?uri=info:doi/10.1371/journal.pone.0004803&imageURI=info:doi/10.1371/journal.pone.0004803.g005}{here}
  \end{center}

euralreboot{7xEX-48RHCY}{0}{72}{Dog has fun.}

%%%%%%%%%%%%%%%%%%%%%%%%%%%%%%%%%%%%%
% properties
%%%%%%%%%%%%%%%%%%%%%%%%%%%%%%%%%%%%%

\section{Properties\ of\ Complex\ Networks}

{block}{A notable feature of large-scale networks:}
    
    
      Graphical renderings are often just a big mess.
              
        
                  
          \includegraphics[height=\textwidth]{nw_purerandom_graphviz01_10}
          
          
          [] 
            $\Leftarrow$ Typical hairball
           
            number of nodes $N$ = 500
           
            number of edges $m$ = 1000
           
            average degree $\tavg{k}$ = 4
          
                  
      And even when renderings somehow look good:\\
      {
      \alertb{``That is a very graphic analogy which aids 
      understanding wonderfully while being,
      strictly speaking, wrong in every possible way''} \\
      {\small
      said Ponder [Stibbons]
      ---\textit{Making Money}, T. Pratchett.
      }
      }
     
      We need to extract \alert{digestible, meaningful aspects}.
    
  

etworks:}
      
    
    
     \alertb{degree distribution}$^\ast$
     assortativity
     homophily
     clustering
     motifs
     modularity
    
    
    
     concurrency
     hierarchical scaling
     network distances
     centrality
     efficiency
     interconnectedness
     robustness
    
    
    

  
   
    Plus coevolution of network structure 
    \\ and processes on networks.
  [$\ast$]
    Degree distribution is the elephant in the room that
    we are now all very aware of...
  

 $P_k$}
    
    
      $P_k$ is the probability that a randomly selected
      node has degree $k$.
    
      $k$ = node degree = number of connections.
    
      \alert{ex 1:}
      \erdosrenyi\ random networks have Poisson degree distributions: \\
      \insertassignmentquestionsoft{05}{5}
      $$ P_k = e^{-\tavg{k}} \frac{\tavg{k}^k}{k!} $$
    
      \alert{ex 2:}
      \alert{``Scale-free'' networks:}
      $P_k \propto k^{-\gamma}$ $\Rightarrow$ `hubs'.
    
      link cost controls skew.
    
      hubs may facilitate or impede contagion.
    
  
 
{itemize}
    
      \erdosrenyi\ random networks are a \alertb{\textit{mathematical construct}}.
    
      `Scale-free' networks are \alert{growing networks} that form
      according to a \alert{plausible mechanism}.
     Randomness is out there, just not to the degree of
       a completely random network.
    
  

{itemize}
     Social networks: \wordwikilink{http://en.wikipedia.org/wiki/Homophily}{Homophily} = birds of a feather
     e.g., degree is standard property for sorting:\\
      measure degree-degree correlations.
    
      \alert{Assortative} network:\cite{newman2002a} 
      similar degree nodes connecting to each other.\\
      {\textit{Often \alertb{social}: company directors, coauthors, actors.}}
    
      \alert{Disassortative} network: high degree nodes connecting to low degree nodes.\\
      {\textit{Often \alertb{techological} or \alertb{biological}: 
        Internet, WWW, protein interactions, neural networks, food webs.}}
    
  

{block}{4. Clustering:}
          
      \includegraphics[width=\textwidth]{clustering-sketch-C1-tp-3.pdf}
      
      
       Your friends tend to know each other.
       Two measures (explained on following slides):
        
         Watts \& Strogatz\cite{watts1998a}
          $$ 
          C_1 
          = 
          \avg{
            \frac{\sum_{j_1 j_2 \in {\cal N}_i} a_{j_1 j_2}}
            {k_i(k_i-1)/2}}_{i}
          $$  
         Newman\cite{newman2003a}
          $$ 
          C_2 
          = 
          \frac{3 \times \textrm{\#triangles}}
          {\textrm{\#triples} }
          $$ 
        
      
      

{block}{}
      
    
%%    \includegraphics[width=\textwidth]{clustering-sketch-C1-tp-3.pdf}\\
    Example network:\\
    \includegraphics[width=\textwidth]{clustering-sketch-example-network-tp-3.pdf}\\
    Calculation of $C_1$:\\
    \includegraphics[width=\textwidth]{clustering-sketch-example-network-C1-calculation-tp-3.pdf}
    
    
     $C_1$ is the \alert{average fraction of 
        pairs of neighbors who are connected}.
     Fraction of pairs of neighbors who are connected is
      $$ \frac{\sum_{j_1 j_2 \in {\cal N}_i} a_{j_1 j_2}}{k_i(k_i-1)/2} $$
      where
      $k_i$ is node $i$'s degree, and 
      ${\cal N}_i$ is the set of $i$'s neighbors.
    
      Averaging over all nodes, we have:\\
      $ C_1 = \frac{1}{n}{\sum_{i=1}^{n}\frac{\sum_{j_1 j_2 \in {\cal N}_i} a_{j_1 j_2}}{k_i(k_i-1)/2}} 
      { = \avg{\frac{\sum_{j_1 j_2 \in {\cal N}_i} a_{j_1 j_2}}{k_i(k_i-1)/2}}_{i} }$
    
    

gles}

  
      
    \centering
    Example network:\\
    \includegraphics[width=\textwidth]{clustering-sketch-example-network-tp-3.pdf}\\
    Triangles:\\
    \includegraphics[width=0.7\textwidth]{clustering-sketch-example-network-triangles-tp-3.pdf}\\
    Triples:\\
    \includegraphics[width=\textwidth]{clustering-sketch-example-network-triples-tp-3.pdf}
    
    
    
      Nodes $i_1$, $i_2$, and $i_3$ form a \alert{triple}
      around $i_1$ if $i_1$ is connected to $i_2$ and $i_3$.
    
      Nodes $i_1$, $i_2$, and $i_3$ form a \alert{triangle}
      if each pair of nodes is connected
     
      The definition
      $ C_2 = \frac{3 \times \textrm{\#triangles}}{\textrm{\#triples}} $
      measures the fraction of \alert{closed triples}
     
      The \alert{`3'} appears because for each
      triangle, we have 3 closed triples.
     
      Social Network Analysis (SNA): fraction of
      \alert{transitive triples}.
    
    

{block}{Sneaky counting for undirected, unweighted networks:}
    
    
      If the path $i$--$j$--$\ell$ exists 
      then $a_{ij} a_{j\ell} = 1$.
    
      Otherwise, $a_{ij} a_{j\ell} = 0$.
    
      We want $i \ne \ell$ for good triples.
    
      In general, a path of $n$ edges between nodes $i_{1}$ and $i_{n}$ travelling
      through nodes $i_{2}$, $i_{3}$, \ldots $i_{n-1}$ exists $\iff$
      $a_{i_{1}i_{2}} a_{i_{2}i_{3}} a_{i_{3}i_{4}} \cdots a_{i_{n-2}i_{n-1}} a_{i_{n-1} i_{n}}$ = 1.
    
      $$ 
      \#\textrm{triples}
      = 
      \frac{1}{2} 
      \left(
        \sum_{i=1}^{N}
        \sum_{\ell=1}^{N}
        \left[ 
          A^2 
        \right]_{i\ell}
        -
        \textrm{Tr}
        A^2
    \right)
      $$
    
      $$ 
      \#\textrm{triangles} 
      = 
      \frac{1}{6} 
      \textrm{Tr} 
      A^3 
      $$
    
  

{itemize}
   For sparse networks, $C_1$ tends to discount
    highly connected nodes.
   $C_2$ is a useful and often preferred variant
   In general, $C_1 \ne C_2$.
   $C_1$ is a global average of a local ratio.
   $C_2$ is a ratio of two global quantities.
  
  

{itemize}
  small, recurring functional subnetworks 
  e.g., Feed Forward Loop:
          
      \begin{center}
        \includegraphics[width=0.45\textwidth]{feedforwardloop-tp.pdf}%
      \end{center}
        Shen-Orr, Uri Alon, \etal\cite{shen-orr2002a}
  %% , Wiggins \etal
  
  

d structure/community detection:}
    \begin{center}
      \begin{tabular}{c}
        \includegraphics[height=0.6\textheight]{ncaa_annotated-tp-10}\\
        Clauset \etal, 2006\cite{clauset2006a}: NCAA football
      \end{tabular}
    \end{center}
  

currency:}
    
     
      transmission of a contagious element
      only occurs during contact
     
      rather obvious but easily missed in a simple model
     
      dynamic property---static networks are not enough
     
      knowledge of previous contacts crucial
     
      beware cumulated network data
     
      Kretzschmar and Morris, 1996\cite{kretzschmar1996a}
     
      ``Temporal networks'' become a concrete area of study 
      for Piranha Physicus in 2013.
    
  

-Strahler ratios:}
    
     Metrics for branching networks:
      
       Method for ordering streams hierarchically
      
        Number: $R_n = N_{\omega}/N_{\omega+1}$ 
      
        Segment length: $R_l = \tavg{l_{\omega+1}}/\tavg{l_{\omega}}$ 
      
        Area/Volume: $R_a = \tavg{a_{\omega+1}}/\tavg{a_{\omega}}$ 
      
    
          
      \begin{center}
        \includegraphics[height=0.4\textheight]{network1}%
        \includegraphics[height=0.4\textheight]{network2} 
        \includegraphics[height=0.3\textheight]{network3}%
      \end{center}
      

etwork distances:}
    \textbf{\alert{(a) shortest path length $d_{ij}$:}}
      
       Fewest number of steps between nodes $i$ and $j$.      
       (Also called the chemical distance between $i$ and $j$.)
      
    
    \textbf{\alert{(b) average path length $\tavg{d_{ij}}$:}}
      
        Average shortest path length in whole network.
        
        Good algorithms exist for calculation.
       
        Weighted links can be accommodated.
      
    

  

etwork distances:}
    
    
      \alert{network diameter $d_{\textrm{max}}$:}\\
      Maximum shortest path length between any two nodes.
    
      \alert{closeness $d_{\textrm{cl}} = [\sum_{ij} d_{ij}^{\ -1} / \binom{n}{2}]^{-1}$:}\\
      Average `distance' between any two nodes.
    
      Closeness handles disconnected networks ($d_{ij}=\infty$)
    
      $d_{\textrm{cl}} = \infty$ only when all nodes are isolated.
    
      Closeness perhaps compresses too much into one number
    
  

trality:}
    
     Many such measures of a node's `importance.'  
     \alert{ex 1:} Degree centrality: $k_i$.
     \alert{ex 2:} Node $i$'s betweenness \\
      = fraction of shortest paths that pass through $i$.
     \alert{ex 3:} Edge $\ell$'s betweenness \\
      = fraction of shortest paths that travel along $\ell$.
     \alert{ex 4:} Recursive centrality: Hubs and Authorities
      (Jon Kleinberg\cite{kleinberg1998a})
    
    
  

terconnected networks and robustness (two for one deal):}
    ``Catastrophic cascade of failures in interdependent networks''\cite{buldyrev2010a}.
    Buldyrev et al., Nature 2010.
    
    \includegraphics[width=0.8\textwidth]{buldyrev2010a_fig1.pdf}
  

{Nutshell}

{block}{Overview Key Points:}
    
    
      The field of complex networks came into
      existence in the late 1990s.
    
      Explosion of papers and interest since 1998/99.
    
      Hardened up much thinking about complex systems.
    
      Specific focus on networks that are 
      \alert{large-scale}, 
      \alertb{sparse}, 
      \alert{natural} or \alert{man-made}, 
      \alertb{evolving} and \alertb{dynamic}, 
      and 
      (crucially) \alert{measurable}.
    
      Three main (blurred) categories: 
      
       
      \alert{Physical} (e.g., river networks),
       
      \alert{Interactional} (e.g., social networks),
       
      \alert{Abstract} (e.g., thesauri).
      
    
    
  



